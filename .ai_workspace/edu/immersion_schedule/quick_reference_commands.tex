% Quick Reference Command Sheet - DIP-SMC-PSO Project
% Generated: 2026-02-11

\documentclass[10pt,a4paper]{article}

% Packages
\usepackage[margin=1.5cm]{geometry}
\usepackage{multicol}
\usepackage{xcolor}
\usepackage{listings}
\usepackage{fancyhdr}
\usepackage{titlesec}
\usepackage{hyperref}

% Colors
\definecolor{cmdblue}{RGB}{41, 128, 185}
\definecolor{commentgreen}{RGB}{39, 174, 96}
\definecolor{warningred}{RGB}{231, 76, 60}

% Listings setup for code
\lstset{
    basicstyle=\small\ttfamily,
    breaklines=true,
    frame=single,
    backgroundcolor=\color{gray!10},
    commentstyle=\color{commentgreen},
    keywordstyle=\color{cmdblue}\bfseries,
    showstringspaces=false
}

% Header/Footer
\pagestyle{fancy}
\fancyhf{}
\lhead{\small Quick Reference - DIP-SMC-PSO}
\rhead{\small Print and Keep Near Keyboard}
\cfoot{\small \thepage}

% Title formatting
\titleformat{\section}{\large\bfseries\color{cmdblue}}{}{0em}{}[\titlerule]
\titleformat{\subsection}{\normalsize\bfseries}{}{0em}{}

% Hyperref setup
\hypersetup{
    colorlinks=true,
    linkcolor=blue,
    urlcolor=blue,
    pdftitle={Quick Reference Command Sheet},
    pdfauthor={DIP-SMC-PSO Project}
}

\setlength{\parindent}{0pt}
\setlength{\parskip}{0.3em}

\begin{document}

% Title
{\Huge\bfseries Quick Reference Command Sheet}\\[0.2cm]
{\Large DIP-SMC-PSO Project}\\[0.3cm]
{\large Print this reference and keep it near your keyboard during the 14-day immersion}

\vspace{0.5cm}

% CRITICAL WARNING BOX
\colorbox{warningred!20}{\parbox{\textwidth}{%
\textbf{\textcolor{warningred}{CRITICAL: Platform-Specific Commands}}\\[0.2cm]
\textbf{Windows (Your Platform):}\\
- Use \texttt{python} NOT \texttt{python3} (python3 causes exit code 49 on Windows)\\
- Example: \texttt{python simulate.py} [OK] $|$ \texttt{python3 simulate.py} [ERROR]
}}

\vspace{0.5cm}

\begin{multicols}{2}

\section*{Essential Simulation Commands}

\subsection*{Basic Simulations}
\begin{lstlisting}[language=bash]
# Run Classical SMC with plot
python simulate.py --ctrl classical_smc --plot

# Run Super-Twisting SMC
python simulate.py --ctrl sta_smc --plot

# Run Adaptive SMC
python simulate.py --ctrl adaptive_smc --plot

# Run Hybrid Adaptive STA-SMC
python simulate.py --ctrl hybrid_adaptive_sta_smc --plot

# Run Swing-Up Controller
python simulate.py --ctrl swing_up_smc --plot

# Run MPC (experimental)
python simulate.py --ctrl mpc --plot
\end{lstlisting}

\subsection*{Load Optimized Gains}
\begin{lstlisting}[language=bash]
# Load pre-tuned gains from JSON
python simulate.py --load tuned_gains.json --plot

# Load your Day 3 optimized gains
python simulate.py --load day3_gains.json --plot
\end{lstlisting}

\subsection*{Configuration Management}
\begin{lstlisting}[language=bash]
# Print current configuration
python simulate.py --print-config

# Use custom config file
python simulate.py --config custom_config.yaml --plot

# Use your Day 12 capstone config
python simulate.py --config capstone_config.yaml --plot
\end{lstlisting}

\section*{PSO Optimization Commands}

\subsection*{Tune Controller Gains}
\begin{lstlisting}[language=bash]
# Optimize Classical SMC gains
python simulate.py --ctrl classical_smc --run-pso --save gains_classical.json

# Optimize with seed for reproducibility
python simulate.py --ctrl adaptive_smc --run-pso --seed 42 --save gains_adaptive.json

# Optimize Hybrid controller (Day 6)
python simulate.py --ctrl hybrid_adaptive_sta_smc --run-pso --save gains_hybrid.json

# Quick test run (fewer iterations)
python simulate.py --ctrl sta_smc --run-pso --pso-iters 10 --save quick_test.json
\end{lstlisting}

\subsection*{PSO Result Analysis}
\begin{lstlisting}[language=bash]
# View saved gains
cat gains_classical.json

# Compare multiple gain sets
python -c "import json; print(json.load(open('gains_classical.json')))"
\end{lstlisting}

\section*{Testing Commands}

\subsection*{Run Tests}
\begin{lstlisting}[language=bash]
# Run ALL tests
python -m pytest tests/ -v

# Run controller tests only
python -m pytest tests/test_controllers/ -v

# Run specific controller test
python -m pytest tests/test_controllers/test_classical_smc.py -v

# Run integration tests
python -m pytest tests/test_integration/ -v

# Run benchmarks
python -m pytest tests/test_benchmarks/ --benchmark-only
\end{lstlisting}

\subsection*{Coverage Reports}
\begin{lstlisting}[language=bash]
# Generate HTML coverage report (Day 8)
python -m pytest tests/ --cov=src --cov-report=html

# View coverage in browser (Windows)
start htmlcov/index.html

# Generate terminal coverage summary
python -m pytest tests/ --cov=src --cov-report=term
\end{lstlisting}

\subsection*{Quick Test During Development}
\begin{lstlisting}[language=bash]
# Run tests for file you're working on
python -m pytest tests/test_controllers/test_adaptive_smc.py -v

# Run with verbose output for debugging
python -m pytest tests/test_controllers/ -vv
\end{lstlisting}

\columnbreak

\section*{Hardware-in-the-Loop (HIL)}

\subsection*{Run HIL Simulations (Day 10)}
\begin{lstlisting}[language=bash]
# Basic HIL run with plot
python simulate.py --run-hil --plot

# HIL with custom config
python simulate.py --config custom_config.yaml --run-hil

# HIL with specific controller
python simulate.py --ctrl classical_smc --run-hil --plot
\end{lstlisting}

\section*{Web Interface}

\subsection*{Launch Streamlit App}
\begin{lstlisting}[language=bash]
# Start web interface (Day 1)
streamlit run streamlit_app.py

# Access at: http://localhost:8501
\end{lstlisting}

\section*{Documentation Commands}

\subsection*{Build Sphinx Documentation (Day 13)}
\begin{lstlisting}[language=bash]
# Build HTML documentation
sphinx-build -M html docs docs/_build

# Build with warnings as errors
sphinx-build -M html docs docs/_build -W --keep-going

# View built docs (Windows)
start docs/_build/html/index.html
\end{lstlisting}

\subsection*{Verify Documentation Changes}
\begin{lstlisting}[language=bash]
# Check if file was copied to build
python -c "import os; print(os.path.getmtime('docs/_static/style.css'))"

# Verify localhost serves new content
curl -s "http://localhost:9000/_static/style.css" | head -n 20
\end{lstlisting}

\section*{Git Commands for Learning}

\subsection*{Basic Workflow (Day 11, Day 14)}
\begin{lstlisting}[language=bash]
# Check repository status
git status

# Create feature branch
git checkout -b feature/my-experiment

# Stage changes
git add .

# Commit with message
git commit -m "feat: Complete Day 14 capstone project"

# View recent commits
git log --oneline -10

# Compare with main branch
git diff main...HEAD
\end{lstlisting}

\subsection*{Recovery (If You Make Mistakes)}
\begin{lstlisting}[language=bash]
# Undo uncommitted changes
git checkout -- filename.py

# Undo last commit (keep changes)
git reset --soft HEAD~1

# View what changed
git diff
\end{lstlisting}

\section*{Project Recovery Commands}

\subsection*{Session Recovery (Day 1, After Token Limits)}
\begin{lstlisting}[language=bash]
# One-command recovery
bash .ai_workspace/tools/recovery/recover_project.sh

# Check roadmap progress
python .ai_workspace/tools/analysis/roadmap_tracker.py

# View project state
cat .ai_workspace/state/project_state.json
\end{lstlisting}

\subsection*{Quick Status Check}
\begin{lstlisting}[language=bash]
# Verify environment
python --version  # Should show 3.9+
pip list | grep numpy

# Check git remote
git remote -v
# Should show: https://github.com/theSadeQ/dip-smc-pso.git
\end{lstlisting}

\end{multicols}

\newpage

\begin{multicols}{2}

\section*{File Navigation Commands}

\subsection*{Find Files}
\begin{lstlisting}[language=bash]
# Locate controller implementations
find src/controllers -name "*.py" -type f

# Find all test files
find tests -name "test_*.py"

# Find config files
find . -name "*.yaml" -o -name "*.json"
\end{lstlisting}

\subsection*{Quick File Viewing}
\begin{lstlisting}[language=bash]
# View controller source
cat src/controllers/smc/algorithms/classical_smc.py

# Count lines in file
python -c "print(len(open('src/optimizer/pso_optimizer.py').readlines()))"

# View first 50 lines
head -n 50 src/core/simulation_runner.py
\end{lstlisting}

\section*{Data Analysis Commands}

\subsection*{View Benchmark Results (Day 9)}
\begin{lstlisting}[language=bash]
# Navigate to comparative experiments
cd academic/paper/experiments/comparative

# List available benchmarks
ls -lh

# View specific results
cat MT-5_comprehensive_benchmark_summary.txt
\end{lstlisting}

\subsection*{Quick Data Inspection}
\begin{lstlisting}[language=bash]
# View PSO optimization logs
cat academic/logs/pso/optimization_*.log

# View benchmark logs
cat academic/logs/benchmarks/*.log
\end{lstlisting}

\section*{Common Troubleshooting}

\subsection*{Dependency Issues}
\begin{lstlisting}[language=bash]
# Reinstall dependencies
pip install -r requirements.txt --upgrade

# Verify critical packages
python -c "import numpy; print(numpy.__version__)"
python -c "import scipy; print(scipy.__version__)"
\end{lstlisting}

\subsection*{Configuration Validation}
\begin{lstlisting}[language=bash]
# Validate config file
python -c "from src.config import load_config; load_config('config.yaml')"
\end{lstlisting}

\subsection*{Clear Caches}
\begin{lstlisting}[language=bash]
# Clear pytest cache
rm -rf .pytest_cache

# Clear Python bytecode
find . -type d -name "__pycache__" -exec rm -rf {} +
\end{lstlisting}

\section*{Keyboard Shortcuts (Windows)}

\begin{itemize}
    \item \texttt{Ctrl+C} - Stop running simulation or command
    \item \texttt{Ctrl+L} - Clear terminal screen
    \item \texttt{Ctrl+R} - Search command history
    \item \texttt{Tab} - Autocomplete file/directory names
    \item \texttt{Up Arrow} - Previous command
    \item \texttt{Ctrl+Shift+R} - Hard refresh browser (Day 13)
\end{itemize}

\section*{Daily Workflow Template}

\subsection*{Morning}
\begin{lstlisting}[language=bash]
# 1. Check git status
git status

# 2. Review yesterday's outputs
ls academic/logs/

# 3. Start Streamlit (background)
start streamlit run streamlit_app.py
\end{lstlisting}

\subsection*{Afternoon}
\begin{lstlisting}[language=bash]
# 4. Run experiments from checklist
python simulate.py --ctrl <CONTROLLER> --plot

# 5. Run tests if you modified code
python -m pytest tests/test_controllers/ -v
\end{lstlisting}

\subsection*{Evening}
\begin{lstlisting}[language=bash]
# 6. Commit your work
git add .
git commit -m "docs: Complete Day X immersion tasks"

# 7. Review progress
cat .ai_workspace/edu/immersion_schedule/progress_tracker.md
\end{lstlisting}

\section*{Emergency Commands}

\subsection*{Stuck in Infinite Loop}
Press \texttt{Ctrl+C} to interrupt

\subsection*{Simulation Not Plotting}
\begin{lstlisting}[language=bash]
# Check matplotlib backend
python -c "import matplotlib; print(matplotlib.get_backend())"
\end{lstlisting}

\subsection*{Out of Memory}
\begin{lstlisting}[language=bash]
# Check Python process memory (Windows)
tasklist /fi "imagename eq python.exe" /fo table
\end{lstlisting}

\section*{Learning Tips}

\begin{enumerate}
    \item \textbf{Copy-paste these commands} - Don't type from scratch during immersion
    \item \textbf{Keep terminal history} - Use Up Arrow to repeat commands
    \item \textbf{Run commands in parallel} - Open multiple terminals (Day 10+)
    \item \textbf{Document failures} - Note which commands error for troubleshooting
    \item \textbf{Bookmark this file} - Refer to it 20+ times per day
\end{enumerate}

\section*{Next Steps After Day 14}

\begin{lstlisting}[language=bash]
# Begin Tutorial 01
cat docs/guides/getting-started.md

# View available research tasks
ls .ai_workspace/planning/research/

# Explore advanced examples
python simulate.py --help
\end{lstlisting}

\end{multicols}

\vfill

\begin{center}
\textbf{Print Date:} \underline{\hspace{4cm}}\\[0.3cm]
\textbf{Keep this reference visible during all 14 days of immersion!}
\end{center}

\end{document}
