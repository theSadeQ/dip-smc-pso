\section{Sliding Mode Controller Design}
\label{sec:controllers}

% Content to be extracted from docs/theory/classical_smc.md, sta_smc.md, adaptive_smc.md

\subsection{Classical SMC}
The sliding surface is defined as:
\begin{equation}
s(\vect{x}, t) = \lambda \vect{e} + \dot{\vect{e}}
\label{eq:sliding_surface}
\end{equation}

The control law ensures $s \to 0$ in finite time:
\begin{equation}
u = -K \sign(s)
\label{eq:classical_control}
\end{equation}

\subsection{Super-Twisting Algorithm (STA-SMC)}
To reduce chattering, STA uses continuous control with second-order sliding mode:
\begin{equation}
u = -k_1 |s|^{1/2} \sign(s) + u_1, \quad \dot{u}_1 = -k_2 \sign(s)
\label{eq:sta_control}
\end{equation}

\subsection{Adaptive SMC}
Gain adaptation handles model uncertainty:
\begin{equation}
\hat{K}(t) = \hat{K}(0) + \gamma \int_0^t |s(\tau)| d\tau
\label{eq:adaptive_gain}
\end{equation}

\subsection{Hybrid Adaptive STA-SMC}
Combines adaptive gains with super-twisting for robust, low-chattering control.

% Comparison table will be added from csv_to_table.py script
