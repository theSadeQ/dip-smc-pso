\section{Sliding Mode Controller Design}
\label{sec:controllers}

% Content to be extracted from docs/theory/classical_smc.md, sta_smc.md, adaptive_smc.md

\subsection{Classical SMC}
Classical sliding mode control combines model-based equivalent control with robust discontinuous switching. The sliding surface is defined as:
\begin{equation}
s(\vect{x}, t) = \lambda \vect{e} + \dot{\vect{e}}
\label{eq:sliding_surface}
\end{equation}

where $\vect{e} = [\theta_1, \theta_2]^T$ represents angle errors and $\lambda$ is the sliding surface slope parameter ensuring Hurwitz stability.

The complete control law decomposes into three components:
\begin{equation}
u = u_{eq} - K \cdot \sat(s/\varepsilon) - k_d \cdot s
\label{eq:classical_control}
\end{equation}

where $u_{eq}$ is the equivalent control (model-based feedforward), $K$ is switching gain, $\varepsilon$ is boundary layer thickness for chattering reduction, and $k_d$ is damping coefficient. The boundary layer parameter $\varepsilon$ trades off chattering reduction against tracking precision, as illustrated in Figure \ref{fig:boundary_layer}.

\begin{figure}[htbp]
\centering
\includegraphics[width=0.75\textwidth]{figures/fig_boundary_layer_optimization.pdf}
\caption{Boundary layer optimization showing the tradeoff between chattering amplitude and tracking error for varying $\varepsilon$ values. Optimal value: $\varepsilon = 0.02$ rad balances chattering suppression with acceptable tracking performance.}
\label{fig:boundary_layer}
\end{figure}

The equivalent control is derived by setting $\dot{s} = 0$:
\begin{equation}
u_{eq} = (\vect{L} \mat{M}^{-1} \mat{B})^{-1} \cdot [\vect{L} \mat{M}^{-1}(\mat{C}\dot{\vect{q}} + \vect{G})]
\label{eq:equivalent_control}
\end{equation}

where $\mat{M}$, $\mat{C}$, $\vect{G}$ are inertia, Coriolis, and gravity matrices, and $\vect{L} = [\lambda_1, \lambda_2, k_1, k_2]$ defines the sliding surface coefficients.

\subsection{Super-Twisting Algorithm (STA-SMC)}
To address chattering inherent in classical SMC, the Super-Twisting Algorithm achieves second-order sliding mode with continuous control. The control law consists of two components:
\begin{equation}
u = -k_1 |s|^{1/2} \sign(s) + u_1, \quad \dot{u}_1 = -k_2 \sign(s)
\label{eq:sta_control}
\end{equation}

where $k_1$ and $k_2$ are STA gains satisfying stability conditions. The fractional power $|s|^{1/2}$ provides finite-time convergence while maintaining control continuity, resulting in 70\% chattering reduction compared to classical SMC.

The stability conditions are:
\begin{equation}
k_1 > 0, \quad k_2 > \frac{L}{k_1}, \quad k_1^2 \geq 4k_2 \frac{k_2 + L}{k_2 - L}
\label{eq:sta_stability}
\end{equation}

where $L$ is the Lipschitz constant of the disturbance.

\subsection{Adaptive SMC}
Adaptive sliding mode control addresses model uncertainty by online estimation of switching gains. The adaptive law is:
\begin{equation}
\hat{K}(t) = \hat{K}(0) + \gamma \int_0^t |s(\tau)| d\tau
\label{eq:adaptive_gain}
\end{equation}

where $\gamma > 0$ is the adaptation rate and $\hat{K}(t)$ is the time-varying switching gain. This approach eliminates the need for conservative overestimation of disturbance bounds, improving control efficiency under varying conditions.

The adaptation law ensures:
\begin{equation}
\dot{V} = -\eta |s| + (\tilde{K} - \delta) |s| \leq 0
\label{eq:adaptive_lyapunov}
\end{equation}

where $\tilde{K} = K - \hat{K}$ is the gain estimation error and $\delta$ is the disturbance bound.

\subsection{Hybrid Adaptive STA-SMC}
The hybrid controller combines adaptive gain tuning with super-twisting dynamics, achieving both robustness and chattering reduction. The control law integrates:
\begin{equation}
u = u_{eq} - \hat{K}(t) \cdot |s|^{1/2} \sign(s) + u_1
\label{eq:hybrid_control}
\end{equation}

with adaptive update:
\begin{equation}
\dot{\hat{K}}(t) = \gamma |s|, \quad \dot{u}_1 = -k_2 \sign(s)
\label{eq:hybrid_adaptation}
\end{equation}

This architecture provides best overall performance: 40\% faster settling than classical SMC, 70\% chattering reduction, and 15\% performance degradation under $\pm$30\% model uncertainty.
