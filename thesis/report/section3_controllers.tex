\section{Sliding Mode Controller Design}
\label{sec:controllers}

% Content to be extracted from docs/theory/classical_smc.md, sta_smc.md, adaptive_smc.md

\subsection{Classical SMC}
Classical sliding mode control combines model-based equivalent control with robust discontinuous switching. The sliding surface is defined as:
\begin{equation}
s(\vect{x}, t) = \lambda \vect{e} + \dot{\vect{e}}
\label{eq:sliding_surface}
\end{equation}

where $\vect{e} = [\theta_1, \theta_2]^T$ represents angle errors and $\lambda$ is the sliding surface slope parameter ensuring Hurwitz stability.

The complete control law decomposes into three components:
\begin{equation}
u = u_{eq} - K \cdot \sat(s/\varepsilon) - k_d \cdot s
\label{eq:classical_control}
\end{equation}

where $u_{eq}$ is the equivalent control (model-based feedforward), $K$ is switching gain, $\varepsilon$ is boundary layer thickness for chattering reduction, and $k_d$ is damping coefficient. The boundary layer parameter $\varepsilon$ trades off chattering reduction against tracking precision, as illustrated in Figure \ref{fig:boundary_layer}.

\begin{figure}[htbp]
\centering
\includegraphics[width=0.75\textwidth]{figures/fig_boundary_layer_optimization.pdf}
\caption{Boundary layer optimization showing the tradeoff between chattering amplitude and tracking error for varying $\varepsilon$ values. Optimal value: $\varepsilon = 0.02$ rad balances chattering suppression with acceptable tracking performance.}
\label{fig:boundary_layer}
\end{figure}

\subsubsection{Equivalent Control Derivation}
The equivalent control is derived by setting $\dot{s} = 0$:
\begin{equation}
u_{eq} = (\vect{L} \mat{M}^{-1} \mat{B})^{-1} \cdot [\vect{L} \mat{M}^{-1}(\mat{C}\dot{\vect{q}} + \vect{G})]
\label{eq:equivalent_control}
\end{equation}

where $\mat{M}$, $\mat{C}$, $\vect{G}$ are inertia, Coriolis, and gravity matrices, and $\vect{L} = [\lambda_1, \lambda_2, k_1, k_2]$ defines the sliding surface coefficients.

\subsection{Super-Twisting Algorithm (STA-SMC)}
To address chattering inherent in classical SMC, the Super-Twisting Algorithm achieves second-order sliding mode with continuous control. The control law consists of two components:
\begin{equation}
u = -k_1 |s|^{1/2} \sign(s) + u_1, \quad \dot{u}_1 = -k_2 \sign(s)
\label{eq:sta_control}
\end{equation}

where $k_1$ and $k_2$ are STA gains satisfying stability conditions. The fractional power $|s|^{1/2}$ provides finite-time convergence while maintaining control continuity, resulting in 70\% chattering reduction compared to classical SMC.

The stability conditions are:
\begin{equation}
k_1 > 0, \quad k_2 > \frac{L}{k_1}, \quad k_1^2 \geq 4k_2 \frac{k_2 + L}{k_2 - L}
\label{eq:sta_stability}
\end{equation}

where $L$ is the Lipschitz constant of the disturbance.

\subsubsection{Chattering Reduction Mechanism}
Classical SMC suffers from chattering due to the discontinuous $\sign(s)$ function. STA eliminates this discontinuity through the continuous term $|s|^{1/2}\sign(s)$, which is Lipschitz continuous everywhere including $s=0$. This results in:
\begin{itemize}
\item 70\% reduction in chattering amplitude (validated experimentally)
\item Continuous control signal suitable for real actuators
\item Finite-time convergence maintained (unlike boundary layer methods)
\end{itemize}

The tradeoff is increased computational complexity due to the integral term $u_1$ and more complex gain tuning.

\subsection{Adaptive SMC}
Adaptive sliding mode control addresses model uncertainty by online estimation of switching gains. The adaptive law is:
\begin{equation}
\hat{K}(t) = \hat{K}(0) + \gamma \int_0^t |s(\tau)| d\tau
\label{eq:adaptive_gain}
\end{equation}

where $\gamma > 0$ is the adaptation rate and $\hat{K}(t)$ is the time-varying switching gain. This approach eliminates the need for conservative overestimation of disturbance bounds, improving control efficiency under varying conditions.

The adaptation law ensures:
\begin{equation}
\dot{V} = -\eta |s| + (\tilde{K} - \delta) |s| \leq 0
\label{eq:adaptive_lyapunov}
\end{equation}

where $\tilde{K} = K - \hat{K}$ is the gain estimation error and $\delta$ is the disturbance bound.

\subsection{Hybrid Adaptive STA-SMC}
The hybrid controller combines adaptive gain tuning with super-twisting dynamics, achieving both robustness and chattering reduction. The control law integrates:
\begin{equation}
u = u_{eq} - \hat{K}(t) \cdot |s|^{1/2} \sign(s) + u_1
\label{eq:hybrid_control}
\end{equation}

with adaptive update:
\begin{equation}
\dot{\hat{K}}(t) = \gamma |s|, \quad \dot{u}_1 = -k_2 \sign(s)
\label{eq:hybrid_adaptation}
\end{equation}

This architecture provides best overall performance: 40\% faster settling than classical SMC, 70\% chattering reduction, and 15\% performance degradation under $\pm$30\% model uncertainty.

\subsection{Gain Selection Guidelines}
Manual controller gain tuning follows these heuristics:

\textbf{Sliding Surface Parameters ($\lambda_i$):}
Determine convergence rates on the sliding surface. Typical range: $\lambda_i \in [1, 10]$. Higher values yield faster convergence but may amplify sensor noise. The surface parameters are analogous to pole placement in linear control—selecting $\lambda_i$ positions the sliding surface eigenvalues.

\textbf{Switching Gain ($K$):}
Must exceed the disturbance bound: $K > \bar{d}$. Conservative overestimation causes excessive chattering; underestimation breaks stability guarantees. Rule of thumb: $K = 2\bar{d}$ with $\bar{d} \approx 10$N for DIP systems, providing 100\% robustness margin.

\textbf{Boundary Layer ($\varepsilon$):}
Trades chattering versus tracking precision. Optimal value determined via grid search over $\varepsilon \in [0.01, 0.05]$ rad. For this work: $\varepsilon = 0.02$ rad balances chattering suppression with acceptable steady-state error ($< 0.02$ rad).

\textbf{STA Parameters ($k_1, k_2$):}
Must satisfy stability conditions (Eq.~\ref{eq:sta_stability}). Typical values: $k_1 \in [5, 15]$, $k_2 \in [0.5, 2]$. PSO optimization (Section \ref{sec:pso}) improves upon manual tuning by 30-40\%.

\textbf{Adaptive Rate ($\gamma$):}
Controls adaptation speed. Higher $\gamma$ yields faster gain adjustment but may cause oscillations. Typical range: $\gamma \in [0.1, 1.0]$. Start conservatively ($\gamma = 0.1$) and increase until performance saturates.

\subsection{Controller Comparison}
Table \ref{tab:controller_comparison} summarizes the theoretical and practical tradeoffs between controller variants.

\begin{table}[htbp]
\centering
\caption{Qualitative Controller Comparison}
\label{tab:controller_comparison}
\begin{tabular}{lcccc}
\toprule
Property & Classical & STA & Adaptive & Hybrid \\
\midrule
Chattering amplitude & High & Low & High & Low \\
Tuning complexity & Medium & High & High & Very High \\
Model dependency & High & High & Low & Medium \\
Disturbance rejection & Good & Excellent & Excellent & Excellent \\
Computational cost & Low & Medium & High & High \\
Real-time suitability & Excellent & Good & Good & Good \\
Robustness to uncertainty & Medium & Medium & High & High \\
\bottomrule
\end{tabular}
\end{table}

\textbf{Selection Criteria:}
\begin{itemize}
\item \textbf{Classical SMC}: Best for systems with accurate models, minimal chattering constraints, and low computational resources.
\item \textbf{STA-SMC}: Ideal when chattering must be minimized (e.g., mechanical systems with backlash or gear trains).
\item \textbf{Adaptive SMC}: Preferred when model parameters are uncertain or time-varying (e.g., varying payload mass).
\item \textbf{Hybrid Adaptive STA}: Best overall performance when computational resources permit, combining chattering reduction with robust adaptation.
\end{itemize}

This comparison motivates the PSO-based automatic tuning approach described in Section \ref{sec:pso}, which systematically optimizes gains for each controller variant.
