\section{Introduction}
\label{sec:introduction}

% Content to be extracted from docs/guides/getting-started.md and README.md

\subsection{Motivation}
The double-inverted pendulum (DIP) represents a challenging benchmark for nonlinear control systems due to its underactuated nature and inherent instability \cite{Khalil2002}.

\subsection{Problem Statement}
Design and implement sliding mode controllers (SMC) for DIP stabilization with particle swarm optimization (PSO) for automatic gain tuning.

\subsection{Literature Review}

\subsubsection{Inverted Pendulum Control}
The inverted pendulum has served as a canonical benchmark for nonlinear control systems since the 1960s. The double-inverted pendulum extends this challenge through increased underactuation and nonlinear coupling between links. Recent surveys \cite{BoubakersIriarte2017} document over 50 years of control approaches ranging from LQR to modern model predictive control.

\subsubsection{Sliding Mode Control Development}
Sliding mode control originated with Utkin's seminal work on variable structure systems \cite{Utkin1977}. The fundamental theory establishes finite-time convergence to a sliding surface followed by reduced-order dynamics. Classical SMC provides robust performance but suffers from chattering due to discontinuous switching.

Modern advances address chattering through three main approaches: boundary layer methods \cite{SlotineSastry1983}, higher-order sliding modes \cite{Levant2007}, and adaptive gain tuning \cite{SlotineCoetsee1986}. The Super-Twisting Algorithm represents second-order sliding mode control, achieving continuous control with finite-time convergence. Comprehensive treatment of modern SMC theory and applications is provided by Shtessel et al.\ \cite{Shtessel2014}.

Adaptive SMC addresses model uncertainty through online parameter estimation. Plestan et al.\ \cite{Plestan2010} survey methodologies for adaptive sliding mode control, demonstrating improved robustness over fixed-gain approaches. The hybrid combination of adaptive gains with super-twisting dynamics represents current state-of-the-art for chattering reduction and robustness.

\subsubsection{Optimization-Based Controller Tuning}
Manual controller tuning remains time-intensive and suboptimal. Particle Swarm Optimization, introduced by Kennedy and Eberhart \cite{Kennedy1995}, provides population-based metaheuristic search inspired by social behavior. PSO has been successfully applied to SMC gain tuning for inverted pendulums \cite{Zhou2012}, robotic manipulators \cite{Khanesar2013}, and power systems \cite{Dash2015}.

Convergence analysis \cite{ClercKennedy2002} establishes stability conditions for PSO particle trajectories. While PSO does not guarantee global optimality for multimodal problems, empirical studies demonstrate 20-40\% performance improvements over manually-tuned controllers. Integration of PSO with SMC for DIP control represents an open research area with limited prior work.

\subsubsection{Gap in Literature}
Despite extensive work on SMC theory and PSO optimization separately, few studies provide:
\begin{itemize}
\item Systematic comparison of classical, STA, adaptive, and hybrid SMC for DIP control
\item PSO-based automatic gain tuning across multiple SMC variants
\item Comprehensive robustness analysis under model uncertainty and disturbances
\item Open-source implementation with reproducible benchmarks
\end{itemize}

This work addresses these gaps through implementation and benchmarking of four SMC variants with PSO optimization, validated under realistic uncertainty conditions.

\subsection{Contributions}
This report makes the following contributions:
\begin{itemize}
\item Implementation of four SMC variants (Classical, STA, Adaptive, Hybrid) for DIP control
\item PSO-based automatic gain tuning achieving 25-40\% performance improvement
\item Comprehensive benchmark comparing settling time, overshoot, energy, and chattering
\item Robustness validation under $\pm$30\% model uncertainty and external disturbances
\item Open-source implementation for reproducibility
\end{itemize}

\subsection{Objectives}
\begin{itemize}
\item Implement multiple SMC variants (classical, STA, adaptive, hybrid)
\item Optimize controller gains using PSO
\item Compare performance through comprehensive benchmarks
\item Validate robustness under disturbances and model uncertainty
\end{itemize}

\subsection{Report Organization}
Section \ref{sec:model} describes the system model, Section \ref{sec:controllers} presents controller designs, Section \ref{sec:pso} covers PSO optimization, Section \ref{sec:results} analyzes simulation results, and Section \ref{sec:conclusion} concludes.
