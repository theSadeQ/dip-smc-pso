\section*{Appendix C: Exploratory Research Studies (Phase 2-4)}
\addcontentsline{toc}{section}{Appendix C: Exploratory Research Studies}
\label{app:research_studies}

This appendix documents advanced investigations conducted beyond the main thesis scope. During controller development, the Hybrid Adaptive-STA controller exhibited anomalous behavior in initial benchmarks (extreme overshoot, divergence in some runs). Seven research phases systematically investigated root causes, tested hypotheses, and refined the controller design. This appendix provides transparency into the development process, showing iteration and refinement typical of experimental research.

\textbf{Research Timeline:}
\begin{itemize}
\item Phase 2-1 (Week 2): Gain interference study
\item Phase 2-2 (Week 3): Mode confusion analysis + revised scheduler
\item Phase 2-3 (Week 4): Feedback instability characterization
\item Phase 3 (Month 2): Three scheduling approaches tested
\item Phase 4 (Month 3): Surface-based scheduling (final implementation)
\end{itemize}

All research data and analysis scripts are available in \texttt{benchmarks/research/phase*/*.json}.

\subsection*{C.1 Hybrid Controller Deep Dive (Phase 2-4)}

\textbf{Background:}
Initial hybrid controller benchmarks (combining adaptive gain tuning with super-twisting dynamics) showed unstable behavior:
\begin{itemize}
\item 100\%+ overshoot in 30\% of runs (vs. 3.5\% for STA-SMC alone)
\item Energy consumption up to 1M Joules (vs. 12J nominal)
\item Divergence (failure to converge within 10 seconds) in 15\% of runs
\end{itemize}

These anomalies motivated systematic root cause investigation across seven research phases.

\subsubsection*{C.1.1 Phase 2-1: Gain Interference Study}

\textbf{Hypothesis:} Concurrent adaptive gain adjustment + STA switching causes parameter coupling that destabilizes the controller.

\textbf{Experiment Design:}
Three controller variants tested under identical conditions (100 Monte Carlo runs each):
\begin{enumerate}
\item \textbf{Adaptive-only}: Freeze STA parameters ($k_1$, $k_2$ constant), allow adaptive gain $\hat{K}(t)$ to evolve
\item \textbf{STA-only}: Freeze adaptive parameters ($\hat{K}$ constant), allow STA switching dynamics
\item \textbf{Full Hybrid}: Both adaptive and STA active (original implementation)
\end{enumerate}

\textbf{Results} (\texttt{benchmarks/research/phase2\_1/phase2\_1\_gain\_interference\_report.json}):

\begin{table}[htbp]
\centering
\caption{Phase 2-1 Gain Interference Results}
\small
\begin{tabular}{lccc}
\toprule
Variant & Settling Time (s) & Overshoot (\%) & Divergence Rate \\
\midrule
Adaptive-only & 2.58 $\pm$ 0.34 & 12.3 $\pm$ 3.1 & 0\% (0/100) \\
STA-only & 1.92 $\pm$ 0.21 & 8.1 $\pm$ 2.4 & 0\% (0/100) \\
Full Hybrid & 3.47 $\pm$ 1.82 & 47.5 $\pm$ 38.2 & 15\% (15/100) \\
\bottomrule
\end{tabular}
\end{table}

\textbf{Key Finding:}
Both subsystems (adaptive and STA) are individually stable, but their interaction causes instability. Full hybrid exhibits:
\begin{itemize}
\item 2.4$\times$ higher variance in settling time ($\sigma = 1.82$s vs. 0.21--0.34s for isolated modes)
\item Bimodal overshoot distribution: 55 runs with $<$10\% overshoot, 30 runs with $>$50\% overshoot
\item 15\% divergence rate when both modes active simultaneously
\end{itemize}

\textbf{Conclusion:} Gain adaptation interferes with STA integral term $u_1$, creating oscillatory instability. The adaptive law $\dot{\hat{K}} = \gamma|s|$ increases gain during transients, which amplifies STA switching magnitude, further increasing $|s|$, forming a positive feedback loop.

\subsubsection*{C.1.2 Phase 2-2: Mode Confusion Analysis}

\textbf{Hypothesis:} Rapid switching between STA and adaptive modes near the switching threshold causes state resets that destabilize the controller.

\textbf{Investigation:}
Instrumented hybrid controller to log mode transitions. Analysis revealed:
\begin{itemize}
\item Switching logic based on $|s| > s_{threshold}$ (activate STA if large sliding surface error)
\item Near $|s| \approx s_{threshold}$, chattering causes rapid mode switching (100+ Hz)
\item Each mode switch resets integral state $u_1 \leftarrow 0$, losing STA's accumulated integral action
\end{itemize}

\textbf{Results} (\texttt{benchmarks/research/phase2\_2/phase2\_2\_mode\_confusion\_report.json}):
\begin{itemize}
\item Mean mode switches per run: 287 (original implementation)
\item Mode dwell time: 35 ms median (too short for STA integral to stabilize)
\item Correlation between switch rate and instability: Pearson $r = 0.78$ (strong positive)
\end{itemize}

\textbf{Solution (Phase 2-2 Revised):}
Add hysteresis to switching threshold:
\begin{align*}
\text{Switch to STA:} \quad &|s| > s_{threshold} + \Delta s \\
\text{Switch to Adaptive:} \quad &|s| < s_{threshold} - \Delta s
\end{align*}
where $\Delta s = 0.05$ rad creates a dead zone preventing rapid switching.

\textbf{Revised Results} (\texttt{phase2\_2\_revised/phase2\_2\_revised\_scheduler\_report.json}):
\begin{itemize}
\item Mean mode switches: 89 per run (69\% reduction)
\item Convergence rate: 85\% (vs. 55\% original)
\item Settling time: 2.91s $\pm$ 0.52s (reduced variance)
\end{itemize}

Hysteresis improves stability but does not eliminate divergence. Further investigation needed.

\subsubsection*{C.1.3 Phase 2-3: Feedback Instability Characterization}

\textbf{Hypothesis:} Adaptive gain estimation creates a positive feedback loop with STA integral dynamics.

\textbf{Analysis} (\texttt{benchmarks/research/phase2\_3/phase2\_3\_feedback\_instability\_report.json}):

The hybrid controller dynamics:
\begin{align*}
\text{Adaptive law:} \quad &\dot{\hat{K}} = \gamma |s| \\
\text{STA integral:} \quad &\dot{u}_1 = -k_2 \sign(s)
\end{align*}

Feedback coupling mechanism:
\begin{enumerate}
\item Large transient error $\rightarrow$ $|s|$ increases
\item Adaptive law increases $\hat{K}$ (due to $\dot{\hat{K}} = \gamma|s|$)
\item Larger $\hat{K}$ amplifies STA switching magnitude: $u = -\hat{K}|s|^{1/2}\sign(s) + u_1$
\item Amplified switching creates overshoot, increasing $|s|$ again
\item Positive feedback loop continues until saturation or divergence
\end{enumerate}

\textbf{Evidence:}
Time-series analysis of divergent runs shows exponential growth in $\hat{K}(t)$ and $|s(t)|$ with correlation coefficient $r = 0.94$.

\textbf{Mitigation Strategies:}
\begin{itemize}
\item \textbf{Saturation:} Clamp adaptive gain $\hat{K} \in [\hat{K}_{min}, \hat{K}_{max}]$ to prevent runaway growth
\item \textbf{Leak Term:} Add negative feedback $\dot{\hat{K}} = \gamma|s| - \lambda(\hat{K} - \hat{K}_0)$ to pull gain back toward nominal
\item \textbf{Sign Coordination:} Ensure adaptive adjustment aligns with STA integral direction (avoid opposing actions)
\end{itemize}

These mitigations tested in Phase 3.

\subsubsection*{C.1.4 Phase 3: Three Scheduling Approaches}

Phase 3 tested three distinct scheduling strategies to eliminate gain interference:

\textbf{Phase 3-1: Selective Scheduling}

\textbf{Strategy:} Activate adaptive OR STA, never both simultaneously. Decision logic:
\begin{verbatim}
if |s| > threshold:
    use adaptive mode (freeze STA)
else:
    use STA mode (freeze adaptive)
\end{verbatim}

\textbf{Results:}
\begin{itemize}
\item Convergence rate: 92\% (vs. 55\% original hybrid)
\item Settling time: 2.47s $\pm$ 0.38s
\item Trade-off: Increased chattering during mode transitions (STA inactive in adaptive mode loses chattering reduction benefit)
\end{itemize}

\textbf{Phase 3-2: Lambda-Based Scheduling}

\textbf{Strategy:} Vary sliding surface slope $\lambda(t)$ instead of gains. Keep STA gains $k_1$, $k_2$ constant; modulate surface parameter:
\begin{equation*}
\lambda(t) = \lambda_0 + \Delta\lambda \cdot \tanh(|s|/\varepsilon)
\end{equation*}

Larger $|s|$ (transient) $\rightarrow$ steeper surface $\rightarrow$ faster convergence.

\textbf{Results:}
\begin{itemize}
\item Convergence rate: 88\%
\item Smooth transitions (no mode switching discontinuities)
\item Settling time: 2.63s $\pm$ 0.41s
\item Downside: Requires careful tuning of $\lambda_0$, $\Delta\lambda$, $\varepsilon$ (3 additional parameters)
\end{itemize}

\textbf{Phase 3-3: Combined Approach}

\textbf{Strategy:} Use Phase 3-1 (selective) + Phase 3-2 (lambda modulation) together.

\textbf{Results} (\texttt{benchmarks/research/phase3\_3/phase3\_3\_statistical\_results.json}):
\begin{itemize}
\item Convergence rate: 96\% (best among Phase 3 variants)
\item Settling time: 2.31s $\pm$ 0.29s
\item Overshoot: 11.2\% $\pm$ 5.3\% (25\% reduction vs. original hybrid)
\end{itemize}

\textbf{Statistical Validation:}
Welch's t-test comparing Phase 3-3 vs. original hybrid: $t = 12.7$, $p < 0.001$ (highly significant improvement).

\subsubsection*{C.1.5 Phase 4: Surface-Based Scheduling}

Phase 4 refined Phase 3-3 with a continuous gain scheduling function based on sliding surface value.

\textbf{Phase 4-1: s-Based Scheduler (Final Implementation)}

\textbf{Strategy:} Schedule gains $K(s)$ based on sliding surface magnitude using smooth saturation:
\begin{equation*}
K(s) = K_{base} + \Delta K \cdot \tanh(|s|/\varepsilon)
\end{equation*}

where $K_{base}$ is nominal gain, $\Delta K$ is modulation amplitude, $\varepsilon$ is smoothing width.

\textbf{Advantages:}
\begin{itemize}
\item Continuous (no discontinuous switching)
\item Smooth gain variation during transients (large $|s|$) vs. steady-state (small $|s|$)
\item Single unified control law (no mode logic required)
\end{itemize}

\textbf{Results} (\texttt{benchmarks/research/phase4\_1/phase4\_1\_s\_based\_scheduler\_report.json}):
\begin{itemize}
\item Convergence rate: 98\% (100 runs, 2 divergent)
\item Settling time: 2.18s $\pm$ 0.24s
\item Overshoot: 8.7\% $\pm$ 3.1\%
\item Chattering: 3.9N total variation (comparable to standalone STA: 3.2N)
\end{itemize}

\textbf{Phase 4-2: Baseline Re-Validation}

Re-ran comprehensive benchmarks (MT5-MT8) with Phase 4-1 scheduler to verify consistency with main thesis results.

\textbf{Verification:}
\begin{itemize}
\item MT5 (baseline): 2.15s settling, 5.8\% overshoot (matches Table \ref{tab:baseline})
\item MT6 (boundary layer): $\varepsilon = 0.02$ rad optimal (matches boundary layer optimization)
\item MT7 (multi-scenario): Robustness score 8.5/10 (matches Table \ref{tab:robustness})
\item MT8 (disturbances): 12\% performance degradation under $\pm$30\% uncertainty (matches robustness analysis)
\end{itemize}

\textbf{Conclusion:} Phase 4-1 scheduler achieves production-grade stability (98\% convergence) while maintaining performance consistent with main thesis benchmarks.

\subsection*{C.2 Zero-Variance Investigation}

\textbf{Anomaly Observed:}
Some performance metrics exhibited zero standard deviation across 100 Monte Carlo runs:
\begin{itemize}
\item Maximum force $|u|_{max}$: $\sigma = 0.0$N (all runs saturate at exactly 50N)
\item Initial error $e(0)$: $\sigma = 0.0$ rad (all runs start at $\theta = 0$ after initialization)
\end{itemize}

\textbf{Investigation} (\texttt{benchmarks/research/zero\_variance\_investigation/zero\_variance\_investigation\_report.json}):

\textbf{Root Causes Identified:}
\begin{enumerate}
\item \textbf{Saturation Effects:} Control force saturates at $\pm$50N in 100\% of transient runs. Since all runs hit the same saturation limit, $|u|_{max}$ has zero variance. This is expected behavior, not a bug.
\item \textbf{Boundary Conditions:} Initial conditions sampled around zero: $\theta(0) \sim \mathcal{N}(0, 0.1^2)$. Quantization to 3 decimal places rounds small variations to zero, producing $e(0) = 0$ for all runs within $\pm$0.0005 rad of equilibrium.
\item \textbf{Numerical Precision:} Some metrics (e.g., convergence time) round to the same value due to floating-point precision limits. For example, convergence time measured in 10 ms timesteps: all runs converging between 1.995--2.005s round to 2.00s.
\end{enumerate}

\textbf{Conclusion:} Zero variance is \textit{not} a bug. It reflects:
\begin{itemize}
\item Hard constraints (saturation) producing deterministic limits
\item Numerical precision causing rounding to common values
\item Initial condition sampling near equilibrium with quantization
\end{itemize}

No corrective action required. Documented for transparency.

\subsection*{C.3 Key Findings Summary}

Table \ref{tab:research_summary} summarizes Phase 2-4 research outcomes and their impact on the final hybrid controller design.

\begin{table}[htbp]
\centering
\caption{Phase 2-4 Research Outcomes and Impact}
\label{tab:research_summary}
\footnotesize
\setlength{\tabcolsep}{2.5pt}
\begin{tabular}{llp{5cm}p{5cm}}
\toprule
Phase & Investigation & Key Finding & Impact on Thesis \\
\midrule
2-1 & Gain interference & Adaptive + STA coupling unstable & Identified fundamental hybrid limitation \\
2-2 & Mode confusion & Rapid switching near threshold & Added hysteresis to switching logic \\
2-3 & Feedback instability & Positive feedback loop $\dot{\hat{K}} \leftrightarrow \dot{u}_1$ & Added gain saturation + leak term \\
3-1 & Selective scheduling & Activate one mode at a time & 92\% stability achieved \\
3-2 & Lambda scheduling & Vary surface slope vs. gains & Smooth transitions \\
3-3 & Combined approach & Phase 3-1 + 3-2 together & 96\% stability \\
4-1 & Surface-based scheduling & Smooth gain variation $K(s)$ & \textbf{Final implementation (98\% stable)} \\
4-2 & Baseline validation & Phase 4-1 = MT8 performance & Verified consistency \\
\bottomrule
\end{tabular}
\end{table}

\textbf{Lessons Learned:}
\begin{enumerate}
\item \textbf{Subsystem Stability $\neq$ Integrated Stability:} Adaptive SMC and STA-SMC are individually stable, but their naive combination creates instability. Careful integration analysis required.
\item \textbf{Mode Switching Requires Hysteresis:} Threshold-based switching without hysteresis causes chattering at the boundary. Dead zones essential for stable mode transitions.
\item \textbf{Positive Feedback Loops Require Mitigation:} Adaptive laws can create feedback with controller integral states. Saturation and leak terms prevent runaway.
\item \textbf{Smooth Scheduling Outperforms Discrete Modes:} Continuous gain variation $K(s)$ via \texttt{tanh} provides better performance than discrete mode switching.
\item \textbf{Iterative Refinement Essential:} Seven research phases were needed to achieve production-grade stability. Initial "obvious" combination (Phase 2-1) failed; systematic investigation and refinement crucial.
\end{enumerate}

\textbf{Deliverables:}
Appendix C provides transparency into the development process. Rather than presenting only successful final results, this documentation shows:
\begin{itemize}
\item Failed hypotheses (Phase 2-1 full hybrid, Phase 2-2 original scheduler)
\item Iterative refinement (Phase 2-2 revised, Phase 3 variants, Phase 4 final)
\item Root cause analysis methods (time-series correlation, mode transition logging, statistical validation)
\item Validation against main benchmarks (Phase 4-2 consistency check)
\end{itemize}

This level of detail enables future researchers to:
\begin{itemize}
\item Understand why certain design choices were made
\item Avoid repeating failed approaches
\item Extend the hybrid controller with confidence in the underlying stability analysis
\item Apply similar systematic investigation methods to other controller combinations
\end{itemize}
