% ============================================================================
% ABSTRACT - PERSIAN (چکیده فارسی)
% ============================================================================

\chapter*{چکیده}
\addcontentsline{toc}{chapter}{چکیده}

\begin{persian}

\noindent
\textbf{پیشینه:}
کنترل مد لغزشی (SMC) پایدارسازی مقاوم برای سیستم‌های غیرخطی کم‌عملگر را فراهم می‌آورد، اما از پدیده چترینگ—نوسانات فرکانس بالا که عملکرد را تخریب کرده و کاربرد صنعتی را محدود می‌کند—رنج می‌برد. در حالی که لایه‌های مرزی تطبیقی می‌توانند چترینگ را کاهش دهند، تنظیم بهینه پارامترها همچنان چالش‌برانگیز است و معمولاً نیازمند تنظیم دستی زمان‌بر یا اتکا به عرض‌های مرزی ثابت است که یا دقت ردیابی یا دامنه چترینگ را به خطر می‌اندازند.

\vspace{0.5cm}

\noindent
\textbf{هدف:}
این پایان‌نامه یک چارچوب بهینه‌سازی ازدحام ذرات (PSO) را برای تنظیم سیستماتیک پارامترهای لایه مرزی تطبیقی برای یک سیستم آونگ دوگانه معکوس (DIP) توسعه می‌دهد که چترینگ را با حفظ عملکرد کنترل به حداقل می‌رساند. ما تحلیل پایداری مبتنی بر لیاپانوف برای لایه مرزی متغیر با زمان ارائه کرده و رویکرد را به طور دقیق در چندین سناریوی آزمایشی اعتبارسنجی می‌کنیم.

\vspace{0.5cm}

\noindent
\textbf{روش‌ها:}
ما یک مکانیسم لایه مرزی تطبیقی طراحی می‌کنیم که در آن عرض مؤثر $\epsilon_{\text{eff}}(t) = \epsilon_{\min} + \alpha|\dot{s}(t)|$ به طور پویا بر اساس سرعت سطح لغزش تنظیم می‌شود. PSO سه پارامتر ($\epsilon_{\min}$، $\alpha$، $\lambda$) را با استفاده از یک تابع برازندگی چندهدفه وزن‌دار چترینگ (۷۰٪ شاخص چترینگ، ۱۵٪ انرژی، ۱۵٪ خطای ردیابی) بهینه می‌کند. اعتبارسنجی از شبیه‌سازی‌های مونت کارلو (۱۰۰-۵۰۰ آزمایش به ازای هر سناریو) در چهار آزمایش استفاده می‌کند: (MT-5) مقایسه کنترل‌کننده پایه، (MT-6) اعتبارسنجی عملکرد اسمی، (MT-7) تست استرس مقاومت (شرایط اولیه $\pm0.3$ رادیان)، و (MT-8) تحلیل رد اغتشاش (اغتشاشات پله‌ای/ضربه‌ای/سینوسی). معنی‌داری آماری از طریق آزمون $t$ ولش، اندازه اثر (Cohen's $d$)، و فواصل اطمینان بوت‌استرپ ارزیابی می‌شود.

\vspace{0.5cm}

\noindent
\textbf{نتایج:}
در شرایط اسمی ($\pm0.05$ رادیان، MT-6)، SMC لایه مرزی تطبیقی بهینه‌شده با PSO به کاهش ۶۶.۵٪ چترینگ (شاخص میانگین: ۰.۰۷۷ در مقابل ۰.۲۳۰، $p<0.001$، $d=5.29$) با جریمه انرژی صفر نسبت به SMC کلاسیک دست یافت. با این حال، تعمیم به شرایط اولیه چالش‌برانگیز ($\pm0.3$ رادیان، MT-7) تخریب چترینگ ۵۰.۴ برابری (میانگین: ۳.۸۸ در مقابل ۰.۰۷۷) و نرخ شکست ۹۰.۲٪ را نشان داد که نشان‌دهنده بیش‌برازش شدید به توزیع آموزش است. تست‌های رد اغتشاش (MT-8) نرخ موفقیت ۰٪ را در همه انواع اغتشاش برای هر دو SMC کلاسیک و تطبیقی نشان دادند که به فقدان عمل انتگرالی و آموزش PSO تک‌سناریویی نسبت داده می‌شود.

\vspace{0.5cm}

\noindent
\textbf{نتیجه‌گیری:}
PSO با موفقیت لایه‌های مرزی تطبیقی را برای کاهش چترینگ اسمی بهینه می‌کند که از طریق اثبات‌های دقیق پایداری لیاپانوف اعتبارسنجی شده است. با این حال، بهینه‌سازی تک‌سناریویی به طور بنیادی مقاومت را محدود می‌کند. راه‌حل‌های پیشنهادی شامل PSO مقاوم چندسناریویی (برازندگی مینی‌مکس در توزیع‌های شرایط اولیه)، توابع برازندگی آگاه از اغتشاش، و طراحی‌های مد لغزشی انتگرالی است. این کار بهترین شیوه‌های روش‌شناختی را برای اعتبارسنجی SMC تثبیت می‌کند و بر تست چندسناریویی و گزارش صادقانه نتایج منفی تأکید دارد.

\vspace{0.5cm}

\noindent
\textbf{کلمات کلیدی:}
کنترل مد لغزشی، کاهش چترینگ، بهینه‌سازی ازدحام ذرات، لایه مرزی تطبیقی، آونگ دوگانه معکوس، کنترل مقاوم، پایداری لیاپانوف، بهینه‌سازی چندهدفه

\end{persian}

\clearpage
