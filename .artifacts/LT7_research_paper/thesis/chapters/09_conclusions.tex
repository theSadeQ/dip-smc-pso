% ============================================================================
% CHAPTER 9: CONCLUSIONS AND FUTURE WORK
% ============================================================================

\chapter{Conclusions and Future Work}
\label{ch:conclusions}

This chapter provides a comprehensive summary of the thesis contributions, answers the research questions posed in Chapter~\ref{chap:introduction}, acknowledges the limitations of the current work, and outlines concrete future research directions to address these limitations. Section~\ref{sec:conclusions_contributions} summarizes the three main contributions. Section~\ref{sec:conclusions_research_answers} directly answers the five research questions using experimental evidence. Section~\ref{sec:conclusions_limitations} acknowledges five critical limitations that contextualize the findings. Section~\ref{sec:conclusions_future_work} proposes five actionable future research directions with expected outcomes. Section~\ref{sec:conclusions_final_remarks} provides closing remarks on the broader implications for the sliding mode control community.

% ============================================================================
\section{Summary of Contributions}
\label{sec:conclusions_contributions}

This thesis makes three primary contributions to the sliding mode control literature, each validated through rigorous experimental evidence and statistical analysis:

% ----------------------------------------------------------------------------
\subsection{PSO-Optimized Adaptive Boundary Layer with Chattering-Weighted Fitness}
\label{subsec:contribution_adaptive_boundary}

We introduced an adaptive boundary layer mechanism:
\begin{equation}
    \epsilon_{\text{eff}} = \epsilon_{\min} + \alpha|\dot{s}|
    \label{eq:conclusions_adaptive_boundary}
\end{equation}
that dynamically adjusts boundary layer thickness based on sliding surface derivative magnitude $|\dot{s}|$. The two parameters $(\epsilon_{\min}, \alpha)$ were systematically optimized using Particle Swarm Optimization with a novel chattering-weighted fitness function:
\begin{equation}
    F = 0.70 \cdot C + 0.15 \cdot T_s + 0.15 \cdot O
    \label{eq:conclusions_fitness}
\end{equation}
where chattering index $C$ (FFT-based, $f > 10$ Hz) receives 70\% weight, settling time $T_s$ and overshoot $O$ each receive 15\% weight.

\textbf{Key Result}: Monte Carlo validation over 100 trials at MT-6 nominal conditions ($\pm0.05$ rad initial conditions) demonstrated \textbf{66.5\% chattering reduction} (6.37 → 2.14, $p < 0.001$, Cohen's $d = 5.29$) compared to fixed boundary layer SMC, with \textbf{zero energy penalty} (energy consumption statistically unchanged, $p = 0.339$). The Cohen's $d$ effect size of 5.29 represents an \textbf{exceptional} effect magnitude by conventional benchmarks ($d > 1.3$ is considered very large in control systems research), far exceeding typical chattering reduction results in the literature ($d \approx 0.5$--$1.0$).

This represents a substantial advancement over prior heuristic adaptive boundary layer methods that lack systematic optimization frameworks and typically achieve modest effect sizes.

% ----------------------------------------------------------------------------
\subsection{Lyapunov Stability Analysis for Time-Varying Boundary Layer}
\label{subsec:contribution_stability}

We provided rigorous theoretical guarantees via Lyapunov's direct method (Chapter~\ref{ch:controller_design}, Theorem~\ref{thm:adaptive_boundary_stability}), proving that the adaptive boundary layer \textit{preserves finite-time convergence to the sliding surface} under standard SMC assumptions:
\begin{itemize}
    \item Matched disturbances ($d(t)$ enters through control channel)
    \item Switching gain dominance ($K > \bar{d}$ where $\bar{d}$ is disturbance bound)
    \item Controllability of system dynamics
    \item Positive surface and boundary layer gains ($\lambda_i > 0$, $\epsilon_{\min} > 0$, $\alpha \geq 0$)
\end{itemize}

\textbf{Key Theoretical Result}: Theorem~\ref{thm:adaptive_boundary_stability} establishes reaching time bounded by:
\begin{equation}
    t_{\text{reach}} \leq \frac{\sqrt{2}|s(0)|}{\beta\eta}
    \label{eq:conclusions_reaching_time}
\end{equation}
where $s(0)$ is initial sliding surface value, $\beta$ is Lyapunov decay rate, and $\eta = K - \bar{d}$ is switching gain margin. Critically, this bound is \textit{independent of the time-varying} $\epsilon_{\text{eff}}(t)$—the adaptive boundary layer does not compromise finite-time convergence guarantees.

This addresses a significant theoretical gap in existing adaptive boundary layer literature, which often proposes dynamic adaptation laws without rigorous Lyapunov proofs. Our stability analysis demonstrates that the adaptation mechanism $\alpha|\dot{s}|$ is compatible with classical SMC stability theory, providing theoretical confidence for practical deployment.

% ----------------------------------------------------------------------------
\subsection{Honest Reporting of Generalization and Disturbance Rejection Failures}
\label{subsec:contribution_negative_results}

Through systematic multi-scenario stress testing (MT-7 robustness analysis, MT-8 disturbance rejection), we identified and quantified two critical limitations that are rarely reported in the SMC optimization literature:

\paragraph{Generalization Failure (MT-7):}
PSO parameters optimized exclusively on $\pm0.05$ rad initial conditions exhibited catastrophic performance degradation when tested under $\pm0.3$ rad initial conditions:
\begin{itemize}
    \item \textbf{50.4$\times$ chattering degradation}: Chattering index increased from 2.14 (optimized distribution) to 107.61 (test distribution)
    \item \textbf{90.2\% failure rate}: Only 49/500 Monte Carlo trials converged successfully
    \item \textbf{Root cause}: Single-scenario overfitting—PSO training on narrow distribution (17\% of operating range) produced brittle parameters
\end{itemize}

This finding directly contradicts the common implicit assumption in SMC literature that parameters optimized under nominal conditions will gracefully degrade under stress. Instead, we observed \textit{catastrophic failure} beyond the training distribution.

\paragraph{Disturbance Rejection Failure (MT-8):}
All controllers (classical SMC, adaptive boundary layer SMC, even baseline controllers) achieved \textbf{0\% convergence} under external disturbances (10~N step force, 30~N$\cdot$s impulse, 8~N sinusoidal):
\begin{itemize}
    \item \textbf{Root cause 1 (PSO)}: Fitness function optimized chattering/settling time without disturbance scenarios—PSO had zero incentive to preserve disturbance rejection
    \item \textbf{Root cause 2 (SMC structure)}: Classical SMC lacks integral action ($\int e \, dt$), preventing rejection of constant disturbances
\end{itemize}

\textbf{Methodological Contribution}: These \textit{negative results}, transparently reported with statistical evidence, provide actionable insights for future robust optimization research. They establish best practices for multi-scenario validation (testing beyond training distributions, including disturbances in fitness evaluation) and challenge the SMC community's tendency to report only success cases. By honestly documenting both achievements \textit{and} failures, this thesis advances the field toward more robust and trustworthy controller design methodologies.

% ============================================================================
\section{Answers to Research Questions}
\label{sec:conclusions_research_answers}

This section directly answers the five research questions posed in Section~\ref{sec:research_objectives} using experimental evidence from Chapters~\ref{ch:results_analysis} and~\ref{ch:discussion}.

% ----------------------------------------------------------------------------
\subsection{RQ1: Chattering Reduction Under Nominal Conditions}
\label{subsec:answer_rq1}

\textbf{Research Question 1}: \textit{How much chattering reduction can PSO-optimized adaptive boundary layer achieve compared to classical SMC under nominal conditions?}

\textbf{Hypothesis}: Adaptive boundary layer will reduce chattering index by $>50$\% with statistical significance ($p<0.01$) and large effect size ($d>0.8$).

\textbf{Answer}: \textbf{Hypothesis CONFIRMED with exceptional strength}. Under MT-6 nominal conditions ($\pm0.05$ rad initial conditions, no disturbances), the PSO-optimized adaptive boundary layer achieved:
\begin{itemize}
    \item \textbf{66.5\% chattering reduction}: Chattering index decreased from $6.37 \pm 0.42$ (fixed boundary layer) to $2.14 \pm 0.18$ (adaptive), exceeding the 50\% threshold
    \item \textbf{Statistical significance}: Welch's $t$-test yielded $p < 0.001$ (far below $p < 0.01$ threshold), indicating results are highly unlikely due to chance
    \item \textbf{Exceptional effect size}: Cohen's $d = 5.29$ (vastly exceeding $d > 0.8$ criterion), classified as "exceptional" by conventional benchmarks
\end{itemize}

The optimized parameters $\epsilon_{\min}^* = 0.0025$ and $\alpha^* = 1.21$ (Section~\ref{sec:pso_process}, Table~\ref{tab:pso_optimal}) represent a near-optimal solution in the 2-dimensional PSO search space for the given fitness function weights.

% ----------------------------------------------------------------------------
\subsection{RQ2: Energy and Tracking Trade-offs}
\label{subsec:answer_rq2}

\textbf{Research Question 2}: \textit{Does adaptive boundary layer chattering reduction come at the cost of increased energy consumption or degraded tracking accuracy?}

\textbf{Hypothesis}: Energy consumption will remain within 10\% of baseline, and tracking error will not increase significantly ($p>0.05$).

\textbf{Answer}: \textbf{Hypothesis CONFIRMED for energy, EXCEEDED expectations for tracking}. MT-6 results (Section~\ref{sec:adaptive_optimization}, Table~\ref{tab:mt6_chattering}) showed:

\paragraph{Energy Consumption:}
\begin{itemize}
    \item Adaptive: $32.58 \pm 1.82$ J, Fixed: $32.81 \pm 1.92$ J
    \item Difference: $-0.7$\% (within 10\% threshold)
    \item Statistical test: $p = 0.339$ (exceeding $p > 0.05$ threshold)
    \item \textbf{Conclusion}: \textit{Zero energy penalty}—chattering reduction achieved without increased control effort
\end{itemize}

\paragraph{Tracking Accuracy (Overshoot):}
\begin{itemize}
    \item Both controllers: Overshoot $\approx 0$ (equilibrium stabilization task)
    \item No statistical difference detected
    \item \textbf{Conclusion}: Tracking accuracy \textit{preserved}
\end{itemize}

This result demonstrates that the adaptive boundary layer mechanism successfully exploits state-space variation (large $\epsilon_{\text{eff}}$ during reaching, small $\epsilon_{\text{eff}}$ during sliding) to reduce chattering without requiring additional control energy. The PSO fitness function's 70-15-15 weighting effectively balanced the competing objectives.

% ----------------------------------------------------------------------------
\subsection{RQ3: Generalization to Diverse Initial Conditions}
\label{subsec:answer_rq3}

\textbf{Research Question 3}: \textit{Do PSO-optimized parameters generalize to initial conditions significantly different from the training distribution?}

\textbf{Hypothesis}: Controllers optimized for $\pm0.05$ rad will maintain <2$\times$ chattering degradation when tested on $\pm0.3$ rad initial conditions.

\textbf{Answer}: \textbf{Hypothesis REJECTED—catastrophic generalization failure observed}. MT-7 robustness stress testing (Section~\ref{sec:robustness_analysis}, Table~\ref{tab:mt7_robustness}) revealed:
\begin{itemize}
    \item \textbf{50.4$\times$ chattering degradation}: Chattering index increased from 2.14 (training distribution) to 107.61 (test distribution), vastly exceeding the 2$\times$ threshold
    \item \textbf{90.2\% failure rate}: Only 49/500 Monte Carlo trials converged (divergence criteria: $|\theta| > 1$ rad or $t > 30$ s)
    \item \textbf{Comparison with fixed boundary layer}: Fixed boundary layer exhibited only 3.3$\times$ degradation (6.37 → 20.82), demonstrating that the adaptive mechanism's benefit \textit{reversed} under stress
\end{itemize}

\textbf{Root Cause Analysis} (Section~\ref{sec:generalization_failure}): Single-scenario PSO optimization created narrow parameter specialization. The training distribution $\pm0.05$ rad represents only 17\% of the $\pm0.3$ rad operating range—PSO had no incentive to preserve performance outside this narrow window. The adaptive parameter $\alpha^* = 1.21$ is tuned to sliding surface dynamics specific to small initial deviations; under large deviations, $|\dot{s}|$ values fall outside the range PSO explored, causing $\epsilon_{\text{eff}}$ to become inappropriately large or small.

This negative result has critical implications for practical deployment and highlights a common pitfall in controller optimization: \textit{optimize and test distributions must overlap significantly}, or brittle overfitting will occur.

% ----------------------------------------------------------------------------
\subsection{RQ4: Disturbance Rejection Robustness}
\label{subsec:answer_rq4}

\textbf{Research Question 4}: \textit{How robust is the optimized controller to external disturbances (step, impulse, sinusoidal)?}

\textbf{Hypothesis}: Controller will achieve $>80$\% convergence rate under all disturbance types with chattering index <2$\times$ baseline.

\textbf{Answer}: \textbf{Hypothesis REJECTED—complete disturbance rejection failure}. MT-8 disturbance analysis (Section~\ref{sec:disturbance_rejection}, Table~\ref{tab:mt8_disturbance}) showed:
\begin{itemize}
    \item \textbf{0\% convergence rate}: ALL controllers (classical, adaptive, fixed) failed to stabilize under any disturbance type:
    \begin{itemize}
        \item 10~N step force: 0/100 successful trials
        \item 30~N$\cdot$s impulse: 0/100 successful trials
        \item 8~N sinusoidal (0.5~Hz): 0/100 successful trials
    \end{itemize}
    \item \textbf{Failure mode}: Persistent steady-state error → divergence (typical failure time 8--15~s)
\end{itemize}

\textbf{Dual Root Cause Analysis} (Section~\ref{sec:disturbance_failure}):
\begin{enumerate}
    \item \textbf{PSO fitness myopia}: The fitness function $F = 0.70 \cdot C + 0.15 \cdot T_s + 0.15 \cdot O$ evaluated performance under \textit{zero disturbances only}. PSO optimized chattering/settling for the easiest scenario; disturbance robustness received zero weight and thus zero optimization attention.

    \item \textbf{Classical SMC structural limitation}: The sliding surface $s = e + \lambda \dot{e}$ lacks integral action ($\int e \, dt$). Without integral term, classical SMC cannot reject constant disturbances—this is a fundamental limitation of the controller \textit{structure}, not the PSO optimization quality.
\end{enumerate}

This result underscores the critical importance of \textit{fitness function design}: whatever scenarios are excluded from fitness evaluation will be ignored by PSO, regardless of their practical importance.

% ----------------------------------------------------------------------------
\subsection{RQ5: Lyapunov Stability for Time-Varying Boundary Layer}
\label{subsec:answer_rq5}

\textbf{Research Question 5}: \textit{Can Lyapunov stability be guaranteed for time-varying $\epsilon_{\text{eff}}(t)$ under standard SMC assumptions?}

\textbf{Hypothesis}: Yes, with reaching time bounded by $t_{\text{reach}} \leq \sqrt{2}|s(0)|/(\beta\eta)$ where $\eta = K - \bar{d}$.

\textbf{Answer}: \textbf{Hypothesis CONFIRMED—rigorous Lyapunov proof provided}. Chapter~\ref{ch:controller_design}, Theorem~\ref{thm:adaptive_boundary_stability} establishes:

\paragraph{Lyapunov Function:} $V(s) = \frac{1}{2}s^2$

\paragraph{Convergence Condition:} Under switching gain dominance $K > \bar{d}$ (disturbance bound), the Lyapunov derivative satisfies:
\begin{equation}
    \dot{V} \leq -\beta\sqrt{V}
    \label{eq:conclusions_lyapunov_condition}
\end{equation}
for some $\beta > 0$, guaranteeing finite-time convergence.

\paragraph{Reaching Time Bound:}
\begin{equation}
    t_{\text{reach}} \leq \frac{\sqrt{2}|s(0)|}{\beta\eta}, \quad \eta = K - \bar{d}
    \label{eq:conclusions_reaching_bound}
\end{equation}

\paragraph{Key Insight:} The reaching time bound is \textit{independent of} $\epsilon_{\text{eff}}(t)$. The adaptive boundary layer only affects behavior \textit{inside} the boundary layer ($|s| \leq \epsilon_{\text{eff}}$); reaching phase dynamics ($|s| > \epsilon_{\text{eff}}$) remain governed by the switching control law with dominance condition $K > \bar{d}$. Thus, time-varying $\epsilon_{\text{eff}}(t)$ does not compromise finite-time convergence.

This theoretical result addresses the research gap identified in Section~\ref{sec:research_gap}: existing adaptive boundary layer proposals often lack rigorous stability proofs. Our Lyapunov analysis demonstrates that state-dependent boundary thickness is theoretically sound under standard SMC assumptions.

% ============================================================================
\section{Acknowledged Limitations}
\label{sec:conclusions_limitations}

This thesis has five primary limitations that contextualize the findings and inform future research directions:

% ----------------------------------------------------------------------------
\subsection{Single-Scenario PSO Optimization}
\label{subsec:limitation_single_scenario}

\textbf{Limitation}: The PSO algorithm optimized parameters exclusively on initial conditions sampled from $\mathcal{U}(-0.05, 0.05)$ rad \textit{without disturbance scenarios}. This narrow training distribution represents only $\approx 17$\% of the $\pm0.3$ rad operating range specified in the problem formulation.

\textbf{Consequence}: Catastrophic generalization failure (MT-7: 50.4$\times$ degradation, 90.2\% failure rate). The optimized parameters $(\epsilon_{\min}^* = 0.0025, \alpha^* = 1.21)$ are highly specialized to small-deviation dynamics and exhibit \textit{negative transfer} to large-deviation scenarios—performance \textit{worse} than non-optimized baseline.

\textbf{Implication}: Multi-scenario robust PSO (Section~\ref{sec:conclusions_robust_pso}) is \textbf{mandatory} for practical deployment. Single-scenario optimization is only acceptable if the operational distribution is \textit{guaranteed} to match the training distribution—a rare situation in real-world mechatronic systems.

% ----------------------------------------------------------------------------
\subsection{Simulation-Only Validation}
\label{subsec:limitation_simulation}

\textbf{Limitation}: All results (Chapters~\ref{ch:experimental_setup}--\ref{ch:results_analysis}) are based on numerical simulation of the nonlinear DIP model using 4th-order Runge-Kutta integration ($\Delta t = 0.001$ s). Real-world hardware exhibits unmodeled dynamics absent from our simulation:
\begin{itemize}
    \item \textbf{Friction}: Coulomb and viscous friction in cart bearings and pendulum joints
    \item \textbf{Backlash}: Mechanical play in gears and linkages
    \item \textbf{Flexible modes}: Elastic deformations in pendulum links (assumed rigid)
    \item \textbf{Sensor noise}: Quantization, thermal drift, electromagnetic interference in encoders
    \item \textbf{Actuator dynamics}: Motor electrical time constants, current limiting, saturation nonlinearities
\end{itemize}

\textbf{Consequence}: Unknown simulation-to-reality gap. Literature suggests 10--30\% performance degradation is typical when transitioning from simulation to hardware \cite{andrychowicz2020learning}. The 66.5\% chattering reduction may decrease to 40--60\% on real hardware.

\textbf{Implication}: Hardware validation on a physical DIP experimental setup (Section~\ref{sec:conclusions_hardware}) is necessary to confirm simulation findings, quantify the reality gap, and identify whether adaptive parameter tuning or model refinement is required.

% ----------------------------------------------------------------------------
\subsection{Classical SMC Without Integral Action}
\label{subsec:limitation_no_integral}

\textbf{Limitation}: The classical SMC structure (Equation~\ref{eq:smc_sliding_surface}) uses sliding surface $s = e + \lambda \dot{e}$ \textit{without integral term} $k_I \int e \, dt$. This prevents rejection of constant disturbances, as evidenced by MT-8's 0\% convergence under step forces.

\textbf{Consequence}: The controller is suitable only for \textit{matched disturbances that are transient or bounded-average}. Applications requiring zero steady-state error under persistent disturbances (e.g., robotic manipulators with gravitational loads, vehicles on inclined surfaces) cannot use this controller structure.

\textbf{Implication}: Integral Sliding Mode Control (ISMC) with PSO-optimized integral gains (Section~\ref{sec:conclusions_ismc_pso}) is required for disturbance-rich environments. However, this increases the parameter search space from 2D ($\epsilon_{\min}, \alpha$) to 3D (add $k_I$), raising computational cost by $\approx 5$--$10\times$ (more PSO iterations needed for convergence in higher dimensions).

% ----------------------------------------------------------------------------
\subsection{Fixed Sliding Surface Gains}
\label{subsec:limitation_fixed_gains}

\textbf{Limitation}: While the adaptive boundary layer parameters $(\epsilon_{\min}, \alpha)$ were optimized via PSO, the sliding surface gains $(k_1, k_2, \lambda_1, \lambda_2)$ and switching gain $K$ were manually selected using classical pole placement and dominance margin criteria. This represents a \textit{partial optimization}—only 2 of 7 total controller parameters were tuned.

\textbf{Consequence}: Potential suboptimality. Manual gain selection may have left performance on the table. Joint optimization of all 7 parameters could potentially:
\begin{itemize}
    \item Further reduce chattering (additional 10--20\% reduction possible)
    \item Improve robustness margins (wider stable region in parameter space)
    \item Reduce energy consumption (optimal switching gain $K$ balancing disturbance rejection vs. control effort)
\end{itemize}

\textbf{Trade-off}: Joint optimization requires 7-dimensional PSO search, increasing computational cost by $\approx 10\times$ (more particles, more iterations). For academic research validating the adaptive boundary layer \textit{concept}, 2D optimization was sufficient. For industrial deployment, 7D joint optimization may be justified.

% ----------------------------------------------------------------------------
\subsection{System-Specific Optimized Parameters}
\label{subsec:limitation_system_specific}

\textbf{Limitation}: The optimized parameters $\epsilon_{\min}^* = 0.0025$ and $\alpha^* = 1.21$ are tailored to the specific DIP configuration used in this thesis:
\begin{itemize}
    \item Cart mass: $m_0 = 1.0$ kg
    \item Link masses: $m_1 = m_2 = 0.1$ kg
    \item Link lengths: $L_1 = L_2 = 0.5$ m
    \item Friction coefficients, actuator limits, etc. (Table~\ref{tab:system_parameters})
\end{itemize}

\textbf{Consequence}: Direct transfer of these numerical values to DIP systems with different mass ratios, lengths, or actuator constraints will likely yield suboptimal performance. For example:
\begin{itemize}
    \item Heavier pendulums → larger inertia → different $|\dot{s}|$ dynamics → $\alpha^*$ mismatch
    \item Shorter links → faster dynamics → different settling time → fitness function weights may need adjustment
\end{itemize}

\textbf{Transferability}: However, the \textit{PSO methodology itself} (adaptive boundary layer structure, chattering-weighted fitness function, Monte Carlo validation protocol) is fully transferable. Practitioners applying this approach to different DIP configurations or other underactuated systems (cart-pole, quadrotor, robotic arm) should:
\begin{enumerate}
    \item Use the same adaptive boundary layer formula (Equation~\ref{eq:adaptive_boundary_formula})
    \item Use the same fitness function structure (70-15-15 weighting) or adjust based on application priorities
    \item Re-run PSO optimization with their specific system model
    \item Expect comparable chattering reduction (50--70\%) if fitness function design is appropriate
\end{enumerate}

% ============================================================================
\section{Future Research Directions}
\label{sec:conclusions_future_work}

We propose five concrete research directions to address the limitations identified in Section~\ref{sec:conclusions_limitations} and advance robust SMC design:

% ----------------------------------------------------------------------------
\subsection{Multi-Scenario Robust PSO (High Priority)}
\label{sec:conclusions_robust_pso}

\textbf{Motivation}: MT-7 demonstrated catastrophic generalization failure when testing outside the training distribution. Single-scenario optimization is fundamentally brittle.

\textbf{Proposed Approach}: Extend the PSO fitness function to include \textit{diverse initial condition distributions and disturbance scenarios} during optimization:

\paragraph{Multi-Distribution Sampling:}
Instead of sampling all particles from $\mathcal{U}(-0.05, 0.05)$ rad, sample from a mixture:
\begin{equation}
    \theta_0 \sim 0.5 \cdot \mathcal{U}(-0.05, 0.05) + 0.3 \cdot \mathcal{U}(-0.3, 0.3) + 0.2 \cdot \mathcal{U}(-0.5, 0.5)
    \label{eq:conclusions_multi_distribution}
\end{equation}
This ensures PSO particles experience small, medium, and large deviations during every fitness evaluation.

\paragraph{Minimax Fitness Function:}
Replace average-case fitness (current approach) with worst-case fitness:
\begin{equation}
    F_{\text{robust}}(\theta) = \max_{\text{scenario } i \in \{1, \dots, N\}} F_i(\theta)
    \label{eq:conclusions_minimax}
\end{equation}
where each scenario $i$ represents a distinct combination of initial conditions and disturbances. This minimax formulation optimizes the \textit{worst-case} performance rather than typical-case, producing parameters that degrade \textit{gracefully} under stress rather than catastrophically.

\paragraph{Expected Outcome:}
\begin{itemize}
    \item Chattering reduction under nominal conditions: 40--50\% (lower than single-scenario 66.5\%, due to trade-off with robustness)
    \item Chattering degradation under stress: <2$\times$ (acceptable graceful degradation vs. current 50.4$\times$ catastrophic failure)
    \item Convergence rate under large deviations: >80\% (vs. current 9.8\%)
\end{itemize}

\paragraph{Computational Cost:}
Evaluating $N = 5$--$10$ scenarios per fitness call increases PSO runtime by $5$--$10\times$. For 30 particles, 100 iterations, 100 Monte Carlo trials per scenario, total runtime $\approx 40$--80 hours on modern multi-core CPU (vs. current 8 hours). This is acceptable for offline optimization in industrial applications.

% ----------------------------------------------------------------------------
\subsection{Disturbance-Aware Fitness Function (High Priority)}
\label{sec:conclusions_disturbance_fitness}

\textbf{Motivation}: MT-8 showed 0\% convergence under disturbances because the fitness function \textit{never evaluated disturbed scenarios}—PSO had zero incentive to preserve disturbance rejection.

\textbf{Proposed Approach}: Redesign the fitness function to explicitly weight both nominal and disturbed performance:
\begin{equation}
    F_{\text{robust}} = 0.50 \cdot C_{\text{nominal}} + 0.20 \cdot C_{\text{disturbed}} + 0.15 \cdot T_s + 0.15 \cdot O
    \label{eq:conclusions_disturbance_fitness}
\end{equation}
where:
\begin{itemize}
    \item $C_{\text{nominal}}$: Chattering under zero disturbances (current metric)
    \item $C_{\text{disturbed}}$: Average chattering under step, impulse, sinusoidal disturbances
    \item Weights: 50\% nominal, 20\% disturbed, 15\% settling, 15\% overshoot
\end{itemize}

\paragraph{Disturbance Test Suite:}
Each fitness evaluation includes:
\begin{enumerate}
    \item Nominal trial: No disturbances (current approach)
    \item Step disturbance: 10~N constant force applied at $t = 5$~s
    \item Impulse disturbance: 30~N$\cdot$s impulse at $t = 5$~s
    \item Sinusoidal disturbance: 8~N amplitude, 0.5~Hz, continuous
\end{enumerate}

\paragraph{Expected Outcome:}
\begin{itemize}
    \item Nominal chattering reduction: 50--60\% (slightly lower than single-scenario due to robustness trade-off)
    \item Disturbance convergence rate: 60--80\% (vs. current 0\%)
    \item Recovery time after disturbance: <5~s (time to re-stabilize after impulse/step)
\end{itemize}

\paragraph{Combined with ISMC:}
This fitness function alone \textit{cannot overcome} the structural limitation of classical SMC (no integral action). For zero steady-state error under constant disturbances, combine this fitness function with Integral SMC (Section~\ref{sec:conclusions_ismc_pso}).

% ----------------------------------------------------------------------------
\subsection{Integral SMC with Joint Parameter Optimization (Medium Priority)}
\label{sec:conclusions_ismc_pso}

\textbf{Motivation}: Classical SMC's lack of integral action prevents disturbance rejection (MT-8 limitation). Integral Sliding Mode Control (ISMC) addresses this structural limitation.

\textbf{Proposed Approach}: Extend the sliding surface to include integral term:
\begin{equation}
    s = e + \lambda \dot{e} + k_I \int e \, dt
    \label{eq:conclusions_ismc_surface}
\end{equation}
and jointly optimize \textit{all} parameters $(k_1, k_2, \lambda_1, \lambda_2, k_I, K, \epsilon_{\min}, \alpha)$ using PSO with disturbance-aware fitness (Equation~\ref{eq:conclusions_disturbance_fitness}).

\paragraph{Parameter Search Space:}
\begin{itemize}
    \item Sliding surface gains: $k_1, k_2 \in [1, 50]$, $\lambda_1, \lambda_2 \in [0.5, 10]$
    \item Integral gain: $k_I \in [0.1, 10]$
    \item Switching gain: $K \in [5, 100]$
    \item Adaptive boundary: $\epsilon_{\min} \in [0.001, 0.05]$, $\alpha \in [0, 5]$
\end{itemize}
Total: 8-dimensional search space (vs. current 2D).

\paragraph{Lyapunov Stability Extension:}
Theorem~\ref{thm:adaptive_boundary_stability} must be extended to account for integral dynamics. The Lyapunov function becomes:
\begin{equation}
    V(s, e_I) = \frac{1}{2}s^2 + \frac{1}{2}k_I e_I^2, \quad e_I = \int e \, dt
    \label{eq:conclusions_ismc_lyapunov}
\end{equation}
Finite-time convergence can still be proven under augmented assumptions (integral wind-up protection, integral gain bounds).

\paragraph{Expected Outcome:}
\begin{itemize}
    \item \textbf{Zero steady-state error} under constant disturbances (integral action eliminates bias)
    \item Disturbance convergence rate: >90\% (combined effect of ISMC structure + optimized parameters)
    \item Chattering reduction: 40--50\% (slightly lower than classical SMC due to integral dynamics introducing additional $\dot{s}$ contributions)
\end{itemize}

\paragraph{Computational Cost:}
8D PSO optimization requires $\approx 200$--300 iterations (vs. 100 for 2D) and 50 particles (vs. 30). Total runtime for multi-scenario fitness: 80--120 hours. However, this one-time offline cost produces production-ready parameters for industrial deployment.

% ----------------------------------------------------------------------------
\subsection{Hardware Validation on Physical DIP (High Priority)}
\label{sec:conclusions_hardware}

\textbf{Motivation}: All results are simulation-only (Section~\ref{subsec:limitation_simulation}). Unknown simulation-to-reality gap may degrade performance by 10--30\%.

\textbf{Proposed Approach}: Implement the PSO-optimized adaptive boundary layer controller on a real-world DIP experimental platform and validate against the same scenarios (MT-5, MT-6, MT-7, MT-8).

\paragraph{Hardware Platform Options:}
\begin{enumerate}
    \item \textbf{Commercial}: Quanser Rotary Inverted Pendulum (linearized DIP, \$8k)
    \item \textbf{Custom-built}: Servo-actuated cart on linear rail with optical encoders (\$3k)
    \item \textbf{Collaborative}: Partner with university lab possessing DIP testbed
\end{enumerate}

\paragraph{Experimental Protocol:}
\begin{enumerate}
    \item \textbf{System Identification}: Measure actual physical parameters (mass, length, friction) using pendulum swing tests; update simulation model
    \item \textbf{Baseline Validation}: Verify classical SMC performs comparably to simulation (establish reality gap baseline)
    \item \textbf{Adaptive Validation}: Test PSO-optimized parameters on hardware across same scenarios
    \item \textbf{Re-optimization (if needed)}: If reality gap >20\%, re-run PSO using updated hardware model
\end{enumerate}

\paragraph{Instrumentation:}
\begin{itemize}
    \item \textbf{Chattering measurement}: Mount accelerometer on cart to measure high-frequency vibrations (FFT analysis)
    \item \textbf{Energy measurement}: Current sensor on motor to measure actual electrical energy consumption
    \item \textbf{High-speed camera}: Capture pendulum motion for trajectory validation
\end{itemize}

\paragraph{Expected Outcome:}
\begin{itemize}
    \item Chattering reduction on hardware: 40--60\% (assuming 10--30\% degradation from 66.5\% simulation result)
    \item Quantified reality gap: Identify which unmodeled dynamics contribute most (friction likely dominant)
    \item Refined simulation model: Update model with identified parameters for future research
\end{itemize}

\paragraph{Publication Impact:}
Hardware validation significantly strengthens publication prospects (IEEE Trans. Control Systems Technology, Control Engineering Practice prefer experimental validation). Combined simulation + hardware results address common reviewer criticism of "simulation-only studies."

% ----------------------------------------------------------------------------
\subsection{Transfer to Other Underactuated Systems (Low Priority)}
\label{sec:conclusions_transfer}

\textbf{Motivation}: Demonstrate generalizability of the PSO-optimized adaptive boundary layer methodology beyond the DIP benchmark.

\textbf{Proposed Systems:}
\begin{enumerate}
    \item \textbf{Single Inverted Pendulum}: Simpler dynamics, faster optimization (validate methodology on canonical system)
    \item \textbf{Cart-Pole (Swing-Up)}: Requires global stabilization (large-angle maneuvers), tests multi-scenario PSO
    \item \textbf{Quadrotor Altitude Control}: Continuous system (vs. discrete DIP), tests chattering impact on aerial vehicle
    \item \textbf{2-DOF Robotic Manipulator}: Coupled nonlinear dynamics, tests scalability to higher-dimensional systems
\end{enumerate}

\paragraph{Research Questions:}
\begin{itemize}
    \item Does the 70-15-15 fitness function weighting transfer to other systems, or should weights be system-specific?
    \item Is the $\epsilon_{\text{eff}} = \epsilon_{\min} + \alpha|\dot{s}|$ formula universally effective, or do some systems benefit from alternative adaptations (e.g., $\epsilon_{\text{eff}} = \epsilon_{\min} + \alpha|s|$)?
    \item Do effect sizes (Cohen's $d$) remain large (>0.8) across diverse systems, or is DIP particularly amenable to adaptive boundary layers?
\end{itemize}

\paragraph{Expected Outcome:}
\begin{itemize}
    \item Establish transferability of chattering reduction (expect 40--70\% reduction across systems)
    \item Identify system-specific tuning requirements (fitness weights, adaptation formula)
    \item Publish methodology paper with cross-system validation (high-impact journal: Automatica, IEEE Trans. Automatic Control)
\end{itemize}

% ============================================================================
\section{Final Remarks}
\label{sec:conclusions_final_remarks}

This thesis demonstrates that PSO-optimized adaptive boundary layers can achieve \textit{dramatic chattering reduction} (66.5\%, Cohen's $d = 5.29$) for sliding mode control under \textit{nominal operating conditions}, with rigorous Lyapunov stability guarantees and zero energy penalty. However, the work also reveals a critical cautionary tale: \textbf{single-scenario optimization produces brittle controllers that fail catastrophically (50.4$\times$ degradation) outside the training distribution}.

These findings motivate two fundamental shifts in SMC research practices that we believe are essential for advancing the field toward industrial deployment:

% ----------------------------------------------------------------------------
\subsection{Robust Multi-Scenario Optimization is Mandatory}

The traditional approach of optimizing controllers under nominal conditions and \textit{hoping} for graceful degradation under stress is insufficient. Our MT-7 results demonstrate that modern optimization algorithms (PSO, genetic algorithms, gradient descent) will exploit any distributional mismatch between training and testing to achieve artificially impressive nominal performance at the cost of catastrophic failures elsewhere.

\textbf{Recommendation}: Future controller optimization research must:
\begin{enumerate}
    \item Include \textit{diverse operating conditions} (initial conditions, disturbances, parameter variations) during fitness evaluation, not just nominal scenarios
    \item Use worst-case or robust fitness formulations (minimax, CVaR) that optimize robustness rather than average performance
    \item Ensure training distributions \textit{cover} or \textit{exceed} anticipated operational distributions (test inside training range, not outside)
\end{enumerate}

This represents a cultural shift from "optimize for best-case performance" to "optimize for acceptable worst-case performance"—a shift already embraced by the robust control and machine learning communities, but not yet widespread in SMC research.

% ----------------------------------------------------------------------------
\subsection{Honest Validation and Reporting Must Become Standard Practice}

The SMC literature exhibits strong publication bias toward positive results. Controllers are typically validated under the same conditions used for optimization, and negative results (failures, degradation, instability) are rarely reported. This practice:
\begin{itemize}
    \item \textbf{Conceals brittleness}: Single-scenario validation cannot detect overfitting or distributional mismatch
    \item \textbf{Wastes research effort}: Future researchers repeat the same mistakes (e.g., single-scenario PSO) because failures go unreported
    \item \textbf{Hinders industrial adoption}: Practitioners lose trust when "proven" algorithms fail in real-world stress conditions
\end{itemize}

\textbf{Recommendation}: The SMC community should establish rigorous validation protocols:
\begin{enumerate}
    \item \textbf{Test beyond training}: Validate controllers under initial conditions 2--10$\times$ larger than training range
    \item \textbf{Include disturbances}: Test step, impulse, sinusoidal, and parameter variations even if not used during optimization
    \item \textbf{Report all results}: Transparently document both successes \textit{and} failures with statistical evidence
    \item \textbf{Negative results are publishable}: Journals should explicitly encourage negative result submissions that provide methodological insights (e.g., "Why PSO-optimized controllers fail under disturbances")
\end{enumerate}

By combining systematic robust optimization, rigorous multi-scenario validation, and honest reporting of both achievements and limitations, this thesis advances the practical deployment of sliding mode control in industrial mechatronic systems and establishes methodological best practices for future controller design research.

% ----------------------------------------------------------------------------
\subsection{Closing Statement}

The journey from theoretical SMC (perfect switching, infinite bandwidth) to practical SMC (discrete-time, actuator limits, sensor noise) is challenging. Chattering represents one of the fundamental barriers, and adaptive boundary layers offer a computationally efficient mitigation strategy. This thesis demonstrates that PSO can systematically optimize adaptive parameters to achieve exceptional nominal performance—but also reveals that optimization without robustness awareness creates fragile solutions.

The path forward requires balancing three competing priorities:
\begin{enumerate}
    \item \textbf{Performance}: Aggressive chattering reduction, fast transients, low energy
    \item \textbf{Robustness}: Graceful degradation under stress, wide operational envelope, disturbance rejection
    \item \textbf{Theoretical guarantees}: Lyapunov stability, finite-time convergence, provable safety bounds
\end{enumerate}

We believe the multi-scenario robust PSO framework proposed in Section~\ref{sec:conclusions_robust_pso}, combined with Integral SMC (Section~\ref{sec:conclusions_ismc_pso}) and hardware validation (Section~\ref{sec:conclusions_hardware}), represents a viable path to practical industrial sliding mode controllers that achieve all three priorities simultaneously.

The SMC community has developed powerful theoretical tools over five decades. The challenge now is to bridge the gap between theory and practice through systematic optimization, honest validation, and a commitment to robustness over fragile peak performance. This thesis represents one step in that direction.

% ============================================================================
% End of Chapter 9
% ============================================================================
