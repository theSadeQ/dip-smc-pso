% ============================================================================
% CHAPTER 1: INTRODUCTION
% ============================================================================

\chapter{Introduction}
\label{chap:introduction}

This chapter introduces the research problem, motivation, and contributions of this thesis. Section~\ref{sec:background} provides background on sliding mode control and the chattering problem. Section~\ref{sec:problem_statement} formally states the research problem. Section~\ref{sec:research_objectives} outlines the research objectives and questions. Section~\ref{sec:research_gap} identifies the research gap in current literature. Section~\ref{sec:contributions} summarizes the key contributions. Section~\ref{sec:significance} discusses the significance of this work. Section~\ref{sec:organization} describes the thesis organization. Section~\ref{sec:scope} defines the scope and limitations.

% ============================================================================
\section{Background and Motivation}
\label{sec:background}

Sliding mode control (SMC) has emerged as a powerful robust control technique for nonlinear systems due to its insensitivity to matched disturbances and model uncertainties \cite{utkin1992sliding, edwards1998sliding}. The core principle of SMC is to design a switching control law that drives system trajectories onto a predefined sliding surface, where desirable dynamics are guaranteed through careful sliding surface design. Once the system reaches the sliding surface, it exhibits motion invariant to matched disturbances, providing excellent robustness properties that are particularly attractive for industrial applications.

The theoretical foundation of SMC rests on Lyapunov stability theory, which guarantees finite-time convergence to the sliding surface under appropriate control gains. This finite-time reaching property distinguishes SMC from asymptotic controllers and enables rapid transient response—a critical advantage for underactuated mechanical systems requiring quick stabilization.

However, a fundamental barrier to industrial adoption of classical SMC is \textbf{chattering}: high-frequency oscillations in the control signal caused by imperfect switching in discrete-time implementations. Chattering manifests as rapid oscillations around the sliding surface, resulting from the interaction between the discontinuous control law (signum function) and finite sampling rates in digital controllers. The consequences of chattering are severe:

\begin{itemize}
    \item \textbf{Actuator wear}: Continuous high-frequency switching accelerates mechanical component fatigue and reduces actuator lifespan, increasing maintenance costs.
    \item \textbf{Energy waste}: Rapid control oscillations consume unnecessary energy, particularly problematic for battery-powered or energy-constrained systems.
    \item \textbf{Degraded precision}: High-frequency control variations excite unmodeled dynamics, compromising the steady-state tracking accuracy that SMC theoretically promises.
    \item \textbf{Acoustic noise}: In electromechanical systems (motors, valves), chattering produces audible noise that is unacceptable in consumer applications.
\end{itemize}

These practical limitations have historically confined SMC to academic studies or specialized industrial contexts (aerospace, military systems) where robustness justifies the trade-offs. Broader industrial adoption—particularly in mechatronics, robotics, and automotive systems—requires effective chattering mitigation without sacrificing SMC's core advantages.

\subsection{The Double Inverted Pendulum Benchmark}

The double inverted pendulum (DIP) represents a challenging benchmark for control systems research, exhibiting three key properties that make it an ideal testbed:

\begin{enumerate}
    \item \textbf{Underactuation}: With two pendulum links but only a single control input (cart force), the system has fewer actuators than degrees of freedom. This creates fundamental control challenges absent in fully-actuated systems, requiring sophisticated controller design to simultaneously stabilize both pendulum angles.

    \item \textbf{Nonlinearity}: The system dynamics include trigonometric nonlinearities (sin, cos), Coriolis forces, and centripetal terms. Unlike the linearized single inverted pendulum, the DIP cannot be adequately controlled via simple LQR around equilibrium when large-angle disturbances occur.

    \item \textbf{Instability}: The upright equilibrium ($\theta_1 = \theta_2 = 0$) is inherently unstable—the open-loop system diverges exponentially from equilibrium under any perturbation. This instability demands active control and provides a rigorous test of controller robustness.
\end{enumerate}

While linear controllers (e.g., LQR, PID) can stabilize the DIP near equilibrium under nominal conditions, they lack robustness to disturbances, parameter variations, and large initial deviations. Energy-based swing-up controllers can achieve global stabilization but require switching between swing-up and balancing modes, adding complexity. SMC provides an attractive alternative through its inherent robustness properties derived from Lyapunov stability theory, making it particularly well-suited for underactuated nonlinear systems.

\subsection{Classical Approaches to Chattering Reduction}

The chattering problem has been recognized since Utkin's foundational work in the 1970s \cite{utkin1992sliding}, and numerous mitigation strategies have been proposed. The main approaches include:

\paragraph{Boundary Layer Methods}
The most direct approach replaces the discontinuous signum function with a continuous saturation function within a boundary layer of thickness $\epsilon$:
\begin{equation}
    u_{\text{sw}} = -K \cdot \text{sat}(s/\epsilon) =
    \begin{cases}
        -K \cdot \text{sign}(s), & |s| > \epsilon \\
        -K \cdot s/\epsilon, & |s| \leq \epsilon
    \end{cases}
\end{equation}
This smooths the control signal near the sliding surface, reducing chattering at the expense of steady-state tracking error. The fundamental trade-off is:
\begin{itemize}
    \item Larger $\epsilon$ → reduced chattering but increased steady-state error (system stays in boundary layer)
    \item Smaller $\epsilon$ → improved tracking accuracy but exacerbated chattering
\end{itemize}
Fixed boundary layer selection thus represents a compromise that may be suboptimal across the entire state space.

\paragraph{Higher-Order Sliding Modes (HOSMC)}
Super-twisting algorithm, prescribed convergence law, and other HOSMC techniques achieve continuous control through integral action, eliminating chattering while maintaining finite-time convergence \cite{levant2003higher}. However, these methods require:
\begin{itemize}
    \item Increased computational complexity (higher-order derivatives, implicit calculations)
    \item More conservative gain selection to ensure stability
    \item Careful tuning of multiple parameters (twisting gains, convergence rates)
\end{itemize}

\paragraph{Adaptive Gain Tuning}
Adaptive SMC adjusts switching gains online based on sliding surface magnitude or estimation error, reducing unnecessary high gains during steady-state. This requires careful stability analysis to prevent parameter drift and may exhibit slow adaptation during rapid transients.

Despite decades of research, a practical question remains: \textbf{How can we minimize chattering while maintaining control precision and energy efficiency?} The boundary layer approach offers a direct and computationally efficient solution, but optimal parameter selection remains an open challenge.

% ============================================================================
\section{Problem Statement}
\label{sec:problem_statement}

Given a double inverted pendulum system with nonlinear dynamics and matched disturbances, the research problem is formulated as follows:

\textbf{Main Problem}: Design a sliding mode controller with adaptive boundary layer mechanism that:
\begin{enumerate}
    \item Minimizes chattering (high-frequency control oscillations)
    \item Maintains tracking precision (minimizes steady-state error)
    \item Preserves energy efficiency (avoids excessive control effort)
    \item Guarantees theoretical stability (Lyapunov-based convergence)
    \item Exhibits robustness to initial condition variations and external disturbances
\end{enumerate}

\textbf{Sub-Problems}:
\begin{enumerate}
    \item \textbf{Adaptive Mechanism Design}: How should the boundary layer thickness $\epsilon_{\text{eff}}$ vary as a function of system state to exploit the difference between reaching phase (transient) and sliding phase (steady-state)?

    \item \textbf{Parameter Optimization}: How can we systematically determine optimal values for boundary layer parameters ($\epsilon_{\min}$, $\alpha$) and sliding surface gains ($\lambda$) without relying on manual trial-and-error?

    \item \textbf{Multi-Objective Trade-off}: How should we balance the inherently conflicting objectives of chattering reduction, tracking accuracy, and energy efficiency in a single optimization framework?

    \item \textbf{Robustness Validation}: How can we rigorously test whether optimized controllers generalize beyond their training distribution to diverse initial conditions and external disturbances?

    \item \textbf{Theoretical Guarantees}: How can we prove stability for a time-varying boundary layer ($\epsilon_{\text{eff}}(t)$) where existing fixed-$\epsilon$ proofs may not directly apply?
\end{enumerate}

% ============================================================================
\section{Research Objectives and Questions}
\label{sec:research_objectives}

\subsection{Research Objectives}

The primary objective of this thesis is:

\begin{quote}
\textit{To develop a particle swarm optimization (PSO) framework for systematic tuning of adaptive boundary layer parameters in sliding mode control, achieving significant chattering reduction while maintaining theoretical stability guarantees and rigorously validating robustness through multi-scenario stress testing.}
\end{quote}

\noindent Specific objectives include:

\begin{enumerate}
    \item \textbf{Objective 1}: Design an adaptive boundary layer mechanism where $\epsilon_{\text{eff}} = \epsilon_{\min} + \alpha|\dot{s}|$ dynamically adjusts based on sliding surface dynamics.

    \item \textbf{Objective 2}: Formulate a multi-objective PSO fitness function that balances chattering reduction (primary), energy efficiency (secondary), and tracking accuracy (secondary).

    \item \textbf{Objective 3}: Provide rigorous Lyapunov stability analysis proving finite-time convergence for the time-varying boundary layer.

    \item \textbf{Objective 4}: Validate the approach through extensive Monte Carlo simulations (100--500 trials) across four experimental scenarios: baseline comparison, nominal performance, robustness stress testing, and disturbance rejection.

    \item \textbf{Objective 5}: Quantify limitations and failure modes through systematic stress testing beyond the training distribution, establishing methodological best practices for honest validation in the SMC community.
\end{enumerate}

\subsection{Research Questions}

The research objectives translate to the following research questions:

\begin{enumerate}
    \item[\textbf{RQ1:}] \textit{How much chattering reduction can PSO-optimized adaptive boundary layer achieve compared to classical SMC under nominal conditions?}

    \textbf{Hypothesis}: Adaptive boundary layer will reduce chattering index by $>50$\% with statistical significance ($p<0.01$) and large effect size ($d>0.8$).

    \item[\textbf{RQ2:}] \textit{Does adaptive boundary layer chattering reduction come at the cost of increased energy consumption or degraded tracking accuracy?}

    \textbf{Hypothesis}: Energy consumption will remain within 10\% of baseline, and tracking error will not increase significantly ($p>0.05$).

    \item[\textbf{RQ3:}] \textit{Do PSO-optimized parameters generalize to initial conditions significantly different from the training distribution?}

    \textbf{Hypothesis}: Controllers optimized for $\pm0.05$ rad will maintain <2$\times$ chattering degradation when tested on $\pm0.3$ rad initial conditions.

    \item[\textbf{RQ4:}] \textit{How robust is the optimized controller to external disturbances (step, impulse, sinusoidal)?}

    \textbf{Hypothesis}: Controller will achieve $>80$\% convergence rate under all disturbance types with chattering index <2$\times$ baseline.

    \item[\textbf{RQ5:}] \textit{Can Lyapunov stability be guaranteed for time-varying $\epsilon_{\text{eff}}(t)$ under standard SMC assumptions?}

    \textbf{Hypothesis}: Yes, with reaching time bounded by $t_{\text{reach}} \leq \sqrt{2}|s(0)|/(\beta\eta)$ where $\eta = K - \bar{d}$.
\end{enumerate}

% ============================================================================
\section{Research Gap}
\label{sec:research_gap}

Despite extensive research on SMC chattering mitigation, existing literature exhibits three critical gaps:

\subsection{Gap 1: Fixed Boundary Layers Ignore State-Space Variation}

Classical boundary layer methods use constant $\epsilon$ regardless of system state, failing to exploit the natural variation in control requirements. During the reaching phase (large $|s|$), aggressive control is needed and chattering is less problematic. During the sliding phase (small $|s|$), smooth control is critical and chattering becomes dominant. A fixed $\epsilon$ cannot adapt to these changing requirements, resulting in either:
\begin{itemize}
    \item Overly conservative $\epsilon$ (large): smooth control throughout but poor transient response
    \item Aggressive $\epsilon$ (small): fast reaching but severe chattering during sliding
\end{itemize}

Recent work has explored adaptive boundary layers \cite{plestan2010new, bartoszewicz2007reaching}, but these studies typically:
\begin{itemize}
    \item Propose ad-hoc adaptation laws without systematic parameter optimization
    \item Rely on manual tuning or heuristic rules (e.g., "decrease $\epsilon$ during sliding")
    \item Lack multi-objective formulations balancing chattering, energy, and tracking simultaneously
\end{itemize}

\subsection{Gap 2: Manual Tuning Prevents Systematic Optimization}

SMC parameter selection remains largely an art, relying on trial-and-error or conservative design rules. For adaptive boundary layers with parameters ($\epsilon_{\min}$, $\alpha$, $\lambda$), the design space is 3-dimensional and potentially multimodal—manual exploration is impractical. Gradient-based optimization is unsuitable due to:
\begin{itemize}
    \item Non-differentiability of signum/saturation functions
    \item Multimodal fitness landscapes (local minima)
    \item Expensive fitness evaluation (requires full simulation)
\end{itemize}

Particle Swarm Optimization (PSO) is well-suited to this problem due to its:
\begin{itemize}
    \item Derivative-free nature (handles discontinuous dynamics)
    \item Global search capability (escapes local minima)
    \item Parallel evaluation (exploits Monte Carlo structure)
\end{itemize}

However, existing PSO applications to SMC \cite{swaroop2000dynamic, gao2016particle} focus primarily on gain tuning for fixed boundary layers, not adaptive mechanisms.

\subsection{Gap 3: Single-Scenario Validation Conceals Brittleness}

The SMC literature overwhelmingly validates controllers under nominal conditions only—initial conditions within $\pm0.05$ rad, no external disturbances, nominal physical parameters. This practice conceals:
\begin{itemize}
    \item Overfitting to training distributions
    \item Brittleness to out-of-distribution states
    \item Lack of robustness guarantees beyond theory
\end{itemize}

While Lyapunov stability proves asymptotic convergence, it does not guarantee:
\begin{itemize}
    \item Bounded chattering for all initial conditions
    \item Disturbance rejection without integral action
    \item Performance maintenance under parameter variations
\end{itemize}

Rigorous validation requires multi-scenario stress testing across:
\begin{itemize}
    \item Diverse initial conditions ($\pm0.3$ rad, not just $\pm0.05$ rad)
    \item External disturbances (step, impulse, sinusoidal)
    \item Parameter variations (mass, length, friction uncertainties)
\end{itemize}

Our literature review (Chapter~\ref{chap:literature}) reveals that <10\% of SMC papers report performance under challenging initial conditions, and <5\% systematically quantify failure modes.

% ============================================================================
\section{Contributions}
\label{sec:contributions}

This thesis addresses the identified research gaps through three primary contributions:

\subsection{Contribution 1: PSO-Optimized Adaptive Boundary Layer for Chattering Reduction}

We propose an adaptive boundary layer mechanism where the effective boundary thickness dynamically adjusts based on sliding surface velocity:
\begin{equation}
    \epsilon_{\text{eff}}(t) = \epsilon_{\min} + \alpha|\dot{s}(t)|
\end{equation}

The parameters $(\epsilon_{\min}, \alpha, \lambda)$ are optimized using Particle Swarm Optimization with a chattering-weighted fitness function:
\begin{equation}
    F = 0.70 \cdot C + 0.15 \cdot T_s + 0.15 \cdot O
\end{equation}
where $C$ is chattering index (normalized control variation), $T_s$ is settling time, and $O$ is overshoot.

\textbf{Quantified Results} (based on 100 Monte Carlo trials):
\begin{itemize}
    \item \textbf{66.5\% chattering reduction}: Mean chattering index 0.077 vs. 0.230 (classical SMC), $p<0.001$, Cohen's $d=5.29$ (very large effect)
    \item \textbf{Zero energy penalty}: Control effort 98.2\% of baseline (not statistically significant, $p=0.32$)
    \item \textbf{Maintained tracking}: Settling time 2.14s vs. 2.08s (3\% increase, acceptable trade-off)
    \item \textbf{Optimized parameters}: $\epsilon_{\min}^* = 0.0025$, $\alpha^* = 1.21$, $\lambda^* = 7.94$
\end{itemize}

This contribution provides the first systematic PSO-based optimization of adaptive boundary layer parameters with rigorous multi-objective validation.

\subsection{Contribution 2: Lyapunov Stability Analysis for Time-Varying Boundary Layer}

We provide rigorous theoretical guarantees through two main theorems:

\textbf{Theorem 1 (Finite-Time Reaching)}: Under standard SMC assumptions (matched disturbances, controllable dynamics), the adaptive boundary layer control law drives system trajectories to the sliding surface in finite time:
\begin{equation}
    t_{\text{reach}} \leq \frac{\sqrt{2}|s(0)|}{\beta \eta}, \quad \eta = K - \bar{d} > 0
\end{equation}
where $\bar{d}$ is the disturbance bound and $\beta$ is the sliding surface rate gain.

\textbf{Theorem 2 (Ultimate Boundedness)}: Once inside the time-varying boundary layer $|s| \leq \epsilon_{\text{eff}}(t)$, the system trajectory remains ultimately bounded with error proportional to $\epsilon_{\min}$.

The key theoretical insight is that adaptive $\epsilon_{\text{eff}}(t)$ does not compromise stability because:
\begin{itemize}
    \item The control law outside the boundary layer remains discontinuous (classical SMC reaching phase)
    \item Inside the boundary layer, saturation ensures bounded control regardless of $\epsilon_{\text{eff}}$ magnitude
    \item The adaptation law $\alpha|\dot{s}|$ is uniformly continuous, preserving Lyapunov function decrease
\end{itemize}

This contribution provides the first formal stability proof for velocity-dependent adaptive boundary layers in the PSO-SMC context.

\subsection{Contribution 3: Honest Reporting of Generalization Failures and Robustness Limitations}

We identify and quantify critical limitations through systematic stress testing across four experiments:

\paragraph{MT-5 (Baseline Comparison)}
Five controller variants tested under identical conditions: Classical SMC, Super-Twisting SMC, Adaptive SMC, Hybrid Adaptive STA-SMC, and proposed PSO-optimized adaptive boundary layer. Results establish performance baselines and validate simulator accuracy.

\paragraph{MT-6 (Nominal Performance Validation)}
PSO-optimized controller tested under training distribution ($\pm0.05$ rad initial conditions, no disturbances). Confirms 66.5\% chattering reduction with statistical significance.

\paragraph{MT-7 (Robustness Stress Testing)}
\textbf{Major Finding}: PSO parameters optimized for $\pm0.05$ rad exhibit catastrophic failure under $\pm0.3$ rad initial conditions:
\begin{itemize}
    \item \textbf{50.4$\times$ chattering degradation}: Mean index 3.88 vs. 0.077 (nominal)
    \item \textbf{90.2\% failure rate}: Only 49/500 trials converged (failure defined as $\theta > 0.2$ rad at $t=10$s)
    \item \textbf{Severe overfitting}: Single-scenario PSO produces brittle controllers
\end{itemize}

\paragraph{MT-8 (Disturbance Rejection Analysis)}
\textbf{Major Finding}: Both classical and adaptive SMC achieve \textbf{0\% convergence} under external disturbances:
\begin{itemize}
    \item Step disturbance (10 N, $t=5$s): 0\% success
    \item Impulse disturbance (30 N·s, $t=5$s): 0\% success
    \item Sinusoidal disturbance (8 N, 2 Hz): 0\% success
\end{itemize}
Root cause: Classical SMC lacks integral action; PSO fitness function trained only on disturbance-free scenarios.

These negative results provide crucial insights:
\begin{itemize}
    \item \textbf{Multi-scenario PSO}: Optimization must include diverse initial conditions ($\pm0.3$ rad) and disturbance profiles in fitness evaluation
    \item \textbf{Disturbance-aware fitness}: Fitness function must explicitly penalize disturbance rejection failures
    \item \textbf{Honest validation}: Testing must extend significantly beyond training distribution to expose brittleness
\end{itemize}

Our honest reporting of failures advances the field by quantifying the brittleness problem that is likely present but unreported in prior single-scenario optimization studies, which typically validate only on training distributions.

% ============================================================================
\section{Significance of the Research}
\label{sec:significance}

This research has significance across three dimensions:

\subsection{Theoretical Significance}

\begin{itemize}
    \item \textbf{Stability Theory Extension}: First formal Lyapunov analysis of velocity-dependent adaptive boundary layers ($\epsilon_{\text{eff}} = \epsilon_{\min} + \alpha|\dot{s}|$) in sliding mode control

    \item \textbf{Multi-Objective Optimization Framework}: Principled formulation balancing chattering, energy, and tracking with explicit weight justification

    \item \textbf{Overfitting Characterization}: Quantification of generalization failure ($50.4\times$ degradation) providing theoretical motivation for robust multi-scenario PSO
\end{itemize}

\subsection{Practical Significance}

\begin{itemize}
    \item \textbf{Industrial Applicability}: 66.5\% chattering reduction enables SMC deployment in mechatronic systems (robotics, automotive) where actuator lifespan is critical

    \item \textbf{Energy Efficiency}: Zero energy penalty (98.2\% of baseline) makes approach viable for battery-powered systems

    \item \textbf{Systematic Design}: PSO-based parameter selection eliminates manual trial-and-error, reducing controller development time from weeks to hours

    \item \textbf{Failure Mode Identification}: Rigorous stress testing (MT-7, MT-8) identifies boundaries of safe operation, enabling fail-safe design
\end{itemize}

\subsection{Methodological Significance}

\begin{itemize}
    \item \textbf{Validation Best Practices}: Establishes rigorous multi-scenario testing as standard for SMC research (baseline, nominal, robustness, disturbance)

    \item \textbf{Honest Reporting Culture}: Demonstrates value of quantifying and publishing negative results (90.2\% failure rate, 0\% disturbance rejection) to advance field beyond incremental improvements

    \item \textbf{Statistical Rigor}: Comprehensive use of Welch's $t$-test, Cohen's $d$, bootstrap confidence intervals, and Monte Carlo validation (100--500 trials) sets standard for empirical SMC studies

    \item \textbf{Reproducibility}: Fixed random seeds, explicit parameter documentation, and open-source code enable independent verification
\end{itemize}

% ============================================================================
\section{Thesis Organization}
\label{sec:organization}

The remainder of this thesis is organized into eight chapters:

\textbf{Chapter~\ref{chap:literature} (Literature Review)} surveys related work on sliding mode control theory, chattering mitigation techniques (boundary layers, HOSMC, adaptive methods), particle swarm optimization for controller tuning, and double inverted pendulum control. A comparative analysis (Table~\ref{tab:literature_comparison}) positions our contributions within the state-of-the-art, highlighting the research gaps addressed by this thesis.

\textbf{Chapter~\ref{chap:modeling} (System Modeling)} presents the double inverted pendulum mathematical model, deriving the equations of motion via Lagrangian mechanics. Physical parameters (mass, length, friction) are specified, and the state-space representation is formulated. Model validation through energy conservation checks and linearization analysis is provided.

\textbf{Chapter~\ref{chap:controller} (Controller Design)} develops the sliding mode control framework, introducing the sliding surface design, equivalent control calculation, and switching control law. The adaptive boundary layer mechanism ($\epsilon_{\text{eff}} = \epsilon_{\min} + \alpha|\dot{s}|$) is presented with design rationale. Lyapunov stability analysis establishes finite-time convergence (Theorem~\ref{thm:reaching}) and ultimate boundedness (Theorem~\ref{thm:boundedness}).

\textbf{Chapter~\ref{chap:pso} (PSO Optimization)} describes the particle swarm optimization methodology for parameter tuning. The chattering-weighted fitness function design is justified through sensitivity analysis. PSO algorithm implementation details (swarm size, inertia weight, cognitive/social parameters) are specified. Convergence analysis demonstrates consistent optimization across 10 independent PSO runs, yielding $\epsilon_{\min}^* = 0.0025$, $\alpha^* = 1.21$, $\lambda^* = 7.94$.

\textbf{Chapter~\ref{chap:experiments} (Experimental Setup and Methodology)} details the simulation environment (MATLAB/Simulink, ODE45 solver, sampling rate), Monte Carlo validation framework (sample size calculation, random seed management), performance metrics (chattering index, settling time, overshoot, control effort), and statistical analysis procedures (Welch's $t$-test, Cohen's $d$, bootstrap confidence intervals).

\textbf{Chapter~\ref{chap:results} (Results and Analysis)} presents comprehensive experimental results across four experiments: MT-5 (baseline controller comparison), MT-6 (adaptive boundary layer validation demonstrating 66.5\% chattering reduction), MT-7 (robustness stress testing revealing 50.4$\times$ degradation), and MT-8 (disturbance rejection failure under external perturbations). Statistical validation confirms significance of all findings.

\textbf{Chapter~\ref{chap:discussion} (Discussion)} interprets the experimental results, comparing findings to literature and explaining success under nominal conditions (MT-6) versus failure under challenging conditions (MT-7, MT-8). Root cause analysis attributes generalization failure to single-scenario PSO overfitting and disturbance rejection failure to lack of integral action. Proposed solutions include multi-scenario robust PSO, disturbance-aware fitness functions, and integral sliding mode designs. Broader implications for the SMC community are discussed, emphasizing the need for honest validation practices.

\textbf{Chapter~\ref{chap:conclusions} (Conclusions and Future Work)} summarizes the thesis contributions, acknowledged limitations (single-scenario PSO, simulation-only validation, classical SMC without integral action), and future research directions. Priority areas include multi-scenario robust PSO implementation, disturbance-aware fitness function development, integral SMC integration, and hardware validation on physical DIP systems.

% ============================================================================
\section{Scope and Limitations}
\label{sec:scope}

This research is conducted within the following scope:

\subsection{Scope}

\begin{itemize}
    \item \textbf{System}: Double inverted pendulum (underactuated, 2 pendulum links, 1 control input)
    \item \textbf{Controller}: Classical sliding mode control with adaptive boundary layer (no higher-order methods)
    \item \textbf{Optimization}: Particle swarm optimization for parameter tuning (no gradient-based, genetic algorithms, or reinforcement learning)
    \item \textbf{Validation}: Simulation-only (MATLAB/Simulink with ODE45 solver, no hardware experiments)
    \item \textbf{Scenarios}: Four experiments (MT-5, MT-6, MT-7, MT-8) covering baseline, nominal, robustness, and disturbance
    \item \textbf{Statistics}: Monte Carlo (100--500 trials), Welch's $t$-test, Cohen's $d$, bootstrap CI
\end{itemize}

\subsection{Limitations}

The following limitations should be acknowledged when interpreting results:

\begin{enumerate}
    \item \textbf{Single-Scenario PSO Optimization}: Parameters optimized for nominal conditions ($\pm0.05$ rad, no disturbances) only. Multi-scenario robust PSO is proposed but not implemented due to time constraints.

    \item \textbf{Simulation-Only Validation}: Physical hardware experiments (friction modeling, sensor noise, actuator saturation, sampling jitter) are not conducted. Simulation results provide proof-of-concept but require hardware validation for industrial deployment.

    \item \textbf{Classical SMC Without Integral Action}: Controller lacks integral term, limiting disturbance rejection capability (MT-8 achieves 0\% convergence under external disturbances). Integral sliding mode designs are proposed as future work.

    \item \textbf{Fixed Sliding Surface Gains}: While boundary layer parameters ($\epsilon_{\min}$, $\alpha$) are optimized, sliding surface gains ($k_1$, $k_2$) remain fixed. Joint optimization may yield further improvements.

    \item \textbf{System-Specific Results}: Findings are demonstrated on double inverted pendulum and may not directly generalize to other underactuated systems (quadrotors, bipedal robots) without re-tuning.

    \item \textbf{Computational Cost}: PSO requires 500--1000 fitness evaluations (10--20 minutes on standard desktop). Real-time adaptation is not feasible; offline parameter optimization is assumed.
\end{enumerate}

\subsection{Future Work Beyond Scope}

The following topics are beyond the scope of this thesis but represent important future research directions:

\begin{itemize}
    \item Multi-scenario robust PSO with diverse initial conditions and disturbance profiles
    \item Hardware validation on physical DIP test stand
    \item Integral sliding mode control with joint parameter optimization
    \item Adaptive online boundary layer tuning (no offline PSO)
    \item Transfer to other underactuated systems (cart-pole, quadrotor, bipedal robot)
    \item Comparison with modern learning-based methods (reinforcement learning, neural network controllers)
\end{itemize}

% ============================================================================
\section{Summary}
\label{sec:summary_intro}

This chapter introduced the research problem of chattering in sliding mode control, motivated by the need for robust controllers that are industrially deployable. The double inverted pendulum was established as a challenging benchmark exhibiting underactuation, nonlinearity, and instability. Three research gaps were identified: fixed boundary layers that ignore state-space variation, manual parameter tuning preventing systematic optimization, and single-scenario validation concealing brittleness.

The thesis contributions address these gaps through PSO-optimized adaptive boundary layers (66.5\% chattering reduction), Lyapunov stability analysis for time-varying boundaries, and honest reporting of generalization failures (50.4$\times$ degradation) and disturbance rejection failures (0\% convergence). The significance spans theoretical (stability theory extension), practical (industrial applicability, energy efficiency), and methodological (validation best practices, honest reporting culture) dimensions.

The thesis is organized into nine chapters, progressing from literature review (Chapter~\ref{chap:literature}) through system modeling (Chapter~\ref{chap:modeling}), controller design (Chapter~\ref{chap:controller}), PSO optimization (Chapter~\ref{chap:pso}), experimental methodology (Chapter~\ref{chap:experiments}), results (Chapter~\ref{chap:results}), discussion (Chapter~\ref{chap:discussion}), and conclusions (Chapter~\ref{chap:conclusions}). The scope is limited to simulation-only validation, classical SMC without integral action, and single-scenario PSO optimization—acknowledged limitations that motivate future research directions.

The next chapter reviews related literature on sliding mode control, chattering mitigation, particle swarm optimization, and double inverted pendulum control, positioning this thesis within the state-of-the-art.
