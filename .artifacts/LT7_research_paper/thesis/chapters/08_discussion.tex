\chapter{Discussion}
\label{ch:discussion}

This chapter interprets the experimental results from Chapter~\ref{ch:results_analysis}, compares our findings to the state-of-the-art literature reviewed in Chapter~\ref{ch:literature_review}, analyzes the root causes of generalization and disturbance rejection failures, proposes solutions for future work, and discusses broader implications for the sliding mode control research community. Section~\ref{sec:mt6_interpretation} examines the mechanisms underlying the 66.5\% chattering reduction achieved in MT-6. Section~\ref{sec:generalization_failure_discussion} analyzes the catastrophic 50.4$\times$ performance degradation observed in MT-7 robustness testing. Section~\ref{sec:disturbance_failure_discussion} investigates the 0\% convergence rate under external disturbances (MT-8). Section~\ref{sec:proposed_solutions} proposes concrete research directions to address these limitations. Section~\ref{sec:broader_implications} discusses methodological implications for the SMC research community, including validation practices and honest reporting of negative results.

% ============================================================================
\section{Interpretation of Adaptive Boundary Layer Success}
\label{sec:mt6_interpretation}

The MT-6 results demonstrate that PSO-optimized adaptive boundary layer parameters achieve \textbf{66.5\% chattering reduction} (6.37 $\to$ 2.14, $p < 0.001$, Cohen's $d = 5.29$) with \textbf{zero energy penalty} ($p = 0.339$) compared to fixed boundary layer SMC. This represents a substantial advancement over recent literature.

% ----------------------------------------------------------------------------
\subsection{Comparison with State-of-the-Art}
\label{subsec:sota_comparison}

Our chattering reduction compares favorably to recent approaches reviewed in Chapter~\ref{ch:literature_review}:

\begin{itemize}
    \item \textbf{Fuzzy-adaptive methods}~\cite{fuzzy_smc_frontiers2024,sfa_smc2024} report qualitative chattering mitigation but lack quantitative metrics. Our 66.5\% reduction provides a concrete benchmark with statistical significance ($p < 0.001$).

    \item \textbf{Higher-order SMC}~\cite{ayinalem2025hosmc,hepso_smc2025} achieve smooth control through integral action at the cost of increased complexity (additional state variables, observers). Our adaptive boundary layer maintains first-order SMC simplicity while achieving comparable chattering reduction through systematic PSO optimization.

    \item \textbf{Hybrid frameworks}~\cite{fuzzy_hybrid_scirep2024} (Fuzzy + SMC) demonstrate chattering reduction but require extensive domain expertise for fuzzy rule design. Our PSO approach automates parameter selection, making the methodology transferable to other systems without manual tuning.
\end{itemize}

\textbf{Key Distinction:} We are the first to report effect size (Cohen's $d = 5.29$), indicating the chattering reduction is not only statistically significant but also profoundly meaningful in practice. This ``very large'' effect size ($d > 0.8$ by Cohen's conventions~\cite{cohen1988statistical}) suggests the adaptive boundary layer fundamentally alters control behavior, not merely provides marginal improvement.

% ----------------------------------------------------------------------------
\subsection{Mechanism Analysis}
\label{subsec:mechanism_analysis}

The 66.5\% chattering reduction stems from three complementary mechanisms:

\textbf{(a) Dynamic Adaptation to System State:}
The formula $\eeff = \emin + \alpha|\dot{s}|$ (Equation~\ref{eq:adaptive_boundary}) automatically adjusts boundary layer thickness based on sliding surface derivative magnitude. During the reaching phase (large $|\dot{s}|$), the boundary layer widens ($\eeff$ increases), smoothing the control input and preventing high-frequency oscillations. Near equilibrium (small $|\dot{s}|$), the boundary layer narrows ($\eeff \approx \emin$), preserving tracking precision.

\textbf{(b) PSO-Optimal Parameter Selection:}
The optimized parameters $\emin^* = 0.0025$ and $\alpha^* = 1.21$ (Equation~\ref{eq:optimized_params}) represent a Pareto-optimal tradeoff between chattering (70\% weight), settling time (15\%), and overshoot (15\%). Manual tuning would unlikely discover this optimal combination, as the fitness landscape is non-convex with multiple local minima (evidenced by PSO's 38.4\% fitness improvement over initial random particles, Figure~\ref{fig:pso_convergence}).

\textbf{(c) No Energy Penalty:}
The zero energy penalty ($p = 0.339$) is critical for industrial deployment. Both fixed and adaptive boundary layers exhibit identical mean control energy (5,232 N$^2 \cdot$s), confirming that chattering reduction does not come at the cost of increased actuator effort. This distinguishes our approach from higher-order SMC methods that inherently require additional control authority.

% ----------------------------------------------------------------------------
\subsection{Practical Implications}
\label{subsec:practical_implications}

For industrial mechatronic systems, the 66.5\% chattering reduction translates to:
\begin{itemize}
    \item \textbf{Extended actuator lifespan:} Reduced high-frequency mechanical stress (wear on bearings, gears, hydraulic valves)~\cite{young1999survey}
    \item \textbf{Improved control precision:} Lower oscillations enable tighter trajectory tracking in robotic applications
    \item \textbf{Energy efficiency:} Chattering amplitude reduction directly correlates with reduced energy waste in control oscillations
\end{itemize}

The statistical robustness (non-overlapping 95\% CIs: Fixed [6.13, 6.61], Adaptive [2.11, 2.16]) ensures the result is reproducible across different initial conditions within the training distribution.

% ============================================================================
\section{Analysis of Generalization Failure}
\label{sec:generalization_failure_discussion}

The MT-7 results reveal a critical limitation: PSO parameters optimized for $\pm 0.05$ rad initial conditions exhibit \textbf{50.4$\times$ chattering degradation} (2.14 $\to$ 107.61) and \textbf{90.2\% failure rate} when tested under $\pm 0.3$ rad initial conditions. This catastrophic generalization failure demands careful analysis.

% ----------------------------------------------------------------------------
\subsection{Root Cause: Single-Scenario Overfitting}
\label{subsec:overfitting_analysis}

The generalization failure stems from \textbf{optimization bias toward the training distribution}:

\textbf{(a) Narrow Search Space:}
PSO particles explored parameter combinations exclusively evaluated on initial conditions sampled from $\mathcal{U}(-0.05, 0.05)$ rad. This narrow distribution represents only $\sim$17\% of the $\pm 0.3$ rad operating range tested in MT-7. The fitness function (Equation~\ref{eq:fitness_total}) provided no incentive to achieve robustness beyond this range, leading to parameters optimized for a specific corner of the state space.

\textbf{(b) Boundary Layer Inadequacy for Large Errors:}
The adaptive formula $\eeff = \emin + \alpha|\dot{s}|$ with $\alpha = 1.21$ produces boundary layer thicknesses appropriate for small initial errors. However, when $|\dot{s}|$ increases by 6$\times$ (due to 6$\times$ larger initial angles), $\eeff$ also increases proportionally. For large $\eeff$, the saturation region $|s| \leq \eeff$ may encompass the entire reachable state space, effectively disabling the switching control and causing divergence.

\textbf{(c) Gain Mismatch:}
The switching gain $K$ optimized for $\pm 0.05$ rad disturbances may be insufficient for the larger matched disturbances equivalent to $\pm 0.3$ rad initial errors. The Lyapunov stability requirement $K > \bar{d}$ (Theorem~\ref{thm:lyapunov_stability}) remains valid, but the effective disturbance bound $\bar{d}$ increases with initial condition magnitude.

% ----------------------------------------------------------------------------
\subsection{Comparison with Literature}
\label{subsec:generalization_literature}

\textbf{Critical Observation:} All PSO-SMC studies reviewed in Chapter~\ref{ch:literature_review}~\cite{ayinalem2025hosmc,hepso_smc2025,quadcopter_smc2025} validated controllers only on the training distribution. None reported robustness testing beyond optimization conditions. Our MT-7 results suggest these controllers likely suffer similar generalization failures if tested outside their training domains, but such failures go unreported.

\textbf{Implication:} The SMC literature exhibits a \textbf{validation gap}---controllers are optimized and validated on identical distributions, providing optimistic performance estimates that do not reflect real-world robustness. Our honest reporting of 50.4$\times$ degradation fills this gap and establishes a precedent for rigorous validation practices.

% ----------------------------------------------------------------------------
\subsection{Why Existing Approaches May Avoid This Issue}
\label{subsec:online_adaptation}

Interestingly, the adaptive boundary layer methods reviewed in Section~\ref{sec:chattering_mitigation} (self-regulated, fuzzy-adaptive) may exhibit better generalization due to \textbf{continuous online adaptation}. Unlike our fixed PSO parameters, these methods adjust boundary layers in real-time based on current tracking error. However, this comes at the cost of:
\begin{itemize}
    \item \textbf{Increased complexity:} Additional adaptation laws, potential parameter drift
    \item \textbf{Stability challenges:} No Lyapunov guarantees for arbitrary adaptation rules
    \item \textbf{Computational burden:} Real-time optimization or fuzzy inference
\end{itemize}

Our approach trades off online adaptability for simplicity and theoretical guarantees (Theorem~\ref{thm:lyapunov_stability}), but exposes the brittleness of single-scenario optimization.

% ============================================================================
\section{Disturbance Rejection Failure Analysis}
\label{sec:disturbance_failure_discussion}

The MT-8 results show \textbf{0\% convergence} across all controllers (Classical, STA, Adaptive) and all disturbance types (step, impulse, sinusoidal). This universal failure reveals a fundamental limitation of nominal-condition optimization.

% ----------------------------------------------------------------------------
\subsection{Root Cause: Fitness Function Myopia}
\label{subsec:fitness_myopia}

The PSO fitness function (Equation~\ref{eq:fitness_total}) optimized for:

\begin{equation}
\label{eq:nominal_fitness}
F = 0.70 \cdot C + 0.15 \cdot T_s + 0.15 \cdot O
\end{equation}

under \textbf{disturbance-free conditions}. The chattering index $C$, settling time $T_s$, and overshoot $O$ were measured assuming no external perturbations. Consequently, the optimizer discovered parameters that minimize chattering for nominal trajectories but provide insufficient robustness margin for disturbance rejection.

% ----------------------------------------------------------------------------
\subsection{Classical SMC Limitation: No Integral Action}
\label{subsec:integral_action}

The classical SMC control law (Equation~\ref{eq:smc_control_law}) lacks integral action:

\begin{equation}
u = u_{\text{eq}} - K \cdot \text{sat}(s/\eeff) - k_d \cdot s
\end{equation}

This structure cannot reject \textbf{constant disturbances} (e.g., 10 N step force). The equivalent control $u_{\text{eq}}$ cancels nominal dynamics, but the switching term $-K \cdot \text{sat}(s/\eeff)$ provides only proportional feedback on the sliding variable. Without an integral term, steady-state errors persist indefinitely.

\textbf{Contrast with Integral SMC:}
An integral sliding surface $s = e + \lambda \int e \, dt$ inherently rejects constant disturbances. The integral term accumulates error over time, forcing the controller to compensate. This explains why existing literature (Section~\ref{sec:smc_fundamentals}) often employs higher-order SMC (which includes implicit integral action via auxiliary states) for disturbance-prone environments~\cite{levant2007principles}.

% ----------------------------------------------------------------------------
\subsection{Comparison with Literature}
\label{subsec:disturbance_literature}

The disturbance rejection literature (Section~\ref{sec:chattering_mitigation}) achieves robustness through:
\begin{itemize}
    \item \textbf{Extended State Observers (ESO):} Estimate and compensate unmatched disturbances~\cite{chen2016disturbance}
    \item \textbf{Disturbance Observers:} Reconstruct external forces and cancel them explicitly
    \item \textbf{Robust optimization:} Include worst-case disturbance scenarios in fitness evaluation
\end{itemize}

Our work demonstrates that \textbf{ignoring disturbances during optimization produces brittle controllers}, even when the control law theoretically possesses disturbance rejection properties (through switching gain $K > \bar{d}$). The practical lesson: robustness must be explicitly optimized, not assumed.

% ============================================================================
\section{Proposed Solutions and Future Directions}
\label{sec:proposed_solutions}

The MT-7 and MT-8 failures motivate three concrete future research directions:

% ----------------------------------------------------------------------------
\subsection{Multi-Scenario Robust PSO}
\label{subsec:multi_scenario_pso}

\textbf{Objective:} Optimize parameters robust to diverse operating conditions.

\textbf{Approach:}
\begin{itemize}
    \item \textbf{Fitness function redesign:} Include multiple initial condition distributions and disturbance scenarios
    \begin{equation}
    \label{eq:robust_minimax_fitness}
    F_{\text{robust}} = \max_{\text{scenario } i} \left( 0.70 \cdot C_i + 0.15 \cdot T_{s,i} + 0.15 \cdot O_i \right)
    \end{equation}
    where the fitness is the \textbf{worst-case performance} across scenarios (minimax optimization).

    \item \textbf{Scenario diversity:} Sample initial conditions from $\mathcal{U}(-0.3, 0.3)$ rad during PSO evaluation, include disturbance rejection trials with step/impulse/sinusoidal forces.

    \item \textbf{Computational cost:} Multi-scenario fitness evaluation increases cost 5--10$\times$ (if using 5--10 scenarios per particle), requiring parallel PSO implementation or reduced swarm size.
\end{itemize}

\textbf{Expected Outcome:} Parameters that generalize beyond training distribution, with graceful degradation rather than catastrophic failure.

% ----------------------------------------------------------------------------
\subsection{Disturbance-Aware Fitness Function}
\label{subsec:disturbance_fitness}

\textbf{Objective:} Explicitly penalize poor disturbance rejection during optimization.

\textbf{Approach:}
\begin{itemize}
    \item \textbf{Augmented fitness:}
    \begin{equation}
    \label{eq:disturbance_fitness}
    F_{\text{robust}} = 0.50 \cdot C_{\text{nominal}} + 0.20 \cdot C_{\text{disturbed}} + 0.15 \cdot T_s + 0.15 \cdot O
    \end{equation}
    where $C_{\text{disturbed}}$ measures chattering/divergence under external perturbations.

    \item \textbf{Disturbance injection:} Apply step/impulse/sinusoidal forces during PSO evaluation trials, penalize divergence heavily (e.g., $C_{\text{disturbed}} = 1000$ if divergence occurs).
\end{itemize}

\textbf{Expected Outcome:} Parameters that balance nominal chattering reduction with disturbance rejection capability.

% ----------------------------------------------------------------------------
\subsection{Integral Sliding Mode Control with PSO}
\label{subsec:integral_smc_pso}

\textbf{Objective:} Eliminate steady-state errors under constant disturbances.

\textbf{Approach:}
\begin{itemize}
    \item \textbf{Integral sliding surface:}
    \begin{equation}
    \label{eq:integral_sliding_surface}
    s = k_1(\dot{\theta}_1 + \lambda_1\theta_1) + k_2(\dot{\theta}_2 + \lambda_2\theta_2) + k_I \int_{0}^{t} (k_1\theta_1 + k_2\theta_2) \, d\tau
    \end{equation}
    The integral term $k_I \int (\cdot)$ provides disturbance rejection.

    \item \textbf{PSO optimization:} Tune gains $(k_1, k_2, \lambda_1, \lambda_2, k_I, \emin, \alpha)$ using disturbance-aware fitness (Equation~\ref{eq:disturbance_fitness}).
\end{itemize}

\textbf{Expected Outcome:} Zero steady-state error under constant disturbances, preserved chattering reduction through adaptive boundary layer.

% ----------------------------------------------------------------------------
\subsection{Hardware Validation}
\label{subsec:hardware_validation}

\textbf{Objective:} Validate simulation results on physical DIP system.

\textbf{Challenges:}
\begin{itemize}
    \item \textbf{Actuator limitations:} Real motors have friction, backlash, bandwidth limits not modeled in simulation
    \item \textbf{Sensor noise:} Angular position measurements contain noise, affecting $\dot{s}$ estimation (Equation~\ref{eq:derivative_filtering})
    \item \textbf{Unmodeled dynamics:} Flexible modes, cable dynamics, air resistance
\end{itemize}

\textbf{Approach:}
\begin{itemize}
    \item Implement controller on real-time embedded system (e.g., dSPACE, NI CompactRIO)
    \item Test under same initial condition distributions ($\pm 0.05$ rad, $\pm 0.3$ rad)
    \item Measure actual chattering using accelerometers on pendulum joints
\end{itemize}

\textbf{Expected Outcome:} Validation of simulation findings or identification of simulation-to-reality gap requiring model refinement.

% ============================================================================
\section{Broader Implications for the SMC Community}
\label{sec:broader_implications}

Our findings carry important lessons for the sliding mode control research community:

% ----------------------------------------------------------------------------
\subsection{Honest Reporting of Negative Results}
\label{subsec:negative_results}

The 50.4$\times$ generalization failure and 0\% disturbance rejection are \textbf{negative results} that many researchers might omit from publications. However, reporting failures is critical for:
\begin{itemize}
    \item \textbf{Reproducibility:} Preventing others from repeating the same mistakes
    \item \textbf{Scientific integrity:} Providing complete picture of controller performance
    \item \textbf{Future progress:} Identifying concrete limitations to address in follow-on work
\end{itemize}

\textbf{Recommendation:} The SMC community should encourage reporting of generalization failures, disturbance rejection limitations, and validation beyond training distributions. Journals and conferences should value honest negative results as contributions, not weaknesses.

% ----------------------------------------------------------------------------
\subsection{Validation Beyond Training Distributions}
\label{subsec:validation_practices}

The ubiquitous practice of validating controllers only on the same distribution used for optimization (observed in all Section~\ref{sec:chattering_mitigation} studies) provides \textbf{optimistically biased performance estimates}. Robust validation requires:
\begin{itemize}
    \item \textbf{Out-of-distribution testing:} Initial conditions 2--10$\times$ larger than training range
    \item \textbf{Cross-validation:} Separate training and test sets for Monte Carlo trials
    \item \textbf{Stress testing:} Extreme scenarios (disturbances, parameter variations, sensor noise)
\end{itemize}

\textbf{Recommendation:} Establish a standard validation protocol for optimized controllers: (1) optimize on distribution A, (2) validate on distribution A (in-distribution performance), (3) validate on distribution B $\gg$ A (out-of-distribution robustness), (4) report both results with clear distinction.

% ----------------------------------------------------------------------------
\subsection{Multi-Objective vs. Single-Objective Optimization}
\label{subsec:pareto_fronts}

Our chattering-weighted fitness function (70-15-15) represents a specific design choice prioritizing chattering reduction. However, different applications may require different tradeoffs:
\begin{itemize}
    \item \textbf{Industrial robots:} Prioritize chattering (wear reduction)
    \item \textbf{Aerospace systems:} Prioritize energy efficiency (battery life)
    \item \textbf{Medical devices:} Prioritize precision (settling time, overshoot)
\end{itemize}

\textbf{Recommendation:} Future PSO-SMC research should report \textbf{Pareto fronts} (multi-objective optimization results) rather than single optimized points, allowing designers to select parameters appropriate for their application requirements.

% ----------------------------------------------------------------------------
\subsection{Theoretical Stability vs. Empirical Robustness}
\label{subsec:theory_vs_practice}

Our Lyapunov stability proof (Theorem~\ref{thm:lyapunov_stability}) guarantees finite-time convergence under Assumptions 1--4 (matched disturbances, $K > \bar{d}$, controllability, positive gains). The MT-6 success validates this theory for nominal conditions. However, the MT-7/MT-8 failures demonstrate that:
\begin{itemize}
    \item \textbf{Theoretical stability $\neq$ practical robustness:} Lyapunov guarantees are asymptotic (infinite time), but finite-time performance depends on gains
    \item \textbf{Assumptions matter:} Violation of matched disturbance assumption (MT-8) or exceeding disturbance bound (MT-7) invalidates guarantees
\end{itemize}

\textbf{Recommendation:} SMC researchers should complement Lyapunov analysis with systematic robustness testing (Monte Carlo, worst-case scenarios, sensitivity analysis) to ensure theoretical guarantees translate to practical performance.

% ============================================================================
\section{Summary}
\label{sec:chapter8_summary}

This chapter provided critical analysis of the experimental results presented in Chapter~\ref{ch:results_analysis}, comparing our findings to state-of-the-art literature, identifying root causes of failures, and proposing concrete solutions:

\textbf{Key Insights:}
\begin{itemize}
    \item \textbf{MT-6 Success Mechanisms} (Section~\ref{sec:mt6_interpretation}): 66.5\% chattering reduction stems from dynamic state-dependent adaptation, PSO-optimal parameter selection, and zero energy penalty, comparing favorably to fuzzy-adaptive, HOSMC, and hybrid approaches with exceptional effect size ($d = 5.29$)

    \item \textbf{MT-7 Generalization Failure} (Section~\ref{sec:generalization_failure_discussion}): 50.4$\times$ degradation results from single-scenario overfitting, narrow training distribution ($\pm 0.05$ rad represents only 17\% of $\pm 0.3$ rad test range), boundary layer inadequacy for large errors, and gain mismatch; literature exhibits validation gap (optimize = test distributions)

    \item \textbf{MT-8 Disturbance Rejection Failure} (Section~\ref{sec:disturbance_failure_discussion}): 0\% convergence stems from fitness function myopia (no disturbance scenarios), lack of integral action in classical SMC structure, and optimization bias toward nominal conditions

    \item \textbf{Proposed Solutions} (Section~\ref{sec:proposed_solutions}): Multi-scenario robust PSO with minimax fitness (Equation~\ref{eq:robust_minimax_fitness}), disturbance-aware fitness function (50-20-15-15 weights, Equation~\ref{eq:disturbance_fitness}), integral SMC with PSO-optimized gains, and hardware validation on physical DIP system

    \item \textbf{Methodological Implications} (Section~\ref{sec:broader_implications}): Encourage honest reporting of negative results, establish out-of-distribution validation protocols (test on 2--10$\times$ larger ranges), report Pareto fronts for multi-objective tradeoffs, complement Lyapunov analysis with empirical robustness testing
\end{itemize}

\textbf{Central Conclusion:} Single-scenario PSO optimization achieves dramatic performance improvements for trained conditions (66.5\% chattering reduction, $p < 0.001$) but fails catastrophically outside the training distribution (50.4$\times$ degradation, 90.2\% failure rate), motivating future research on robust multi-scenario optimization and rigorous validation practices for the SMC research community. The honest presentation of both successes and failures provides a complete, reproducible picture that advances the field toward more robust controller design methodologies.

Chapter~\ref{ch:future_work} concludes the thesis with a summary of contributions, explicit acknowledgment of limitations, answers to the research questions posed in Chapter~\ref{ch:introduction}, and a detailed roadmap for future work addressing the identified gaps.
