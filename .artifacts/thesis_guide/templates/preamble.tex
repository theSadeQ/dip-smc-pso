% ==============================================================================
% PREAMBLE - Package Configuration and Custom Commands
% ==============================================================================
% Contains all package imports, page layout, typography settings,
% and custom LaTeX commands for the thesis.
% ==============================================================================

% ==============================================================================
% ESSENTIAL PACKAGES
% ==============================================================================

% Document layout and geometry
\usepackage[letterpaper,margin=1in]{geometry}
\usepackage{setspace}
\onehalfspacing  % 1.5 line spacing for readability

% Font and typography
\usepackage{times}           % Times New Roman font (similar to Times)
\usepackage[T1]{fontenc}     % Font encoding
\usepackage[utf8]{inputenc}  % Input encoding for special characters
\usepackage{microtype}       % Improved typography (spacing, kerning)

% Mathematics
\usepackage{amsmath}         % Advanced math environments
\usepackage{amssymb}         % Math symbols
\usepackage{amsfonts}        % Math fonts
\usepackage{mathtools}       % Extensions to amsmath
\usepackage{bm}              % Bold math symbols

% Figures and graphics
\usepackage{graphicx}        % Include images
\usepackage{float}           % Better float control ([H] option)
\usepackage{subcaption}      % Subfigures and subtables
\usepackage{wrapfig}         % Wrap text around figures

% Tables
\usepackage{booktabs}        % Professional table rules
\usepackage{longtable}       % Multi-page tables
\usepackage{multirow}        % Multi-row table cells
\usepackage{array}           % Extended column formatting

% Colors
\usepackage[usenames,dvipsnames,table]{xcolor}

% Algorithms and code listings
\usepackage{algorithm}       % Algorithm environment
\usepackage{algpseudocode}   % Pseudocode formatting
\usepackage{listings}        % Source code listings

% Citations and bibliography
\usepackage{cite}            % Improved citations
\usepackage{natbib}          % Natural sciences bibliography

% Cross-references
\usepackage{hyperref}        % Hyperlinks (load near end)
\usepackage{cleveref}        % Intelligent cross-references (load after hyperref)

% Additional utilities
\usepackage{textcomp}        % Additional text symbols
\usepackage{gensymb}         % Generic symbols (degree, etc.)
\usepackage{siunitx}         % SI units formatting
\usepackage{enumitem}        % Customize lists

% ==============================================================================
% PAGE LAYOUT
% ==============================================================================

% Header and footer
\usepackage{fancyhdr}
\pagestyle{fancy}
\fancyhf{}  % Clear all header/footer
\fancyhead[LE,RO]{\thepage}  % Page numbers on outer edges
\fancyhead[RE]{\leftmark}    % Chapter on even pages
\fancyhead[LO]{\rightmark}   % Section on odd pages
\renewcommand{\headrulewidth}{0.5pt}

% Chapter and section formatting
\usepackage{titlesec}
\titleformat{\chapter}[display]
  {\normalfont\huge\bfseries}{\chaptertitlename\ \thechapter}{20pt}{\Huge}
\titlespacing*{\chapter}{0pt}{0pt}{40pt}

% ==============================================================================
% HYPERREF CONFIGURATION
% ==============================================================================

\hypersetup{
    colorlinks=true,
    linkcolor=blue,
    citecolor=blue,
    filecolor=magenta,
    urlcolor=cyan,
    pdftitle={Sliding Mode Control of Double-Inverted Pendulum with PSO},
    pdfauthor={Your Name},
    pdfsubject={Master's Thesis - Control Systems},
    pdfkeywords={sliding mode control, double inverted pendulum, PSO, optimization},
    bookmarksdepth=3,
    bookmarksopen=true
}

% ==============================================================================
% LISTINGS (CODE) CONFIGURATION
% ==============================================================================

\lstset{
    basicstyle=\ttfamily\small,
    keywordstyle=\color{blue}\bfseries,
    commentstyle=\color{gray}\itshape,
    stringstyle=\color{red},
    numbers=left,
    numberstyle=\tiny\color{gray},
    stepnumber=1,
    numbersep=8pt,
    backgroundcolor=\color{gray!10},
    frame=single,
    frameround=tttt,
    rulecolor=\color{black},
    breaklines=true,
    breakatwhitespace=false,
    tabsize=4,
    captionpos=b,
    xleftmargin=2em,
    framexleftmargin=1.5em
}

% Python syntax highlighting
\lstdefinestyle{python}{
    language=Python,
    morekeywords={self, as, with, yield}
}

% ==============================================================================
% CUSTOM MATH COMMANDS
% ==============================================================================

% Vectors (bold lowercase)
\newcommand{\vect}[1]{\bm{#1}}

% Matrices (bold uppercase)
\newcommand{\mat}[1]{\bm{#1}}

% Real numbers
\newcommand{\Real}{\mathbb{R}}

% Norm
\newcommand{\norm}[1]{\left\|#1\right\|}

% Absolute value
\newcommand{\abs}[1]{\left|#1\right|}

% Derivative
\newcommand{\diff}[2]{\frac{d#1}{d#2}}

% Partial derivative
\newcommand{\pdiff}[2]{\frac{\partial#1}{\partial#2}}

% Dot notation (time derivative)
\newcommand{\ddt}[1]{\dot{#1}}

% Double dot (second derivative)
\newcommand{\dddt}[1]{\ddot{#1}}

% Sign function
\DeclareMathOperator{\sign}{sign}

% Saturation function
\DeclareMathOperator{\sat}{sat}

% State vector (common notation)
\newcommand{\state}{\vect{x}}
\newcommand{\control}{\vect{u}}
\newcommand{\output}{\vect{y}}

% ==============================================================================
% CUSTOM ENVIRONMENTS
% ==============================================================================

% Theorem environment
\usepackage{amsthm}
\newtheorem{theorem}{Theorem}[chapter]
\newtheorem{lemma}[theorem]{Lemma}
\newtheorem{proposition}[theorem]{Proposition}
\newtheorem{corollary}[theorem]{Corollary}

% Definition environment
\theoremstyle{definition}
\newtheorem{definition}[theorem]{Definition}
\newtheorem{example}[theorem]{Example}

% Remark environment
\theoremstyle{remark}
\newtheorem{remark}[theorem]{Remark}
\newtheorem{note}[theorem]{Note}

% ==============================================================================
% SIUNITX CONFIGURATION (SI Units)
% ==============================================================================

\sisetup{
    inter-unit-product = \ensuremath{{}\cdot{}},
    per-mode = symbol,
    separate-uncertainty = true,
    uncertainty-mode = separate,
    range-phrase = --,
    range-units = single
}

% ==============================================================================
% CLEVEREF CONFIGURATION (Smart Cross-References)
% ==============================================================================

\crefname{equation}{Eq.}{Eqs.}
\Crefname{equation}{Equation}{Equations}
\crefname{figure}{Fig.}{Figs.}
\Crefname{figure}{Figure}{Figures}
\crefname{table}{Tab.}{Tabs.}
\Crefname{table}{Table}{Tables}
\crefname{chapter}{Chapter}{Chapters}
\crefname{section}{Section}{Sections}
\crefname{appendix}{Appendix}{Appendices}
\crefname{algorithm}{Algorithm}{Algorithms}

% ==============================================================================
% END OF PREAMBLE
% ==============================================================================
