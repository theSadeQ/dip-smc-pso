\documentclass[11pt]{article}
\usepackage[utf8]{inputenc}
\usepackage{amsmath,amssymb}
\usepackage[margin=1in]{geometry}

\title{Section 3\textcolon Controller Design}
\date{December 25, 2025}

\begin{document}
\maketitle

\section{Controller Design}

This section presents the control law design for each of the seven SMC variants evaluated in this study. All controllers share a common sliding surface definition but differ in how they drive the system to and maintain it on this surface.

\subsection{Sliding Surface (Common to All SMC Variants)}

Definition:

The sliding surface $\sigma: \mathbb{R}^6 \to \mathbb{R}$ combines pendulum angle errors and their derivatives:


where:
- $\lambda 1, \lambda 2 > 0$ - position error weights
- $k 1, k 2 > 0$ - velocity error weights

Physical Interpretation:

The sliding surface represents a weighted combination of pendulum state errors. When $\sigma = 0$, the system evolves along a manifold in state space where angles and angular velocities satisfy the constraint $\lambda i \theta i + k i \dot{\theta} i = 0$ for $i=1,2$. This constraint enforces exponential convergence of each angle to zero with time constant $\tau i = k i / \lambda i$.

Design Philosophy:

- Reaching Phase: Drive system toward sliding surface ($\sigma \to 0$)
- Sliding Phase: Maintain system on surface ($\sigma = 0$), ensuring exponential convergence to equilibrium
- Steady-State: System remains at equilibrium ($\theta 1 = \theta 2 = 0$)

---

\subsubsection{Controller Architecture Overview}

All seven SMC variants in this study share a common architecture pattern but differ in specific implementation of the control law and how they handle uncertainties.

Figure 3.1: Common SMC architecture for DIP stabilization


Controller Family Tree:


Architectural Differences:


[TABLE - See Markdown version for details]


This architectural overview provides context for understanding design tradeoffs: simplicity (Classical) vs performance (STA) vs adaptability (Adaptive/Hybrid).

---

\subsection{Classical Sliding Mode Control}

Control Law:


where:
- $u {\text{eq}}$ - equivalent control (model-based feedforward)
- $K > 0$ - switching gain (drives system to sliding surface)
- $\epsilon > 0$ - boundary layer width (chattering reduction)
- $k d \geq 0$ - derivative gain (damping)
- $\text{sat}(\cdot)$ - saturation function (continuous approximation of sign function)

Equivalent Control:

The equivalent control compensates for known dynamics:


where:
- $L = [0, k 1, k 2]$ - sliding surface gradient vector
- $M, C, G$ - inertia, Coriolis, gravity matrices from Section 2
- $B = [1, 0, 0]^T$ - control input matrix

Saturation Function (Boundary Layer):

Two options implemented:

- Hyperbolic Tangent (Default):
   Smooth transition, maintains control authority near $\sigma=0$

- Linear Saturation:
   Piecewise linear, sharper switching

Design Parameters:


[TABLE - See Markdown version for details]


Advantages:
- Simple implementation (6 gains)
- Fastest computation (18.5 mus, Section 7.1)
- Well-understood theory
- Good energy efficiency (12.4 J, Section 7.4)

Disadvantages:
- Moderate chattering (index 8.2, Section 7.3)
- Larger overshoot (5.8percent, Section 7.2)
- Boundary layer introduces steady-state error

Implementation Notes:

Discretization (dt = 0.01s, 100 Hz control loop):

The continuous-time control law must be discretized for digital implementation:

- Sliding Surface: Direct substitution (no discretization error)

- Equivalent Control: Use backward differentiation for stability

- Saturation Function: tanh is inherently continuous, no discretization needed

Numerical Stability:

- Matrix Inversion: M(q) is always invertible (positive definite) but can become ill-conditioned for large theta. Use LU decomposition (scipy.linalg.solve) instead of explicit inv(M)
- Overflow Prevention: Clip intermediate calculations: u eq limited to $\pm$100N before adding switching term
- Derivative Estimation: Use filtered backward difference for thetȧ (Butterworth 2nd-order, 20 Hz cutoff) to reduce noise amplification

Computational Breakdown (18.5 mus total):


[TABLE - See Markdown version for details]


Common Pitfalls:

- Chattering from small epsilon: Setting epsilon<0.01 causes high-frequency switching (>50 Hz). Stay above epsilon$\geq$0.02 for dt=0.01s.
- Instability from large k d: Derivative gain k d>5.0 can cause oscillations due to noise amplification in thetȧ estimates.
- Steady-state error from large epsilon: Boundary layer epsilon>0.1 introduces ~5percent steady-state error in theta. Tune epsilon to balance chattering vs accuracy.
- Matrix inversion failure: For |theta|>π/2, M(q) becomes poorly conditioned. Always check condition number: cond(M) < 1000.

Figure 3.2: \textbf{Classical SMC} block diagram


Signal Flow:
- Measure state x = [x, theta₁, theta₂, ẋ, thetȧ₁, thetȧ₂]ᵀ
- Compute sliding surface $\sigma$ = lambda₁theta₁ + lambda₂theta₂ + k₁thetȧ₁ + k₂thetȧ₂
- Compute equivalent control u eq (model-based feedforward)
- Compute switching term: -K·sat(sigma/epsilon)
- Compute derivative damping: -k d·sigma
- Sum all terms: u = u eq - K·sat(sigma/epsilon) - k d·sigma
- Apply saturation: u sat = clip(u, -20N, +20N)

---

\subsection{Super-Twisting Algorithm (STA-SMC)}

Control Law:

STA employs a continuous 2nd-order sliding mode algorithm:


where:
- $K 1, K 2 > 0$ - STA algorithm gains (satisfy Lyapunov conditions)
- $z$ - integral state (provides continuous control action)
- $\text{sign}(\sigma)$ - smoothed via saturation function: $\text{sign}(\sigma) \approx \tanh(\sigma/\epsilon)$

Key Features:

- Continuous Control: Unlike classical SMC, $u {\text{STA}}$ is continuous (no discontinuity at $\sigma=0$)
- Finite-Time Convergence: Guaranteed convergence to $\sigma=0$ in finite time (not just asymptotic)
- Chattering Reduction: Continuous action inherently eliminates chattering

Gain Selection (Lyapunov-Based):

For stability, gains must satisfy:


where $\bar{d}$ is the upper bound on disturbances.

Convergence Time Estimate:

Upper bound on reaching time:


Design Parameters:


[TABLE - See Markdown version for details]


Advantages:
- Best overall performance (1.82s settling, 2.3percent overshoot)
- Lowest chattering (index 2.1, 74percent reduction vs Classical)
- Most energy-efficient (11.8 J)
- Finite-time convergence guarantee

Disadvantages:
- +31percent compute overhead vs Classical (24.2 mus)
- More complex gain tuning (Lyapunov conditions)
- Less intuitive than classical SMC

Figure 3.3: Super-Twisting Algorithm (STA) block diagram


Signal Flow:
- Measure state x = [x, theta₁, theta₂, ẋ, thetȧ₁, thetȧ₂]ᵀ
- Compute sliding surface $\sigma$ = lambda₁theta₁ + lambda₂theta₂ + k₁thetȧ₁ + k₂thetȧ₂
- Compute equivalent control u eq (model-based feedforward)
- Compute proportional term: -K₁|sigma|^(1/2)·sign(sigma)
- Compute integral state: ż = -K₂·sign(sigma)
- Sum STA terms: u STA = -K₁|sigma|^(1/2)·sign(sigma) + z
- Total control: u = u eq + u STA
- Apply saturation: u sat = clip(u, -20N, +20N)

Implementation Notes:

Discretization (dt = 0.01s):

- Fractional Power Term: |sigma|^(1/2) can cause numerical issues for small sigma. Use safety threshold:

- Integral State Update: Use backward Euler for stability:

- Sign Function Smoothing: Replace discontinuous sign with smooth saturation:

Numerical Stability:

- Integral Windup: Clip z to prevent unbounded growth: z ∈ [-100, +100]
- Division by Zero: Check |sigma| > epsilon min before computing fractional power
- Overflow Protection: Clip u STA before adding to u eq: u STA ∈ [-50N, +50N]

Common Pitfalls:

- Instability from violating Lyapunov conditions: Ensure K₁² $\geq$ 2K₂d̄ where d̄ is disturbance bound (~1.0 for DIP)
- Integral windup: Without anti-windup (z clamping), integral state can grow unbounded during saturation
- Chattering from small epsilon: If epsilon<0.005, sign function becomes too sharp -> high-frequency switching
- Slow convergence from small K₁: If K₁<8.0, reaching time increases beyond acceptable limits (>5s)

---

\subsection{Adaptive Sliding Mode Control}

Control Law:


where:
- $K(t)$ - time-varying adaptive gain
- $\gamma > 0$ - adaptation rate (increase when $|\sigma|$ large)
- $\beta > 0$ - leak rate (decay toward $K {\text{init}}$ when $|\sigma|$ small)
- $\delta > 0$ - dead-zone threshold
- $K {\text{init}}$ - nominal gain value

Adaptation Mechanism:

- Outside Dead-Zone ($|\sigma| > \delta$): Gain increases proportionally to sliding surface magnitude, providing more control authority when far from surface
- Inside Dead-Zone ($|\sigma| \leq \delta$): Gain decays toward nominal value, preventing unbounded growth

Bounded Gain Constraint:


Prevents gain saturation or underflow.

Design Parameters:


[TABLE - See Markdown version for details]


Advantages:
- Adapts to model uncertainty online
- Predicted best robustness to parameter errors (15percent tolerance, Section 8.1)
- Bounded gains prevent instability

Disadvantages:
- Slowest settling (2.35s, Section 7.2)
- Highest chattering (index 9.7, Section 7.3)
- Highest energy (13.6 J, +15percent vs STA)
- Most complex computation (31.6 mus)

---

\subsection{Hybrid Adaptive STA-SMC}

Control Law:

Hybrid controller switches between STA mode and Adaptive mode based on sliding surface magnitude:


where:
- $u {\text{STA}}$ - STA control law (Section 3.3)
- $u {\text{Adaptive}}$ - Adaptive control law (Section 3.4)
- $\sigma {\text{switch}}$ - mode switching threshold

Switching Logic:

- Reaching Phase ($|\sigma|$ large): Use STA for fast, chattering-free convergence
- Sliding Phase ($|\sigma|$ small): Use Adaptive for robustness to model uncertainty
- Hysteresis: Implement hysteresis band to prevent chattering between modes

Mode Transition:


where $\Delta$ is hysteresis margin.

Design Parameters:


[TABLE - See Markdown version for details]


Advantages:
- Balanced performance (1.95s settling, 3.5percent overshoot)
- Best predicted robustness (16percent model uncertainty tolerance)
- Good disturbance rejection (89percent attenuation)
- Combines STA speed with Adaptive robustness

Disadvantages:
- Complex switching logic requires validation
- Moderate compute overhead (26.8 mus)
- Requires tuning both STA and Adaptive gains

Figure 3.4: Hybrid Adaptive \textbf{STA-SMC} with mode switching


Signal Flow:
- Measure state x = [x, theta₁, theta₂, ẋ, thetȧ₁, thetȧ₂]ᵀ
- Compute sliding surface $\sigma$ = lambda₁theta₁ + lambda₂theta₂ + k₁thetȧ₁ + k₂thetȧ₂
- Compute equivalent control u eq (model-based feedforward)
- Evaluate mode selector:
   - If |sigma| > $\sigma$ switch + Delta -> Mode = STA
   - If |sigma| < $\sigma$ switch - Delta -> Mode = Adaptive
   - Otherwise -> Keep previous mode (hysteresis)
- Compute control based on mode:
   - STA mode: u sw = -K₁|sigma|^(1/2)·sign(sigma) + z
   - Adaptive mode: u sw = -K(t)·sat(sigma/epsilon) - k d·sigma
- Total control: u = u eq + u sw
- Apply saturation: u sat = clip(u, -20N, +20N)

Implementation Notes:

Mode Switching Logic (Critical for Safety):

- Hysteresis Implementation:

- State Continuity: When switching modes, ensure control continuity:
   - Transfer integral state z from STA to Adaptive K(t)
   - Use smooth transition: u[k] = alpha·u STA + (1-alpha)·u Adaptive where $\alpha$ ∈ [0,1] based on hysteresis position

- Mode Initialization:
   - Start in STA mode (typical for large initial errors)
   - Initialize z=0, K(t)=K init
   - Track mode transitions for debugging

Numerical Stability:

- Bumpless Transfer: During mode switch, match initial conditions:
  - STA->Adaptive: Set K(t) = current equivalent switching gain
  - Adaptive->STA: Set z = accumulated adaptive correction
- Anti-Windup: Reset integral states (z or K) if control saturates for >100ms
- Mode Chattering Prevention: Enforce minimum dwell time (50ms) in each mode

Common Pitfalls:

- Mode chattering: If Delta too small (<0.005), controller oscillates between modes -> instability
- Discontinuous control: Without bumpless transfer, u jumps at mode switches -> excites high-frequency dynamics
- Incorrect state initialization: Forgetting to transfer integral states causes transient spikes (>20percent overshoot)
- Hysteresis too large: If Delta > $\sigma$ switch/2, mode never switches -> defeats hybrid design purpose

---

\subsection{Swing-Up SMC}

Two-Phase Control:

Swing-up SMC operates in two distinct modes:

Phase 1: Swing-Up (Energy-Based Control)

When total system energy $E < E {\text{threshold}}$:


where:
- $k {\text{swing}} > 0$ - swing-up gain
- Energy pumping: Adds energy when $\cos(\theta 1) \dot{\theta} 1 > 0$ (constructive phase)

Phase 2: Stabilization (SMC)

When $E \geq E {\text{threshold}}$ and $|\theta 1|, |\theta 2| < \theta {\text{switch}}$:


Uses any SMC variant (typically Classical or STA) for stabilization.

Energy Calculation:


Mode Transition Logic:


Design Parameters:


[TABLE - See Markdown version for details]


Advantages:
- Global controller (works from any initial condition)
- Can bring pendulum from downward to upward position
- Combines energy-based and model-based control

Disadvantages:
- Complex mode logic requires careful tuning
- Swing-up phase performance not guaranteed (heuristic energy pumping)
- Not applicable to small perturbation stabilization (this study's focus)

---

\subsection{Model Predictive Control (MPC)}

Optimization Problem:

At each time step, solve finite-horizon optimal control problem:


where:
- $N$ - prediction horizon (number of future time steps)
- $Q, R, Q f$ - state, input, terminal cost weight matrices
- $f(\cdot, \cdot)$ - discretized nonlinear dynamics (Section 2)
- $u {\max}$ - actuator limit

Linearization (For Computational Efficiency):

Approximate nonlinear dynamics around current trajectory:


where $A(k), B(k)$ are Jacobians computed via finite differences.

Implementation:

Uses `cvxpy` library to solve quadratic program (QP) at each time step.

Design Parameters:


[TABLE - See Markdown version for details]


Advantages:
- Explicit handling of constraints (actuator limits, state bounds)
- Optimal control over finite horizon
- Can incorporate future reference trajectories

Disadvantages:
- Computationally expensive (requires external optimizer)
- Not self-contained (depends on `cvxpy`)
- Real-time feasibility questionable for 10 kHz control
- Excluded from main comparative analysis (dependency issue)

---

\subsection{Summary and Comparison}

Table 3.1: Controller Characteristics Comparison


[TABLE - See Markdown version for details]


Convergence Guarantees:


[TABLE - See Markdown version for details]


Design Complexity:

- Simplest: \textbf{Classical SMC} (6 scalar gains)
- Moderate: STA SMC (2 gains + Lyapunov conditions), \textbf{Adaptive SMC} (5 gains + adaptation law)
- Complex: Hybrid STA (8 gains + switching logic)
- Most Complex: Swing-Up SMC (energy calculation + mode transitions), MPC (weight matrices + optimization)


Computational Complexity Analysis:

Table 3.2: Detailed Computational Breakdown


[TABLE - See Markdown version for details]


Common Operations (All Controllers):
- M, C, G Evaluation: 8.2 mus, ~120 FLOPs (inertia matrix, Coriolis, gravity)
- Matrix Inversion: 4.1 mus, ~60 FLOPs (3x3 LU decomposition for M^{-1})
- Overhead: 1.3-1.5 mus (function calls, memory access, state copying)

Controller-Specific Costs:

- \textbf{Classical SMC} (4.9 mus control law):
   - Sliding surface sigma: 0.9 mus (10 FLOPs: 4 multiplies + 3 adds)
   - Equivalent control u eq: 2.8 mus (40 FLOPs: matrix-vector products)
   - Switching term: 1.2 mus (5 FLOPs: saturation + multiply)
   - Bottleneck: u eq calculation (58percent of control law time)

- STA SMC (10.6 mus control law):
   - Sliding surface sigma: 0.9 mus (same as Classical)
   - Equivalent control u eq: 2.8 mus (same as Classical)
   - Fractional power |sigma|^{1/2}: 3.2 mus (sqrt operation ~50 cycles)
   - Integral state update ż: 2.1 mus (sign function + integration)
   - Sign smoothing (tanh): 1.6 mus (~40 cycles for tanh approximation)
   - Bottleneck: Fractional power term (30percent of control law time)

- \textbf{Adaptive SMC} (17.8 mus control law):
   - Sliding surface sigma: 0.9 mus
   - Equivalent control u eq: 2.8 mus
   - Switching term: 1.2 mus (same as Classical)
   - Gain adaptation update: 8.4 mus (dead-zone check, conditional update, bounds checking)
   - State history management: 4.5 mus (circular buffer for derivative estimation)
   - Bottleneck: Gain adaptation (47percent of control law time)

- Hybrid STA (13.2 mus control law):
   - Sliding surface sigma: 0.9 mus
   - Equivalent control u eq: 2.8 mus
   - Mode selector logic: 2.1 mus (hysteresis check, mode transitions)
   - Dual control law computation: 6.2 mus (compute both STA and Adaptive in parallel)
   - Bumpless transfer: 1.2 mus (state continuity during mode switch)
   - Bottleneck: Dual control law (47percent of control law time)

- Swing-Up SMC (8.5 mus control law):
   - Energy calculation: 3.8 mus (kinetic + potential energy terms)
   - Mode selector: 0.8 mus (energy threshold check)
   - Swing-up term: 1.4 mus (k swing  cos(theta₁)  thetȧ₁)
   - SMC stabilizer: 2.5 mus (simplified \textbf{Classical SMC})
   - Bottleneck: Energy calculation (45percent of control law time)

Real-Time Feasibility (100 Hz Control Loop):


[TABLE - See Markdown version for details]


Notes:
- All SMC variants have >99.6percent timing margin -> safe for 100 Hz deployment
- MPC requires optimization solver (10-50 iterations) -> not real-time feasible without warm-start
- Worst-case timing (\textbf{Adaptive SMC}): 31.6 mus << 10 ms deadline (0.32percent utilization)

Scalability to Faster Control Loops:


[TABLE - See Markdown version for details]


Observations:
- SMC variants scale to 5 kHz (200 mus budget) with >84percent margin (Classical) or >84percent margin (Adaptive)
- \textbf{Classical SMC} fastest -> best for high-frequency applications (robotics: 1-10 kHz)
- MPC limited to <100 Hz without hardware acceleration (GPU, FPGA)


\subsection{Parameter Tuning Guidelines}

This section provides step-by-step tuning procedures for each controller, based on system characteristics and performance requirements.

General Tuning Principles:

- Start Conservative: Begin with small gains, increase gradually until performance meets requirements
- One Parameter at a Time: Change single parameter, observe response, iterate
- Measure Performance: Track settling time, overshoot, chattering index after each change
- Document Baseline: Record initial parameters and performance for comparison

System Characterization (Required Before Tuning):

Before tuning any controller, characterize the DIP system:
- Mass ratios: m₁/m₀, m₂/m₀ (affects inertia coupling)
- Length ratios: L₁/L cart, L₂/L₁ (affects angular dynamics)
- Natural frequencies: omega₁ ~= √(g/L₁), omega₂ ~= √(g/L₂) (sets response timescales)
- Disturbance levels: Measure typical external force magnitudes d̄ (wind, friction)
- Actuator limits: u max (typically $\pm$20N for DIP)

---

3.9.1 \textbf{Classical SMC} Tuning Procedure

Step 1: Design Sliding Surface (lambda₁, lambda₂, k₁, k₂)

- Choose convergence rates based on natural frequencies:
   Rule: 2x natural frequency provides good damping without excessive speed

- Choose sliding gains for critically damped surface:
   Rule: k i = lambda i/2 gives critically damped sliding variable dynamics

Step 2: Tune Switching Gain K

- Estimate disturbance bound: d̄ = max observed |disturbance| (typically 0.5-1.5 for DIP)
- Set initial K = 1.5·d̄ (50percent margin)
- Simulate and observe:
   - If oscillations persist -> increase K by 20percent
   - If chattering excessive -> decrease K by 10percent, increase epsilon
- Final K typically 1.2-2.0x disturbance bound

Step 3: Tune Boundary Layer epsilon

- Start with epsilon = 0.05 (large boundary layer, low chattering)
- Gradually decrease epsilon while monitoring chattering index:
- If chattering index > 15 -> stop, increase epsilon
- Final epsilon typically 0.02-0.05 for DIP (balance accuracy vs chattering)

Step 4: Tune Derivative Gain k d

- Start with k d = 0 (no damping)
- Increase k d in steps of 0.5 until overshoot < 5percent
- Typical range: k d ∈ [1.0, 3.0]
- Warning: k d > 5.0 amplifies sensor noise -> instability

Expected Performance (after tuning):
- Settling time: 2.0-2.5s
- Overshoot: 5-8percent
- Chattering index: 7-10
- Computation: 18.5 mus

---

3.9.2 \textbf{STA-SMC} Tuning Procedure

Step 1: Estimate Disturbance Bound d̄

Same as \textbf{Classical SMC} (typically 0.5-1.5 for DIP)

Step 2: Apply Lyapunov Conditions

- Choose K₂ to dominate disturbances:
   For d̄=1.0, epsilon=0.01 -> K₂ > 200
   Practical choice: K₂ = 250 (25percent margin)

- Choose K₁ to satisfy stability:
   For K₂=250, d̄=1.0 -> K₁ > √(500) ~= 22.4
   Practical choice: K₁ = 30 (34percent margin)

Step 3: Tune for Performance

- Start with Lyapunov-based values (K₁=30, K₂=250)
- If convergence too slow -> increase K₁ by 20percent
- If chattering observed -> decrease K₁ by 10percent, increase epsilon
- Final gains typically: K₁ ∈ [12, 20], K₂ ∈ [8, 15] (after PSO optimization)

Step 4: Adjust Sign Function Smoothing epsilon

- Start with epsilon = 0.01 (tight smoothing)
- If chattering index > 5 -> increase epsilon to 0.02
- STA should achieve chattering index < 3 with epsilon=0.01

Expected Performance (after tuning):
- Settling time: 1.8-2.0s
- Overshoot: 2-4percent
- Chattering index: 1-3
- Computation: 24.2 mus

---

3.9.3 \textbf{Adaptive SMC} Tuning Procedure

Step 1: Set Initial Gain K init

Choose K init = 1.2·d̄ (similar to \textbf{Classical SMC} switching gain)

Step 2: Tune Adaptation Rate gamma

- Start with gamma = 5.0 (moderate adaptation)
- Simulate with large disturbance (e.g., 50percent parameter error)
- If tracking error persists -> increase gamma by 50percent
- If gain K(t) oscillates -> decrease gamma by 25percent
- Final gamma typically 3.0-7.0

Step 3: Tune Leak Rate beta

- Start with $\beta$ = 0.1 (slow decay)
- If K(t) grows unbounded -> increase $\beta$ to 0.2
- If K(t) doesn't adapt fast enough -> decrease $\beta$ to 0.05
- Final $\beta$ typically 0.05-0.15

Step 4: Set Dead-Zone delta

- Choose delta = 2epsilon (twice boundary layer width)
- Ensures adaptation stops when on sliding surface
- Typical delta = 0.01-0.02

Step 5: Set Gain Bounds

- Lower bound: K min = 0.5·K init (prevent gain collapse)
- Upper bound: K max = 5·K init (prevent excessive control effort)
- Typical: K min=5.0, K max=50.0

Expected Performance (after tuning):
- Settling time: 2.3-2.5s
- Overshoot: 4-6percent
- Chattering index: 9-11
- Robustness: 15percent model uncertainty tolerance

---

3.9.4 Hybrid Adaptive \textbf{STA-SMC} Tuning Procedure

Step 1: Tune STA and Adaptive Controllers Independently

Follow Sections 3.9.2 and 3.9.3 to obtain nominal gains for both modes.

Step 2: Set Switching Threshold $\sigma$ switch

- Analyze typical sliding variable range during transient response
- Choose $\sigma$ switch at 50-70percent of peak |sigma| during reaching phase
- Typical: $\sigma$ switch = 0.05 (5percent of initial error)

Step 3: Set Hysteresis Margin Delta

- Start with Delta = $\sigma$ switch/5 (20percent hysteresis band)
- If mode chattering observed -> increase Delta by 50percent
- If mode switches too infrequently -> decrease Delta by 25percent
- Final Delta typically 0.01-0.02 (10-20percent of $\sigma$ switch)

Step 4: Verify Bumpless Transfer

- Simulate mode transitions and check control discontinuity:
- If Deltau > 0.2·u max -> adjust state initialization logic
- Target: Deltau < 0.1·u max (bumpless transfer)

Step 5: Test Robustness Across Modes

- Simulate with:
   - Large initial errors (test STA mode)
   - Model uncertainty (test Adaptive mode)
   - Mode transitions (test hysteresis)
- Verify no chattering at mode boundaries

Expected Performance (after tuning):
- Settling time: 1.9-2.1s
- Overshoot: 3-5percent
- Chattering index: 4-6
- Robustness: 16percent model uncertainty tolerance

---

3.9.5 Common Tuning Pitfalls


[TABLE - See Markdown version for details]


---

3.9.6 PSO-Based Automated Tuning (Recommended)

Manual tuning can be labor-intensive. PSO optimization (Section 5) automates the process:

Advantages:
- Explores parameter space systematically (swarm-based search)
- Optimizes multi-objective cost (settling time + overshoot + chattering)
- Finds near-optimal gains in 50-100 iterations (~10 minutes)

Procedure:
- Define parameter bounds (e.g., K ∈ [5, 30], epsilon ∈ [0.01, 0.1])
- Choose cost function: J = w₁·t settle + w₂·overshoot + w₃·chattering
- Run PSO with 20 particles, 50 iterations
- Verify performance on validation scenarios (different initial conditions)

Typical Results:
- \textbf{Classical SMC}: K=15.0, epsilon=0.02, k d=2.0 -> 18percent better than manual tuning
- STA SMC: K₁=12.0, K₂=8.0, epsilon=0.01 -> 22percent better performance
- Hybrid STA: $\sigma$ switch=0.05, Delta=0.01 -> optimal mode switching

See Section 5 for complete PSO methodology.


---


\end{document}
