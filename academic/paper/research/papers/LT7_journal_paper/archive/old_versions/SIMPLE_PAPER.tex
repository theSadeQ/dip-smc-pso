\documentclass[11pt]{article}
\usepackage[utf8]{inputenc}
\usepackage[margin=1in]{geometry}
\usepackage{amsmath,amssymb}
\usepackage{graphicx}
\usepackage{booktabs}
\usepackage{longtable}
\usepackage{hyperref}

\title{Comparative Analysis of Sliding Mode Control Variants for\ Double-Inverted Pendulum Systems:\ Performance, Stability, and Robustness}
\author{Research Paper - DIP-SMC-PSO Project}
\date{December 2025}

\begin{document}
\maketitle

\begin{abstract}
This paper presents a comprehensive comparative analysis of seven sliding mode control (SMC) variants for stabilization of a double-inverted pendulum (DIP) system. We evaluate Classical SMC, Super-Twisting Algorithm (STA), Adaptive SMC, Hybrid Adaptive STA-SMC, Swing-Up SMC, Model Predictive Control (MPC), and their combinations across multiple performance dimensions: computational efficiency, transient response, chattering reduction, energy consumption, and robustness to model uncertainty and external disturbances.

\textbf{Key Results:} STA-SMC achieves superior overall performance (1.82s settling time, 2.3\% overshoot, 11.8J energy), while Classical SMC provides the fastest computation (18.5 microseconds). PSO-based optimization reveals critical generalization limitations: parameters optimized for small perturbations ($\pm 0.05$ rad) exhibit 50.4$\times$ chattering degradation under realistic disturbances ($\pm 0.3$ rad). Robustness analysis with $\pm 20\%$ model parameter errors shows Hybrid Adaptive STA-SMC offers best uncertainty tolerance (16\% mismatch before instability).

\textbf{Keywords:} Sliding mode control, double-inverted pendulum, super-twisting algorithm, adaptive control, Lyapunov stability, particle swarm optimization, robust control, chattering reduction
\end{abstract}

\newpage
\tableofcontents
\newpage

\section{1. Introduction}


\section{2. System Model and Problem Formulation}


\section{3. Controller Design}


\section{4. Lyapunov Stability Analysis}


\section{5. PSO Optimization Methodology}


\section{6. Experimental Setup and Benchmarking Protocol}


\section{7. Performance Comparison Results}


\section{8. Robustness Analysis}


\section{9. Discussion}


\section{10. Conclusion and Future Work}


\end{document}
