%%%%%%%%%%%%%%%%%%%%%%%%%%%%%%%%%%%%%%%%%%%%%%%%%%%%%%%%%%%%%%%%%%%%%%%%%%%%%%%%
% PREFACE
%%%%%%%%%%%%%%%%%%%%%%%%%%%%%%%%%%%%%%%%%%%%%%%%%%%%%%%%%%%%%%%%%%%%%%%%%%%%%%%%

\chapter*{Preface}
\addcontentsline{toc}{chapter}{Preface}

This textbook emerges from several years of research and teaching in sliding mode control (SMC) for underactuated systems. It bridges the gap between theoretical foundations and practical implementation, providing a complete learning path from mathematical preliminaries to production-ready Python code.

\section*{Goals and Audience}

This book is designed for:

\begin{itemize}
    \item \textbf{Graduate students} in control engineering, robotics, or mechatronics seeking rigorous treatment of SMC with hands-on implementation
    \item \textbf{Engineers} transitioning from classical control to robust nonlinear techniques
    \item \textbf{Researchers} requiring validated Python implementations for benchmarking and extension
    \item \textbf{Self-learners} with basic control theory background (Laplace transforms, state-space models) who want to master SMC
\end{itemize}

\section*{Prerequisites}

Readers should have:
\begin{itemize}
    \item Linear algebra (matrices, eigenvalues)
    \item Differential equations (ODEs, linearization)
    \item Basic control theory (stability, transfer functions)
    \item Python programming (NumPy, basic OOP)
\end{itemize}

For complete beginners, we recommend the preparatory roadmap in \texttt{.ai\_workspace/edu/beginner-roadmap.md}, which provides 125-150 hours of foundational study before starting this textbook.

\section*{How to Use This Book}

\textbf{Learning Paths}:

\begin{enumerate}
    \item \textbf{Theory-First Path} (graduate students): Read Chapters 1-7 sequentially, derive all Lyapunov proofs, complete exercises, then implement in Python (Chapters 8-11).

    \item \textbf{Implementation-First Path} (engineers): Skim Chapters 1-2, jump to Chapter 11 (software), run examples, then return to theory as needed.

    \item \textbf{Research Path} (PhD candidates): Focus on Chapters 8-10 (benchmarking, PSO, advanced topics), extend to novel controllers, compare with provided baselines.
\end{enumerate}

\textbf{Chapter Organization}:
\begin{itemize}
    \item \textbf{Chapters 1-2}: Foundations (DIP dynamics, Lagrangian mechanics, Lyapunov stability)
    \item \textbf{Chapters 3-6}: Core SMC algorithms (Classical, STA, Adaptive, Hybrid)
    \item \textbf{Chapter 7}: PSO optimization theory and application
    \item \textbf{Chapter 8}: Comprehensive benchmarking and experimental validation
    \item \textbf{Chapters 9-10}: Advanced topics (robustness, benchmarking, PSO results)
    \item \textbf{Chapter 11}: Production-quality Python implementation
    \item \textbf{Chapter 12}: Real-world case studies
\end{itemize}

\section*{Software Repository}

All Python code, configuration files, and datasets are available at:

\begin{center}
\url{https://github.com/theSadeQ/dip-smc-pso}
\end{center}

The repository includes:
\begin{itemize}
    \item Production controllers (\texttt{src/controllers/})
    \item PSO optimizer (\texttt{src/optimizer/})
    \item Dynamics models (\texttt{src/plant/})
    \item Comprehensive test suite (\texttt{tests/}, 100+ tests)
    \item Benchmarking data (\texttt{benchmarks/})
    \item Streamlit web interface
\end{itemize}

\section*{Acknowledgments}

This work was supported by [Institution/Grant]. We thank [Advisors, Collaborators] for invaluable feedback. Special thanks to the open-source community for NumPy, SciPy, and PySwarms libraries.

\vspace{1cm}

\noindent [Author Name] \\
[Institution] \\
January 2026
