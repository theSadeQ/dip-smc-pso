%%%%%%%%%%%%%%%%%%%%%%%%%%%%%%%%%%%%%%%%%%%%%%%%%%%%%%%%%%%%%%%%%%%%%%%%%%%%%%%%
% CHAPTER 6: HYBRID ADAPTIVE STA-SMC
%%%%%%%%%%%%%%%%%%%%%%%%%%%%%%%%%%%%%%%%%%%%%%%%%%%%%%%%%%%%%%%%%%%%%%%%%%%%%%%%

\chapter{Hybrid Adaptive STA-SMC}
\label{ch:hybrid_smc}

\begin{chapterabstract}
This chapter presents a hybrid controller combining super-twisting algorithm (\cref{ch:super_twisting}) with adaptive gain scheduling (\cref{ch:adaptive_smc}). The hybrid approach achieves finite-time convergence, chattering reduction, and robustness to model uncertainty simultaneously. We derive dual-gain adaptation laws, lambda scheduling for state-dependent sliding surfaces, and mode-switching logic. Experimental validation demonstrates best overall performance: 94\% success rate, lowest energy consumption (0.9 J), and minimal chattering (1.0 N/s).
\end{chapterabstract}

%===============================================================================
\section{Motivation for Hybrid Control}
%===============================================================================

Individual controllers offer distinct advantages:
\begin{itemize}
    \item \textbf{STA-SMC}: Finite-time convergence, 56\% chattering reduction
    \item \textbf{Adaptive SMC}: 92\% robustness to model uncertainty
\end{itemize}

\textbf{Hybrid Goal}: Combine both benefits to achieve:
\begin{enumerate}
    \item Finite-time convergence from STA
    \item Model uncertainty robustness from adaptation
    \item Optimal energy efficiency
\end{enumerate}

%===============================================================================
\section{Hybrid Control Architecture}
%===============================================================================

\subsection{Control Law Structure}

\begin{equation}
u(t) = u_{\text{eq}}(t) - K_1(t) \sqrt{|s|} \sat(s/\epsilon) + z(t) - k_d s
\label{eq:hybrid_control_law}
\end{equation}

where:
\begin{itemize}
    \item $K_1(t)$ is the adaptive continuous gain
    \item $z(t)$ evolves according to $\dot{z} = -K_2(t) \sat(s/\epsilon)$ (adaptive integral gain)
\end{itemize}

\subsection{Dual-Gain Adaptation Laws}

\begin{align}
\dot{K}_1(t) &= \gamma_1 \sqrt{|s|} (|s| - \delta)_+ \sign(s) - \alpha_1 K_1 \label{eq:K1_adaptation} \\
\dot{K}_2(t) &= \gamma_2 (|s| - \delta)_+ \sign(s) - \alpha_2 K_2 \label{eq:K2_adaptation}
\end{align}
\coderef{src/controllers/smc/hybrid_adaptive_sta_smc.py}{201}

\textbf{Coupling}: $K_1$ and $K_2$ must satisfy STA stability conditions (\cref{eq:sta_stability_conditions}) at all times to maintain finite-time convergence.

See \pyfile{src/controllers/smc/hybrid\_adaptive\_sta\_smc.py} for complete implementation combining STA dynamics with dual-gain adaptation. Additional scheduling utilities: \pyfile{src/controllers/adaptive\_gain\_scheduler.py} and \pyfile{src/controllers/sliding\_surface\_scheduler.py}.

%===============================================================================
\section{Lambda Scheduling for Adaptive Sliding Surface}
%===============================================================================

\subsection{State-Dependent Surface}

Instead of fixed $\lambda_1, \lambda_2$, we use state-dependent scheduling:

\begin{equation}
\lambda_i(t) = \lambda_i^0 \cdot f(\|\vect{\theta}\|)
\end{equation}

where $f(\cdot)$ is a scheduling function. One effective choice is:

\begin{equation}
f(\|\vect{\theta}\|) = 1 + \beta \exp\left( -\frac{\|\vect{\theta}\|^2}{2\sigma^2} \right)
\end{equation}

\textbf{Effect}: Larger $\lambda$ near equilibrium (faster local convergence), smaller $\lambda$ far from equilibrium (reduced overshoot).

%===============================================================================
\section{Experimental Validation}
%===============================================================================

\subsection{Performance Comparison}

\begin{table}[ht]
\centering
\caption{Hybrid Adaptive STA-SMC vs. Individual Controllers}
\label{tab:hybrid_performance}
\begin{tabular}{lcccc}
\toprule
\textbf{Metric} & \textbf{Classical} & \textbf{STA} & \textbf{Adaptive} & \textbf{Hybrid} \\
\midrule
Settling time (s) & 1.82 & 1.65 & 2.10 & \textbf{1.58} \\
Energy (J) & 1.2 & 1.0 & 1.4 & \textbf{0.9} \\
Chattering (N/s) & 2.5 & 1.1 & 2.8 & \textbf{1.0} \\
Success rate (\%) & 85 & 88 & 92 & \textbf{94} \\
Computation ($\mu$s) & 12 & 15 & 18 & 22 \\
\bottomrule
\end{tabular}
\end{table}

\textbf{Result}: Hybrid achieves best performance across all metrics except computation time (still $<50$ $\mu$s for real-time control). For comprehensive benchmark comparisons including statistical significance testing, see \cref{ch:benchmarking}.

\begin{figure}[ht]
\centering
\includegraphics[width=0.8\textwidth]{figures/ch06_hybrid_adaptive_sta/hybrid_adaptive_sta_smc_convergence.png}
\caption{Hybrid adaptive STA-SMC transient response. The controller achieves fastest settling time ($t_s = 1.58$ s), lowest energy ($0.9$ J), and minimal chattering ($1.0$ N/s). The dual-gain adaptation (inset) shows $K_1$ evolving from 5.0 to 9.5 N and $K_2$ from 5.0 to 7.2 N to compensate for $\pm 20\%$ mass uncertainty.}
\label{fig:hybrid_convergence}
\end{figure}

\subsection{Energy Efficiency Analysis}

Energy consumption is a critical metric for embedded control systems where battery life or thermal constraints are limiting factors. The hybrid controller demonstrates superior energy efficiency across all operating conditions.

\begin{figure}[ht]
\centering
\includegraphics[width=0.85\textwidth]{figures/ch06_hybrid_adaptive_sta/energy_hybrid.png}
\caption[Hybrid Controller Energy Efficiency]{Cumulative energy consumption comparison across four controller types over 100 Monte Carlo trials. Hybrid adaptive STA-SMC (green) achieves lowest total energy ($0.9 \pm 0.15$ J), representing 25\% savings compared to classical SMC ($1.2 \pm 0.2$ J). The energy reduction results from: (1) continuous control signal eliminating chattering-induced energy waste, (2) adaptive gains preventing over-control during nominal conditions, (3) finite-time convergence minimizing settling transients. Shaded regions show 95\% confidence intervals. The hybrid controller maintains energy efficiency even under $\pm 20\%$ parameter uncertainty, demonstrating robust performance.}
\label{fig:hybrid_energy}
\end{figure}

\subsection{Three-Phase Performance Comparison}

Controller performance varies across three distinct phases of the stabilization task: initial swing-up, transient stabilization, and steady-state regulation. The following analysis decomposes performance metrics by phase.

\begin{figure}[ht]
\centering
\includegraphics[width=0.9\textwidth]{figures/ch06_hybrid_adaptive_sta/phase3_3_phase_comparison.png}
\caption[Three-Phase Performance Comparison]{Phase-decomposed performance analysis showing settling time (left), chattering amplitude (center), and energy consumption (right) across three phases: Phase 1 (swing-up, $t \in [0, 0.5]$ s), Phase 2 (transient, $t \in [0.5, 2.0]$ s), Phase 3 (steady-state, $t > 2.0$ s). The hybrid controller (green) excels in all phases: Phase 1 achieves fastest swing-up ($0.45$ s vs. $0.58$ s for classical SMC), Phase 2 exhibits minimal overshoot ($2.1\%$ vs. $4.2\%$ for classical SMC), Phase 3 maintains lowest chattering ($0.8$ N/s vs. $2.5$ N/s for classical SMC). Error bars show standard deviation over 100 trials. This comprehensive analysis demonstrates that the hybrid approach does not sacrifice performance in any operating regime.}
\label{fig:hybrid_three_phase}
\end{figure}

%===============================================================================
\section{Summary}
%===============================================================================

\begin{keybox}
\textbf{Hybrid Controller Benefits:}
\begin{itemize}
    \item \textbf{Best settling time}: 1.58 s (9\% faster than classical SMC)
    \item \textbf{Lowest energy}: 0.9 J (25\% reduction vs. classical SMC)
    \item \textbf{Minimal chattering}: 1.0 N/s (60\% reduction vs. classical SMC)
    \item \textbf{Highest robustness}: 94\% success rate under 20\% uncertainty
\end{itemize}
\end{keybox}

\textbf{Trade-off}: 83\% increased computation time ($22$ $\mu$s vs. $12$ $\mu$s) still enables 10 kHz real-time control.

\textbf{Next Steps}: \cref{ch:pso} demonstrates PSO-based multi-objective optimization of hybrid controller gains. Real-world validation and hardware-in-the-loop testing are presented in \cref{ch:case_studies}.

%===============================================================================
% END OF CHAPTER 6
%===============================================================================
