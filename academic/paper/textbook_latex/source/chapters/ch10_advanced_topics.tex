%%%%%%%%%%%%%%%%%%%%%%%%%%%%%%%%%%%%%%%%%%%%%%%%%%%%%%%%%%%%%%%%%%%%%%%%%%%%%%%%
% CHAPTER 10: ADVANCED TOPICS - ROBUSTNESS & MODEL UNCERTAINTY
%%%%%%%%%%%%%%%%%%%%%%%%%%%%%%%%%%%%%%%%%%%%%%%%%%%%%%%%%%%%%%%%%%%%%%%%%%%%%%%%

\chapter{Advanced Topics: Robustness and Model Uncertainty}
\label{ch:advanced_topics}

\begin{chapterabstract}
This chapter evaluates controller robustness beyond nominal conditions through systematic disturbance rejection tests (MT-8) and model uncertainty analysis (LT-6). We demonstrate that PSO-optimized gains successfully reject step and impulse disturbances (96-99\% overshoot reduction), present adaptive gain scheduling results (11-40.6\% chattering reduction for oscillatory disturbances), and analyze performance degradation under $\pm 10\%$/$\pm 20\%$ parameter variations. The results establish design guidelines for industrial deployment in uncertain environments.
\end{chapterabstract}

%===============================================================================
\section{Introduction}
%===============================================================================

Chapters 8-9 established performance baselines and PSO optimization effectiveness under ideal conditions (perfect model, no external disturbances, small perturbations). Real-world deployment requires controllers that maintain performance under:

\begin{enumerate}
    \item \textbf{External Disturbances:} Wind gusts on outdoor robots, floor vibrations on indoor platforms, contact forces during manipulation.

    \item \textbf{Model Uncertainty:} Manufacturing tolerances cause $\pm 5-10\%$ variations in mass/length parameters. Payload changes alter system dynamics.

    \item \textbf{Sensor Noise:} Encoders provide noisy angle measurements, accelerometers drift over time.

    \item \textbf{Actuator Limitations:} Saturation at $u_{\max} = 150$ N, bandwidth limits prevent instantaneous torque changes.
\end{enumerate}

This chapter focuses on disturbance rejection (Section \ref{sec:disturbance_rejection}) and model uncertainty (Section \ref{sec:model_uncertainty}), leaving sensor noise and actuator dynamics for future work.

\subsection{Research Questions}

\begin{enumerate}
    \item \textbf{RQ1 (Disturbance Rejection):} How much external force can controllers tolerate before divergence?

    \item \textbf{RQ2 (Recovery Time):} How quickly do controllers return to equilibrium after disturbance removal?

    \item \textbf{RQ3 (Model Uncertainty):} What parameter error magnitude ($\pm 10\%$, $\pm 20\%$, $\pm 30\%$) causes performance degradation or failure?

    \item \textbf{RQ4 (Controller Ranking):} Which controller demonstrates superior robustness across all scenarios?

    \item \textbf{RQ5 (Adaptive Scheduling):} Can state-magnitude-based gain scheduling improve chattering under disturbances?
\end{enumerate}

%===============================================================================
\section{Disturbance Rejection Analysis (MT-8)}
\label{sec:disturbance_rejection}
%===============================================================================

\subsection{Disturbance Scenarios}

We evaluated three disturbance types with varying magnitudes:

\begin{enumerate}
    \item \textbf{Step Disturbance:} Constant force applied at $t = 2.0$ s until simulation end ($t = 10.0$ s).
        \begin{itemize}
            \item Magnitudes: 5 N, 10 N, 15 N, 20 N
            \item Physical interpretation: Constant horizontal push (e.g., wind)
        \end{itemize}

    \item \textbf{Impulse Disturbance:} Brief high-magnitude force pulse.
        \begin{itemize}
            \item Magnitude: 30 N (duration: 0.1 s at $t = 2.0$ s)
            \item Physical interpretation: Impact (e.g., collision)
        \end{itemize}

    \item \textbf{Sinusoidal Disturbance:} Periodic oscillating force.
        \begin{itemize}
            \item Amplitude: 5 N, Frequency: 1 Hz (starting at $t = 1.0$ s)
            \item Physical interpretation: Vibration (e.g., floor oscillations)
        \end{itemize}
\end{enumerate}

\subsection{Disturbance Rejection Results}

Table \ref{tab:disturbance_rejection} presents post-PSO disturbance rejection performance.

\begin{table}[htbp]
\centering
\caption{MT-8 Disturbance Rejection Performance (PSO-Optimized Gains)}
\label{tab:disturbance_rejection}
\begin{tabular}{lcccc}
\toprule
\textbf{Controller} & \textbf{Step 10N (deg)} & \textbf{Impulse 30N (deg)} & \textbf{Recovery (s)} & \textbf{Converged?} \\
\midrule
Classical SMC       & 8.2 & 12.5 & 2.8 & Yes \\
STA-SMC             & 6.8 & 10.1 & 2.3 & Yes \\
Adaptive SMC        & 9.1 & 13.7 & 3.1 & Yes \\
Hybrid Adaptive STA & 7.5 & 11.3 & 2.6 & Yes \\
\bottomrule
\end{tabular}
\end{table}

\textbf{Key Findings:}
\begin{itemize}
    \item \textbf{All controllers converge} after disturbance removal (robust PSO essential, see Chapter 9).
    \item \textbf{STA-SMC shows best disturbance rejection} (6.8$^\circ$ step, 10.1$^\circ$ impulse, 2.3 s recovery).
    \item \textbf{Hybrid Adaptive STA achieves balanced performance} (7.5$^\circ$ step, 11.3$^\circ$ impulse, 2.6 s recovery).
    \item \textbf{Adaptive SMC slightly worse} (9.1$^\circ$ step, 13.7$^\circ$ impulse) due to transient adaptation phase.
\end{itemize}

\subsection{Comparison to Baseline (Pre-PSO) Performance}

Table \ref{tab:disturbance_baseline_comparison} quantifies improvement from robust PSO.

\begin{table}[htbp]
\centering
\caption{Disturbance Rejection Improvement: Pre-PSO vs Post-PSO}
\label{tab:disturbance_baseline_comparison}
\begin{tabular}{lccc}
\toprule
\textbf{Controller} & \textbf{Baseline (deg)} & \textbf{Post-PSO (deg)} & \textbf{Improvement} \\
\midrule
Classical SMC       & 187.3 & 8.2 & 95.6\% \\
STA-SMC             & 269.3 & 6.8 & 97.5\% \\
Adaptive SMC        & 267.7 & 9.1 & 96.6\% \\
Hybrid Adaptive STA & 625.2 & 7.5 & 98.8\% \\
\bottomrule
\end{tabular}
\end{table}

\textbf{Critical Insight:} Robust PSO (50\% nominal + 50\% disturbed fitness) achieves 95.6-98.8\% overshoot reduction. This validates the necessity of disturbance-aware optimization (Chapter 9, Section 9.3).

\subsection{Disturbance Magnitude Sensitivity}

Figure \ref{fig:disturbance_magnitude} shows overshoot versus disturbance magnitude.

\begin{figure}[htbp]
\centering
\includegraphics[width=0.8\textwidth]{figures/ch09_robustness/disturbance_rejection.png}
\caption{Maximum overshoot versus step disturbance magnitude (5-20 N) for all four controllers. STA-SMC (orange) maintains lowest overshoot across all magnitudes (6.8-18.3$^\circ$). Hybrid adaptive STA (red) shows similar trend (7.5-19.1$^\circ$). Classical SMC (blue) and adaptive SMC (green) exhibit slightly higher overshoots but remain stable up to 20 N. All controllers diverge at 25 N (not shown), indicating shared robustness limit.}
\label{fig:disturbance_magnitude}
\end{figure}

\textbf{Robustness Limits:}
\begin{itemize}
    \item All controllers stable up to 20 N step disturbance.
    \item Controllers diverge at 25 N (angles exceed 90$^\circ$, loss of control).
    \item Linear degradation: $\sim$0.5-0.7$^\circ$ overshoot increase per 1 N disturbance.
\end{itemize}

%===============================================================================
\section{Adaptive Gain Scheduling for Disturbance Rejection (MT-8 Enhancement)}
%===============================================================================

\subsection{Motivation: Disturbance-Dependent Chattering}

Observation from MT-8 baseline experiments: chattering amplitude varies significantly across disturbance types:
\begin{itemize}
    \item Step disturbance: High chattering (large steady-state sliding variable)
    \item Impulse disturbance: Transient chattering spike (brief high-magnitude error)
    \item Sinusoidal disturbance: Periodic chattering (tracking oscillations)
\end{itemize}

\textbf{Hypothesis:} State-magnitude-based gain scheduling can reduce chattering by adapting gains to current system state.

\subsection{Adaptive Scheduler Design}

\textbf{Gain Scheduling Law:}
\begin{equation}
k_i(t) = k_{i,\text{base}} \cdot \left(1 + \alpha \cdot \frac{|\mathbf{x}(t)|}{\|\mathbf{x}_{\text{ref}}\|}\right)
\end{equation}

where:
\begin{itemize}
    \item $k_{i,\text{base}}$: PSO-optimized baseline gains (Table 9.2)
    \item $\alpha \in [0.1, 1.0]$: Scheduling aggressiveness (PSO-tuned)
    \item $|\mathbf{x}(t)|$: State magnitude $\sqrt{x^2 + \theta_1^2 + \theta_2^2 + \dot{x}^2 + \dot{\theta}_1^2 + \dot{\theta}_2^2}$
    \item $\|\mathbf{x}_{\text{ref}}\|$: Reference threshold (0.1 rad = 5.7$^\circ$)
\end{itemize}

\textbf{Scheduling Logic:}
\begin{itemize}
    \item When $|\mathbf{x}| \ll \|\mathbf{x}_{\text{ref}}\|$: Use baseline gains (near equilibrium).
    \item When $|\mathbf{x}| \gg \|\mathbf{x}_{\text{ref}}\|$: Increase gains by factor $(1 + \alpha)$ (large error).
\end{itemize}

\subsection{Adaptive Scheduling Results}

Table \ref{tab:adaptive_scheduling} presents classical SMC chattering reduction with adaptive scheduling.

\begin{table}[htbp]
\centering
\caption{MT-8 Adaptive Scheduling Results for Classical SMC}
\label{tab:adaptive_scheduling}
\begin{tabular}{lccc}
\toprule
\textbf{Disturbance Type} & \textbf{Baseline Chat. (N)} & \textbf{Scheduled Chat. (N)} & \textbf{Reduction} \\
\midrule
Step 10N        & 8.45 & 7.52 & 11.0\% \\
Impulse 30N     & 12.38 & 7.36 & 40.6\% \\
Sinusoidal 5N   & 6.82 & 5.98 & 12.3\% \\
\bottomrule
\end{tabular}
\end{table}

\textbf{Key Findings:}
\begin{itemize}
    \item \textbf{Impulse disturbances benefit most} (40.6\% chattering reduction) due to transient high-magnitude states.
    \item \textbf{Step and sinusoidal disturbances show modest improvement} (11.0-12.3\%) as steady-state magnitude remains bounded.
    \item \textbf{Overall reduction: 21.3\% average} across three disturbance types.
\end{itemize}

\subsection{Critical Limitation: Overshoot Penalty}

Table \ref{tab:adaptive_scheduling_tradeoff} reveals significant overshoot penalty for step disturbances.

\begin{table}[htbp]
\centering
\caption{Adaptive Scheduling Trade-off: Chattering vs Overshoot}
\label{tab:adaptive_scheduling_tradeoff}
\begin{tabular}{lcccc}
\toprule
\textbf{Disturbance} & \textbf{Chat. Reduction} & \textbf{Overshoot Change} & \textbf{Settling Change} \\
\midrule
Step 10N        & -11.0\% & \textcolor{red}{+354\%} & +18.2\% \\
Impulse 30N     & -40.6\% & +8.1\% & +2.3\% \\
Sinusoidal 5N   & -12.3\% & +6.7\% & +1.1\% \\
\bottomrule
\end{tabular}
\end{table}

\textbf{Critical Finding:} Adaptive scheduling causes \textbf{+354\% overshoot increase} for step disturbances (from 8.2$^\circ$ to 37.2$^\circ$), negating chattering benefits.

\textbf{Root Cause:} Gain scheduling increases gains during transient phase (large $|\mathbf{x}|$), causing aggressive control action that overshoots equilibrium before gains reduce.

\subsection{Hardware-in-the-Loop (HIL) Validation}

MT-8 conducted 120 HIL trials on physical DIP testbed:

\begin{table}[htbp]
\centering
\caption{MT-8 HIL Validation Results (Classical SMC, 120 Trials)}
\label{tab:hil_validation}
\begin{tabular}{lccc}
\toprule
\textbf{Metric} & \textbf{Simulation} & \textbf{HIL} & \textbf{Sim-Hardware Gap} \\
\midrule
Step 10N Overshoot     & 8.2$^\circ$ & 9.7$^\circ$ & +18.3\% \\
Impulse 30N Overshoot  & 12.5$^\circ$ & 14.1$^\circ$ & +12.8\% \\
Recovery Time          & 2.8 s & 3.2 s & +14.3\% \\
Chattering (baseline)  & 8.45 N & 11.23 N & +32.9\% \\
Chattering (scheduled) & 7.52 N & 10.01 N & +33.1\% \\
\bottomrule
\end{tabular}
\end{table}

\textbf{Sim-Hardware Gap Analysis:}
\begin{itemize}
    \item \textbf{12-18\% overshoot increase:} Attributable to actuator dynamics (0.05 s delay), sensor quantization (0.01$^\circ$ encoder resolution).
    \item \textbf{33\% chattering increase:} Real actuators exhibit higher-frequency oscillations than simulated ideal motor.
    \item \textbf{Qualitative trends preserved:} Adaptive scheduling reduces chattering in hardware (10.01 N vs 11.23 N), validating simulation predictions.
\end{itemize}

\subsection{Deployment Recommendation}

\textbf{Use adaptive gain scheduling ONLY for:**
\begin{itemize}
    \item \textbf{Sinusoidal/oscillatory disturbances} (12.3\% chattering reduction, 6.7\% overshoot penalty).
    \item \textbf{Impulse disturbances} (40.6\% chattering reduction, 8.1\% overshoot penalty) if settling time not critical.
\end{itemize}

\textbf{Do NOT use for:**
\begin{itemize}
    \item \textbf{Step disturbances} (354\% overshoot penalty far exceeds 11\% chattering benefit).
    \item \textbf{Overshoot-critical applications} (e.g., precision robotics, aerospace).
\end{itemize}

\textbf{Future Work:} Disturbance-aware scheduling (detect disturbance type, enable scheduling only for impulse/sinusoidal). See \texttt{MT8\_ADAPTIVE\_SCHEDULING\_SUMMARY.md} for details.

%===============================================================================
\section{Model Uncertainty Analysis (LT-6)}
\label{sec:model_uncertainty}
%===============================================================================

\subsection{Uncertainty Scenarios}

We perturbed three parameter categories with $\pm 10\%$, $\pm 20\%$ variations:

\begin{enumerate}
    \item \textbf{Mass Parameters:}
        \begin{itemize}
            \item $m_0$: Cart mass (nominal: 1.5 kg)
            \item $m_1$: Pendulum 1 mass (nominal: 0.2 kg)
            \item $m_2$: Pendulum 2 mass (nominal: 0.1 kg)
        \end{itemize}

    \item \textbf{Length Parameters:}
        \begin{itemize}
            \item $l_1$: Pendulum 1 length (nominal: 0.3 m)
            \item $l_2$: Pendulum 2 length (nominal: 0.2 m)
        \end{itemize}

    \item \textbf{Inertia Parameters:}
        \begin{itemize}
            \item $I_1$: Pendulum 1 inertia (nominal: 0.009 kg$\cdot$m$^2$)
            \item $I_2$: Pendulum 2 inertia (nominal: 0.002 kg$\cdot$m$^2$)
        \end{itemize}
\end{enumerate}

\textbf{Total Scenarios:} 28 single-parameter variations + 4 combined worst-case = 32 per controller.

\textbf{Physical Interpretation:}
\begin{itemize}
    \item $\pm 10\%$ mass: Manufacturing tolerance (realistic)
    \item $\pm 20\%$ mass: Payload change (load attached to cart or pendulum tip)
    \item $\pm 20\%$ length: Different pendulum configuration (adjustable links)
\end{itemize}

\subsection{Robustness Ranking}

Table \ref{tab:model_uncertainty_ranking} presents robustness scores under parameter variations.

\begin{table}[htbp]
\centering
\caption{LT-6 Robustness Ranking (Lower score = better robustness)}
\label{tab:model_uncertainty_ranking}
\begin{tabular}{lcccc}
\toprule
\textbf{Controller} & \textbf{$\pm 10\%$ Success} & \textbf{$\pm 20\%$ Success} & \textbf{Robustness Score} & \textbf{Rank} \\
\midrule
Adaptive SMC        & 99\% & 96\% & 97.5 / 100 & 1 \\
Hybrid Adaptive STA & 98\% & 94\% & 96.0 / 100 & 2 \\
STA-SMC             & 97\% & 89\% & 93.0 / 100 & 3 \\
Classical SMC       & 95\% & 78\% & 86.5 / 100 & 4 \\
\bottomrule
\end{tabular}
\end{table}

\textbf{Key Findings:}
\begin{itemize}
    \item \textbf{Adaptive SMC most robust} (97.5/100 score) due to online parameter estimation compensating for model mismatch.
    \item \textbf{Hybrid Adaptive STA second} (96.0/100 score), combining STA robustness with adaptive gain adjustment.
    \item \textbf{Classical SMC least robust} (86.5/100 score), fixed gains cannot adapt to parameter changes.
    \item \textbf{All controllers tolerate $\pm 10\%$ errors} (95-99\% success), validating robustness for manufacturing tolerances.
\end{itemize}

\subsection{Performance Degradation Analysis}

Table \ref{tab:model_uncertainty_degradation} quantifies settling time and overshoot degradation.

\begin{table}[htbp]
\centering
\caption{Performance Degradation Under Model Uncertainty (Mean $\pm$ Std)}
\label{tab:model_uncertainty_degradation}
\begin{tabular}{lcccc}
\toprule
\textbf{Controller} & \textbf{Nominal $t_s$ (s)} & \textbf{$\pm 20\%$ $t_s$ (s)} & \textbf{Degradation} \\
\midrule
Classical SMC       & 2.15 $\pm$ 0.18 & 2.83 $\pm$ 0.42 & +31.6\% \\
STA-SMC             & 1.82 $\pm$ 0.15 & 2.18 $\pm$ 0.31 & +19.8\% \\
Adaptive SMC        & 2.35 $\pm$ 0.21 & 2.47 $\pm$ 0.23 & +5.1\% \\
Hybrid Adaptive STA & 1.95 $\pm$ 0.16 & 2.23 $\pm$ 0.27 & +14.4\% \\
\midrule
\textbf{Controller} & \textbf{Nominal $M_p$ (\%)} & \textbf{$\pm 20\%$ $M_p$ (\%)} & \textbf{Degradation} \\
\midrule
Classical SMC       & 5.8 $\pm$ 0.8 & 8.9 $\pm$ 1.7 & +53.4\% \\
STA-SMC             & 2.3 $\pm$ 0.4 & 3.8 $\pm$ 0.9 & +65.2\% \\
Adaptive SMC        & 8.2 $\pm$ 1.1 & 9.1 $\pm$ 1.3 & +11.0\% \\
Hybrid Adaptive STA & 3.5 $\pm$ 0.5 & 5.2 $\pm$ 1.1 & +48.6\% \\
\bottomrule
\end{tabular}
\end{table}

\textbf{Key Observations:}
\begin{itemize}
    \item \textbf{Adaptive SMC exhibits minimal degradation} (5.1\% settling time, 11.0\% overshoot), validating adaptive law effectiveness.
    \item \textbf{Classical SMC degrades significantly} (31.6\% settling time, 53.4\% overshoot) due to fixed gains.
    \item \textbf{STA-SMC shows moderate degradation} (19.8\% settling time) but overshoot penalty is high (65.2\%).
    \item \textbf{Hybrid balances robustness and nominal performance} (14.4\% settling time, 48.6\% overshoot).
\end{itemize}

\subsection{Worst-Case Parameter Combinations}

Table \ref{tab:worst_case_uncertainty} analyzes combined worst-case scenarios.

\begin{table}[htbp]
\centering
\caption{Worst-Case Model Uncertainty Scenarios (All Parameters $\pm 20\%$)}
\label{tab:worst_case_uncertainty}
\begin{tabular}{lcccc}
\toprule
\textbf{Scenario} & \textbf{Classical SMC} & \textbf{STA-SMC} & \textbf{Adaptive SMC} & \textbf{Hybrid} \\
\midrule
All masses +20\%    & Fail (div.) & Converge (4.2 s) & Converge (3.1 s) & Converge (3.7 s) \\
All masses -20\%    & Converge (3.8 s) & Converge (2.9 s) & Converge (2.8 s) & Converge (2.5 s) \\
All lengths +20\%   & Fail (div.) & Fail (div.) & Converge (4.5 s) & Converge (5.2 s) \\
All lengths -20\%   & Converge (2.9 s) & Converge (2.3 s) & Converge (2.5 s) & Converge (2.1 s) \\
\bottomrule
\multicolumn{5}{l}{\footnotesize div. = divergence (angles $> 90^\circ$)}
\end{tabular}
\end{table}

\textbf{Critical Insight:} Only adaptive controllers (Adaptive SMC, Hybrid Adaptive STA) handle combined +20\% errors. Fixed-gain controllers (Classical, STA) diverge when multiple parameters increase simultaneously.

\textbf{Physical Explanation:} Combined mass increases cause total system inertia to exceed design assumptions. Adaptive controllers compensate via gain adaptation, while fixed-gain controllers apply insufficient control effort.

\subsection{Parameter Sensitivity Heatmap}

Figure \ref{fig:parameter_sensitivity} visualizes settling time degradation across all parameter variations.

\begin{figure}[htbp]
\centering
\includegraphics[width=0.8\textwidth]{figures/ch09_robustness/model_uncertainty.png}
\caption{Parameter sensitivity heatmap for settling time degradation (Classical SMC, Adaptive SMC). Each cell shows settling time for $\pm 10\%$, $\pm 20\%$ variations in seven parameters (masses, lengths, inertias). Classical SMC (left) exhibits large red regions (divergence or >5 s settling) for +20\% mass/length errors. Adaptive SMC (right) shows predominantly blue/green (stable settling $< 3$ s), demonstrating superior robustness. Dark red cells indicate controller failure (angles exceeded 90$^\circ$ before settling).}
\label{fig:parameter_sensitivity}
\end{figure}

\textbf{Heatmap Interpretation:}
\begin{itemize}
    \item \textbf{Classical SMC:} Sensitive to mass/length increases (+20\% region), robust to decreases.
    \item \textbf{Adaptive SMC:} Uniform blue/green across all scenarios, minimal parameter sensitivity.
    \item \textbf{Inertia variations:} Less critical than mass/length (both controllers tolerate $\pm 20\%$).
\end{itemize}

%===============================================================================
\section{Sensor Noise Robustness (Future Work)}
%===============================================================================

\textbf{Scope:} LT-6 evaluated model parameter uncertainty. Sensor noise robustness remains for future investigation.

\subsection{Proposed Sensor Noise Scenarios}

\begin{enumerate}
    \item \textbf{Angle Measurement Noise:}
        \begin{itemize}
            \item Source: Encoder quantization, electromagnetic interference
            \item Model: $\theta_{\text{meas}} = \theta_{\text{true}} + \mathcal{N}(0, \sigma_\theta^2)$
            \item Magnitudes: $\sigma_\theta \in \{0.01^\circ, 0.1^\circ, 0.5^\circ\}$ (encoder resolution limits)
        \end{itemize}

    \item \textbf{Velocity Estimation Noise:}
        \begin{itemize}
            \item Source: Numerical differentiation amplifies noise
            \item Model: $\dot{\theta}_{\text{meas}} = \dot{\theta}_{\text{true}} + \mathcal{N}(0, \sigma_{\dot{\theta}}^2)$
            \item Mitigation: Low-pass filter (cutoff 50 Hz) or Kalman filter
        \end{itemize}

    \item \textbf{Communication Delays:}
        \begin{itemize}
            \item Source: CAN bus latency, wireless transmission delays
            \item Model: Measurement available at $t + \tau_{\text{delay}}$, where $\tau \in \{0, 5, 10, 20\}$ ms
        \end{itemize}
\end{enumerate}

\textbf{Expected Results:}
\begin{itemize}
    \item STA-SMC more sensitive to noise (continuous control law amplifies high-frequency content).
    \item Classical SMC boundary layer acts as low-pass filter, reducing noise sensitivity.
    \item Adaptive SMC may exhibit gain oscillations if noise corrupts adaptation law.
\end{itemize}

\textbf{Recommendation:} Combine adaptive SMC with Kalman filter for state estimation in noisy environments.

%===============================================================================
\section{Controller Selection Guide for Uncertain Environments}
%===============================================================================

\subsection{Decision Matrix}

Table \ref{tab:controller_selection_uncertainty} provides deployment recommendations based on uncertainty type.

\begin{table}[htbp]
\centering
\caption{Controller Selection for Uncertain Environments}
\label{tab:controller_selection_uncertainty}
\begin{tabular}{lp{10cm}}
\toprule
\textbf{Scenario} & \textbf{Recommended Controller (Rationale)} \\
\midrule
\textbf{Step Disturbances} & STA-SMC (6.8$^\circ$ overshoot, 2.3 s recovery) or Hybrid (7.5$^\circ$ overshoot, balanced) \\
\midrule
\textbf{Impulse Disturbances} & Classical SMC with adaptive scheduling (40.6\% chattering reduction) or STA-SMC (10.1$^\circ$ overshoot) \\
\midrule
\textbf{Sinusoidal Disturbances} & Classical SMC with adaptive scheduling (12.3\% chattering reduction, acceptable 6.7\% overshoot penalty) \\
\midrule
\textbf{Model Uncertainty $\pm 10\%$} & Any controller (all achieve 95-99\% success) \\
\midrule
\textbf{Model Uncertainty $\pm 20\%$} & Adaptive SMC (96\% success, 5.1\% degradation) or Hybrid (94\% success, 14.4\% degradation) \\
\midrule
\textbf{Combined Uncertainty +20\%} & Adaptive SMC only (Classical/STA diverge) \\
\midrule
\textbf{Unknown Disturbance Type} & Hybrid Adaptive STA (balanced across all scenarios, 96.0/100 robustness score) \\
\midrule
\textbf{Overshoot-Critical} & STA-SMC (2.3\% overshoot nominal, 3.8\% under uncertainty) or Hybrid (3.5\% nominal, 5.2\% under uncertainty) \\
\midrule
\textbf{Energy-Constrained} & Classical SMC (9{,}843 N$^2\cdot$s, 20$\times$ more efficient than others, Chapter 8) \\
\bottomrule
\end{tabular}
\end{table}

\subsection{Industrial Deployment Checklist}

\textbf{Pre-Deployment Validation:}
\begin{enumerate}
    \item \textbf{Disturbance Profiling:} Measure expected disturbance types and magnitudes (accelerometer on prototype).
    \item \textbf{Parameter Identification:} Measure actual system parameters (masses, lengths) with $\pm 5\%$ uncertainty budget.
    \item \textbf{Worst-Case Simulation:} Test controller with combined +20\% errors + maximum disturbance.
    \item \textbf{Hardware-in-the-Loop:} Validate on physical testbed (expect 12-33\% sim-hardware gap, Table \ref{tab:hil_validation}).
    \item \textbf{Safety Margins:} Design for 2$\times$ expected disturbance magnitude (if 10 N expected, test up to 20 N).
\end{enumerate}

\textbf{Runtime Monitoring:}
\begin{enumerate}
    \item \textbf{Angle Limits:} Trigger emergency stop if $|\theta_i| > 45^\circ$ (approaching uncontrollable region).
    \item \textbf{Control Saturation:} Log frequency of $|u| = u_{\max}$ events (indicates insufficient actuator capacity).
    \item \textbf{Settling Time Timeout:} If system fails to settle within 5 s, trigger safe shutdown.
    \item \textbf{Chattering Monitoring:} If chattering amplitude $> 15$ N (20\% of $u_{\max}$), reduce gains or enable adaptive scheduling.
\end{enumerate}

%===============================================================================
\section{Summary and Key Insights}
%===============================================================================

\subsection{Disturbance Rejection (MT-8)}

\begin{enumerate}
    \item \textbf{Robust PSO essential:} Default gains fail catastrophically (187-625$^\circ$ overshoots). Post-PSO controllers achieve 95.6-98.8\% improvement.

    \item \textbf{STA-SMC best overall:} 6.8$^\circ$ step overshoot, 10.1$^\circ$ impulse overshoot, 2.3 s recovery time.

    \item \textbf{Adaptive scheduling reduces chattering:} 11-40.6\% reduction depending on disturbance type, but \textbf{+354\% overshoot penalty for step disturbances}.

    \item \textbf{Robustness limit:} All controllers diverge at 25 N step disturbance (actuator saturation).

    \item \textbf{HIL validation confirms trends:} 12-33\% sim-hardware gap attributable to actuator dynamics and sensor noise.
\end{enumerate}

\subsection{Model Uncertainty (LT-6)}

\begin{enumerate}
    \item \textbf{Adaptive SMC most robust:} 97.5/100 robustness score, tolerates $\pm 20\%$ parameter variations with only 5.1\% settling time degradation.

    \item \textbf{Fixed-gain controllers vulnerable:} Classical SMC and STA-SMC diverge under combined +20\% errors (masses, lengths).

    \item \textbf{Manufacturing tolerances manageable:} All controllers tolerate $\pm 10\%$ errors (95-99\% success), sufficient for typical manufacturing tolerances.

    \item \textbf{Payload changes require adaptation:} If payload adds >20\% mass, use Adaptive SMC or Hybrid to ensure stability.

    \item \textbf{Inertia less critical:} Controllers tolerate $\pm 20\%$ inertia errors better than mass/length errors.
\end{enumerate}

\subsection{Design Guidelines for Robust Deployment}

\begin{enumerate}
    \item \textbf{Use Robust PSO:} Always optimize with 50\% nominal + 50\% disturbed fitness (Chapter 9).

    \item \textbf{Choose controller based on dominant uncertainty:}
        \begin{itemize}
            \item Disturbances: STA-SMC or Hybrid Adaptive STA
            \item Model uncertainty: Adaptive SMC or Hybrid Adaptive STA
            \item Unknown/mixed: Hybrid Adaptive STA (best all-around)
        \end{itemize}

    \item \textbf{Enable adaptive scheduling conditionally:}
        \begin{itemize}
            \item For sinusoidal/impulse disturbances: Yes (11-40.6\% chattering reduction)
            \item For step disturbances: No (+354\% overshoot penalty)
        \end{itemize}

    \item \textbf{Allocate 2$\times$ disturbance margin:} Design for worst-case = 2$\times$ expected disturbance magnitude.

    \item \textbf{Validate on hardware:} Expect 12-33\% performance degradation due to actuator dynamics, sensor noise, and model mismatch.
\end{enumerate}

%===============================================================================
\section{Future Research Directions}
%===============================================================================

\begin{enumerate}
    \item \textbf{Sensor Noise Robustness:} Evaluate controllers under encoder quantization noise ($0.01-0.5^\circ$), communication delays (5-20 ms), and Kalman filter integration.

    \item \textbf{Disturbance-Aware Scheduling:} Develop online disturbance classifier (step vs impulse vs sinusoidal) to enable/disable adaptive scheduling dynamically.

    \item \textbf{Multi-Objective Robustness Optimization:} Use MOPSO to generate Pareto frontiers for chattering-overshoot-robustness trade-offs.

    \item \textbf{Learning-Based Adaptation:} Train neural network to predict optimal gains based on system state and disturbance history (online learning).

    \item \textbf{Actuator Dynamics Modeling:} Extend dynamics model to include motor bandwidth limits (0.05 s delay), backlash, and friction. Re-optimize gains for realistic actuators.

    \item \textbf{Long-Duration Tests:} Current experiments run 10 s simulations. Industrial deployment requires hours-long tests to identify slow parameter drift, thermal effects, and wear.

    \item \textbf{Multi-Disturbance Scenarios:} Test controllers under simultaneous step + sinusoidal disturbances (e.g., wind + floor vibration).
\end{enumerate}

%===============================================================================
\section{Conclusion}
%===============================================================================

This chapter established that robust PSO optimization (Chapter 9) enables controllers to operate in uncertain environments. Key achievements:

\begin{enumerate}
    \item \textbf{Disturbance Rejection Validation:} 95.6-98.8\% improvement over default gains, demonstrating necessity of robust fitness.

    \item \textbf{Adaptive Scheduling Results:} 11-40.6\% chattering reduction for impulse/sinusoidal disturbances, but +354\% overshoot penalty for step disturbances.

    \item \textbf{Model Uncertainty Analysis:} Adaptive SMC tolerates $\pm 20\%$ parameter errors with 5.1\% degradation, while fixed-gain controllers diverge under combined errors.

    \item \textbf{HIL Validation:} Confirmed 12-33\% sim-hardware gap, validating simulation predictions on physical hardware.

    \item \textbf{Controller Selection Guide:} Provided decision matrix (Table \ref{tab:controller_selection_uncertainty}) for industrial deployment.
\end{enumerate}

\textbf{Recommendation for Deployment:} Use Hybrid Adaptive STA-SMC with robust PSO optimization for best all-around robustness (96.0/100 score, balanced disturbance rejection and model uncertainty tolerance).
