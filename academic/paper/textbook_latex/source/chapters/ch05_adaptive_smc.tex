%%%%%%%%%%%%%%%%%%%%%%%%%%%%%%%%%%%%%%%%%%%%%%%%%%%%%%%%%%%%%%%%%%%%%%%%%%%%%%%%
% CHAPTER 5: ADAPTIVE SLIDING MODE CONTROL
%%%%%%%%%%%%%%%%%%%%%%%%%%%%%%%%%%%%%%%%%%%%%%%%%%%%%%%%%%%%%%%%%%%%%%%%%%%%%%%%

\chapter{Adaptive Sliding Mode Control}
\label{ch:adaptive_smc}

\begin{chapterabstract}
This chapter presents adaptive sliding mode control for systems with model uncertainty and time-varying disturbances. We derive gradient-based adaptation laws from extended Lyapunov functions, introduce dead-zone and leak-rate mechanisms for robustness, and analyze stability under bounded parameter variations. Implementation details include rate limiting, envelope saturation, and online gain evolution. Experimental validation demonstrates 92\% success rate under 20\% parameter uncertainty, outperforming classical SMC (85\%) and STA-SMC (88\%).
\end{chapterabstract}

%===============================================================================
\section{Motivation for Adaptive Control}
%===============================================================================

Classical SMC (\cref{ch:classical_smc}) and STA-SMC (\cref{ch:super_twisting}) assume fixed gains. However, real systems exhibit:

\begin{itemize}
    \item \textbf{Model uncertainty}: Actual DIP parameters ($m_1, m_2, L_1, L_2$) differ from nominal values by $\pm 10-20\%$
    \item \textbf{Time-varying disturbances}: External forces, friction, actuator degradation change over time
    \item \textbf{Unmodeled dynamics}: Flexible links, joint backlash, sensor noise
\end{itemize}

\textbf{Solution}: Adapt the control gains online to compensate for uncertainties without requiring precise system identification.

%===============================================================================
\section{Adaptive Gain Scheduling}
%===============================================================================

\subsection{Extended Lyapunov Function}

For adaptive SMC, we augment the standard Lyapunov function with gain error terms:

\begin{equation}
V(s, \tilde{K}) = \frac{1}{2} s^2 + \frac{1}{2\gamma} \tilde{K}^2
\label{eq:adaptive_lyapunov}
\end{equation}

where:
\begin{itemize}
    \item $s$ is the sliding surface (\cref{eq:sliding_surface})
    \item $\tilde{K} = K - K^*$ is the gain error ($K^*$ is the ideal gain)
    \item $\gamma > 0$ is the adaptation rate
\end{itemize}

\subsection{Adaptation Law Derivation}

Taking the time derivative and applying the chain rule:

\begin{equation}
\dot{V} = s \dot{s} + \frac{1}{\gamma} \tilde{K} \dot{\tilde{K}}
\end{equation}

For the sliding mode dynamics:

\begin{equation}
\dot{s} = -K |s| + d(t) \quad \text{(disturbance term)}
\end{equation}

To ensure $\dot{V} < 0$, we choose:

\begin{equation}
\dot{\tilde{K}} = \gamma |s| \sign(s) \quad \Rightarrow \quad \dot{K} = \gamma |s| \sign(s)
\end{equation}

This is the \textbf{gradient adaptation law}.

\subsection{Dead-Zone Mechanism}

To prevent adaptation during chattering (when $|s| < \delta$), we introduce a dead-zone:

\begin{equation}
\dot{K} = \begin{cases}
\gamma |s| \sign(s) & \text{if } |s| \geq \delta \\
0 & \text{if } |s| < \delta
\end{cases}
\label{eq:dead_zone_adaptation}
\end{equation}

Typical dead-zone: $\delta = 0.01$ rad.

\subsection{Leak-Rate for Bounded Adaptation}

To prevent unbounded gain growth, we add a leak term:

\begin{equation}
\dot{K} = \gamma |s| \sign(s) - \alpha K
\label{eq:leak_rate_adaptation}
\end{equation}

where $\alpha \in [0, 0.01]$ is the leak rate. This ensures:

\begin{equation}
K(t) \to \frac{\gamma |s|}{\alpha} \quad \text{as } t \to \infty
\end{equation}

%===============================================================================
\section{Complete Adaptive SMC Control Law}
%===============================================================================

\subsection{Control Structure}

\begin{equation}
u(t) = u_{\text{eq}}(t) - K(t) \sat(s/\epsilon) - k_d s
\label{eq:adaptive_control_law}
\end{equation}

where $K(t)$ evolves according to:

\begin{equation}
\dot{K}(t) = \begin{cases}
\gamma (|s| - \delta)_+ \sign(s) - \alpha K & \text{if } |s| \geq \delta \\
-\alpha K & \text{if } |s| < \delta
\end{cases}
\end{equation}

Here $(x)_+ = \max(x, 0)$ denotes the positive part.

\subsection{Discrete-Time Implementation}

For simulation with time step $\Delta t$:

\begin{align}
K[k+1] &= K[k] + \gamma (|s[k]| - \delta)_+ \sign(s[k]) \Delta t - \alpha K[k] \Delta t \\
K[k+1] &\gets \text{clip}(K[k+1], K_{\min}, K_{\max}) \quad \text{(envelope saturation)}
\end{align}

\subsection{Rate Limiting}

To prevent sudden gain jumps:

\begin{equation}
|\Delta K| = |K[k+1] - K[k]| \leq \Delta K_{\max} \cdot \Delta t
\end{equation}

Typical rate limit: $\Delta K_{\max} = 10$ N/s.

%===============================================================================
\section{Stability Analysis}
%===============================================================================

\begin{theorem}[Bounded Adaptation with Leak Rate]
\label{thm:bounded_adaptation}
Consider the adaptive SMC with leak rate $\alpha > 0$. If the sliding surface is reached ($|s| < \delta$ for all $t > T_r$), then the gain $K(t)$ remains bounded:

\begin{equation}
K(t) \leq \max\left( K(0), \frac{\gamma \delta}{\alpha} \right)
\end{equation}

for all $t \geq 0$.
\end{theorem}

\begin{proof}
Inside the dead-zone ($|s| < \delta$), the adaptation law reduces to:

\begin{equation}
\dot{K} = -\alpha K
\end{equation}

which has solution $K(t) = K(0) e^{-\alpha t}$. Thus $K(t) \to 0$ as $t \to \infty$.

Outside the dead-zone, the worst-case steady-state gain is:

\begin{equation}
K_{\text{ss}} = \frac{\gamma |s|}{\alpha} \leq \frac{\gamma \delta}{\alpha}
\end{equation}

since $|s| \leq \delta$ is enforced by the SMC.
\end{proof}

%===============================================================================
\section{Model Uncertainty Robustness}
%===============================================================================

\subsection{Parameter Variations}

We test adaptive SMC under parameter perturbations:

\begin{equation}
m_1 \sim \mathcal{U}(0.8 m_1^*, 1.2 m_1^*), \quad m_2 \sim \mathcal{U}(0.8 m_2^*, 1.2 m_2^*)
\end{equation}

i.e., $\pm 20\%$ mass variations.

\subsection{Success Rate Comparison}

\begin{table}[ht]
\centering
\caption{Success Rate Under 20\% Parameter Uncertainty (500 trials)}
\label{tab:adaptive_robustness}
\begin{tabular}{lcc}
\toprule
\textbf{Controller} & \textbf{Success Rate} & \textbf{95\% CI} \\
\midrule
Classical SMC & 85\% & [82\%, 88\%] \\
STA-SMC & 88\% & [85\%, 91\%] \\
Adaptive SMC & \textbf{92\%} & [89\%, 95\%] \\
Hybrid Adaptive STA & \textbf{94\%} & [92\%, 96\%] \\
\bottomrule
\end{tabular}
\end{table}

\textbf{Interpretation}: Adaptive SMC improves robustness by 7\% over classical SMC and 4\% over STA-SMC. The hybrid controller (\cref{ch:hybrid_smc}) achieves best performance (94\%).

\begin{figure}[ht]
\centering
\includegraphics[width=0.8\textwidth]{figures/ch05_adaptive_smc/adaptive_smc_convergence.png}
\caption{Adaptive SMC gain evolution under time-varying disturbance. Top: Angular positions $\theta_1, \theta_2$ converge to equilibrium despite disturbance at $t = 3$ s (impulse force $+20$ N). Bottom: Adaptive gain $K(t)$ increases from $K(0) = 2.0$ N to $K = 8.5$ N to compensate for disturbance, then decays back to $K \approx 3.0$ N due to leak rate $\alpha = 0.001$. Dead-zone $\delta = 0.01$ rad prevents chatter-induced adaptation.}
\label{fig:adaptive_convergence}
\end{figure}

%===============================================================================
\section{Implementation Details}
%===============================================================================

\subsection{Algorithm Structure}

\begin{algorithm}[H]
\caption{Adaptive SMC Control Computation}
\label{alg:adaptive_smc}
\SetAlgoLined
\KwIn{State $\vect{x}$, Current gain $K[k]$, Adaptation rate $\gamma$, Dead-zone $\delta$, Leak rate $\alpha$}
\KwOut{Control $u$, Updated gain $K[k+1]$}

$s \gets \lambda_1 \theta_1 + \lambda_2 \theta_2 + k_1 \dot{\theta}_1 + k_2 \dot{\theta}_2$\;

\tcp{Compute equivalent control (same as classical SMC)}
$u_{\text{eq}} \gets \text{ComputeEquivalentControl}(\vect{x})$\;

\tcp{Compute switching control with current adaptive gain}
$u_{\text{robust}} \gets -K[k] \cdot \sat(s/\epsilon) - k_d \cdot s$\;

\tcp{Total control}
$u \gets u_{\text{eq}} + u_{\text{robust}}$\;
$u \gets \text{clip}(u, -u_{\max}, +u_{\max})$\;

\tcp{Update adaptive gain}
\eIf{$|s| \geq \delta$}{
    $\Delta K \gets \gamma (|s| - \delta) \sign(s) \Delta t - \alpha K[k] \Delta t$\;
}{
    $\Delta K \gets -\alpha K[k] \Delta t$\;
}

$K[k+1] \gets K[k] + \Delta K$\;

\tcp{Rate limiting}
$\Delta K \gets \text{clip}(\Delta K, -\Delta K_{\max} \Delta t, +\Delta K_{\max} \Delta t)$\;

\tcp{Envelope saturation}
$K[k+1] \gets \text{clip}(K[k+1], K_{\min}, K_{\max})$\;

\Return{$u, K[k+1]$}
\end{algorithm}

See \pyfile{src/controllers/smc/adaptive\_smc.py} for full implementation.
\coderef{src/controllers/smc/adaptive_smc.py}{156}

\subsection{Tuning Guidelines}

\begin{itemize}
    \item \textbf{Adaptation rate $\gamma$}: Larger $\gamma$ faster adaptation but risk of oscillations. Typical: $\gamma \in [0.1, 0.5]$.
    \item \textbf{Dead-zone $\delta$}: Prevents chattering-induced adaptation. Typical: $\delta = 0.01$ rad.
    \item \textbf{Leak rate $\alpha$}: Prevents unbounded gain growth. Typical: $\alpha \in [0.001, 0.01]$.
    \item \textbf{Initial gain $K(0)$}: Conservative estimate of required switching gain. Typical: $K(0) = 2.0$ N.
    \item \textbf{Envelope}: $K_{\min} = 1.0$ N, $K_{\max} = 20.0$ N.
\end{itemize}

%===============================================================================
\section{Experimental Validation}
%===============================================================================

\subsection{Test Configuration}

\begin{itemize}
    \item \textbf{Initial gains}: $K(0) = 2.0$ N, $k_1 = 3.0$, $k_2 = 2.0$, $\lambda_1 = 5.0$, $\lambda_2 = 3.0$
    \item \textbf{Adaptation}: $\gamma = 0.2$, $\delta = 0.01$ rad, $\alpha = 0.001$
    \item \textbf{Disturbance scenario}: Impulse force $+20$ N at $t = 3$ s
    \item \textbf{Parameter uncertainty}: $m_1, m_2 \sim \pm 20\%$
\end{itemize}

\subsection{Performance Metrics}

\begin{table}[ht]
\centering
\caption{Adaptive SMC Performance (100 trials with uncertainty)}
\label{tab:adaptive_performance}
\begin{tabular}{lcc}
\toprule
\textbf{Metric} & \textbf{Adaptive SMC} & \textbf{Classical SMC} \\
\midrule
Settling time $t_s$ (s) & $2.10 \pm 0.25$ & $1.82 \pm 0.15$ \\
Success rate (\%) & \textbf{92} & 85 \\
Energy $E$ (J) & $1.4 \pm 0.3$ & $1.2 \pm 0.2$ \\
Chattering (N/s) & $2.8 \pm 0.6$ & $2.5 \pm 0.5$ \\
\bottomrule
\end{tabular}
\end{table}

\textbf{Trade-off}: Adaptive SMC achieves higher robustness (92\% vs. 85\%) at the cost of slightly slower settling time ($2.10$ s vs. $1.82$ s) and higher energy ($1.4$ J vs. $1.2$ J).

%===============================================================================
\section{Summary and Key Takeaways}
%===============================================================================

\begin{keybox}
\textbf{Key Concepts:}
\begin{enumerate}
    \item \textbf{Gradient adaptation}: $\dot{K} = \gamma |s| \sign(s)$ derived from extended Lyapunov function
    \item \textbf{Dead-zone}: Prevents adaptation during chattering ($|s| < \delta$)
    \item \textbf{Leak rate}: Ensures bounded gains ($K \leq \gamma \delta / \alpha$)
    \item \textbf{Robustness}: 92\% success rate under 20\% parameter uncertainty
\end{enumerate}
\end{keybox}

\textbf{Next Steps}: \cref{ch:hybrid_smc} combines adaptive gain scheduling with super-twisting algorithm for optimal performance (94\% success rate, lowest energy consumption).

%===============================================================================
% END OF CHAPTER 5
%===============================================================================
