% ==============================================================================
% EPISODE 008: RESEARCH OUTPUTS AND PUBLICATIONS
% ==============================================================================
% ==============================================================================
% MASTER TEMPLATE FOR PODCAST EPISODE CHEATSHEET PDFs
% ==============================================================================
% Purpose: Beginner-friendly, colorful, multi-page study guides (2-4 pages)
% Target Audience: Complete beginners (Path 0 learners)
% Visual Style: Infographic-style with vibrant colors, icons, callout boxes
% ==============================================================================

\documentclass[11pt,a4paper]{article}

% ------------------------------------------------------------------------------
% GEOMETRY & LAYOUT
% ------------------------------------------------------------------------------
\usepackage[top=1.5cm, bottom=1.5cm, left=1.5cm, right=1.5cm, headheight=14pt]{geometry}
\usepackage{multicol}
\usepackage{fancyhdr}
\pagestyle{fancy}

% ------------------------------------------------------------------------------
% COLOR PALETTE (Vibrant & Beginner-Friendly)
% ------------------------------------------------------------------------------
\usepackage{xcolor}
\definecolor{primary}{RGB}{41, 128, 185}      % Blue - Main concepts, headers
\definecolor{secondary}{RGB}{39, 174, 96}     % Green - Success, examples
\definecolor{accent}{RGB}{230, 126, 34}       % Orange - Important notes
\definecolor{warning}{RGB}{231, 76, 60}       % Red - Common pitfalls
\definecolor{background}{RGB}{236, 240, 241}  % Light gray - Callout boxes
\definecolor{codeblock}{RGB}{44, 62, 80}      % Dark blue-gray - Code background
\definecolor{highlight}{RGB}{241, 196, 15}    % Yellow - Highlights

% ------------------------------------------------------------------------------
% TIKZ & GRAPHICS
% ------------------------------------------------------------------------------
\usepackage{tikz}
\usetikzlibrary{shapes, arrows, positioning, calc, shadows, decorations.pathreplacing, backgrounds, fit}
\usepackage{graphicx}
\usepackage{float}

% TikZ styles for consistent diagrams
\tikzstyle{block} = [rectangle, draw, fill=primary!20, text width=5em, text centered, rounded corners, minimum height=3em, drop shadow]
\tikzstyle{arrow} = [thick,->,>=stealth]
\tikzstyle{process} = [rectangle, draw, fill=secondary!20, text width=6em, text centered, rounded corners, minimum height=3em]
\tikzstyle{decision} = [diamond, draw, fill=accent!20, text width=4.5em, text badly centered, inner sep=0pt]
\tikzstyle{cloud} = [ellipse, draw, fill=background, text width=5em, text centered, minimum height=2.5em]

% ------------------------------------------------------------------------------
% BOXES & CALLOUTS (tcolorbox)
% ------------------------------------------------------------------------------
\usepackage{tcolorbox}
\tcbuselibrary{skins, breakable, raster}

% Key Concept Box (Blue)
\newtcolorbox{keypoint}{
    enhanced,
    colback=primary!10,
    colframe=primary,
    fonttitle=\bfseries,
    title=\faLightbulb\ Key Concept,
    attach boxed title to top left={yshift=-2mm, xshift=5mm},
    boxed title style={colback=primary},
    breakable
}

% Example Box (Green)
\newtcolorbox{example}{
    enhanced,
    colback=secondary!10,
    colframe=secondary,
    fonttitle=\bfseries,
    title=\faCode\ Example,
    attach boxed title to top left={yshift=-2mm, xshift=5mm},
    boxed title style={colback=secondary},
    breakable
}

% Warning Box (Red)
\newtcolorbox{warning}{
    enhanced,
    colback=warning!10,
    colframe=warning,
    fonttitle=\bfseries,
    title=\faExclamationTriangle\ Common Pitfall,
    attach boxed title to top left={yshift=-2mm, xshift=5mm},
    boxed title style={colback=warning},
    breakable
}

% Tip Box (Orange)
\newtcolorbox{tip}{
    enhanced,
    colback=accent!10,
    colframe=accent,
    fonttitle=\bfseries,
    title=\faLightbulb\ Pro Tip,
    attach boxed title to top left={yshift=-2mm, xshift=5mm},
    boxed title style={colback=accent},
    breakable
}

% Summary Box (Light gray)
\newtcolorbox{summary}{
    enhanced,
    colback=background,
    colframe=codeblock,
    fonttitle=\bfseries,
    title=\faListUl\ Quick Summary,
    attach boxed title to top left={yshift=-2mm, xshift=5mm},
    boxed title style={colback=codeblock},
    breakable
}

% ------------------------------------------------------------------------------
% ICONS & SYMBOLS (fontawesome5)
% ------------------------------------------------------------------------------
\usepackage{fontawesome5}

% Custom icon commands for consistency
\newcommand{\iconKey}{\textcolor{primary}{\faLightbulb}}
\newcommand{\iconCode}{\textcolor{secondary}{\faCode}}
\newcommand{\iconWarning}{\textcolor{warning}{\faExclamationTriangle}}
\newcommand{\iconTip}{\textcolor{accent}{\faInfoCircle}}
\newcommand{\iconLink}{\textcolor{primary}{\faLink}}
\newcommand{\iconBook}{\textcolor{secondary}{\faBook}}
\newcommand{\iconTarget}{\textcolor{accent}{\faBullseye}}

% ------------------------------------------------------------------------------
% TYPOGRAPHY & FONTS
% ------------------------------------------------------------------------------
\usepackage{lmodern}
\usepackage[T1]{fontenc}
\usepackage[utf8]{inputenc}

% Section styling
\usepackage{titlesec}
\titleformat{\section}{\Large\bfseries\color{primary}}{\thesection}{1em}{}[\titlerule]
\titleformat{\subsection}{\large\bfseries\color{secondary}}{\thesubsection}{1em}{}

% ------------------------------------------------------------------------------
% CODE LISTINGS
% ------------------------------------------------------------------------------
\usepackage{listings}
\lstset{
    basicstyle=\ttfamily\footnotesize\color{white},
    backgroundcolor=\color{codeblock},
    keywordstyle=\color{primary!80},
    commentstyle=\color{secondary!60}\itshape,
    stringstyle=\color{accent!80},
    numbers=left,
    numberstyle=\tiny\color{white!50},
    stepnumber=1,
    numbersep=8pt,
    frame=single,
    rulecolor=\color{codeblock},
    breaklines=true,
    breakatwhitespace=true,
    tabsize=4,
    captionpos=b
}

% Python-specific styling
\lstdefinestyle{python}{
    language=Python,
    morekeywords={self, def, class, import, from, as, return, if, else, elif, for, while, True, False, None}
}

% YAML-specific styling
\lstdefinestyle{yaml}{
    basicstyle=\ttfamily\footnotesize,
    showstringspaces=false,
    commentstyle=\color{secondary!60}\itshape,
    keywordstyle=\color{primary!80}
}

% ------------------------------------------------------------------------------
% HYPERLINKS
% ------------------------------------------------------------------------------
\usepackage{hyperref}
\hypersetup{
    colorlinks=true,
    linkcolor=primary,
    urlcolor=secondary,
    citecolor=accent
}

% ------------------------------------------------------------------------------
% HEADER & FOOTER CUSTOMIZATION
% ------------------------------------------------------------------------------
\fancyhf{}
\fancyhead[L]{\textcolor{primary}{\textbf{DIP-SMC-PSO Podcast Cheatsheet}}}
\fancyhead[R]{\textcolor{secondary}{\episodetitle}}
\fancyfoot[C]{\textcolor{primary}{\thepage}}
\renewcommand{\headrulewidth}{0.5pt}
\renewcommand{\footrulewidth}{0.5pt}

% ------------------------------------------------------------------------------
% CUSTOM COMMANDS
% ------------------------------------------------------------------------------

% Episode title command (to be defined in each episode file)
\newcommand{\episodetitle}{Episode Title}

% Learning objective command
\newcommand{\learningobjective}[1]{%
    \begin{center}
    \begin{tcolorbox}[colback=highlight!30, colframe=accent, width=0.9\textwidth]
    \iconTarget\ \textbf{Learning Objective:} #1
    \end{tcolorbox}
    \end{center}
}

% Quick reference command (for formulas/code snippets)
\newcommand{\quickref}[2]{%
    \begin{tcolorbox}[colback=background, colframe=primary, title=\faBookmark\ #1]
    #2
    \end{tcolorbox}
}

% Resource link command
\newcommand{\resourcelink}[2]{%
    \iconLink\ \href{#1}{\textcolor{secondary}{#2}}
}

% ------------------------------------------------------------------------------
% TITLE PAGE FORMATTING
% ------------------------------------------------------------------------------
\usepackage{afterpage}

\newcommand{\makeepisodetitle}[4]{%
    \begin{titlepage}
    \begin{tikzpicture}[remember picture, overlay]
        % Background gradient
        \fill[primary!20] (current page.south west) rectangle (current page.north east);

        % Title box
        \node[
            fill=white,
            rounded corners=10pt,
            drop shadow,
            text width=0.8\paperwidth,
            align=center
        ] at (current page.center) {
            \Huge\bfseries\color{primary} #1 \\[1em]
            \Large\color{secondary} #2 \\[2em]
            \large\color{codeblock} Part #3 $\cdot$ Duration: #4 \\[1em]
            \normalsize\color{codeblock} \textit{Beginner-Friendly Visual Study Guide}
        };

        % Footer
        \node[
            anchor=south,
            text width=0.9\paperwidth,
            align=center
        ] at (current page.south) {
            \color{primary}\rule{0.8\paperwidth}{0.5pt} \\[0.5em]
            \small\color{codeblock}
            \textbf{Repository:} \url{https://github.com/theSadeQ/dip-smc-pso} \\
            \textbf{Documentation:} academic/paper/presentations/podcasts/episodes/ \\[1em]
        };
    \end{tikzpicture}
    \end{titlepage}
}

% ------------------------------------------------------------------------------
% MATH & EQUATIONS
% ------------------------------------------------------------------------------
\usepackage{amsmath, amssymb, amsthm}
\usepackage{mathtools}

% Equation box (highlighted equations)
\newcommand{\eqbox}[1]{%
    \begin{tcolorbox}[colback=primary!5, colframe=primary, boxrule=1pt]
    \begin{equation}
    #1
    \end{equation}
    \end{tcolorbox}
}

% ------------------------------------------------------------------------------
% TABLES
% ------------------------------------------------------------------------------
\usepackage{booktabs}
\usepackage{array}
\usepackage{multirow}
\usepackage{colortbl}

% Custom table colors
\newcommand{\tableheadcolor}{\rowcolor{primary!30}}

% ------------------------------------------------------------------------------
% END OF PREAMBLE
% ------------------------------------------------------------------------------

% ==============================================================================
% REUSABLE TIKZ COMPONENTS FOR PODCAST CHEATSHEETS
% ==============================================================================
% Purpose: Common diagrams, flowcharts, and visual elements
% Usage: % ==============================================================================
% REUSABLE TIKZ COMPONENTS FOR PODCAST CHEATSHEETS
% ==============================================================================
% Purpose: Common diagrams, flowcharts, and visual elements
% Usage: % ==============================================================================
% REUSABLE TIKZ COMPONENTS FOR PODCAST CHEATSHEETS
% ==============================================================================
% Purpose: Common diagrams, flowcharts, and visual elements
% Usage: \input{../templates/tikz_components.tex} in each episode file
% ==============================================================================

\usepackage{tikz}
\usetikzlibrary{shapes, arrows, positioning, calc, shadows, decorations.pathreplacing, backgrounds, fit}

% ------------------------------------------------------------------------------
% SYSTEM ARCHITECTURE DIAGRAM (7 Controllers + Plant + Optimizer)
% ------------------------------------------------------------------------------
\newcommand{\systemarchitecture}{%
\begin{tikzpicture}[node distance=2cm, auto]
    % Define styles
    \tikzstyle{component} = [rectangle, draw, fill=primary!20, text width=3.5cm, text centered, rounded corners, minimum height=1.2cm, drop shadow]
    \tikzstyle{controller} = [rectangle, draw, fill=secondary!20, text width=2.8cm, text centered, rounded corners, minimum height=1cm]
    \tikzstyle{arrow} = [thick,->,>=stealth, color=primary!70]

    % Central components
    \node[component] (sim) {\textbf{Simulation Engine}\\Runner + Context};
    \node[component, below=1cm of sim] (plant) {\textbf{Plant Model}\\DIP Dynamics};
    \node[component, right=2.5cm of sim] (opt) {\textbf{PSO Optimizer}\\Gain Tuner};

    % Controllers (left side, vertical stack)
    \node[controller, left=2.5cm of sim, yshift=2.5cm] (ctrl1) {Classical SMC};
    \node[controller, below=0.3cm of ctrl1] (ctrl2) {STA-SMC};
    \node[controller, below=0.3cm of ctrl2] (ctrl3) {Adaptive SMC};
    \node[controller, below=0.3cm of ctrl3] (ctrl4) {Hybrid STA};
    \node[controller, below=0.3cm of ctrl4] (ctrl5) {Swing-Up};
    \node[controller, below=0.3cm of ctrl5] (ctrl6) {PID};
    \node[controller, below=0.3cm of ctrl6] (ctrl7) {MPC (Exp)};

    % Controller factory box (grouping)
    \node[draw, dashed, fit=(ctrl1) (ctrl7), fill=secondary!5, rounded corners, inner sep=5pt, label=above:\textbf{Controller Factory}] (factory) {};

    % Monitoring & Visualization
    \node[component, right=2.5cm of plant] (mon) {\textbf{Monitoring}\\Metrics + Latency};
    \node[component, above=0.5cm of mon] (viz) {\textbf{Visualization}\\Plots + Animation};

    % Arrows
    \draw[arrow] (factory) -- node[above] {state} (sim);
    \draw[arrow] (sim) -- node[right] {control} (plant);
    \draw[arrow] (plant) -- (sim);
    \draw[arrow] (opt) -- node[above] {tuned gains} (sim);
    \draw[arrow] (sim) -- (mon);
    \draw[arrow] (mon) -- (viz);

\end{tikzpicture}
}

% ------------------------------------------------------------------------------
% DOUBLE INVERTED PENDULUM PHYSICAL DIAGRAM
% ------------------------------------------------------------------------------
\newcommand{\dipphysical}{%
\begin{tikzpicture}[scale=0.8]
    % Cart
    \draw[fill=primary!30, draw=primary, thick] (-1,0) rectangle (1,0.6);
    \node at (0,0.3) {\small Cart};

    % Wheels
    \draw[fill=codeblock, draw=codeblock] (-0.6,-0.2) circle (0.2);
    \draw[fill=codeblock, draw=codeblock] (0.6,-0.2) circle (0.2);

    % Track
    \draw[thick, <->] (-3,-0.5) -- (3,-0.5);
    \node at (0,-0.8) {\small Track (x-axis)};

    % First pendulum (link 1)
    \draw[thick, color=secondary] (0,0.6) -- ++(60:2.5) coordinate (p1);
    \draw[fill=secondary!30, draw=secondary] (p1) circle (0.3);
    \node[color=secondary] at (p1) {$m_1$};
    \node[color=secondary] at (0.8,1.5) {$\theta_1$};

    % Second pendulum (link 2)
    \draw[thick, color=accent] (p1) -- ++(50:2) coordinate (p2);
    \draw[fill=accent!30, draw=accent] (p2) circle (0.25);
    \node[color=accent] at (p2) {$m_2$};
    \node[color=accent] at ($(p1)+(0.6,1.2)$) {$\theta_2$};

    % Force arrow
    \draw[->, ultra thick, color=warning] (-1.5,0.3) -- (-2.5,0.3);
    \node[color=warning] at (-2,-0.2) {$u$ (Force)};

    % Gravity arrow
    \draw[->, thick, color=codeblock] (3.5,2) -- (3.5,0.5);
    \node at (4,1.2) {$g$};

    % Annotations
    \node[color=secondary] at (-2,2) {Link 1: $\ell_1, m_1, I_1$};
    \node[color=accent] at (-2,1.4) {Link 2: $\ell_2, m_2, I_2$};
    \node[color=primary] at (-2,0.8) {Cart: $M, x$};

\end{tikzpicture}
}

% ------------------------------------------------------------------------------
% PHASE PORTRAIT (Reaching → Sliding → Chattering)
% ------------------------------------------------------------------------------
\newcommand{\phaseportrait}{%
\begin{tikzpicture}[scale=0.9]
    % Axes
    \draw[->] (-3,0) -- (3,0) node[right] {$x_1$ (position)};
    \draw[->] (0,-2.5) -- (0,2.5) node[above] {$x_2$ (velocity)};

    % Sliding surface (diagonal line)
    \draw[thick, color=secondary, dashed] (-2.5,-2) -- (2.5,2);
    \node[color=secondary] at (2.8,2.2) {$s=0$};
    \node[color=secondary] at (2.5,1.3) {\small Sliding Surface};

    % Reaching phase trajectory
    \draw[thick, color=primary, ->] (-2.5,1.5) .. controls (-1.5,0.8) and (-1,0.3) .. (-0.5,0);
    \node[color=primary] at (-2.2,1.8) {\textbf{1. Reaching}};

    % Sliding phase trajectory
    \draw[thick, color=accent, ->] (-0.5,0) -- (0.5,0.5);
    \node[color=accent] at (0,0.8) {\textbf{2. Sliding}};

    % Chattering region (zigzag near origin)
    \draw[thick, color=warning, decoration={zigzag, segment length=2mm, amplitude=0.5mm}, decorate] (0.5,0.5) -- (1.5,1.5);
    \node[color=warning] at (1.8,1.2) {\textbf{3. Chattering}};

    % Equilibrium point
    \fill[color=codeblock] (0,0) circle (3pt);
    \node[below right] at (0,0) {\small Equilibrium};

\end{tikzpicture}
}

% ------------------------------------------------------------------------------
% PSO CONVERGENCE FLOWCHART
% ------------------------------------------------------------------------------
\newcommand{\psoflowchart}{%
\begin{tikzpicture}[node distance=1.5cm]
    \node[block, fill=primary!30] (init) {\textbf{Initialize}\\50 particles\\Random positions};
    \node[process, below of=init] (eval) {\textbf{Evaluate}\\Fitness (cost)};
    \node[process, below of=eval] (update) {\textbf{Update}\\Velocity + Position};
    \node[decision, below of=update, yshift=-0.5cm] (conv) {\textbf{Converged?}};
    \node[block, fill=secondary!30, below of=conv, yshift=-0.5cm] (done) {\textbf{Done}\\Return best gains};

    \draw[arrow] (init) -- (eval);
    \draw[arrow] (eval) -- (update);
    \draw[arrow] (update) -- (conv);
    \draw[arrow] (conv) -- node[right] {Yes} (done);
    \draw[arrow] (conv.west) -- ++(-2,0) |- node[near start, above] {No} (eval.west);

    % Annotations
    \node[right=1.5cm of eval, text width=3cm, align=left, color=codeblock] {
        \small Run simulation\\
        Calculate settling time\\
        Compute cost = $f(gains)$
    };
    \node[right=1.5cm of update, text width=3cm, align=left, color=codeblock] {
        \small $v \leftarrow w \cdot v + c_1 r_1 (p - x) + c_2 r_2 (g - x)$\\
        $x \leftarrow x + v$
    };
\end{tikzpicture}
}

% ------------------------------------------------------------------------------
% CONTROL LOOP TIMING (10ms cycle)
% ------------------------------------------------------------------------------
\newcommand{\controllooptiming}{%
\begin{tikzpicture}[scale=1.2]
    % Timeline
    \draw[thick, ->] (0,0) -- (11,0) node[right] {Time (ms)};

    % Time markers
    \foreach \x in {0,1,2,5,10}
        \draw (\x,0.1) -- (\x,-0.1) node[below] {\x};

    % Sensor read (0-0.1ms)
    \draw[fill=primary!30, draw=primary] (0,0.3) rectangle (0.1,0.8);
    \node[above, color=primary] at (0.05,0.8) {\tiny Sensor};

    % Controller compute (0.1-1.0ms)
    \draw[fill=secondary!30, draw=secondary] (0.1,0.3) rectangle (1.0,0.8);
    \node[above, color=secondary] at (0.55,0.8) {\small Controller};

    % Actuation (1.0-1.1ms)
    \draw[fill=accent!30, draw=accent] (1.0,0.3) rectangle (1.1,0.8);
    \node[above, color=accent] at (1.05,0.8) {\tiny Act};

    % Plant integration (1.1-10ms)
    \draw[fill=warning!20, draw=warning] (1.1,0.3) rectangle (10,0.8);
    \node[above, color=warning] at (5.5,0.8) {\small Plant Integration (RK45)};

    % Annotations
    \draw[dashed] (0,-0.5) -- (0,0.3);
    \draw[dashed] (10,-0.5) -- (10,0.3);
    \draw[<->, thick] (0,-0.5) -- (10,-0.5);
    \node[below] at (5,-0.5) {\textbf{10ms deadline (100 Hz)}};

\end{tikzpicture}
}

% ------------------------------------------------------------------------------
% FACTORY PATTERN (Vending Machine Analogy)
% ------------------------------------------------------------------------------
\newcommand{\factorypattern}{%
\begin{tikzpicture}[scale=0.9]
    % Vending machine frame
    \draw[thick, fill=background, rounded corners] (-0.5,-2) rectangle (3.5,3);
    \node at (1.5,2.5) {\textbf{Controller Factory}};

    % Selection buttons
    \foreach \y/\name/\color in {1.5/Classical/primary, 0.8/STA-SMC/secondary, 0.1/Adaptive/accent, -0.6/Hybrid/warning} {
        \node[circle, draw=\color, fill=\color!30, minimum size=0.6cm] at (0.5,\y) {};
        \node[right, color=codeblock] at (1,\y) {\small \name};
    }

    % Output slot
    \draw[fill=codeblock!20] (0.5,-1.5) rectangle (2.5,-1);
    \node[color=codeblock] at (1.5,-1.8) {\small Output};

    % User and controller object
    \node[left=1cm of 0.5,1.5] (user) {\faUser};
    \node[right=1cm of 2.5,-1.2, block, fill=primary!30] (controller) {Controller\\Object};

    % Arrows
    \draw[->, thick, color=primary] (user) -- (0.3,1.5);
    \draw[->, thick, color=secondary] (2.5,-1.2) -- (controller);

    % Annotation
    \node[below, color=codeblock, text width=5cm, align=center] at (1.5,-2.5) {
        \small Press button $\rightarrow$ Get controller instance
    };

\end{tikzpicture}
}

% ------------------------------------------------------------------------------
% GIT WORKFLOW (Time Machine)
% ------------------------------------------------------------------------------
\newcommand{\gittimeline}{%
\begin{tikzpicture}[scale=0.8]
    % Timeline
    \draw[thick, ->] (0,0) -- (10,0) node[right] {Time};

    % Commits (save points)
    \foreach \x/\label in {1/Oct 2024, 3/Dec 2024, 5/Mar 2025, 7/Nov 2025} {
        \fill[color=primary] (\x,0) circle (3pt);
        \node[above] at (\x,0.3) {\small \label};
        \draw[->, dashed, color=secondary] (\x,-0.3) -- (\x,-1);
        \node[below, color=secondary, text width=1.5cm, align=center] at (\x,-1.3) {\tiny Snapshot};
    }

    % Current position
    \fill[color=warning] (7,0) circle (5pt);
    \node[below=0.5cm, color=warning] at (7,0) {\textbf{You are here}};

    % Time travel arrows
    \draw[<-, ultra thick, color=accent] (7,-2) -- (3,-2);
    \node[below, color=accent] at (5,-2) {\small Time travel to Dec 2024};

    % Annotation
    \node[color=codeblock, text width=8cm, align=center] at (5,-3) {
        \small Git lets you travel to any save point in project history
    };

\end{tikzpicture}
}

% ------------------------------------------------------------------------------
% DIRECTORY STRUCTURE (Zoning Laws)
% ------------------------------------------------------------------------------
\newcommand{\directoryzoning}{%
\begin{tikzpicture}[
    level 1/.style={sibling distance=4cm, level distance=2cm},
    level 2/.style={sibling distance=2cm, level distance=1.5cm},
    every node/.style={draw, rounded corners, font=\small}
]
    \node[fill=primary!20] {Root (Project)}
        child {node[fill=secondary!20] {src/ \faBuilding\\Business District}
            child {node[fill=secondary!10] {controllers/}}
            child {node[fill=secondary!10] {plant/}}
        }
        child {node[fill=accent!20] {scripts/ \faWrench\\Workshop}
            child {node[fill=accent!10] {benchmarks/}}
            child {node[fill=accent!10] {testing/}}
        }
        child {node[fill=warning!20] {tests/ \faCheckCircle\\QA District}
            child {node[fill=warning!10] {unit/}}
            child {node[fill=warning!10] {integration/}}
        };
\end{tikzpicture}
}

% ==============================================================================
% END OF TIKZ COMPONENTS
% ==============================================================================
 in each episode file
% ==============================================================================

\usepackage{tikz}
\usetikzlibrary{shapes, arrows, positioning, calc, shadows, decorations.pathreplacing, backgrounds, fit}

% ------------------------------------------------------------------------------
% SYSTEM ARCHITECTURE DIAGRAM (7 Controllers + Plant + Optimizer)
% ------------------------------------------------------------------------------
\newcommand{\systemarchitecture}{%
\begin{tikzpicture}[node distance=2cm, auto]
    % Define styles
    \tikzstyle{component} = [rectangle, draw, fill=primary!20, text width=3.5cm, text centered, rounded corners, minimum height=1.2cm, drop shadow]
    \tikzstyle{controller} = [rectangle, draw, fill=secondary!20, text width=2.8cm, text centered, rounded corners, minimum height=1cm]
    \tikzstyle{arrow} = [thick,->,>=stealth, color=primary!70]

    % Central components
    \node[component] (sim) {\textbf{Simulation Engine}\\Runner + Context};
    \node[component, below=1cm of sim] (plant) {\textbf{Plant Model}\\DIP Dynamics};
    \node[component, right=2.5cm of sim] (opt) {\textbf{PSO Optimizer}\\Gain Tuner};

    % Controllers (left side, vertical stack)
    \node[controller, left=2.5cm of sim, yshift=2.5cm] (ctrl1) {Classical SMC};
    \node[controller, below=0.3cm of ctrl1] (ctrl2) {STA-SMC};
    \node[controller, below=0.3cm of ctrl2] (ctrl3) {Adaptive SMC};
    \node[controller, below=0.3cm of ctrl3] (ctrl4) {Hybrid STA};
    \node[controller, below=0.3cm of ctrl4] (ctrl5) {Swing-Up};
    \node[controller, below=0.3cm of ctrl5] (ctrl6) {PID};
    \node[controller, below=0.3cm of ctrl6] (ctrl7) {MPC (Exp)};

    % Controller factory box (grouping)
    \node[draw, dashed, fit=(ctrl1) (ctrl7), fill=secondary!5, rounded corners, inner sep=5pt, label=above:\textbf{Controller Factory}] (factory) {};

    % Monitoring & Visualization
    \node[component, right=2.5cm of plant] (mon) {\textbf{Monitoring}\\Metrics + Latency};
    \node[component, above=0.5cm of mon] (viz) {\textbf{Visualization}\\Plots + Animation};

    % Arrows
    \draw[arrow] (factory) -- node[above] {state} (sim);
    \draw[arrow] (sim) -- node[right] {control} (plant);
    \draw[arrow] (plant) -- (sim);
    \draw[arrow] (opt) -- node[above] {tuned gains} (sim);
    \draw[arrow] (sim) -- (mon);
    \draw[arrow] (mon) -- (viz);

\end{tikzpicture}
}

% ------------------------------------------------------------------------------
% DOUBLE INVERTED PENDULUM PHYSICAL DIAGRAM
% ------------------------------------------------------------------------------
\newcommand{\dipphysical}{%
\begin{tikzpicture}[scale=0.8]
    % Cart
    \draw[fill=primary!30, draw=primary, thick] (-1,0) rectangle (1,0.6);
    \node at (0,0.3) {\small Cart};

    % Wheels
    \draw[fill=codeblock, draw=codeblock] (-0.6,-0.2) circle (0.2);
    \draw[fill=codeblock, draw=codeblock] (0.6,-0.2) circle (0.2);

    % Track
    \draw[thick, <->] (-3,-0.5) -- (3,-0.5);
    \node at (0,-0.8) {\small Track (x-axis)};

    % First pendulum (link 1)
    \draw[thick, color=secondary] (0,0.6) -- ++(60:2.5) coordinate (p1);
    \draw[fill=secondary!30, draw=secondary] (p1) circle (0.3);
    \node[color=secondary] at (p1) {$m_1$};
    \node[color=secondary] at (0.8,1.5) {$\theta_1$};

    % Second pendulum (link 2)
    \draw[thick, color=accent] (p1) -- ++(50:2) coordinate (p2);
    \draw[fill=accent!30, draw=accent] (p2) circle (0.25);
    \node[color=accent] at (p2) {$m_2$};
    \node[color=accent] at ($(p1)+(0.6,1.2)$) {$\theta_2$};

    % Force arrow
    \draw[->, ultra thick, color=warning] (-1.5,0.3) -- (-2.5,0.3);
    \node[color=warning] at (-2,-0.2) {$u$ (Force)};

    % Gravity arrow
    \draw[->, thick, color=codeblock] (3.5,2) -- (3.5,0.5);
    \node at (4,1.2) {$g$};

    % Annotations
    \node[color=secondary] at (-2,2) {Link 1: $\ell_1, m_1, I_1$};
    \node[color=accent] at (-2,1.4) {Link 2: $\ell_2, m_2, I_2$};
    \node[color=primary] at (-2,0.8) {Cart: $M, x$};

\end{tikzpicture}
}

% ------------------------------------------------------------------------------
% PHASE PORTRAIT (Reaching → Sliding → Chattering)
% ------------------------------------------------------------------------------
\newcommand{\phaseportrait}{%
\begin{tikzpicture}[scale=0.9]
    % Axes
    \draw[->] (-3,0) -- (3,0) node[right] {$x_1$ (position)};
    \draw[->] (0,-2.5) -- (0,2.5) node[above] {$x_2$ (velocity)};

    % Sliding surface (diagonal line)
    \draw[thick, color=secondary, dashed] (-2.5,-2) -- (2.5,2);
    \node[color=secondary] at (2.8,2.2) {$s=0$};
    \node[color=secondary] at (2.5,1.3) {\small Sliding Surface};

    % Reaching phase trajectory
    \draw[thick, color=primary, ->] (-2.5,1.5) .. controls (-1.5,0.8) and (-1,0.3) .. (-0.5,0);
    \node[color=primary] at (-2.2,1.8) {\textbf{1. Reaching}};

    % Sliding phase trajectory
    \draw[thick, color=accent, ->] (-0.5,0) -- (0.5,0.5);
    \node[color=accent] at (0,0.8) {\textbf{2. Sliding}};

    % Chattering region (zigzag near origin)
    \draw[thick, color=warning, decoration={zigzag, segment length=2mm, amplitude=0.5mm}, decorate] (0.5,0.5) -- (1.5,1.5);
    \node[color=warning] at (1.8,1.2) {\textbf{3. Chattering}};

    % Equilibrium point
    \fill[color=codeblock] (0,0) circle (3pt);
    \node[below right] at (0,0) {\small Equilibrium};

\end{tikzpicture}
}

% ------------------------------------------------------------------------------
% PSO CONVERGENCE FLOWCHART
% ------------------------------------------------------------------------------
\newcommand{\psoflowchart}{%
\begin{tikzpicture}[node distance=1.5cm]
    \node[block, fill=primary!30] (init) {\textbf{Initialize}\\50 particles\\Random positions};
    \node[process, below of=init] (eval) {\textbf{Evaluate}\\Fitness (cost)};
    \node[process, below of=eval] (update) {\textbf{Update}\\Velocity + Position};
    \node[decision, below of=update, yshift=-0.5cm] (conv) {\textbf{Converged?}};
    \node[block, fill=secondary!30, below of=conv, yshift=-0.5cm] (done) {\textbf{Done}\\Return best gains};

    \draw[arrow] (init) -- (eval);
    \draw[arrow] (eval) -- (update);
    \draw[arrow] (update) -- (conv);
    \draw[arrow] (conv) -- node[right] {Yes} (done);
    \draw[arrow] (conv.west) -- ++(-2,0) |- node[near start, above] {No} (eval.west);

    % Annotations
    \node[right=1.5cm of eval, text width=3cm, align=left, color=codeblock] {
        \small Run simulation\\
        Calculate settling time\\
        Compute cost = $f(gains)$
    };
    \node[right=1.5cm of update, text width=3cm, align=left, color=codeblock] {
        \small $v \leftarrow w \cdot v + c_1 r_1 (p - x) + c_2 r_2 (g - x)$\\
        $x \leftarrow x + v$
    };
\end{tikzpicture}
}

% ------------------------------------------------------------------------------
% CONTROL LOOP TIMING (10ms cycle)
% ------------------------------------------------------------------------------
\newcommand{\controllooptiming}{%
\begin{tikzpicture}[scale=1.2]
    % Timeline
    \draw[thick, ->] (0,0) -- (11,0) node[right] {Time (ms)};

    % Time markers
    \foreach \x in {0,1,2,5,10}
        \draw (\x,0.1) -- (\x,-0.1) node[below] {\x};

    % Sensor read (0-0.1ms)
    \draw[fill=primary!30, draw=primary] (0,0.3) rectangle (0.1,0.8);
    \node[above, color=primary] at (0.05,0.8) {\tiny Sensor};

    % Controller compute (0.1-1.0ms)
    \draw[fill=secondary!30, draw=secondary] (0.1,0.3) rectangle (1.0,0.8);
    \node[above, color=secondary] at (0.55,0.8) {\small Controller};

    % Actuation (1.0-1.1ms)
    \draw[fill=accent!30, draw=accent] (1.0,0.3) rectangle (1.1,0.8);
    \node[above, color=accent] at (1.05,0.8) {\tiny Act};

    % Plant integration (1.1-10ms)
    \draw[fill=warning!20, draw=warning] (1.1,0.3) rectangle (10,0.8);
    \node[above, color=warning] at (5.5,0.8) {\small Plant Integration (RK45)};

    % Annotations
    \draw[dashed] (0,-0.5) -- (0,0.3);
    \draw[dashed] (10,-0.5) -- (10,0.3);
    \draw[<->, thick] (0,-0.5) -- (10,-0.5);
    \node[below] at (5,-0.5) {\textbf{10ms deadline (100 Hz)}};

\end{tikzpicture}
}

% ------------------------------------------------------------------------------
% FACTORY PATTERN (Vending Machine Analogy)
% ------------------------------------------------------------------------------
\newcommand{\factorypattern}{%
\begin{tikzpicture}[scale=0.9]
    % Vending machine frame
    \draw[thick, fill=background, rounded corners] (-0.5,-2) rectangle (3.5,3);
    \node at (1.5,2.5) {\textbf{Controller Factory}};

    % Selection buttons
    \foreach \y/\name/\color in {1.5/Classical/primary, 0.8/STA-SMC/secondary, 0.1/Adaptive/accent, -0.6/Hybrid/warning} {
        \node[circle, draw=\color, fill=\color!30, minimum size=0.6cm] at (0.5,\y) {};
        \node[right, color=codeblock] at (1,\y) {\small \name};
    }

    % Output slot
    \draw[fill=codeblock!20] (0.5,-1.5) rectangle (2.5,-1);
    \node[color=codeblock] at (1.5,-1.8) {\small Output};

    % User and controller object
    \node[left=1cm of 0.5,1.5] (user) {\faUser};
    \node[right=1cm of 2.5,-1.2, block, fill=primary!30] (controller) {Controller\\Object};

    % Arrows
    \draw[->, thick, color=primary] (user) -- (0.3,1.5);
    \draw[->, thick, color=secondary] (2.5,-1.2) -- (controller);

    % Annotation
    \node[below, color=codeblock, text width=5cm, align=center] at (1.5,-2.5) {
        \small Press button $\rightarrow$ Get controller instance
    };

\end{tikzpicture}
}

% ------------------------------------------------------------------------------
% GIT WORKFLOW (Time Machine)
% ------------------------------------------------------------------------------
\newcommand{\gittimeline}{%
\begin{tikzpicture}[scale=0.8]
    % Timeline
    \draw[thick, ->] (0,0) -- (10,0) node[right] {Time};

    % Commits (save points)
    \foreach \x/\label in {1/Oct 2024, 3/Dec 2024, 5/Mar 2025, 7/Nov 2025} {
        \fill[color=primary] (\x,0) circle (3pt);
        \node[above] at (\x,0.3) {\small \label};
        \draw[->, dashed, color=secondary] (\x,-0.3) -- (\x,-1);
        \node[below, color=secondary, text width=1.5cm, align=center] at (\x,-1.3) {\tiny Snapshot};
    }

    % Current position
    \fill[color=warning] (7,0) circle (5pt);
    \node[below=0.5cm, color=warning] at (7,0) {\textbf{You are here}};

    % Time travel arrows
    \draw[<-, ultra thick, color=accent] (7,-2) -- (3,-2);
    \node[below, color=accent] at (5,-2) {\small Time travel to Dec 2024};

    % Annotation
    \node[color=codeblock, text width=8cm, align=center] at (5,-3) {
        \small Git lets you travel to any save point in project history
    };

\end{tikzpicture}
}

% ------------------------------------------------------------------------------
% DIRECTORY STRUCTURE (Zoning Laws)
% ------------------------------------------------------------------------------
\newcommand{\directoryzoning}{%
\begin{tikzpicture}[
    level 1/.style={sibling distance=4cm, level distance=2cm},
    level 2/.style={sibling distance=2cm, level distance=1.5cm},
    every node/.style={draw, rounded corners, font=\small}
]
    \node[fill=primary!20] {Root (Project)}
        child {node[fill=secondary!20] {src/ \faBuilding\\Business District}
            child {node[fill=secondary!10] {controllers/}}
            child {node[fill=secondary!10] {plant/}}
        }
        child {node[fill=accent!20] {scripts/ \faWrench\\Workshop}
            child {node[fill=accent!10] {benchmarks/}}
            child {node[fill=accent!10] {testing/}}
        }
        child {node[fill=warning!20] {tests/ \faCheckCircle\\QA District}
            child {node[fill=warning!10] {unit/}}
            child {node[fill=warning!10] {integration/}}
        };
\end{tikzpicture}
}

% ==============================================================================
% END OF TIKZ COMPONENTS
% ==============================================================================
 in each episode file
% ==============================================================================

\usepackage{tikz}
\usetikzlibrary{shapes, arrows, positioning, calc, shadows, decorations.pathreplacing, backgrounds, fit}

% ------------------------------------------------------------------------------
% SYSTEM ARCHITECTURE DIAGRAM (7 Controllers + Plant + Optimizer)
% ------------------------------------------------------------------------------
\newcommand{\systemarchitecture}{%
\begin{tikzpicture}[node distance=2cm, auto]
    % Define styles
    \tikzstyle{component} = [rectangle, draw, fill=primary!20, text width=3.5cm, text centered, rounded corners, minimum height=1.2cm, drop shadow]
    \tikzstyle{controller} = [rectangle, draw, fill=secondary!20, text width=2.8cm, text centered, rounded corners, minimum height=1cm]
    \tikzstyle{arrow} = [thick,->,>=stealth, color=primary!70]

    % Central components
    \node[component] (sim) {\textbf{Simulation Engine}\\Runner + Context};
    \node[component, below=1cm of sim] (plant) {\textbf{Plant Model}\\DIP Dynamics};
    \node[component, right=2.5cm of sim] (opt) {\textbf{PSO Optimizer}\\Gain Tuner};

    % Controllers (left side, vertical stack)
    \node[controller, left=2.5cm of sim, yshift=2.5cm] (ctrl1) {Classical SMC};
    \node[controller, below=0.3cm of ctrl1] (ctrl2) {STA-SMC};
    \node[controller, below=0.3cm of ctrl2] (ctrl3) {Adaptive SMC};
    \node[controller, below=0.3cm of ctrl3] (ctrl4) {Hybrid STA};
    \node[controller, below=0.3cm of ctrl4] (ctrl5) {Swing-Up};
    \node[controller, below=0.3cm of ctrl5] (ctrl6) {PID};
    \node[controller, below=0.3cm of ctrl6] (ctrl7) {MPC (Exp)};

    % Controller factory box (grouping)
    \node[draw, dashed, fit=(ctrl1) (ctrl7), fill=secondary!5, rounded corners, inner sep=5pt, label=above:\textbf{Controller Factory}] (factory) {};

    % Monitoring & Visualization
    \node[component, right=2.5cm of plant] (mon) {\textbf{Monitoring}\\Metrics + Latency};
    \node[component, above=0.5cm of mon] (viz) {\textbf{Visualization}\\Plots + Animation};

    % Arrows
    \draw[arrow] (factory) -- node[above] {state} (sim);
    \draw[arrow] (sim) -- node[right] {control} (plant);
    \draw[arrow] (plant) -- (sim);
    \draw[arrow] (opt) -- node[above] {tuned gains} (sim);
    \draw[arrow] (sim) -- (mon);
    \draw[arrow] (mon) -- (viz);

\end{tikzpicture}
}

% ------------------------------------------------------------------------------
% DOUBLE INVERTED PENDULUM PHYSICAL DIAGRAM
% ------------------------------------------------------------------------------
\newcommand{\dipphysical}{%
\begin{tikzpicture}[scale=0.8]
    % Cart
    \draw[fill=primary!30, draw=primary, thick] (-1,0) rectangle (1,0.6);
    \node at (0,0.3) {\small Cart};

    % Wheels
    \draw[fill=codeblock, draw=codeblock] (-0.6,-0.2) circle (0.2);
    \draw[fill=codeblock, draw=codeblock] (0.6,-0.2) circle (0.2);

    % Track
    \draw[thick, <->] (-3,-0.5) -- (3,-0.5);
    \node at (0,-0.8) {\small Track (x-axis)};

    % First pendulum (link 1)
    \draw[thick, color=secondary] (0,0.6) -- ++(60:2.5) coordinate (p1);
    \draw[fill=secondary!30, draw=secondary] (p1) circle (0.3);
    \node[color=secondary] at (p1) {$m_1$};
    \node[color=secondary] at (0.8,1.5) {$\theta_1$};

    % Second pendulum (link 2)
    \draw[thick, color=accent] (p1) -- ++(50:2) coordinate (p2);
    \draw[fill=accent!30, draw=accent] (p2) circle (0.25);
    \node[color=accent] at (p2) {$m_2$};
    \node[color=accent] at ($(p1)+(0.6,1.2)$) {$\theta_2$};

    % Force arrow
    \draw[->, ultra thick, color=warning] (-1.5,0.3) -- (-2.5,0.3);
    \node[color=warning] at (-2,-0.2) {$u$ (Force)};

    % Gravity arrow
    \draw[->, thick, color=codeblock] (3.5,2) -- (3.5,0.5);
    \node at (4,1.2) {$g$};

    % Annotations
    \node[color=secondary] at (-2,2) {Link 1: $\ell_1, m_1, I_1$};
    \node[color=accent] at (-2,1.4) {Link 2: $\ell_2, m_2, I_2$};
    \node[color=primary] at (-2,0.8) {Cart: $M, x$};

\end{tikzpicture}
}

% ------------------------------------------------------------------------------
% PHASE PORTRAIT (Reaching → Sliding → Chattering)
% ------------------------------------------------------------------------------
\newcommand{\phaseportrait}{%
\begin{tikzpicture}[scale=0.9]
    % Axes
    \draw[->] (-3,0) -- (3,0) node[right] {$x_1$ (position)};
    \draw[->] (0,-2.5) -- (0,2.5) node[above] {$x_2$ (velocity)};

    % Sliding surface (diagonal line)
    \draw[thick, color=secondary, dashed] (-2.5,-2) -- (2.5,2);
    \node[color=secondary] at (2.8,2.2) {$s=0$};
    \node[color=secondary] at (2.5,1.3) {\small Sliding Surface};

    % Reaching phase trajectory
    \draw[thick, color=primary, ->] (-2.5,1.5) .. controls (-1.5,0.8) and (-1,0.3) .. (-0.5,0);
    \node[color=primary] at (-2.2,1.8) {\textbf{1. Reaching}};

    % Sliding phase trajectory
    \draw[thick, color=accent, ->] (-0.5,0) -- (0.5,0.5);
    \node[color=accent] at (0,0.8) {\textbf{2. Sliding}};

    % Chattering region (zigzag near origin)
    \draw[thick, color=warning, decoration={zigzag, segment length=2mm, amplitude=0.5mm}, decorate] (0.5,0.5) -- (1.5,1.5);
    \node[color=warning] at (1.8,1.2) {\textbf{3. Chattering}};

    % Equilibrium point
    \fill[color=codeblock] (0,0) circle (3pt);
    \node[below right] at (0,0) {\small Equilibrium};

\end{tikzpicture}
}

% ------------------------------------------------------------------------------
% PSO CONVERGENCE FLOWCHART
% ------------------------------------------------------------------------------
\newcommand{\psoflowchart}{%
\begin{tikzpicture}[node distance=1.5cm]
    \node[block, fill=primary!30] (init) {\textbf{Initialize}\\50 particles\\Random positions};
    \node[process, below of=init] (eval) {\textbf{Evaluate}\\Fitness (cost)};
    \node[process, below of=eval] (update) {\textbf{Update}\\Velocity + Position};
    \node[decision, below of=update, yshift=-0.5cm] (conv) {\textbf{Converged?}};
    \node[block, fill=secondary!30, below of=conv, yshift=-0.5cm] (done) {\textbf{Done}\\Return best gains};

    \draw[arrow] (init) -- (eval);
    \draw[arrow] (eval) -- (update);
    \draw[arrow] (update) -- (conv);
    \draw[arrow] (conv) -- node[right] {Yes} (done);
    \draw[arrow] (conv.west) -- ++(-2,0) |- node[near start, above] {No} (eval.west);

    % Annotations
    \node[right=1.5cm of eval, text width=3cm, align=left, color=codeblock] {
        \small Run simulation\\
        Calculate settling time\\
        Compute cost = $f(gains)$
    };
    \node[right=1.5cm of update, text width=3cm, align=left, color=codeblock] {
        \small $v \leftarrow w \cdot v + c_1 r_1 (p - x) + c_2 r_2 (g - x)$\\
        $x \leftarrow x + v$
    };
\end{tikzpicture}
}

% ------------------------------------------------------------------------------
% CONTROL LOOP TIMING (10ms cycle)
% ------------------------------------------------------------------------------
\newcommand{\controllooptiming}{%
\begin{tikzpicture}[scale=1.2]
    % Timeline
    \draw[thick, ->] (0,0) -- (11,0) node[right] {Time (ms)};

    % Time markers
    \foreach \x in {0,1,2,5,10}
        \draw (\x,0.1) -- (\x,-0.1) node[below] {\x};

    % Sensor read (0-0.1ms)
    \draw[fill=primary!30, draw=primary] (0,0.3) rectangle (0.1,0.8);
    \node[above, color=primary] at (0.05,0.8) {\tiny Sensor};

    % Controller compute (0.1-1.0ms)
    \draw[fill=secondary!30, draw=secondary] (0.1,0.3) rectangle (1.0,0.8);
    \node[above, color=secondary] at (0.55,0.8) {\small Controller};

    % Actuation (1.0-1.1ms)
    \draw[fill=accent!30, draw=accent] (1.0,0.3) rectangle (1.1,0.8);
    \node[above, color=accent] at (1.05,0.8) {\tiny Act};

    % Plant integration (1.1-10ms)
    \draw[fill=warning!20, draw=warning] (1.1,0.3) rectangle (10,0.8);
    \node[above, color=warning] at (5.5,0.8) {\small Plant Integration (RK45)};

    % Annotations
    \draw[dashed] (0,-0.5) -- (0,0.3);
    \draw[dashed] (10,-0.5) -- (10,0.3);
    \draw[<->, thick] (0,-0.5) -- (10,-0.5);
    \node[below] at (5,-0.5) {\textbf{10ms deadline (100 Hz)}};

\end{tikzpicture}
}

% ------------------------------------------------------------------------------
% FACTORY PATTERN (Vending Machine Analogy)
% ------------------------------------------------------------------------------
\newcommand{\factorypattern}{%
\begin{tikzpicture}[scale=0.9]
    % Vending machine frame
    \draw[thick, fill=background, rounded corners] (-0.5,-2) rectangle (3.5,3);
    \node at (1.5,2.5) {\textbf{Controller Factory}};

    % Selection buttons
    \foreach \y/\name/\color in {1.5/Classical/primary, 0.8/STA-SMC/secondary, 0.1/Adaptive/accent, -0.6/Hybrid/warning} {
        \node[circle, draw=\color, fill=\color!30, minimum size=0.6cm] at (0.5,\y) {};
        \node[right, color=codeblock] at (1,\y) {\small \name};
    }

    % Output slot
    \draw[fill=codeblock!20] (0.5,-1.5) rectangle (2.5,-1);
    \node[color=codeblock] at (1.5,-1.8) {\small Output};

    % User and controller object
    \node[left=1cm of 0.5,1.5] (user) {\faUser};
    \node[right=1cm of 2.5,-1.2, block, fill=primary!30] (controller) {Controller\\Object};

    % Arrows
    \draw[->, thick, color=primary] (user) -- (0.3,1.5);
    \draw[->, thick, color=secondary] (2.5,-1.2) -- (controller);

    % Annotation
    \node[below, color=codeblock, text width=5cm, align=center] at (1.5,-2.5) {
        \small Press button $\rightarrow$ Get controller instance
    };

\end{tikzpicture}
}

% ------------------------------------------------------------------------------
% GIT WORKFLOW (Time Machine)
% ------------------------------------------------------------------------------
\newcommand{\gittimeline}{%
\begin{tikzpicture}[scale=0.8]
    % Timeline
    \draw[thick, ->] (0,0) -- (10,0) node[right] {Time};

    % Commits (save points)
    \foreach \x/\label in {1/Oct 2024, 3/Dec 2024, 5/Mar 2025, 7/Nov 2025} {
        \fill[color=primary] (\x,0) circle (3pt);
        \node[above] at (\x,0.3) {\small \label};
        \draw[->, dashed, color=secondary] (\x,-0.3) -- (\x,-1);
        \node[below, color=secondary, text width=1.5cm, align=center] at (\x,-1.3) {\tiny Snapshot};
    }

    % Current position
    \fill[color=warning] (7,0) circle (5pt);
    \node[below=0.5cm, color=warning] at (7,0) {\textbf{You are here}};

    % Time travel arrows
    \draw[<-, ultra thick, color=accent] (7,-2) -- (3,-2);
    \node[below, color=accent] at (5,-2) {\small Time travel to Dec 2024};

    % Annotation
    \node[color=codeblock, text width=8cm, align=center] at (5,-3) {
        \small Git lets you travel to any save point in project history
    };

\end{tikzpicture}
}

% ------------------------------------------------------------------------------
% DIRECTORY STRUCTURE (Zoning Laws)
% ------------------------------------------------------------------------------
\newcommand{\directoryzoning}{%
\begin{tikzpicture}[
    level 1/.style={sibling distance=4cm, level distance=2cm},
    level 2/.style={sibling distance=2cm, level distance=1.5cm},
    every node/.style={draw, rounded corners, font=\small}
]
    \node[fill=primary!20] {Root (Project)}
        child {node[fill=secondary!20] {src/ \faBuilding\\Business District}
            child {node[fill=secondary!10] {controllers/}}
            child {node[fill=secondary!10] {plant/}}
        }
        child {node[fill=accent!20] {scripts/ \faWrench\\Workshop}
            child {node[fill=accent!10] {benchmarks/}}
            child {node[fill=accent!10] {testing/}}
        }
        child {node[fill=warning!20] {tests/ \faCheckCircle\\QA District}
            child {node[fill=warning!10] {unit/}}
            child {node[fill=warning!10] {integration/}}
        };
\end{tikzpicture}
}

% ==============================================================================
% END OF TIKZ COMPONENTS
% ==============================================================================


% Episode-specific title
\renewcommand{\episodetitle}{E008: Research \& Publications}

\begin{document}

% ==============================================================================
% TITLE PAGE
% ==============================================================================
\makeepisodetitle{E008: Research Outputs and Publications}{From 105,000 Lines of Code to 71-Page Submission-Ready Paper}{2}{15-20 minutes}

% ==============================================================================
% PAGE 1: PHASE 5 ROADMAP
% ==============================================================================
\learningobjective{Understand the 72-hour research roadmap, execution of quick wins / medium-term / long-term tasks, paper evolution (v1.0 → v2.1), automation workflows, and submission preparation}

\section*{The Research Challenge}

\begin{keypoint}
\textbf{Question:} Can 105,000 lines of working code count as research?

\textbf{Answer:} NO! Research requires communication - peer-reviewed papers, reproducible experiments, stability proofs.

\textbf{Goal:} Transform code into submission-ready publication
\end{keypoint}

\subsection*{Phase 5: The 72-Hour Research Roadmap}

\begin{tcolorbox}[colback=primary!5, colframe=primary, title=\faRocket\ Research Timeline]
\textbf{Duration:} 72 hours total effort over 8 weeks (Oct 29 - Nov 7, 2025)

\textbf{Structure:} Three-tier risk management approach

\begin{tabular}{llll}
\toprule
\textbf{Tier} & \textbf{Hours} & \textbf{Weeks} & \textbf{Strategy} \\
\midrule
Quick Wins & 8 (actual: 16) & Week 1 & Build confidence fast \\
Medium-Term & 18 & Weeks 2-4 & Provide substance \\
Long-Term & 46 & Months 2-3 & Real contributions \\
\textbf{TOTAL} & \textbf{72} & \textbf{8 weeks} & \textbf{Momentum → Depth} \\
\bottomrule
\end{tabular}
\end{tcolorbox}

\begin{tip}
\textbf{Why 3 tiers?} Risk management!

Start with 46-hour task $\Rightarrow$ Burnout before seeing results

Start with quick wins $\Rightarrow$ Early momentum $\Rightarrow$ Confidence for long-term work
\end{tip}

\section*{Quick Wins Week: 5 Tasks in Week 1}

\begin{multicols}{2}

\textbf{QW-1: SMC Theory Guide (8 hours)}
\begin{itemize}
    \item 800-1,200 lines explaining SMC fundamentals
    \item Classical SMC, super-twisting, adaptive
    \item Became Section 2 of final paper
\end{itemize}

\textbf{QW-2: Baseline Benchmarks (3 hours)}
\begin{itemize}
    \item Run all 7 controllers with default gains
    \item Extract 4 metrics: settling time, overshoot, energy, chattering
    \item Classical: 2.5s settling, 12\% overshoot
    \item STA: 2.1s settling, 8\% overshoot
\end{itemize}

\columnbreak

\textbf{QW-3: PSO Visualization (2 hours)}
\begin{itemize}
    \item Log particle positions every iteration
    \item Generate scatter plot animation
    \item Shows convergence to optimal gains
\end{itemize}

\textbf{QW-4: Chattering Metrics (2 hours)}
\begin{itemize}
    \item FFT-based high-frequency analysis
    \item Quantify vibration/oscillation
\end{itemize}

\textbf{QW-5: Status Tracking (1 hour)}
\begin{itemize}
    \item Update project docs showing Phase 5 progress
\end{itemize}

\end{multicols}

\begin{warning}
\textbf{Estimated vs Actual:} Planned 8 hours, took 16 hours

\textbf{Why?} Mathematical derivations are slow (rigorous proofs take time)

\textbf{Payoff:} Theory guide became Section 2 with minimal revisions
\end{warning}

% ==============================================================================
% PAGE 2: MEDIUM-TERM TASKS
% ==============================================================================
\newpage

\section*{Medium-Term Tasks: Benchmarks \& Optimization}

\subsection*{MT-5: Comprehensive Benchmark (The Core Study)}

\begin{keypoint}
\textbf{Goal:} Systematically compare all 7 controllers with statistical rigor

\textbf{Scale:} 700 simulations (7 controllers × 100 Monte Carlo trials each)
\end{keypoint}

\begin{tcolorbox}[colback=secondary!10, colframe=secondary, title=\faFlask\ MT-5 Four-Stage Process]
\textbf{Stage 1: Experimental Design}
\begin{itemize}
    \item 7 controllers to test
    \item 4 metrics to measure (settling time, overshoot, energy, chattering)
    \item 100 Monte Carlo trials per controller
    \item Total: 700 simulations
\end{itemize}

\textbf{Stage 2: Data Collection}
\begin{itemize}
    \item Batch simulation script (overnight, 12 hours on 8-core)
    \item Each sim: 10s simulated time, 2s wall time (Numba JIT)
    \item Output: HDF5 file (105 MB compressed)
\end{itemize}

\textbf{Stage 3: Statistical Analysis}
\begin{itemize}
    \item Bootstrap 95\% CI with 2,000 resamples
    \item Welch's t-test for 21 pairwise comparisons
    \item Bonferroni correction (corrected $\alpha = 0.0024$)
    \item Result: All differences statistically significant!
\end{itemize}

\textbf{Stage 4: Figure Generation \& Ranking}
\begin{itemize}
    \item Bar charts for each metric with error bars
    \item Heatmap showing rank matrix (rows=controllers, cols=metrics)
    \item Ranking: Hybrid Adaptive STA 1st, STA 2nd, Adaptive 3rd, Classical 4th
    \item Became Figure 13 in LT-7 paper
\end{itemize}
\end{tcolorbox}

\subsection*{MT-5 Debugging: Two Critical Issues}

\begin{warning}
\textbf{Issue 1: Trial 347 Divergence}

Pendulum fell during simulation! If uncaught, entire benchmark invalid.

\textbf{Root cause:} PSO found gains on edge of stability region

\textbf{Fix:} Added stability margin constraint to cost function
\end{warning}

\begin{warning}
\textbf{Issue 2: Memory Leak}

Loading 700 result files consumed 4 GB RAM!

\textbf{Fix:} Streaming reads - process one trial at a time

\textbf{Result:} Memory dropped to 200 MB
\end{warning}

\subsection*{MT-6: Boundary Layer Optimization (The Negative Result)}

\begin{example}
\textbf{Hypothesis:} Optimize boundary layer thickness to reduce chattering by 60-80\%

\textbf{Initial Result:} $\delta = 0.05$ rad gives 60-80\% reduction - CELEBRATE!

\textbf{Deep Dive Validation:} Re-ran with different FFT cutoffs (5 Hz, 10 Hz, 15 Hz)
\begin{itemize}
    \item At 5 Hz: 90\% reduction
    \item At 10 Hz: 66.5\% reduction
    \item At 15 Hz: 12\% reduction
\end{itemize}

\textbf{RED FLAG:} Metric too sensitive to arbitrary parameter!

\textbf{True Result:} 3.7\% reduction (baseline already near-optimal)

\textbf{Conclusion:} Negative result, but valuable - prevents future wasted effort
\end{example}

\begin{tip}
\textbf{Lesson:} Negative results ARE results! They guide future researchers away from dead ends.
\end{tip}

% ==============================================================================
% PAGE 3: MT-7, MT-8, LONG-TERM TASKS
% ==============================================================================
\newpage

\subsection*{MT-7: Robust PSO Validation}

\begin{keypoint}
\textbf{Goal:} Prove PSO finds consistent gains, not lucky outliers

\textbf{Method:} Run PSO with 100 different random seeds

\textbf{Result:} All converge to similar gains (coefficient of variation < 5\%)

\textbf{Conclusion:} PSO optimization is reproducible!
\end{keypoint}

\subsection*{MT-8: Disturbance Rejection Analysis}

\begin{tcolorbox}[colback=accent!10, colframe=accent, title=\faExclamationTriangle\ Five-Step Protocol]
\textbf{Step 1:} Baseline simulation (2s equilibrium, no disturbance)

\textbf{Step 2:} Apply 10-Newton step force at $t = 2$ s (20\% of max control)

\textbf{Step 3:} Measure recovery time (when cart returns to $\pm 0.01$ m of setpoint)

\textbf{Step 4:} Repeat for 100 Monte Carlo trials with varied initial conditions

\textbf{Step 5:} Statistical analysis - mean recovery time with 95\% CI
\end{tcolorbox}

\textbf{Results by Controller:}

\begin{tabular}{lll}
\toprule
\textbf{Controller} & \textbf{Recovery Time} & \textbf{Why?} \\
\midrule
Adaptive SMC & $0.8 \pm 0.05$ s & Increases gains when error spikes \\
Hybrid Adaptive STA & $0.85 \pm 0.06$ s & Adaptive + smooth control \\
STA-SMC & $1.1 \pm 0.08$ s & Fixed gains, no adaptation \\
Classical SMC & $1.5 \pm 0.12$ s & Fixed gains, chatters \\
Swing-up & $2.2 \pm 0.18$ s & Optimized for large angles, not disturbances \\
\bottomrule
\end{tabular}

\textbf{Paper Integration:} Figure 9 in LT-7 - time series showing cart position recovery

\section*{Long-Term Deliverables: 46 Hours, 3 Tasks}

\subsection*{LT-4: Lyapunov Stability Proofs}

\begin{keypoint}
\textbf{Goal:} Prove each controller drives system to equilibrium + quantify convergence rates

\textbf{Output:} $\sim$1,000 lines of mathematical derivations (Section 4 of paper)
\end{keypoint}

\textbf{Proof Structure (Example: Classical SMC):}

\begin{multicols}{2}

\textbf{Part 1: Define Lyapunov Function}
$$V(s) = \frac{1}{2} s^2$$

Always non-negative, zero only on sliding surface

\textbf{Part 2: Compute Time Derivative}
$$\dot{V} = s \cdot \dot{s}$$

\columnbreak

\textbf{Part 3: Substitute Control Law}
$$u = -K \cdot \text{sign}(s)$$
$$\dot{V} = -K |s| < 0$$

Negative definite when $s \neq 0$!

\textbf{Part 4: Convergence Rate}
$$t_{\text{reach}} = \frac{V(0)}{K}$$

Larger $K$ → faster reaching
\end{multicols}

\textbf{Rigor Level:} Formal but not machine-verified (Coq/Isabelle)

Rigorous enough for peer review in control systems journals

\textbf{Total Proofs:} 7 controllers × 3-5 pages each = $\sim$1,000 lines LaTeX

\textbf{References:} Utkin 1977, Levant 1993, Slotine 1991, Moreno 2008

% ==============================================================================
% PAGE 4: LT-6, LT-7, PAPER STRUCTURE
% ==============================================================================
\newpage

\subsection*{LT-6: Model Uncertainty Robustness Testing}

\begin{tcolorbox}[colback=warning!10, colframe=warning, title=\faExclamationTriangle\ Six-Step Protocol]
\textbf{Step 1:} Identify uncertain parameters (masses, lengths, friction)

\textbf{Step 2:} Define perturbation levels ($\pm 5\%, 10\%, 15\%, 20\%, 25\%, 30\%$)

\textbf{Step 3:} Generate perturbed models (all params randomly varied)

\textbf{Step 4:} Run simulations with nominal gains on perturbed models

\textbf{Step 5:} Measure performance degradation vs nominal

\textbf{Step 6:} Statistical analysis (50 Monte Carlo trials per level)
\end{tcolorbox}

\textbf{Results:}

\begin{multicols}{2}

\textbf{Adaptive Controllers (Robust):}
\begin{itemize}
    \item 10\% perturbation: +5\% settling time
    \item 20\% perturbation: +12\% settling
    \item 30\% perturbation: +22\% settling
    \item \textbf{Graceful linear degradation}
\end{itemize}

\columnbreak

\textbf{Classical SMC (Fragile):}
\begin{itemize}
    \item 10\% perturbation: +8\% settling
    \item 20\% perturbation: +35\% settling
    \item 30\% perturbation: \textbf{60\% FAILURE RATE!}
    \item Fixed gains cannot compensate
\end{itemize}

\end{multicols}

\textbf{Paper Integration:} Section 7.2 - Robustness to Model Uncertainty, Figure 10 degradation curves

\subsection*{LT-7: Research Paper Evolution}

\begin{keypoint}
\textbf{Final Output:} Submission-ready version 2.1

\textbf{Stats:} 71 pages (two-column IEEE format), 14 figures, 39 references
\end{keypoint}

\textbf{Version History:}

\begin{tabular}{lll}
\toprule
\textbf{Version} & \textbf{Pages} & \textbf{Key Changes} \\
\midrule
v0.5 & - & MPC attempt abandoned (50× too slow), never released \\
v1.0 & 50 pages & 8 figures, weak intro, no Lyapunov, thin results \\
v1.5 & 65 pages & +15 pages: 7-page intro (39 refs), 10-page Lyapunov, 12 figures \\
v2.0 & 70 pages & 8-page discussion rewrite, LT-6 integration, 14 figures \\
v2.1 & 71 pages & \textbf{SUBMISSION-READY}: 37 typos fixed, 18 unclear sentences, 47 cross-refs verified \\
\bottomrule
\end{tabular}

\textbf{Timeline:} Version 1.0 → 2.1 in 5 weeks
\begin{itemize}
    \item v1.0 → v1.5: 3 weeks (major additions)
    \item v1.5 → v2.0: 1.5 weeks (discussion rewrite)
    \item v2.0 → v2.1: 4 days (polish)
\end{itemize}

% ==============================================================================
% PAGE 5: PAPER STRUCTURE
% ==============================================================================
\newpage

\section*{LT-7 Paper Structure: Nine Sections}

\begin{tcolorbox}[colback=primary!5, colframe=primary, title=\faBook\ Complete Paper Organization]
\textbf{Section 1: Introduction}
\begin{itemize}
    \item Motivation - Why SMC for underactuated systems?
    \item Related work - Review of existing DIP control methods
    \item Contributions - What this paper adds
\end{itemize}

\textbf{Section 2: Controller Overview}
\begin{itemize}
    \item Mathematical descriptions of 7 SMC variants
    \item Classical, super-twisting, adaptive, hybrid adaptive STA, swing-up, terminal, integral
\end{itemize}

\textbf{Section 3: PSO Methodology}
\begin{itemize}
    \item Multi-objective cost function (settling time + energy + chattering)
    \item Swarm parameters: 50 particles, 100 iterations, inertia 0.7
    \item Validation: 100 random seeds for reproducibility
\end{itemize}

\textbf{Section 4: Lyapunov Analysis}
\begin{itemize}
    \item Stability proofs for each controller
    \item Convergence rate estimates
    \item Boundary layer tradeoffs
\end{itemize}

\textbf{Section 5: Experimental Setup}
\begin{itemize}
    \item DIP model parameters (masses, lengths, friction)
    \item Simulation details (timestep, duration, integrator)
    \item Initial conditions for benchmarks
\end{itemize}
\end{tcolorbox}

\begin{tcolorbox}[colback=secondary!10, colframe=secondary, title=\faChartLine\ Results \& Discussion]
\textbf{Section 6: Performance Comparison}
\begin{itemize}
    \item MT-5 comprehensive benchmark results
    \item Table with all 7 controllers (mean + 95\% CI)
    \item Figures 5-8: Bar charts for each metric
    \item Statistical analysis: Welch's t-test ($p < 0.001$ for all pairs)
\end{itemize}

\textbf{Section 7: Robustness Analysis}
\begin{itemize}
    \item Subsection 7.1: Disturbance rejection (MT-8) - Figure 9
    \item Subsection 7.2: Model uncertainty (LT-6) - Figure 10
\end{itemize}

\textbf{Section 8: Discussion}
\begin{itemize}
    \item Insights: Tradeoffs between chattering and tracking accuracy
    \item Why adaptive controllers outperform under uncertainty
    \item Practical considerations: Computational cost, sensor noise, tuning
    \item Limitations: Simulation only, not physical hardware
\end{itemize}

\textbf{Section 9: Conclusions and Future Work}
\begin{itemize}
    \item Summary of contributions (systematic comparison, PSO tuning, Lyapunov proofs)
    \item Key findings (Hybrid Adaptive STA best, adaptive essential for uncertain systems)
    \item Future work: Formal verification, learning-based tuning, embedded deployment
\end{itemize}
\end{tcolorbox}

% ==============================================================================
% PAGE 6: FIGURES & AUTOMATION
% ==============================================================================
\newpage

\section*{Fourteen Publication-Ready Figures}

\begin{keypoint}
\textbf{Selection Criteria:}
\begin{enumerate}
    \item Does it support a claim in the text?
    \item Does it add new information (not just repeat table)?
    \item Does it meet quality standards? (Vector format, 300 DPI, 10-12pt fonts, colorblind-safe)
\end{enumerate}
\end{keypoint}

\textbf{Figure List:}

\begin{multicols}{2}

\textbf{Architecture \& Theory:}
\begin{enumerate}
    \item Architecture block diagram
    \item Boundary layer illustration
    \item STA phase portrait
\end{enumerate}

\textbf{Performance Metrics:}
\begin{enumerate}
    \setcounter{enumi}{3}
    \item PSO convergence
    \item Settling time comparison
    \item Overshoot comparison
    \item Energy consumption
    \item Chattering frequency
\end{enumerate}

\columnbreak

\textbf{Robustness:}
\begin{enumerate}
    \setcounter{enumi}{8}
    \item Disturbance rejection time series
    \item Model uncertainty degradation
    \item Lyapunov stability regions
\end{enumerate}

\textbf{Statistical Validation:}
\begin{enumerate}
    \setcounter{enumi}{11}
    \item Monte Carlo histograms
    \item Controller ranking heatmap
    \item Pareto frontier
\end{enumerate}

\end{multicols}

\section*{Automation Workflow: Data to Publication in 3 Minutes}

\begin{tcolorbox}[colback=accent!10, colframe=accent, title=\faCogs\ Seven-Stage Pipeline]
\textbf{Stage 1: Data Collection}
\begin{itemize}
    \item 700 trials saved to HDF5 (105 MB)
    \item Includes state trajectory, control signal, timestamps, metadata
\end{itemize}

\textbf{Stage 2: Metric Computation (45 seconds)}
\begin{itemize}
    \item Load HDF5, compute settling time, overshoot, energy, chattering
    \item Output: CSV file (1.2 MB)
\end{itemize}

\textbf{Stage 3: Statistical Analysis}
\begin{itemize}
    \item Bootstrap CI (2,000 resamples per controller)
    \item Welch's t-tests with Bonferroni correction
    \item Output: JSON with means, CIs, p-values
\end{itemize}

\textbf{Stage 4: Figure Generation (60 seconds)}
\begin{itemize}
    \item Read JSON, create 14 plots with Matplotlib
    \item Consistent styling (colorblind-safe palette)
    \item Save as vector PDFs + high-res PNGs
\end{itemize}

\textbf{Stage 5: LaTeX Integration}
\begin{itemize}
    \item Each figure has metadata JSON (caption, data source, script, commit hash)
    \item Python script generates \texttt{\textbackslash includegraphics} + \texttt{\textbackslash caption} + \texttt{\textbackslash label}
\end{itemize}

\textbf{Stage 6: Document Compilation}
\begin{itemize}
    \item Run pdflatex, pull in all 14 figures
    \item Output: Submission PDF (8.2 MB)
\end{itemize}

\textbf{Stage 7: Reproducibility Validation}
\begin{itemize}
    \item Delete all figures, re-run pipeline
    \item Verify 14/14 PDFs are bit-for-bit identical
\end{itemize}
\end{tcolorbox}

\textbf{Total Time:} 3 minutes (vs 2 hours manual, 40× faster!)

% ==============================================================================
% PAGE 7: VERSION CONTROL & COLLABORATION
% ==============================================================================
\newpage

\section*{Version Control for Research: Five Practices}

\begin{keypoint}
\textbf{Goal:} Make every paper revision traceable and reproducible
\end{keypoint}

\begin{multicols}{2}

\textbf{Practice 1: Granular Commits}
\begin{lstlisting}[style=yaml, numbers=none]
git commit -m "Added Lyapunov
proofs for STA-SMC (Section 4.2)"

git commit -m "Fixed typo in
equation 17"
\end{lstlisting}

Every significant change = one commit

\textbf{Practice 2: Tag Paper Versions}
\begin{lstlisting}[style=yaml, numbers=none]
git tag v1.0-draft
git tag v1.0-submission-ieee-tcst
git tag v1.1-revision-1
\end{lstlisting}

Each tag captures exact code, data, figures

\columnbreak

\textbf{Practice 3: Embed Git Hash in PDF}

LaTeX footer includes:
\begin{lstlisting}[style=yaml, numbers=none]
\footnotesize{Git commit:
\texttt{\gitHash}}
\end{lstlisting}

Auto-generated from \texttt{git rev-parse HEAD}

Traceability: PDF → Repository state

\textbf{Practice 4: Git LFS for Large Files}
\begin{lstlisting}[style=yaml, numbers=none]
git lfs track "*.pdf"
git add figures/*.pdf
git commit -m "Add Figure 5-8"
\end{lstlisting}

Binary PDFs (5-15 MB total) stored efficiently

\end{multicols}

\textbf{Practice 5: Reproducibility README}

File: \texttt{academic/paper/experiments/REPRODUCTION.md}

Lists exact commands to regenerate every figure from data

\begin{warning}
\textbf{Example Value:} 6 months after submission, reviewer requests Figure 10 data

\textbf{Without Git tags:} Which data file? Which script version? \textbf{SCREWED!}

\textbf{With Git tags:} Check submission tag → commit 964dc438 → Find exact data file → Send to reviewer → \textbf{CRISIS AVERTED!}
\end{warning}

\section*{Collaboration Workflows: Four Mechanisms}

\begin{tcolorbox}[colback=secondary!10, colframe=secondary, title=\faUsers\ Multi-Author Coordination]
\textbf{Mechanism 1: Shared Git Repository}
\begin{itemize}
    \item All co-authors access \texttt{github.com/theSadeQ/dip-smc-pso.git}
    \item Create branch → Submit pull request → Primary author reviews → Merge
\end{itemize}

\textbf{Mechanism 2: Automated Figure Gallery}
\begin{itemize}
    \item Script generates HTML page with thumbnails + captions (14 figures)
    \item Deployed to GitHub Pages: \texttt{theSadeQ.github.io/dip-smc-pso/figures}
    \item Updates automatically via GitHub Actions on every push
    \item Co-authors view latest figures without cloning repo!
\end{itemize}

\textbf{Mechanism 3: Dropbox/Google Drive Sync}
\begin{itemize}
    \item For non-Git users: \texttt{/Shared/research\_paper/} folder
    \item Cron job syncs from Git to Dropbox every hour
\end{itemize}

\textbf{Mechanism 4: Overleaf Integration}
\begin{itemize}
    \item Upload LaTeX source + figures to Overleaf project
    \item Co-authors edit in browser
    \item Changes sync back to Git via Overleaf-GitHub integration
\end{itemize}
\end{tcolorbox}

% ==============================================================================
% PAGE 8: BIBLIOGRAPHY & SUBMISSION
% ==============================================================================
\newpage

\section*{Comprehensive Bibliography: 39 References}

\begin{keypoint}
\textbf{Three-Source Strategy:} Foundational papers + DIP-specific + PSO/optimization
\end{keypoint}

\begin{multicols}{2}

\textbf{Source 1: Foundational Papers}
\begin{itemize}
    \item \textbf{Utkin 1977} - "Variable structure systems with sliding modes" (15,000+ citations)
    \item \textbf{Levant 1993} - Introduces super-twisting algorithm
    \item \textbf{Slotine 1991} - "Applied Nonlinear Control" textbook (adaptive SMC)
    \item \textbf{Moreno 2008} - Hybrid adaptive STA theory
\end{itemize}

\textbf{Source 2: DIP-Specific Control}
\begin{itemize}
    \item \textbf{Fantoni 2002} - "Non-linear Control for Underactuated Systems"
    \item \textbf{Glück 2013} - Triple pendulum swing-up
    \item \textbf{Zhong 2001} - Energy-based DIP control
    \item Recent 2020-2025 papers from IEEE TCST
\end{itemize}

\columnbreak

\textbf{Source 3: PSO \& Optimization}
\begin{itemize}
    \item \textbf{Kennedy 1995} - Original PSO paper (100,000+ citations)
    \item \textbf{Clerc 2002} - PSO convergence analysis
    \item \textbf{Coello 2004} - Multi-objective PSO theory
    \item \textbf{Gad 2022} - PSO applications survey
\end{itemize}

\end{multicols}

\textbf{Systematic Search Process:}

\begin{enumerate}
    \item Google Scholar: "sliding mode control double inverted pendulum" (2,400 results)
    \item Filter by citations: Sort by most cited, read top 10 (foundational papers)
    \item Follow citation chains: If Paper A cites B and B looks relevant, read B
    \item Check recent work: Filter last 5 years, find state-of-the-art
    \item Journal alignment: Search target journal (IEEE TCST) for similar papers
    \item Result: Read 120 papers → Select 39 gold nuggets
\end{enumerate}

\textbf{Citation Management:} BibTeX file (\texttt{references.bib}) with auto-generated bibliography

\section*{Submission Package: Five Components}

\begin{tcolorbox}[colback=warning!10, colframe=warning, title=\faFileArchive\ Journal Submission Checklist]
\textbf{Component 1: Main Manuscript PDF}
\begin{itemize}
    \item 71 pages, IEEE two-column format, 8.2 MB
\end{itemize}

\textbf{Component 2: Individual Figure Files}
\begin{itemize}
    \item All 14 figures as separate high-res PDFs
    \item Zipped archive: 12 MB (for typesetting)
\end{itemize}

\textbf{Component 3: Supplementary Materials}
\begin{itemize}
    \item 15 pages of additional plots, data tables, extended proofs
    \item Material that doesn't fit in main paper
\end{itemize}

\textbf{Component 4: Cover Letter}
\begin{itemize}
    \item One page addressed to editor
    \item Why work is significant, how it fits journal scope
    \item Suggest 3 expert reviewers (no conflicts of interest)
\end{itemize}

\textbf{Component 5: Author Forms}
\begin{itemize}
    \item Copyright transfer agreement
    \item Conflict of interest statement
    \item Author contributions
    \item ORCID IDs for all authors
\end{itemize}
\end{tcolorbox}

\begin{warning}
\textbf{Miss ONE component?} Submission rejected before review!

\textbf{Use checklist to prevent this!}
\end{warning}

% ==============================================================================
% PAGE 9: KEY TAKEAWAYS & QUICK REFERENCE
% ==============================================================================
\newpage

\section*{Key Takeaways}

\begin{summary}
\textbf{72-Hour Roadmap:} 3 tiers for momentum (quick wins 8h → medium 18h → long-term 46h)

\textbf{Quick Wins (Week 1):} Theory docs, baselines, PSO viz, chattering metrics (planned 8h, actual 16h)

\textbf{MT-5 Benchmark:} 700 sims, 105 MB HDF5, bootstrap CI, Welch's t-test, ranking (Hybrid Adaptive STA 1st)

\textbf{MT-6 Negative Result:} Boundary layer 60-80\% reduction was biased metric (true: 3.7\%)

\textbf{MT-8 Disturbance:} Adaptive recovers in 0.8s, Classical in 1.5s (10N step force)

\textbf{LT-4 Lyapunov:} 7 proofs × 3-5 pages = 1,000 lines LaTeX (Section 4 of paper)

\textbf{LT-6 Uncertainty:} Adaptive graceful (22\% degradation @ 30\%), Classical fails (60\% @ 30\%)

\textbf{LT-7 Paper Evolution:} v1.0 (50 pgs) → v1.5 (+15 pgs) → v2.0 (+5 pgs) → v2.1 (71 pgs, submission-ready)

\textbf{14 Figures:} Architecture, theory, performance metrics (5), robustness (3), statistical validation (3)

\textbf{Automation:} 3 minutes data → PDF (vs 2 hours manual, 40× faster, 14/14 bit-identical)

\textbf{Version Control:} 5 practices (granular commits, git tags, embedded hash, LFS, reproducibility README)

\textbf{Collaboration:} Git repo + figure gallery + Dropbox + Overleaf (4 mechanisms)

\textbf{Bibliography:} 39 references from 120 papers (foundational + DIP + PSO)

\textbf{Submission Package:} 5 components (manuscript + figures + supplementary + cover letter + forms)
\end{summary}

\subsection*{Quick Reference: Research Workflow Commands}

\quickref{Run Research Tasks}{
\begin{lstlisting}[style=yaml, numbers=none]
# MT-5 Comprehensive Benchmark
python academic/paper/experiments/mt5_benchmark.py

# MT-8 Disturbance Rejection
python academic/paper/experiments/mt8_disturbance.py

# LT-6 Model Uncertainty
python academic/paper/experiments/lt6_uncertainty.py
\end{lstlisting}
}

\quickref{Generate Paper Figures}{
\begin{lstlisting}[style=yaml, numbers=none]
# All 14 figures from data (60 seconds)
python academic/paper/experiments/generate_figures.py

# Verify reproducibility
python academic/paper/experiments/verify_reproducibility.py
\end{lstlisting}
}

\quickref{Version Control for Papers}{
\begin{lstlisting}[style=yaml, numbers=none]
# Tag paper version
git tag v2.1-submission-ieee-tcst

# Embed commit hash in PDF footer
\footnotesize{Git: \texttt{\gitHash}}

# Find exact data for figure
git checkout v2.1-submission-ieee-tcst
cat REPRODUCTION.md | grep "Figure 10"
\end{lstlisting}
}

\subsection*{What's Next?}

\begin{keypoint}
\textbf{E009: Educational Materials \& Learning Paths}

Beginner roadmap (125-150 hours), 5 learning paths (complete novice → advanced researcher), documentation for different audiences

\textbf{Remember:} Research is communication. Code alone is not enough - publish, document, make reproducible!
\end{keypoint}

\end{document}
