% ==============================================================================
% EPISODE 009: EDUCATIONAL MATERIALS AND LEARNING PATHS
% ==============================================================================
% ==============================================================================
% MASTER TEMPLATE FOR PODCAST EPISODE CHEATSHEET PDFs
% ==============================================================================
% Purpose: Beginner-friendly, colorful, multi-page study guides (2-4 pages)
% Target Audience: Complete beginners (Path 0 learners)
% Visual Style: Infographic-style with vibrant colors, icons, callout boxes
% ==============================================================================

\documentclass[11pt,a4paper]{article}

% ------------------------------------------------------------------------------
% GEOMETRY & LAYOUT
% ------------------------------------------------------------------------------
\usepackage[top=1.5cm, bottom=1.5cm, left=1.5cm, right=1.5cm, headheight=14pt]{geometry}
\usepackage{multicol}
\usepackage{fancyhdr}
\pagestyle{fancy}

% ------------------------------------------------------------------------------
% COLOR PALETTE (Vibrant & Beginner-Friendly)
% ------------------------------------------------------------------------------
\usepackage{xcolor}
\definecolor{primary}{RGB}{41, 128, 185}      % Blue - Main concepts, headers
\definecolor{secondary}{RGB}{39, 174, 96}     % Green - Success, examples
\definecolor{accent}{RGB}{230, 126, 34}       % Orange - Important notes
\definecolor{warning}{RGB}{231, 76, 60}       % Red - Common pitfalls
\definecolor{background}{RGB}{236, 240, 241}  % Light gray - Callout boxes
\definecolor{codeblock}{RGB}{44, 62, 80}      % Dark blue-gray - Code background
\definecolor{highlight}{RGB}{241, 196, 15}    % Yellow - Highlights

% ------------------------------------------------------------------------------
% TIKZ & GRAPHICS
% ------------------------------------------------------------------------------
\usepackage{tikz}
\usetikzlibrary{shapes, arrows, positioning, calc, shadows, decorations.pathreplacing, backgrounds, fit}
\usepackage{graphicx}
\usepackage{float}

% TikZ styles for consistent diagrams
\tikzstyle{block} = [rectangle, draw, fill=primary!20, text width=5em, text centered, rounded corners, minimum height=3em, drop shadow]
\tikzstyle{arrow} = [thick,->,>=stealth]
\tikzstyle{process} = [rectangle, draw, fill=secondary!20, text width=6em, text centered, rounded corners, minimum height=3em]
\tikzstyle{decision} = [diamond, draw, fill=accent!20, text width=4.5em, text badly centered, inner sep=0pt]
\tikzstyle{cloud} = [ellipse, draw, fill=background, text width=5em, text centered, minimum height=2.5em]

% ------------------------------------------------------------------------------
% BOXES & CALLOUTS (tcolorbox)
% ------------------------------------------------------------------------------
\usepackage{tcolorbox}
\tcbuselibrary{skins, breakable, raster}

% Key Concept Box (Blue)
\newtcolorbox{keypoint}{
    enhanced,
    colback=primary!10,
    colframe=primary,
    fonttitle=\bfseries,
    title=\faLightbulb\ Key Concept,
    attach boxed title to top left={yshift=-2mm, xshift=5mm},
    boxed title style={colback=primary},
    breakable
}

% Example Box (Green)
\newtcolorbox{example}{
    enhanced,
    colback=secondary!10,
    colframe=secondary,
    fonttitle=\bfseries,
    title=\faCode\ Example,
    attach boxed title to top left={yshift=-2mm, xshift=5mm},
    boxed title style={colback=secondary},
    breakable
}

% Warning Box (Red)
\newtcolorbox{warning}{
    enhanced,
    colback=warning!10,
    colframe=warning,
    fonttitle=\bfseries,
    title=\faExclamationTriangle\ Common Pitfall,
    attach boxed title to top left={yshift=-2mm, xshift=5mm},
    boxed title style={colback=warning},
    breakable
}

% Tip Box (Orange)
\newtcolorbox{tip}{
    enhanced,
    colback=accent!10,
    colframe=accent,
    fonttitle=\bfseries,
    title=\faLightbulb\ Pro Tip,
    attach boxed title to top left={yshift=-2mm, xshift=5mm},
    boxed title style={colback=accent},
    breakable
}

% Summary Box (Light gray)
\newtcolorbox{summary}{
    enhanced,
    colback=background,
    colframe=codeblock,
    fonttitle=\bfseries,
    title=\faListUl\ Quick Summary,
    attach boxed title to top left={yshift=-2mm, xshift=5mm},
    boxed title style={colback=codeblock},
    breakable
}

% ------------------------------------------------------------------------------
% ICONS & SYMBOLS (fontawesome5)
% ------------------------------------------------------------------------------
\usepackage{fontawesome5}

% Custom icon commands for consistency
\newcommand{\iconKey}{\textcolor{primary}{\faLightbulb}}
\newcommand{\iconCode}{\textcolor{secondary}{\faCode}}
\newcommand{\iconWarning}{\textcolor{warning}{\faExclamationTriangle}}
\newcommand{\iconTip}{\textcolor{accent}{\faInfoCircle}}
\newcommand{\iconLink}{\textcolor{primary}{\faLink}}
\newcommand{\iconBook}{\textcolor{secondary}{\faBook}}
\newcommand{\iconTarget}{\textcolor{accent}{\faBullseye}}

% ------------------------------------------------------------------------------
% TYPOGRAPHY & FONTS
% ------------------------------------------------------------------------------
\usepackage{lmodern}
\usepackage[T1]{fontenc}
\usepackage[utf8]{inputenc}

% Section styling
\usepackage{titlesec}
\titleformat{\section}{\Large\bfseries\color{primary}}{\thesection}{1em}{}[\titlerule]
\titleformat{\subsection}{\large\bfseries\color{secondary}}{\thesubsection}{1em}{}

% ------------------------------------------------------------------------------
% CODE LISTINGS
% ------------------------------------------------------------------------------
\usepackage{listings}
\lstset{
    basicstyle=\ttfamily\footnotesize\color{white},
    backgroundcolor=\color{codeblock},
    keywordstyle=\color{primary!80},
    commentstyle=\color{secondary!60}\itshape,
    stringstyle=\color{accent!80},
    numbers=left,
    numberstyle=\tiny\color{white!50},
    stepnumber=1,
    numbersep=8pt,
    frame=single,
    rulecolor=\color{codeblock},
    breaklines=true,
    breakatwhitespace=true,
    tabsize=4,
    captionpos=b
}

% Python-specific styling
\lstdefinestyle{python}{
    language=Python,
    morekeywords={self, def, class, import, from, as, return, if, else, elif, for, while, True, False, None}
}

% YAML-specific styling
\lstdefinestyle{yaml}{
    basicstyle=\ttfamily\footnotesize,
    showstringspaces=false,
    commentstyle=\color{secondary!60}\itshape,
    keywordstyle=\color{primary!80}
}

% ------------------------------------------------------------------------------
% HYPERLINKS
% ------------------------------------------------------------------------------
\usepackage{hyperref}
\hypersetup{
    colorlinks=true,
    linkcolor=primary,
    urlcolor=secondary,
    citecolor=accent
}

% ------------------------------------------------------------------------------
% HEADER & FOOTER CUSTOMIZATION
% ------------------------------------------------------------------------------
\fancyhf{}
\fancyhead[L]{\textcolor{primary}{\textbf{DIP-SMC-PSO Podcast Cheatsheet}}}
\fancyhead[R]{\textcolor{secondary}{\episodetitle}}
\fancyfoot[C]{\textcolor{primary}{\thepage}}
\renewcommand{\headrulewidth}{0.5pt}
\renewcommand{\footrulewidth}{0.5pt}

% ------------------------------------------------------------------------------
% CUSTOM COMMANDS
% ------------------------------------------------------------------------------

% Episode title command (to be defined in each episode file)
\newcommand{\episodetitle}{Episode Title}

% Learning objective command
\newcommand{\learningobjective}[1]{%
    \begin{center}
    \begin{tcolorbox}[colback=highlight!30, colframe=accent, width=0.9\textwidth]
    \iconTarget\ \textbf{Learning Objective:} #1
    \end{tcolorbox}
    \end{center}
}

% Quick reference command (for formulas/code snippets)
\newcommand{\quickref}[2]{%
    \begin{tcolorbox}[colback=background, colframe=primary, title=\faBookmark\ #1]
    #2
    \end{tcolorbox}
}

% Resource link command
\newcommand{\resourcelink}[2]{%
    \iconLink\ \href{#1}{\textcolor{secondary}{#2}}
}

% ------------------------------------------------------------------------------
% TITLE PAGE FORMATTING
% ------------------------------------------------------------------------------
\usepackage{afterpage}

\newcommand{\makeepisodetitle}[4]{%
    \begin{titlepage}
    \begin{tikzpicture}[remember picture, overlay]
        % Background gradient
        \fill[primary!20] (current page.south west) rectangle (current page.north east);

        % Title box
        \node[
            fill=white,
            rounded corners=10pt,
            drop shadow,
            text width=0.8\paperwidth,
            align=center
        ] at (current page.center) {
            \Huge\bfseries\color{primary} #1 \\[1em]
            \Large\color{secondary} #2 \\[2em]
            \large\color{codeblock} Part #3 $\cdot$ Duration: #4 \\[1em]
            \normalsize\color{codeblock} \textit{Beginner-Friendly Visual Study Guide}
        };

        % Footer
        \node[
            anchor=south,
            text width=0.9\paperwidth,
            align=center
        ] at (current page.south) {
            \color{primary}\rule{0.8\paperwidth}{0.5pt} \\[0.5em]
            \small\color{codeblock}
            \textbf{Repository:} \url{https://github.com/theSadeQ/dip-smc-pso} \\
            \textbf{Documentation:} academic/paper/presentations/podcasts/episodes/ \\[1em]
        };
    \end{tikzpicture}
    \end{titlepage}
}

% ------------------------------------------------------------------------------
% MATH & EQUATIONS
% ------------------------------------------------------------------------------
\usepackage{amsmath, amssymb, amsthm}
\usepackage{mathtools}

% Equation box (highlighted equations)
\newcommand{\eqbox}[1]{%
    \begin{tcolorbox}[colback=primary!5, colframe=primary, boxrule=1pt]
    \begin{equation}
    #1
    \end{equation}
    \end{tcolorbox}
}

% ------------------------------------------------------------------------------
% TABLES
% ------------------------------------------------------------------------------
\usepackage{booktabs}
\usepackage{array}
\usepackage{multirow}
\usepackage{colortbl}

% Custom table colors
\newcommand{\tableheadcolor}{\rowcolor{primary!30}}

% ------------------------------------------------------------------------------
% END OF PREAMBLE
% ------------------------------------------------------------------------------

% ==============================================================================
% REUSABLE TIKZ COMPONENTS FOR PODCAST CHEATSHEETS
% ==============================================================================
% Purpose: Common diagrams, flowcharts, and visual elements
% Usage: % ==============================================================================
% REUSABLE TIKZ COMPONENTS FOR PODCAST CHEATSHEETS
% ==============================================================================
% Purpose: Common diagrams, flowcharts, and visual elements
% Usage: % ==============================================================================
% REUSABLE TIKZ COMPONENTS FOR PODCAST CHEATSHEETS
% ==============================================================================
% Purpose: Common diagrams, flowcharts, and visual elements
% Usage: \input{../templates/tikz_components.tex} in each episode file
% ==============================================================================

\usepackage{tikz}
\usetikzlibrary{shapes, arrows, positioning, calc, shadows, decorations.pathreplacing, backgrounds, fit}

% ------------------------------------------------------------------------------
% SYSTEM ARCHITECTURE DIAGRAM (7 Controllers + Plant + Optimizer)
% ------------------------------------------------------------------------------
\newcommand{\systemarchitecture}{%
\begin{tikzpicture}[node distance=2cm, auto]
    % Define styles
    \tikzstyle{component} = [rectangle, draw, fill=primary!20, text width=3.5cm, text centered, rounded corners, minimum height=1.2cm, drop shadow]
    \tikzstyle{controller} = [rectangle, draw, fill=secondary!20, text width=2.8cm, text centered, rounded corners, minimum height=1cm]
    \tikzstyle{arrow} = [thick,->,>=stealth, color=primary!70]

    % Central components
    \node[component] (sim) {\textbf{Simulation Engine}\\Runner + Context};
    \node[component, below=1cm of sim] (plant) {\textbf{Plant Model}\\DIP Dynamics};
    \node[component, right=2.5cm of sim] (opt) {\textbf{PSO Optimizer}\\Gain Tuner};

    % Controllers (left side, vertical stack)
    \node[controller, left=2.5cm of sim, yshift=2.5cm] (ctrl1) {Classical SMC};
    \node[controller, below=0.3cm of ctrl1] (ctrl2) {STA-SMC};
    \node[controller, below=0.3cm of ctrl2] (ctrl3) {Adaptive SMC};
    \node[controller, below=0.3cm of ctrl3] (ctrl4) {Hybrid STA};
    \node[controller, below=0.3cm of ctrl4] (ctrl5) {Swing-Up};
    \node[controller, below=0.3cm of ctrl5] (ctrl6) {PID};
    \node[controller, below=0.3cm of ctrl6] (ctrl7) {MPC (Exp)};

    % Controller factory box (grouping)
    \node[draw, dashed, fit=(ctrl1) (ctrl7), fill=secondary!5, rounded corners, inner sep=5pt, label=above:\textbf{Controller Factory}] (factory) {};

    % Monitoring & Visualization
    \node[component, right=2.5cm of plant] (mon) {\textbf{Monitoring}\\Metrics + Latency};
    \node[component, above=0.5cm of mon] (viz) {\textbf{Visualization}\\Plots + Animation};

    % Arrows
    \draw[arrow] (factory) -- node[above] {state} (sim);
    \draw[arrow] (sim) -- node[right] {control} (plant);
    \draw[arrow] (plant) -- (sim);
    \draw[arrow] (opt) -- node[above] {tuned gains} (sim);
    \draw[arrow] (sim) -- (mon);
    \draw[arrow] (mon) -- (viz);

\end{tikzpicture}
}

% ------------------------------------------------------------------------------
% DOUBLE INVERTED PENDULUM PHYSICAL DIAGRAM
% ------------------------------------------------------------------------------
\newcommand{\dipphysical}{%
\begin{tikzpicture}[scale=0.8]
    % Cart
    \draw[fill=primary!30, draw=primary, thick] (-1,0) rectangle (1,0.6);
    \node at (0,0.3) {\small Cart};

    % Wheels
    \draw[fill=codeblock, draw=codeblock] (-0.6,-0.2) circle (0.2);
    \draw[fill=codeblock, draw=codeblock] (0.6,-0.2) circle (0.2);

    % Track
    \draw[thick, <->] (-3,-0.5) -- (3,-0.5);
    \node at (0,-0.8) {\small Track (x-axis)};

    % First pendulum (link 1)
    \draw[thick, color=secondary] (0,0.6) -- ++(60:2.5) coordinate (p1);
    \draw[fill=secondary!30, draw=secondary] (p1) circle (0.3);
    \node[color=secondary] at (p1) {$m_1$};
    \node[color=secondary] at (0.8,1.5) {$\theta_1$};

    % Second pendulum (link 2)
    \draw[thick, color=accent] (p1) -- ++(50:2) coordinate (p2);
    \draw[fill=accent!30, draw=accent] (p2) circle (0.25);
    \node[color=accent] at (p2) {$m_2$};
    \node[color=accent] at ($(p1)+(0.6,1.2)$) {$\theta_2$};

    % Force arrow
    \draw[->, ultra thick, color=warning] (-1.5,0.3) -- (-2.5,0.3);
    \node[color=warning] at (-2,-0.2) {$u$ (Force)};

    % Gravity arrow
    \draw[->, thick, color=codeblock] (3.5,2) -- (3.5,0.5);
    \node at (4,1.2) {$g$};

    % Annotations
    \node[color=secondary] at (-2,2) {Link 1: $\ell_1, m_1, I_1$};
    \node[color=accent] at (-2,1.4) {Link 2: $\ell_2, m_2, I_2$};
    \node[color=primary] at (-2,0.8) {Cart: $M, x$};

\end{tikzpicture}
}

% ------------------------------------------------------------------------------
% PHASE PORTRAIT (Reaching → Sliding → Chattering)
% ------------------------------------------------------------------------------
\newcommand{\phaseportrait}{%
\begin{tikzpicture}[scale=0.9]
    % Axes
    \draw[->] (-3,0) -- (3,0) node[right] {$x_1$ (position)};
    \draw[->] (0,-2.5) -- (0,2.5) node[above] {$x_2$ (velocity)};

    % Sliding surface (diagonal line)
    \draw[thick, color=secondary, dashed] (-2.5,-2) -- (2.5,2);
    \node[color=secondary] at (2.8,2.2) {$s=0$};
    \node[color=secondary] at (2.5,1.3) {\small Sliding Surface};

    % Reaching phase trajectory
    \draw[thick, color=primary, ->] (-2.5,1.5) .. controls (-1.5,0.8) and (-1,0.3) .. (-0.5,0);
    \node[color=primary] at (-2.2,1.8) {\textbf{1. Reaching}};

    % Sliding phase trajectory
    \draw[thick, color=accent, ->] (-0.5,0) -- (0.5,0.5);
    \node[color=accent] at (0,0.8) {\textbf{2. Sliding}};

    % Chattering region (zigzag near origin)
    \draw[thick, color=warning, decoration={zigzag, segment length=2mm, amplitude=0.5mm}, decorate] (0.5,0.5) -- (1.5,1.5);
    \node[color=warning] at (1.8,1.2) {\textbf{3. Chattering}};

    % Equilibrium point
    \fill[color=codeblock] (0,0) circle (3pt);
    \node[below right] at (0,0) {\small Equilibrium};

\end{tikzpicture}
}

% ------------------------------------------------------------------------------
% PSO CONVERGENCE FLOWCHART
% ------------------------------------------------------------------------------
\newcommand{\psoflowchart}{%
\begin{tikzpicture}[node distance=1.5cm]
    \node[block, fill=primary!30] (init) {\textbf{Initialize}\\50 particles\\Random positions};
    \node[process, below of=init] (eval) {\textbf{Evaluate}\\Fitness (cost)};
    \node[process, below of=eval] (update) {\textbf{Update}\\Velocity + Position};
    \node[decision, below of=update, yshift=-0.5cm] (conv) {\textbf{Converged?}};
    \node[block, fill=secondary!30, below of=conv, yshift=-0.5cm] (done) {\textbf{Done}\\Return best gains};

    \draw[arrow] (init) -- (eval);
    \draw[arrow] (eval) -- (update);
    \draw[arrow] (update) -- (conv);
    \draw[arrow] (conv) -- node[right] {Yes} (done);
    \draw[arrow] (conv.west) -- ++(-2,0) |- node[near start, above] {No} (eval.west);

    % Annotations
    \node[right=1.5cm of eval, text width=3cm, align=left, color=codeblock] {
        \small Run simulation\\
        Calculate settling time\\
        Compute cost = $f(gains)$
    };
    \node[right=1.5cm of update, text width=3cm, align=left, color=codeblock] {
        \small $v \leftarrow w \cdot v + c_1 r_1 (p - x) + c_2 r_2 (g - x)$\\
        $x \leftarrow x + v$
    };
\end{tikzpicture}
}

% ------------------------------------------------------------------------------
% CONTROL LOOP TIMING (10ms cycle)
% ------------------------------------------------------------------------------
\newcommand{\controllooptiming}{%
\begin{tikzpicture}[scale=1.2]
    % Timeline
    \draw[thick, ->] (0,0) -- (11,0) node[right] {Time (ms)};

    % Time markers
    \foreach \x in {0,1,2,5,10}
        \draw (\x,0.1) -- (\x,-0.1) node[below] {\x};

    % Sensor read (0-0.1ms)
    \draw[fill=primary!30, draw=primary] (0,0.3) rectangle (0.1,0.8);
    \node[above, color=primary] at (0.05,0.8) {\tiny Sensor};

    % Controller compute (0.1-1.0ms)
    \draw[fill=secondary!30, draw=secondary] (0.1,0.3) rectangle (1.0,0.8);
    \node[above, color=secondary] at (0.55,0.8) {\small Controller};

    % Actuation (1.0-1.1ms)
    \draw[fill=accent!30, draw=accent] (1.0,0.3) rectangle (1.1,0.8);
    \node[above, color=accent] at (1.05,0.8) {\tiny Act};

    % Plant integration (1.1-10ms)
    \draw[fill=warning!20, draw=warning] (1.1,0.3) rectangle (10,0.8);
    \node[above, color=warning] at (5.5,0.8) {\small Plant Integration (RK45)};

    % Annotations
    \draw[dashed] (0,-0.5) -- (0,0.3);
    \draw[dashed] (10,-0.5) -- (10,0.3);
    \draw[<->, thick] (0,-0.5) -- (10,-0.5);
    \node[below] at (5,-0.5) {\textbf{10ms deadline (100 Hz)}};

\end{tikzpicture}
}

% ------------------------------------------------------------------------------
% FACTORY PATTERN (Vending Machine Analogy)
% ------------------------------------------------------------------------------
\newcommand{\factorypattern}{%
\begin{tikzpicture}[scale=0.9]
    % Vending machine frame
    \draw[thick, fill=background, rounded corners] (-0.5,-2) rectangle (3.5,3);
    \node at (1.5,2.5) {\textbf{Controller Factory}};

    % Selection buttons
    \foreach \y/\name/\color in {1.5/Classical/primary, 0.8/STA-SMC/secondary, 0.1/Adaptive/accent, -0.6/Hybrid/warning} {
        \node[circle, draw=\color, fill=\color!30, minimum size=0.6cm] at (0.5,\y) {};
        \node[right, color=codeblock] at (1,\y) {\small \name};
    }

    % Output slot
    \draw[fill=codeblock!20] (0.5,-1.5) rectangle (2.5,-1);
    \node[color=codeblock] at (1.5,-1.8) {\small Output};

    % User and controller object
    \node[left=1cm of 0.5,1.5] (user) {\faUser};
    \node[right=1cm of 2.5,-1.2, block, fill=primary!30] (controller) {Controller\\Object};

    % Arrows
    \draw[->, thick, color=primary] (user) -- (0.3,1.5);
    \draw[->, thick, color=secondary] (2.5,-1.2) -- (controller);

    % Annotation
    \node[below, color=codeblock, text width=5cm, align=center] at (1.5,-2.5) {
        \small Press button $\rightarrow$ Get controller instance
    };

\end{tikzpicture}
}

% ------------------------------------------------------------------------------
% GIT WORKFLOW (Time Machine)
% ------------------------------------------------------------------------------
\newcommand{\gittimeline}{%
\begin{tikzpicture}[scale=0.8]
    % Timeline
    \draw[thick, ->] (0,0) -- (10,0) node[right] {Time};

    % Commits (save points)
    \foreach \x/\label in {1/Oct 2024, 3/Dec 2024, 5/Mar 2025, 7/Nov 2025} {
        \fill[color=primary] (\x,0) circle (3pt);
        \node[above] at (\x,0.3) {\small \label};
        \draw[->, dashed, color=secondary] (\x,-0.3) -- (\x,-1);
        \node[below, color=secondary, text width=1.5cm, align=center] at (\x,-1.3) {\tiny Snapshot};
    }

    % Current position
    \fill[color=warning] (7,0) circle (5pt);
    \node[below=0.5cm, color=warning] at (7,0) {\textbf{You are here}};

    % Time travel arrows
    \draw[<-, ultra thick, color=accent] (7,-2) -- (3,-2);
    \node[below, color=accent] at (5,-2) {\small Time travel to Dec 2024};

    % Annotation
    \node[color=codeblock, text width=8cm, align=center] at (5,-3) {
        \small Git lets you travel to any save point in project history
    };

\end{tikzpicture}
}

% ------------------------------------------------------------------------------
% DIRECTORY STRUCTURE (Zoning Laws)
% ------------------------------------------------------------------------------
\newcommand{\directoryzoning}{%
\begin{tikzpicture}[
    level 1/.style={sibling distance=4cm, level distance=2cm},
    level 2/.style={sibling distance=2cm, level distance=1.5cm},
    every node/.style={draw, rounded corners, font=\small}
]
    \node[fill=primary!20] {Root (Project)}
        child {node[fill=secondary!20] {src/ \faBuilding\\Business District}
            child {node[fill=secondary!10] {controllers/}}
            child {node[fill=secondary!10] {plant/}}
        }
        child {node[fill=accent!20] {scripts/ \faWrench\\Workshop}
            child {node[fill=accent!10] {benchmarks/}}
            child {node[fill=accent!10] {testing/}}
        }
        child {node[fill=warning!20] {tests/ \faCheckCircle\\QA District}
            child {node[fill=warning!10] {unit/}}
            child {node[fill=warning!10] {integration/}}
        };
\end{tikzpicture}
}

% ==============================================================================
% END OF TIKZ COMPONENTS
% ==============================================================================
 in each episode file
% ==============================================================================

\usepackage{tikz}
\usetikzlibrary{shapes, arrows, positioning, calc, shadows, decorations.pathreplacing, backgrounds, fit}

% ------------------------------------------------------------------------------
% SYSTEM ARCHITECTURE DIAGRAM (7 Controllers + Plant + Optimizer)
% ------------------------------------------------------------------------------
\newcommand{\systemarchitecture}{%
\begin{tikzpicture}[node distance=2cm, auto]
    % Define styles
    \tikzstyle{component} = [rectangle, draw, fill=primary!20, text width=3.5cm, text centered, rounded corners, minimum height=1.2cm, drop shadow]
    \tikzstyle{controller} = [rectangle, draw, fill=secondary!20, text width=2.8cm, text centered, rounded corners, minimum height=1cm]
    \tikzstyle{arrow} = [thick,->,>=stealth, color=primary!70]

    % Central components
    \node[component] (sim) {\textbf{Simulation Engine}\\Runner + Context};
    \node[component, below=1cm of sim] (plant) {\textbf{Plant Model}\\DIP Dynamics};
    \node[component, right=2.5cm of sim] (opt) {\textbf{PSO Optimizer}\\Gain Tuner};

    % Controllers (left side, vertical stack)
    \node[controller, left=2.5cm of sim, yshift=2.5cm] (ctrl1) {Classical SMC};
    \node[controller, below=0.3cm of ctrl1] (ctrl2) {STA-SMC};
    \node[controller, below=0.3cm of ctrl2] (ctrl3) {Adaptive SMC};
    \node[controller, below=0.3cm of ctrl3] (ctrl4) {Hybrid STA};
    \node[controller, below=0.3cm of ctrl4] (ctrl5) {Swing-Up};
    \node[controller, below=0.3cm of ctrl5] (ctrl6) {PID};
    \node[controller, below=0.3cm of ctrl6] (ctrl7) {MPC (Exp)};

    % Controller factory box (grouping)
    \node[draw, dashed, fit=(ctrl1) (ctrl7), fill=secondary!5, rounded corners, inner sep=5pt, label=above:\textbf{Controller Factory}] (factory) {};

    % Monitoring & Visualization
    \node[component, right=2.5cm of plant] (mon) {\textbf{Monitoring}\\Metrics + Latency};
    \node[component, above=0.5cm of mon] (viz) {\textbf{Visualization}\\Plots + Animation};

    % Arrows
    \draw[arrow] (factory) -- node[above] {state} (sim);
    \draw[arrow] (sim) -- node[right] {control} (plant);
    \draw[arrow] (plant) -- (sim);
    \draw[arrow] (opt) -- node[above] {tuned gains} (sim);
    \draw[arrow] (sim) -- (mon);
    \draw[arrow] (mon) -- (viz);

\end{tikzpicture}
}

% ------------------------------------------------------------------------------
% DOUBLE INVERTED PENDULUM PHYSICAL DIAGRAM
% ------------------------------------------------------------------------------
\newcommand{\dipphysical}{%
\begin{tikzpicture}[scale=0.8]
    % Cart
    \draw[fill=primary!30, draw=primary, thick] (-1,0) rectangle (1,0.6);
    \node at (0,0.3) {\small Cart};

    % Wheels
    \draw[fill=codeblock, draw=codeblock] (-0.6,-0.2) circle (0.2);
    \draw[fill=codeblock, draw=codeblock] (0.6,-0.2) circle (0.2);

    % Track
    \draw[thick, <->] (-3,-0.5) -- (3,-0.5);
    \node at (0,-0.8) {\small Track (x-axis)};

    % First pendulum (link 1)
    \draw[thick, color=secondary] (0,0.6) -- ++(60:2.5) coordinate (p1);
    \draw[fill=secondary!30, draw=secondary] (p1) circle (0.3);
    \node[color=secondary] at (p1) {$m_1$};
    \node[color=secondary] at (0.8,1.5) {$\theta_1$};

    % Second pendulum (link 2)
    \draw[thick, color=accent] (p1) -- ++(50:2) coordinate (p2);
    \draw[fill=accent!30, draw=accent] (p2) circle (0.25);
    \node[color=accent] at (p2) {$m_2$};
    \node[color=accent] at ($(p1)+(0.6,1.2)$) {$\theta_2$};

    % Force arrow
    \draw[->, ultra thick, color=warning] (-1.5,0.3) -- (-2.5,0.3);
    \node[color=warning] at (-2,-0.2) {$u$ (Force)};

    % Gravity arrow
    \draw[->, thick, color=codeblock] (3.5,2) -- (3.5,0.5);
    \node at (4,1.2) {$g$};

    % Annotations
    \node[color=secondary] at (-2,2) {Link 1: $\ell_1, m_1, I_1$};
    \node[color=accent] at (-2,1.4) {Link 2: $\ell_2, m_2, I_2$};
    \node[color=primary] at (-2,0.8) {Cart: $M, x$};

\end{tikzpicture}
}

% ------------------------------------------------------------------------------
% PHASE PORTRAIT (Reaching → Sliding → Chattering)
% ------------------------------------------------------------------------------
\newcommand{\phaseportrait}{%
\begin{tikzpicture}[scale=0.9]
    % Axes
    \draw[->] (-3,0) -- (3,0) node[right] {$x_1$ (position)};
    \draw[->] (0,-2.5) -- (0,2.5) node[above] {$x_2$ (velocity)};

    % Sliding surface (diagonal line)
    \draw[thick, color=secondary, dashed] (-2.5,-2) -- (2.5,2);
    \node[color=secondary] at (2.8,2.2) {$s=0$};
    \node[color=secondary] at (2.5,1.3) {\small Sliding Surface};

    % Reaching phase trajectory
    \draw[thick, color=primary, ->] (-2.5,1.5) .. controls (-1.5,0.8) and (-1,0.3) .. (-0.5,0);
    \node[color=primary] at (-2.2,1.8) {\textbf{1. Reaching}};

    % Sliding phase trajectory
    \draw[thick, color=accent, ->] (-0.5,0) -- (0.5,0.5);
    \node[color=accent] at (0,0.8) {\textbf{2. Sliding}};

    % Chattering region (zigzag near origin)
    \draw[thick, color=warning, decoration={zigzag, segment length=2mm, amplitude=0.5mm}, decorate] (0.5,0.5) -- (1.5,1.5);
    \node[color=warning] at (1.8,1.2) {\textbf{3. Chattering}};

    % Equilibrium point
    \fill[color=codeblock] (0,0) circle (3pt);
    \node[below right] at (0,0) {\small Equilibrium};

\end{tikzpicture}
}

% ------------------------------------------------------------------------------
% PSO CONVERGENCE FLOWCHART
% ------------------------------------------------------------------------------
\newcommand{\psoflowchart}{%
\begin{tikzpicture}[node distance=1.5cm]
    \node[block, fill=primary!30] (init) {\textbf{Initialize}\\50 particles\\Random positions};
    \node[process, below of=init] (eval) {\textbf{Evaluate}\\Fitness (cost)};
    \node[process, below of=eval] (update) {\textbf{Update}\\Velocity + Position};
    \node[decision, below of=update, yshift=-0.5cm] (conv) {\textbf{Converged?}};
    \node[block, fill=secondary!30, below of=conv, yshift=-0.5cm] (done) {\textbf{Done}\\Return best gains};

    \draw[arrow] (init) -- (eval);
    \draw[arrow] (eval) -- (update);
    \draw[arrow] (update) -- (conv);
    \draw[arrow] (conv) -- node[right] {Yes} (done);
    \draw[arrow] (conv.west) -- ++(-2,0) |- node[near start, above] {No} (eval.west);

    % Annotations
    \node[right=1.5cm of eval, text width=3cm, align=left, color=codeblock] {
        \small Run simulation\\
        Calculate settling time\\
        Compute cost = $f(gains)$
    };
    \node[right=1.5cm of update, text width=3cm, align=left, color=codeblock] {
        \small $v \leftarrow w \cdot v + c_1 r_1 (p - x) + c_2 r_2 (g - x)$\\
        $x \leftarrow x + v$
    };
\end{tikzpicture}
}

% ------------------------------------------------------------------------------
% CONTROL LOOP TIMING (10ms cycle)
% ------------------------------------------------------------------------------
\newcommand{\controllooptiming}{%
\begin{tikzpicture}[scale=1.2]
    % Timeline
    \draw[thick, ->] (0,0) -- (11,0) node[right] {Time (ms)};

    % Time markers
    \foreach \x in {0,1,2,5,10}
        \draw (\x,0.1) -- (\x,-0.1) node[below] {\x};

    % Sensor read (0-0.1ms)
    \draw[fill=primary!30, draw=primary] (0,0.3) rectangle (0.1,0.8);
    \node[above, color=primary] at (0.05,0.8) {\tiny Sensor};

    % Controller compute (0.1-1.0ms)
    \draw[fill=secondary!30, draw=secondary] (0.1,0.3) rectangle (1.0,0.8);
    \node[above, color=secondary] at (0.55,0.8) {\small Controller};

    % Actuation (1.0-1.1ms)
    \draw[fill=accent!30, draw=accent] (1.0,0.3) rectangle (1.1,0.8);
    \node[above, color=accent] at (1.05,0.8) {\tiny Act};

    % Plant integration (1.1-10ms)
    \draw[fill=warning!20, draw=warning] (1.1,0.3) rectangle (10,0.8);
    \node[above, color=warning] at (5.5,0.8) {\small Plant Integration (RK45)};

    % Annotations
    \draw[dashed] (0,-0.5) -- (0,0.3);
    \draw[dashed] (10,-0.5) -- (10,0.3);
    \draw[<->, thick] (0,-0.5) -- (10,-0.5);
    \node[below] at (5,-0.5) {\textbf{10ms deadline (100 Hz)}};

\end{tikzpicture}
}

% ------------------------------------------------------------------------------
% FACTORY PATTERN (Vending Machine Analogy)
% ------------------------------------------------------------------------------
\newcommand{\factorypattern}{%
\begin{tikzpicture}[scale=0.9]
    % Vending machine frame
    \draw[thick, fill=background, rounded corners] (-0.5,-2) rectangle (3.5,3);
    \node at (1.5,2.5) {\textbf{Controller Factory}};

    % Selection buttons
    \foreach \y/\name/\color in {1.5/Classical/primary, 0.8/STA-SMC/secondary, 0.1/Adaptive/accent, -0.6/Hybrid/warning} {
        \node[circle, draw=\color, fill=\color!30, minimum size=0.6cm] at (0.5,\y) {};
        \node[right, color=codeblock] at (1,\y) {\small \name};
    }

    % Output slot
    \draw[fill=codeblock!20] (0.5,-1.5) rectangle (2.5,-1);
    \node[color=codeblock] at (1.5,-1.8) {\small Output};

    % User and controller object
    \node[left=1cm of 0.5,1.5] (user) {\faUser};
    \node[right=1cm of 2.5,-1.2, block, fill=primary!30] (controller) {Controller\\Object};

    % Arrows
    \draw[->, thick, color=primary] (user) -- (0.3,1.5);
    \draw[->, thick, color=secondary] (2.5,-1.2) -- (controller);

    % Annotation
    \node[below, color=codeblock, text width=5cm, align=center] at (1.5,-2.5) {
        \small Press button $\rightarrow$ Get controller instance
    };

\end{tikzpicture}
}

% ------------------------------------------------------------------------------
% GIT WORKFLOW (Time Machine)
% ------------------------------------------------------------------------------
\newcommand{\gittimeline}{%
\begin{tikzpicture}[scale=0.8]
    % Timeline
    \draw[thick, ->] (0,0) -- (10,0) node[right] {Time};

    % Commits (save points)
    \foreach \x/\label in {1/Oct 2024, 3/Dec 2024, 5/Mar 2025, 7/Nov 2025} {
        \fill[color=primary] (\x,0) circle (3pt);
        \node[above] at (\x,0.3) {\small \label};
        \draw[->, dashed, color=secondary] (\x,-0.3) -- (\x,-1);
        \node[below, color=secondary, text width=1.5cm, align=center] at (\x,-1.3) {\tiny Snapshot};
    }

    % Current position
    \fill[color=warning] (7,0) circle (5pt);
    \node[below=0.5cm, color=warning] at (7,0) {\textbf{You are here}};

    % Time travel arrows
    \draw[<-, ultra thick, color=accent] (7,-2) -- (3,-2);
    \node[below, color=accent] at (5,-2) {\small Time travel to Dec 2024};

    % Annotation
    \node[color=codeblock, text width=8cm, align=center] at (5,-3) {
        \small Git lets you travel to any save point in project history
    };

\end{tikzpicture}
}

% ------------------------------------------------------------------------------
% DIRECTORY STRUCTURE (Zoning Laws)
% ------------------------------------------------------------------------------
\newcommand{\directoryzoning}{%
\begin{tikzpicture}[
    level 1/.style={sibling distance=4cm, level distance=2cm},
    level 2/.style={sibling distance=2cm, level distance=1.5cm},
    every node/.style={draw, rounded corners, font=\small}
]
    \node[fill=primary!20] {Root (Project)}
        child {node[fill=secondary!20] {src/ \faBuilding\\Business District}
            child {node[fill=secondary!10] {controllers/}}
            child {node[fill=secondary!10] {plant/}}
        }
        child {node[fill=accent!20] {scripts/ \faWrench\\Workshop}
            child {node[fill=accent!10] {benchmarks/}}
            child {node[fill=accent!10] {testing/}}
        }
        child {node[fill=warning!20] {tests/ \faCheckCircle\\QA District}
            child {node[fill=warning!10] {unit/}}
            child {node[fill=warning!10] {integration/}}
        };
\end{tikzpicture}
}

% ==============================================================================
% END OF TIKZ COMPONENTS
% ==============================================================================
 in each episode file
% ==============================================================================

\usepackage{tikz}
\usetikzlibrary{shapes, arrows, positioning, calc, shadows, decorations.pathreplacing, backgrounds, fit}

% ------------------------------------------------------------------------------
% SYSTEM ARCHITECTURE DIAGRAM (7 Controllers + Plant + Optimizer)
% ------------------------------------------------------------------------------
\newcommand{\systemarchitecture}{%
\begin{tikzpicture}[node distance=2cm, auto]
    % Define styles
    \tikzstyle{component} = [rectangle, draw, fill=primary!20, text width=3.5cm, text centered, rounded corners, minimum height=1.2cm, drop shadow]
    \tikzstyle{controller} = [rectangle, draw, fill=secondary!20, text width=2.8cm, text centered, rounded corners, minimum height=1cm]
    \tikzstyle{arrow} = [thick,->,>=stealth, color=primary!70]

    % Central components
    \node[component] (sim) {\textbf{Simulation Engine}\\Runner + Context};
    \node[component, below=1cm of sim] (plant) {\textbf{Plant Model}\\DIP Dynamics};
    \node[component, right=2.5cm of sim] (opt) {\textbf{PSO Optimizer}\\Gain Tuner};

    % Controllers (left side, vertical stack)
    \node[controller, left=2.5cm of sim, yshift=2.5cm] (ctrl1) {Classical SMC};
    \node[controller, below=0.3cm of ctrl1] (ctrl2) {STA-SMC};
    \node[controller, below=0.3cm of ctrl2] (ctrl3) {Adaptive SMC};
    \node[controller, below=0.3cm of ctrl3] (ctrl4) {Hybrid STA};
    \node[controller, below=0.3cm of ctrl4] (ctrl5) {Swing-Up};
    \node[controller, below=0.3cm of ctrl5] (ctrl6) {PID};
    \node[controller, below=0.3cm of ctrl6] (ctrl7) {MPC (Exp)};

    % Controller factory box (grouping)
    \node[draw, dashed, fit=(ctrl1) (ctrl7), fill=secondary!5, rounded corners, inner sep=5pt, label=above:\textbf{Controller Factory}] (factory) {};

    % Monitoring & Visualization
    \node[component, right=2.5cm of plant] (mon) {\textbf{Monitoring}\\Metrics + Latency};
    \node[component, above=0.5cm of mon] (viz) {\textbf{Visualization}\\Plots + Animation};

    % Arrows
    \draw[arrow] (factory) -- node[above] {state} (sim);
    \draw[arrow] (sim) -- node[right] {control} (plant);
    \draw[arrow] (plant) -- (sim);
    \draw[arrow] (opt) -- node[above] {tuned gains} (sim);
    \draw[arrow] (sim) -- (mon);
    \draw[arrow] (mon) -- (viz);

\end{tikzpicture}
}

% ------------------------------------------------------------------------------
% DOUBLE INVERTED PENDULUM PHYSICAL DIAGRAM
% ------------------------------------------------------------------------------
\newcommand{\dipphysical}{%
\begin{tikzpicture}[scale=0.8]
    % Cart
    \draw[fill=primary!30, draw=primary, thick] (-1,0) rectangle (1,0.6);
    \node at (0,0.3) {\small Cart};

    % Wheels
    \draw[fill=codeblock, draw=codeblock] (-0.6,-0.2) circle (0.2);
    \draw[fill=codeblock, draw=codeblock] (0.6,-0.2) circle (0.2);

    % Track
    \draw[thick, <->] (-3,-0.5) -- (3,-0.5);
    \node at (0,-0.8) {\small Track (x-axis)};

    % First pendulum (link 1)
    \draw[thick, color=secondary] (0,0.6) -- ++(60:2.5) coordinate (p1);
    \draw[fill=secondary!30, draw=secondary] (p1) circle (0.3);
    \node[color=secondary] at (p1) {$m_1$};
    \node[color=secondary] at (0.8,1.5) {$\theta_1$};

    % Second pendulum (link 2)
    \draw[thick, color=accent] (p1) -- ++(50:2) coordinate (p2);
    \draw[fill=accent!30, draw=accent] (p2) circle (0.25);
    \node[color=accent] at (p2) {$m_2$};
    \node[color=accent] at ($(p1)+(0.6,1.2)$) {$\theta_2$};

    % Force arrow
    \draw[->, ultra thick, color=warning] (-1.5,0.3) -- (-2.5,0.3);
    \node[color=warning] at (-2,-0.2) {$u$ (Force)};

    % Gravity arrow
    \draw[->, thick, color=codeblock] (3.5,2) -- (3.5,0.5);
    \node at (4,1.2) {$g$};

    % Annotations
    \node[color=secondary] at (-2,2) {Link 1: $\ell_1, m_1, I_1$};
    \node[color=accent] at (-2,1.4) {Link 2: $\ell_2, m_2, I_2$};
    \node[color=primary] at (-2,0.8) {Cart: $M, x$};

\end{tikzpicture}
}

% ------------------------------------------------------------------------------
% PHASE PORTRAIT (Reaching → Sliding → Chattering)
% ------------------------------------------------------------------------------
\newcommand{\phaseportrait}{%
\begin{tikzpicture}[scale=0.9]
    % Axes
    \draw[->] (-3,0) -- (3,0) node[right] {$x_1$ (position)};
    \draw[->] (0,-2.5) -- (0,2.5) node[above] {$x_2$ (velocity)};

    % Sliding surface (diagonal line)
    \draw[thick, color=secondary, dashed] (-2.5,-2) -- (2.5,2);
    \node[color=secondary] at (2.8,2.2) {$s=0$};
    \node[color=secondary] at (2.5,1.3) {\small Sliding Surface};

    % Reaching phase trajectory
    \draw[thick, color=primary, ->] (-2.5,1.5) .. controls (-1.5,0.8) and (-1,0.3) .. (-0.5,0);
    \node[color=primary] at (-2.2,1.8) {\textbf{1. Reaching}};

    % Sliding phase trajectory
    \draw[thick, color=accent, ->] (-0.5,0) -- (0.5,0.5);
    \node[color=accent] at (0,0.8) {\textbf{2. Sliding}};

    % Chattering region (zigzag near origin)
    \draw[thick, color=warning, decoration={zigzag, segment length=2mm, amplitude=0.5mm}, decorate] (0.5,0.5) -- (1.5,1.5);
    \node[color=warning] at (1.8,1.2) {\textbf{3. Chattering}};

    % Equilibrium point
    \fill[color=codeblock] (0,0) circle (3pt);
    \node[below right] at (0,0) {\small Equilibrium};

\end{tikzpicture}
}

% ------------------------------------------------------------------------------
% PSO CONVERGENCE FLOWCHART
% ------------------------------------------------------------------------------
\newcommand{\psoflowchart}{%
\begin{tikzpicture}[node distance=1.5cm]
    \node[block, fill=primary!30] (init) {\textbf{Initialize}\\50 particles\\Random positions};
    \node[process, below of=init] (eval) {\textbf{Evaluate}\\Fitness (cost)};
    \node[process, below of=eval] (update) {\textbf{Update}\\Velocity + Position};
    \node[decision, below of=update, yshift=-0.5cm] (conv) {\textbf{Converged?}};
    \node[block, fill=secondary!30, below of=conv, yshift=-0.5cm] (done) {\textbf{Done}\\Return best gains};

    \draw[arrow] (init) -- (eval);
    \draw[arrow] (eval) -- (update);
    \draw[arrow] (update) -- (conv);
    \draw[arrow] (conv) -- node[right] {Yes} (done);
    \draw[arrow] (conv.west) -- ++(-2,0) |- node[near start, above] {No} (eval.west);

    % Annotations
    \node[right=1.5cm of eval, text width=3cm, align=left, color=codeblock] {
        \small Run simulation\\
        Calculate settling time\\
        Compute cost = $f(gains)$
    };
    \node[right=1.5cm of update, text width=3cm, align=left, color=codeblock] {
        \small $v \leftarrow w \cdot v + c_1 r_1 (p - x) + c_2 r_2 (g - x)$\\
        $x \leftarrow x + v$
    };
\end{tikzpicture}
}

% ------------------------------------------------------------------------------
% CONTROL LOOP TIMING (10ms cycle)
% ------------------------------------------------------------------------------
\newcommand{\controllooptiming}{%
\begin{tikzpicture}[scale=1.2]
    % Timeline
    \draw[thick, ->] (0,0) -- (11,0) node[right] {Time (ms)};

    % Time markers
    \foreach \x in {0,1,2,5,10}
        \draw (\x,0.1) -- (\x,-0.1) node[below] {\x};

    % Sensor read (0-0.1ms)
    \draw[fill=primary!30, draw=primary] (0,0.3) rectangle (0.1,0.8);
    \node[above, color=primary] at (0.05,0.8) {\tiny Sensor};

    % Controller compute (0.1-1.0ms)
    \draw[fill=secondary!30, draw=secondary] (0.1,0.3) rectangle (1.0,0.8);
    \node[above, color=secondary] at (0.55,0.8) {\small Controller};

    % Actuation (1.0-1.1ms)
    \draw[fill=accent!30, draw=accent] (1.0,0.3) rectangle (1.1,0.8);
    \node[above, color=accent] at (1.05,0.8) {\tiny Act};

    % Plant integration (1.1-10ms)
    \draw[fill=warning!20, draw=warning] (1.1,0.3) rectangle (10,0.8);
    \node[above, color=warning] at (5.5,0.8) {\small Plant Integration (RK45)};

    % Annotations
    \draw[dashed] (0,-0.5) -- (0,0.3);
    \draw[dashed] (10,-0.5) -- (10,0.3);
    \draw[<->, thick] (0,-0.5) -- (10,-0.5);
    \node[below] at (5,-0.5) {\textbf{10ms deadline (100 Hz)}};

\end{tikzpicture}
}

% ------------------------------------------------------------------------------
% FACTORY PATTERN (Vending Machine Analogy)
% ------------------------------------------------------------------------------
\newcommand{\factorypattern}{%
\begin{tikzpicture}[scale=0.9]
    % Vending machine frame
    \draw[thick, fill=background, rounded corners] (-0.5,-2) rectangle (3.5,3);
    \node at (1.5,2.5) {\textbf{Controller Factory}};

    % Selection buttons
    \foreach \y/\name/\color in {1.5/Classical/primary, 0.8/STA-SMC/secondary, 0.1/Adaptive/accent, -0.6/Hybrid/warning} {
        \node[circle, draw=\color, fill=\color!30, minimum size=0.6cm] at (0.5,\y) {};
        \node[right, color=codeblock] at (1,\y) {\small \name};
    }

    % Output slot
    \draw[fill=codeblock!20] (0.5,-1.5) rectangle (2.5,-1);
    \node[color=codeblock] at (1.5,-1.8) {\small Output};

    % User and controller object
    \node[left=1cm of 0.5,1.5] (user) {\faUser};
    \node[right=1cm of 2.5,-1.2, block, fill=primary!30] (controller) {Controller\\Object};

    % Arrows
    \draw[->, thick, color=primary] (user) -- (0.3,1.5);
    \draw[->, thick, color=secondary] (2.5,-1.2) -- (controller);

    % Annotation
    \node[below, color=codeblock, text width=5cm, align=center] at (1.5,-2.5) {
        \small Press button $\rightarrow$ Get controller instance
    };

\end{tikzpicture}
}

% ------------------------------------------------------------------------------
% GIT WORKFLOW (Time Machine)
% ------------------------------------------------------------------------------
\newcommand{\gittimeline}{%
\begin{tikzpicture}[scale=0.8]
    % Timeline
    \draw[thick, ->] (0,0) -- (10,0) node[right] {Time};

    % Commits (save points)
    \foreach \x/\label in {1/Oct 2024, 3/Dec 2024, 5/Mar 2025, 7/Nov 2025} {
        \fill[color=primary] (\x,0) circle (3pt);
        \node[above] at (\x,0.3) {\small \label};
        \draw[->, dashed, color=secondary] (\x,-0.3) -- (\x,-1);
        \node[below, color=secondary, text width=1.5cm, align=center] at (\x,-1.3) {\tiny Snapshot};
    }

    % Current position
    \fill[color=warning] (7,0) circle (5pt);
    \node[below=0.5cm, color=warning] at (7,0) {\textbf{You are here}};

    % Time travel arrows
    \draw[<-, ultra thick, color=accent] (7,-2) -- (3,-2);
    \node[below, color=accent] at (5,-2) {\small Time travel to Dec 2024};

    % Annotation
    \node[color=codeblock, text width=8cm, align=center] at (5,-3) {
        \small Git lets you travel to any save point in project history
    };

\end{tikzpicture}
}

% ------------------------------------------------------------------------------
% DIRECTORY STRUCTURE (Zoning Laws)
% ------------------------------------------------------------------------------
\newcommand{\directoryzoning}{%
\begin{tikzpicture}[
    level 1/.style={sibling distance=4cm, level distance=2cm},
    level 2/.style={sibling distance=2cm, level distance=1.5cm},
    every node/.style={draw, rounded corners, font=\small}
]
    \node[fill=primary!20] {Root (Project)}
        child {node[fill=secondary!20] {src/ \faBuilding\\Business District}
            child {node[fill=secondary!10] {controllers/}}
            child {node[fill=secondary!10] {plant/}}
        }
        child {node[fill=accent!20] {scripts/ \faWrench\\Workshop}
            child {node[fill=accent!10] {benchmarks/}}
            child {node[fill=accent!10] {testing/}}
        }
        child {node[fill=warning!20] {tests/ \faCheckCircle\\QA District}
            child {node[fill=warning!10] {unit/}}
            child {node[fill=warning!10] {integration/}}
        };
\end{tikzpicture}
}

% ==============================================================================
% END OF TIKZ COMPONENTS
% ==============================================================================


% Episode-specific title
\renewcommand{\episodetitle}{E009: Educational Materials}

\begin{document}

% ==============================================================================
% TITLE PAGE
% ==============================================================================
\makeepisodetitle{E009: Educational Materials \& Learning Paths}{From Zero to Research: 5 Paths, 44 Episodes, 985 Files}{2}{15-20 minutes}

% ==============================================================================
% PAGE 1: THE EDUCATIONAL CHALLENGE
% ==============================================================================
\learningobjective{Understand the 5 learning paths (Path 0-4), beginner roadmap (125-150 hrs), NotebookLM podcast series (44 episodes), documentation navigation (985 files), and educational philosophy}

\section*{The Educational Challenge: Audience Diversity}

\begin{keypoint}
\textbf{Question:} How do you teach DIP-SMC to complete beginners AND advanced researchers?

\textbf{Answer:} Five learning paths for five user types, each with tailored documentation
\end{keypoint}

\subsection*{Five User Personas}

\begin{tcolorbox}[colback=primary!5, colframe=primary, title=\faUsers\ User Types \& Entry Points]
\textbf{The Student (Path 0):}
\begin{itemize}
    \item \textbf{Background:} Zero coding/control theory knowledge
    \item \textbf{Need:} ~125 hours of prerequisite study (about a semester)
    \item \textbf{Entry:} \texttt{.ai\_workspace/edu/beginner-roadmap.md}
\end{itemize}

\textbf{The Experimenter (Path 1):}
\begin{itemize}
    \item \textbf{Background:} Knows Python, wants quick results
    \item \textbf{Need:} Run first simulation in 1-2 hours
    \item \textbf{Entry:} \texttt{docs/guides/tutorial\_01\_first\_simulation.md}
\end{itemize}

\textbf{The Engineer (Path 2):}
\begin{itemize}
    \item \textbf{Background:} Understands basic control theory
    \item \textbf{Need:} Compare controllers, tune gains (4-8 hours)
    \item \textbf{Entry:} Tutorials 02-03 (comparison \& PSO)
\end{itemize}

\textbf{The Researcher (Path 3):}
\begin{itemize}
    \item \textbf{Background:} Wants to implement custom algorithms
    \item \textbf{Need:} Research workflows, experiment design (8-12 hours)
    \item \textbf{Entry:} Tutorials 04-05 (custom controllers \& research)
\end{itemize}

\textbf{The Expert (Path 4):}
\begin{itemize}
    \item \textbf{Background:} Experienced developer, wants to contribute
    \item \textbf{Need:} Architectural mastery (12+ hours)
    \item \textbf{Entry:} \texttt{docs/architecture/} + source code
\end{itemize}
\end{tcolorbox}

\begin{warning}
\textbf{Common Mistake:} Forcing everyone through the same linear path

\textbf{Reality:} Learners arrive with different backgrounds - need multiple entry points!
\end{warning}

% ==============================================================================
% PAGE 2: PATH 0 - BEGINNER ROADMAP
% ==============================================================================
\newpage

\section*{Path 0: Complete Beginner Roadmap}

\begin{keypoint}
\textbf{Target:} ZERO prerequisites (never coded, never seen physics/control theory)

\textbf{Duration:} 125-150 hours over 4-6 months (about a semester)

\textbf{Location:} \texttt{.ai\_workspace/edu/beginner-roadmap.md}
\end{keypoint}

\subsection*{Five Phases: Foundation to Mastery}

\begin{tcolorbox}[colback=secondary!10, colframe=secondary, title=\faGraduationCap\ Phase 1: Foundations (40 hours)]
\textbf{Module 1: Computing Basics (10 hours)}
\begin{itemize}
    \item What is an OS? File systems, command line fundamentals
    \item Text editors vs IDEs
\end{itemize}

\textbf{Module 2: Python Programming (15 hours)}
\begin{itemize}
    \item Variables, loops, functions, data structures
    \item NumPy arrays and slicing
    \item Matplotlib for plotting
\end{itemize}

\textbf{Module 3: Physics Review (10 hours)}
\begin{itemize}
    \item Newton's laws, force and torque
    \item Energy and momentum
\end{itemize}

\textbf{Module 4: Mathematics (5 hours)}
\begin{itemize}
    \item Linear algebra: vectors, matrices, dot products
    \item Trigonometry for angles
\end{itemize}
\end{tcolorbox}

\begin{tcolorbox}[colback=accent!10, colframe=accent, title=\faBook\ Phase 2: Core Concepts (30 hours)]
\textbf{Module 1: Control Theory Fundamentals (15 hours)}
\begin{itemize}
    \item What is a control system? Open-loop vs closed-loop
    \item PID control, stability concepts
\end{itemize}

\textbf{Module 2: Sliding Mode Control (10 hours)}
\begin{itemize}
    \item Why SMC? Reaching law and sliding surface design
    \item Chattering problem, super-twisting algorithm
\end{itemize}

\textbf{Module 3: Optimization Basics (5 hours)}
\begin{itemize}
    \item What is optimization? Cost functions and constraints
    \item Introduction to PSO (particle swarms, global vs local search)
\end{itemize}
\end{tcolorbox}

\subsection*{Phases 3-5: Practice to Mastery}

\begin{multicols}{2}

\textbf{Phase 3: Hands-On Practice (25 hours)}
\begin{itemize}
    \item Run first DIP simulation
    \item Experiment with controller parameters
    \item Visualize results, understand metrics
\end{itemize}

\textbf{Phase 4: Advancing Skills (30 hours)}
\begin{itemize}
    \item Advanced Python (OOP, typing, testing)
    \item Reading source code
    \item Understanding simulation engine
\end{itemize}

\columnbreak

\textbf{Phase 5: Mastery \& Specialization (25-75 hours, branching)}

\textbf{Track A: Research}
\begin{itemize}
    \item Paper reading, experiment design
\end{itemize}

\textbf{Track B: Development}
\begin{itemize}
    \item Custom controllers, new features
\end{itemize}

\textbf{Track C: Deployment}
\begin{itemize}
    \item Embedded systems, HIL
\end{itemize}

\end{multicols}

\begin{tip}
\textbf{Status:} Phases 1-2 complete (~2,000 lines), Phases 3-5 outlined (500 lines)

\textbf{Graduation:} Phase 5 completion → Path 1 (Tutorial 01)
\end{tip}

% ==============================================================================
% PAGE 3: PATHS 1-4 & TUTORIALS
% ==============================================================================
\newpage

\section*{Learning Paths 1-4: Progressive Mastery}

\begin{tcolorbox}[colback=primary!5, colframe=primary, title=\faRoad\ Four Progressive Paths]
\textbf{Path 1: Quick Start (1-2 hours)}
\begin{itemize}
    \item \textbf{Target:} Knows Python, wants immediate results
    \item \textbf{Material:} Tutorial 01 - First Simulation
    \item \textbf{Steps:} Install → Run \texttt{python simulate.py --ctrl classical\_smc --plot} → Interpret results
    \item \textbf{Outcome:} See DIP stabilize in 10 seconds, high-level understanding
\end{itemize}

\textbf{Path 2: Intermediate (4-8 hours)}
\begin{itemize}
    \item \textbf{Target:} Understands basic control theory, wants to compare/tune
    \item \textbf{Material:} Tutorials 02-03
    \item Tutorial 02: Controller comparison (7 controllers × 4 metrics, understand tradeoffs)
    \item Tutorial 03: PSO optimization (define cost function, run PSO, validate)
\end{itemize}

\textbf{Path 3: Advanced (8-12 hours)}
\begin{itemize}
    \item \textbf{Target:} Wants to implement custom controllers or run research
    \item \textbf{Material:} Tutorials 04-05
    \item Tutorial 04: Custom controller (extend base class, factory integration, tests)
    \item Tutorial 05: Research workflows (reproduce MT-5/MT-8/LT-7, experimental design, publication figures)
\end{itemize}

\textbf{Path 4: Expert (12+ hours)}
\begin{itemize}
    \item \textbf{Target:} Understand architecture, contribute to project
    \item \textbf{Material:} Source code deep dive, \texttt{docs/architecture/}
    \item Design patterns, testing standards, contribution guidelines
\end{itemize}
\end{tcolorbox}

\subsection*{Tutorial System: Five Tutorials}

\begin{tabular}{p{2.5cm}|p{1.5cm}|p{7.5cm}}
\tableheadcolor
\textbf{Tutorial} & \textbf{Hours} & \textbf{What You Learn} \\
\midrule
01: First Sim & 1-2 & Run simulation with zero theory. Observe stabilization, modify initial conditions \\
\midrule
02: Comparison & 4 & Run all 7 controllers, compare metrics (settling time, overshoot, energy, chattering). Understand tradeoffs. \\
\midrule
03: PSO Tuning & 4 & Tune controller gains automatically. Observe PSO convergence, validate results. \\
\midrule
04: Custom Ctrl & 8 & Implement new controller from scratch. Integrate with codebase, add tests. \\
\midrule
05: Research & 12 & Reproduce MT-5 benchmark. Understand experimental design, generate publication figures. \\
\end{tabular}

\begin{tip}
\textbf{Cross-References:} Each path links to next level

Path 0 Phase 5 → Tutorial 01 (Path 1) → Tutorial 02-03 (Path 2) → Tutorial 04-05 (Path 3) → Architecture docs (Path 4)
\end{tip}

% ==============================================================================
% PAGE 4: NOTEBOOKLM PODCAST SERIES
% ==============================================================================
\newpage

\section*{NotebookLM Podcast Series: Audio Learning}

\begin{keypoint}
\textbf{Purpose:} Convert 125-hour beginner roadmap to podcast audio for commute/exercise learning

\textbf{Series:} 44 episodes, ~40 hours audio, ~125 hours content (3× compression)

\textbf{Status:} All episodes complete (November 2025)
\end{keypoint}

\subsection*{Series Structure by Phase}

\begin{tcolorbox}[colback=secondary!10, colframe=secondary, title=\faHeadphones\ 44 Episodes Across 4 Phases]
\textbf{Phase 1: Foundations (11 episodes, 4 hours audio, 40 hours content)}
\begin{itemize}
    \item E001: Computing Basics
    \item E002-E003: Python Fundamentals Parts 1-2 (variables, loops, functions, classes)
    \item E004: NumPy and Matplotlib
    \item E005: Physics Review (Newton's laws)
    \item E006: Linear Algebra
    \item E007: Trigonometry for Angles
    \item E008-E011: Practice exercises, Q\&A, walkthroughs
\end{itemize}

\textbf{Phase 2: Core Concepts (12 episodes, 5 hours audio, 30 hours content)}
\begin{itemize}
    \item Control systems intro, PID control, stability
    \item SMC fundamentals, super-twisting, adaptive control
    \item PSO basics, cost functions
\end{itemize}

\textbf{Phase 3: Hands-On (8 episodes, 2.5 hours audio, 25 hours content)}
\begin{itemize}
    \item First simulation walkthrough
    \item Parameter experimentation
    \item Visualization techniques
    \item Performance metrics interpretation
\end{itemize}

\textbf{Phase 4: Advancing Skills (13 episodes, 12-15 hours audio, 30 hours content)}
\begin{itemize}
    \item OOP Python, type hints, testing
    \item Reading source code
    \item Simulation engine deep dive
    \item Controller architecture
\end{itemize}
\end{tcolorbox}

\begin{warning}
\textbf{Phase 5 Excluded:} Branching structure (3 tracks) incompatible with linear podcast format

\textbf{Solution:} Phase 5 is documentation-only with specific papers, code examples, advanced tutorials
\end{warning}

\subsection*{TTS Optimization: Making Math Speakable}

\begin{example}
\textbf{Technique 1: Verbalize All Math}

LaTeX: \texttt{\$v = w v + c\_1 (p - x)\$}

Audio: "velocity equals inertia times velocity plus cognitive coefficient times the difference between personal best and position"

\textbf{Listeners hear words, not symbols!}
\end{example}

\begin{example}
\textbf{Technique 2: Spell Out Greek Letters}

First mention: "theta (that's T-H-E-T-A), the angle of the first pendulum link"

Subsequent mentions: "theta increases to 0.2 radians"

\textbf{Don't assume pronunciation!}
\end{example}

\begin{example}
\textbf{Technique 3: Enhanced Narratives}

\textbf{Analogies:} "Sliding mode control is like a ball rolling down a valley. The sliding surface is the valley floor."

\textbf{Progressive Revelation:} Introduce simple terms → Add details gradually

\textbf{Retention:} Summarize every 5 minutes, repeat key points at episode end

\textbf{Example:} Episode E002 explains variables 3 times with increasing depth
\begin{enumerate}
    \item "Containers for values"
    \item "Memory addresses with labels"
    \item "Type system and mutability"
\end{enumerate}
\end{example}

% ==============================================================================
% PAGE 5: DOCUMENTATION NAVIGATION
% ==============================================================================
\newpage

\section*{Documentation Navigation: 985 Files System}

\begin{keypoint}
\textbf{Challenge:} How do 5 user types find relevant content among 985 files?

\textbf{Solution:} Master navigation hub (\texttt{NAVIGATION.md}) as "library front desk"
\end{keypoint}

\subsection*{Master Hub: Four Entry Modes}

\begin{tcolorbox}[colback=accent!10, colframe=accent, title=\faMap\ NAVIGATION.md Entry Points]
\textbf{Mode 1: "I Want To..." (Intent-Based, 6 categories)}
\begin{itemize}
    \item "I want to learn the basics" → Path 0-1
    \item "I want to compare controllers" → Tutorial 02
    \item "I want to optimize gains" → Tutorial 03 + PSO docs
    \item "I want to understand the code" → Architecture docs
    \item "I want to run research experiments" → Research workflow docs
    \item "I want to deploy on hardware" → HIL + embedded guides
\end{itemize}

\textbf{Mode 2: Persona-Based (4 user types)}
\begin{itemize}
    \item \textbf{Beginners} → Path 0
    \item \textbf{Researchers} → Paths 2-3 + research tasks
    \item \textbf{Developers} → Path 4 + architecture
    \item \textbf{Educators} → Teaching materials + slides
\end{itemize}

\textbf{Mode 3: Category Index Directory (43 specialized indexes)}
\begin{itemize}
    \item Guides index (5 tutorials)
    \item Theory index (SMC fundamentals, Lyapunov proofs)
    \item Architecture index (design patterns, module structure)
    \item Educational materials index
\end{itemize}

\textbf{Mode 4: Visual Navigation Tools}
\begin{itemize}
    \item Interactive sitemaps
    \item Dependency graphs
    \item Learning journey flowcharts
\end{itemize}
\end{tcolorbox}

\begin{tip}
\textbf{Think Library Front Desk, Not Card Catalog}

\textbf{Without Master Hub:} Wander through 985 files randomly

\textbf{With Master Hub:} Tell front desk what you need → Get personalized map showing 5-10 relevant files in 30 seconds

Each of those 5 files links to deeper material if you want more
\end{tip}

\subsection*{Documentation Statistics}

\begin{multicols}{2}

\textbf{Total Files:} 985 documentation files
\begin{itemize}
    \item 814 files in \texttt{docs/}
    \item 171 files in \texttt{.ai\_workspace/}
\end{itemize}

\textbf{Navigation Systems:} 11 total
\begin{itemize}
    \item NAVIGATION.md (master hub)
    \item docs/index.md (Sphinx HTML)
    \item guides/INDEX.md
    \item README.md
    \item 3 visual sitemaps
    \item 2 interactive demos
\end{itemize}

\columnbreak

\textbf{Category Indexes:} 43 index.md files
\begin{itemize}
    \item Across all documentation domains
\end{itemize}

\textbf{Learning Paths:} 5 paths
\begin{itemize}
    \item Path 0: 125-150 hrs (semester)
    \item Path 1: 1-2 hrs
    \item Path 2: 4-8 hrs
    \item Path 3: 8-12 hrs
    \item Path 4: 12+ hrs
\end{itemize}

\end{multicols}

% ==============================================================================
% PAGE 6: SPHINX DOCUMENTATION
% ==============================================================================
\newpage

\section*{Sphinx Documentation System}

\begin{keypoint}
\textbf{Purpose:} Generate searchable HTML docs from 814 files in \texttt{docs/}

\textbf{Build:} \texttt{sphinx-build -M html docs docs/\_build}

\textbf{Serve:} \texttt{python -m http.server 9000 --directory docs/\_build/html}
\end{keypoint}

\subsection*{Five Major Sections}

\begin{tcolorbox}[colback=primary!5, colframe=primary, title=\faBook\ Sphinx Structure]
\textbf{Section 1: Guides}
\begin{itemize}
    \item Tutorials 01-05
    \item Getting started, installation
\end{itemize}

\textbf{Section 2: Theory}
\begin{itemize}
    \item SMC fundamentals
    \item Lyapunov stability proofs
    \item PSO optimization theory
    \item DIP dynamics
\end{itemize}

\textbf{Section 3: Architecture}
\begin{itemize}
    \item Module design
    \item Controller factory pattern
    \item Simulation engine internals
    \item Testing strategy
\end{itemize}

\textbf{Section 4: API Reference}
\begin{itemize}
    \item Auto-generated from docstrings
    \item Covers all 358 source files
    \item Function signatures, parameters, return types
\end{itemize}

\textbf{Section 5: Research}
\begin{itemize}
    \item 72-hour roadmap
    \item Tasks MT-5 through LT-7
    \item Reproduction guides
    \item Experiment documentation
\end{itemize}
\end{tcolorbox}

\subsection*{Documentation Quality Standards}

\begin{warning}
\textbf{Quality Metric:} < 5 AI-ish patterns per file

\textbf{Detection:} \texttt{python scripts/docs/detect\_ai\_patterns.py --file <file.md>}

\textbf{Patterns to Avoid:}
\begin{itemize}
    \item "Let's explore..." (too conversational)
    \item "comprehensive" without metrics (vague)
    \item "delve into" (overused AI phrase)
    \item Excessive enthusiasm (!!!)
\end{itemize}

\textbf{Target:} Direct technical writing in API docs, conversational OK in tutorials
\end{warning}

\subsection*{Auto-Rebuild Triggers}

\begin{example}
\textbf{Files That Trigger Rebuild:}
\begin{itemize}
    \item Sphinx source: \texttt{docs/*.md}, \texttt{docs/**/*.rst}
    \item Static assets: \texttt{docs/\_static/*.css}, \texttt{docs/\_static/*.js}
    \item Configuration: \texttt{docs/conf.py}, \texttt{docs/\_templates/*}
    \item Navigation: \texttt{docs/index.rst}, any \texttt{toctree} directives
\end{itemize}

\textbf{After Changes:} Always rebuild, verify with \texttt{curl}, tell user to hard refresh browser (Ctrl+Shift+R)
\end{example}

% ==============================================================================
% PAGE 7: AUDIENCE SEGMENTATION
% ==============================================================================
\newpage

\section*{Audience Segmentation Strategy}

\begin{keypoint}
\textbf{Goal:} Ensure each audience type finds relevant content without drowning in 985 files
\end{keypoint}

\subsection*{Four Segmentation Mechanisms}

\begin{tcolorbox}[colback=secondary!10, colframe=secondary, title=\faUsers\ Mechanism 1: Explicit Signposting]
\textbf{In README.md:}
\begin{itemize}
    \item "Complete beginners: start with \texttt{.ai\_workspace/edu/beginner-roadmap.md}"
    \item "Python users: start with \texttt{docs/guides/tutorial\_01\_first\_simulation.md}"
    \item "Control theorists: start with \texttt{docs/theory/smc\_fundamentals.md}"
    \item "Researchers: start with \texttt{.ai\_workspace/planning/research/72\_HOUR\_ROADMAP.md}"
\end{itemize}

\textbf{Clear entry points prevent wandering!}
\end{tcolorbox}

\begin{tcolorbox}[colback=accent!10, colframe=accent, title=\faMapSigns\ Mechanism 2: Breadcrumbs in Every File]
\textbf{Header Format:}
\begin{itemize}
    \item \textbf{Audience:} Beginners / Intermediate / Advanced Researchers / Developers
    \item \textbf{Prerequisites:} Python basics, control theory / None / Lyapunov theory
\end{itemize}

\textbf{Prevents Beginners from Getting Lost in Advanced Material}
\end{tcolorbox}

\subsection*{Progressive Disclosure \& Layered Docs}

\begin{multicols}{2}

\textbf{Mechanism 3: Progressive Disclosure}

\textbf{Tutorial 01:} How to run simulation (no math)

\textbf{Tutorial 02:} Performance metrics (no derivations)

\textbf{Tutorial 03:} PSO cost function (with equations)

\textbf{Tutorial 04:} Full controller implementation (Lyapunov proofs)

\textbf{Information density increases gradually}

\columnbreak

\textbf{Mechanism 4: Layered Documentation}

\textbf{Layer 1:} Quick reference cards (1 page)

\textbf{Layer 2:} Tutorial guides (5-10 pages)

\textbf{Layer 3:} Theory deep dives (20+ pages)

\textbf{Layer 4:} Source code (annotated with detailed comments)

\textbf{Beginners stay in Layers 1-2, Experts read Layer 4}

\end{multicols}

% ==============================================================================
% PAGE 8: INTERACTIVE LEARNING & ORGANIZATION
% ==============================================================================
\newpage

\section*{Interactive Learning Components}

\begin{tcolorbox}[colback=primary!5, colframe=primary, title=\faGamepad\ Four Interactive Components]
\textbf{Component 1: Streamlit UI (Operational)}
\begin{itemize}
    \item Launch: \texttt{streamlit run streamlit\_app.py}
    \item Web interface: DIP animation, controller parameter sliders, real-time metrics
    \item User adjusts gains → Sees immediate effect on stabilization
    \item \textbf{No coding required!}
\end{itemize}

\textbf{Component 2: Jupyter Notebooks (Planned)}
\begin{itemize}
    \item Combine code + text + visualizations
    \item Users execute cells step-by-step, see intermediate results
    \item Experiment with modifications
\end{itemize}

\textbf{Component 3: Practice Exercises with Solutions (Planned)}
\begin{itemize}
    \item Each tutorial: 5-10 exercises
    \item Example: "Change initial angle from 0.1 to 0.3 rad, predict if controller stabilizes, run, verify"
    \item Solutions in \texttt{docs/solutions/} with worked examples
\end{itemize}

\textbf{Component 4: Self-Assessment Quizzes (Planned)}
\begin{itemize}
    \item Multiple choice testing comprehension
    \item Example: "Which controller has lowest chattering? A) Classical, B) STA, C) Adaptive, D) Hybrid Adaptive STA"
    \item Answers with explanations
\end{itemize}
\end{tcolorbox}

\section*{Educational Content Organization}

\begin{keypoint}
\textbf{Three Locations with Clear Separation:}
\end{keypoint}

\begin{multicols}{2}

\textbf{Location 1: \texttt{.ai\_workspace/edu/}}
\begin{itemize}
    \item \textbf{Content:} Prerequisite materials
    \item beginner-roadmap.md (Path 0)
    \item Future: intermediate roadmap, cheatsheets, video curriculum
    \item \textbf{Audience:} Complete beginners building foundations
\end{itemize}

\textbf{Location 2: \texttt{docs/guides/}}
\begin{itemize}
    \item \textbf{Content:} Project-specific tutorials
    \item tutorial\_01 through tutorial\_05
    \item getting-started.md, installation.md
    \item \textbf{Audience:} Users learning this specific project (Paths 1-3)
\end{itemize}

\columnbreak

\textbf{Location 3: \texttt{docs/theory/}}
\begin{itemize}
    \item \textbf{Content:} Control theory deep dives
    \item smc\_fundamentals.md
    \item lyapunov\_proofs.md
    \item pso\_theory.md
    \item \textbf{Audience:} Users wanting rigorous math (Paths 3-4)
\end{itemize}

\end{multicols}

\subsection*{Cross-Reference Structure}

\begin{example}
\textbf{Linking Between Levels:}
\begin{itemize}
    \item \textbf{beginner-roadmap.md} Phase 5 → \textbf{tutorial\_01} (graduation exercise)
    \item \textbf{tutorial\_01} → \textbf{smc\_fundamentals.md} (understand the math)
    \item \textbf{tutorial\_05} → \textbf{72\_HOUR\_ROADMAP.md} (full research workflow)
\end{itemize}

\textbf{Users can navigate up (advanced) or down (foundational) easily}
\end{example}

% ==============================================================================
% PAGE 9: FUTURE CONTENT & PHILOSOPHY
% ==============================================================================
\newpage

\section*{Future Educational Content (Planned)}

\begin{tcolorbox}[colback=secondary!10, colframe=secondary, title=\faRocket\ Seven Categories]
\textbf{1. Intermediate Roadmap (40 hours)}

For users with Python basics wanting advanced control theory without 125-hour beginner path

Covers: State-space, observability/controllability, LQR, nonlinear control, Lyapunov theory

\textbf{2. Quick Reference Cheatsheets}

One-page PDFs: Python syntax, Git commands, CLI usage, controller selection guide, PSO tuning tips

\textbf{3. Video Curriculum}

Curated YouTube playlists (not creating videos, organizing existing free resources)
\begin{itemize}
    \item "Learn Python in 15 hours" (MIT OpenCourseWare)
    \item "Control systems basics" (Brian Douglas)
    \item "Sliding mode control tutorial" (Slotine lectures)
\end{itemize}

\textbf{4. Exercise Solutions with Worked Examples}

Not just answers - step-by-step derivations

Example: "Why does this controller fail?" shows Lyapunov analysis proving instability

\textbf{5. FAQ for Beginners}

"What is a Lyapunov function?", "Why do pendulums swing up not down?", "How to choose PSO particle count?"

Answers with minimal jargon

\textbf{6. Interactive Demos (JavaScript)}

Web page with DIP animation + sliders for mass, length, gains

Runs in browser, no installation - useful for classroom teaching

\textbf{7. Community Contribution Opportunities}

"Good first issue" tags in GitHub, documentation improvements, controller implementation challenges

Turn learners into contributors
\end{tcolorbox}

\begin{warning}
\textbf{Why Not Implemented?}

Resource constraints - Phase 5 focused on research (11 tasks)

Beginner roadmap Phases 1-2 + NotebookLM podcasts are substantial solo efforts

Future work depends on community interest and contributions
\end{warning}

\section*{Learning Measurement: Five Mechanisms}

\begin{multicols}{2}

\textbf{1. Progress Tracking Checklists}

Each module has checkbox: "- [ ] Completed Python Module 2"

Users check boxes, see completion percentage

\textbf{2. Self-Assessment Quizzes (Planned)}

Score 8/10 or higher to proceed

Test prerequisite knowledge before advanced topics

\textbf{3. Skill Validation Checkpoints}

Tutorial 01 ends: "If you can run a simulation and interpret the plot, you completed Path 1"

Tutorial 05 ends: "If you can reproduce MT-5 within 10\% error, you're ready for independent research"

Clear pass/fail criteria

\columnbreak

\textbf{4. Common Misconception Identification}

Documentation includes "Common Mistakes" sections

Example: "Many beginners think theta\_1 is cart position -- it's the angle of the first link"

Addresses errors proactively

\textbf{5. Feedback Loop for Improvement}

GitHub issues tagged "documentation feedback"

Users report confusing sections, suggest improvements

Maintainers update docs based on feedback

\end{multicols}

% ==============================================================================
% PAGE 10: EDUCATIONAL PHILOSOPHY & KEY TAKEAWAYS
% ==============================================================================
\newpage

\section*{Educational Philosophy: Five Principles}

\begin{summary}
\textbf{Principle 1: Understanding Over Coverage}

Don't teach everything - teach foundational concepts deeply, provide references for advanced topics

Better to master 20\% than superficially touch 100\%

\textbf{Principle 2: Scaffolded Learning from Foundations to Mastery}

Path 0 (prerequisites) → Path 1 (hands-on) → Path 2 (theory) → Path 3 (research) → Path 4 (architecture)

Each level builds on previous

\textbf{Principle 3: Multiple Modalities for Different Learners}

Text (docs) + Audio (podcasts) + Visual (Streamlit UI) + Interactive (Jupyter) + Hands-on (tutorials)

Some learn by reading, others by listening, others by doing

\textbf{Principle 4: Audience-Appropriate Language}

Tutorial 01: "the pendulum swings up and balances" (beginner-friendly)

API reference: "state vector converges to origin under Lyapunov stability" (technical precision)

\textbf{Principle 5: Practice-First Approach}

Tutorial 01: Run simulation BEFORE explaining theory

Tutorial 02: Compare controllers BEFORE reading equations

Understanding comes from experience, not just reading
\end{summary}

\section*{Key Takeaways}

\begin{tcolorbox}[colback=accent!10, colframe=accent, title=\faCheckCircle\ Summary]
\textbf{5 Learning Paths:} Path 0 (Student, ~125 hrs, semester), Path 1 (Experimenter, 1-2 hrs), Path 2 (Engineer, 4-8 hrs), Path 3 (Researcher, 8-12 hrs), Path 4 (Expert, 12+ hrs)

\textbf{Path 0 Roadmap:} 5 phases - Foundations (40 hrs), Core Concepts (30 hrs), Hands-On (25 hrs), Advancing Skills (30 hrs), Mastery/Specialization (25-75 hrs, 3 tracks)

\textbf{NotebookLM Podcasts:} 44 episodes, 40 hours audio, 125 hours content (3× compression)

\textbf{TTS Optimization:} (1) Verbalize all math, (2) Spell out Greek letters, (3) Enhanced narratives (analogies, progressive revelation, retention summaries)

\textbf{Navigation (985 files):} Master hub (NAVIGATION.md) routes users to relevant 5-10 files via 4 entry modes (intent, persona, category, visual)

\textbf{Tutorials:} 5 tutorials - 01: First Sim (1-2h), 02: Comparison (4h), 03: PSO (4h), 04: Custom Ctrl (8h), 05: Research (12h)

\textbf{Sphinx Docs:} 814 files → 5 sections (Guides, Theory, Architecture, API, Research). Build: \texttt{sphinx-build}, Serve: \texttt{http.server}

\textbf{Segmentation:} (1) Explicit signposting (README), (2) Breadcrumbs (audience + prerequisites in every file), (3) Progressive disclosure (Tutorial 01 no math → Tutorial 04 Lyapunov), (4) Layered docs (1-page → 20-page → source code)

\textbf{Interactive:} (1) Streamlit UI (operational), (2) Jupyter notebooks (planned), (3) Practice exercises (planned), (4) Quizzes (planned)

\textbf{Organization:} (1) \texttt{.ai\_workspace/edu/} (prerequisites), (2) \texttt{docs/guides/} (project tutorials), (3) \texttt{docs/theory/} (rigorous math)

\textbf{Future:} Intermediate roadmap, cheatsheets, video curriculum, exercise solutions, FAQ, interactive demos, community contributions

\textbf{Philosophy:} (1) Understanding > coverage, (2) Scaffolded learning, (3) Multiple modalities, (4) Audience-appropriate language, (5) Practice-first
\end{tcolorbox}

\subsection*{What's Next?}

\begin{keypoint}
\textbf{E010: Documentation System \& Navigation}

985 files organized across 11 navigation systems, Sphinx HTML build process, NAVIGATION.md architecture, how documentation scales

\textbf{Remember:} Education is not about transferring knowledge - it's about creating conditions for understanding to emerge!
\end{keypoint}

\end{document}
