% =============================================================================
% Episode 017: Multi-agent Orchestration & Checkpoint Recovery
% Series: DIP-SMC-PSO Professional Toolkit
% Phase: 3 (Professional Practice)
% Duration: ~25 minutes | Pages: 12-14 | Complexity: Professional
% Dependencies: E015 (architectural standards), E016 (documentation quality)
% =============================================================================

% ==============================================================================
% MASTER TEMPLATE FOR PODCAST EPISODE CHEATSHEET PDFs
% ==============================================================================
% Purpose: Beginner-friendly, colorful, multi-page study guides (2-4 pages)
% Target Audience: Complete beginners (Path 0 learners)
% Visual Style: Infographic-style with vibrant colors, icons, callout boxes
% ==============================================================================

\documentclass[11pt,a4paper]{article}

% ------------------------------------------------------------------------------
% GEOMETRY & LAYOUT
% ------------------------------------------------------------------------------
\usepackage[top=1.5cm, bottom=1.5cm, left=1.5cm, right=1.5cm, headheight=14pt]{geometry}
\usepackage{multicol}
\usepackage{fancyhdr}
\pagestyle{fancy}

% ------------------------------------------------------------------------------
% COLOR PALETTE (Vibrant & Beginner-Friendly)
% ------------------------------------------------------------------------------
\usepackage{xcolor}
\definecolor{primary}{RGB}{41, 128, 185}      % Blue - Main concepts, headers
\definecolor{secondary}{RGB}{39, 174, 96}     % Green - Success, examples
\definecolor{accent}{RGB}{230, 126, 34}       % Orange - Important notes
\definecolor{warning}{RGB}{231, 76, 60}       % Red - Common pitfalls
\definecolor{background}{RGB}{236, 240, 241}  % Light gray - Callout boxes
\definecolor{codeblock}{RGB}{44, 62, 80}      % Dark blue-gray - Code background
\definecolor{highlight}{RGB}{241, 196, 15}    % Yellow - Highlights

% ------------------------------------------------------------------------------
% TIKZ & GRAPHICS
% ------------------------------------------------------------------------------
\usepackage{tikz}
\usetikzlibrary{shapes, arrows, positioning, calc, shadows, decorations.pathreplacing, backgrounds, fit}
\usepackage{graphicx}
\usepackage{float}

% TikZ styles for consistent diagrams
\tikzstyle{block} = [rectangle, draw, fill=primary!20, text width=5em, text centered, rounded corners, minimum height=3em, drop shadow]
\tikzstyle{arrow} = [thick,->,>=stealth]
\tikzstyle{process} = [rectangle, draw, fill=secondary!20, text width=6em, text centered, rounded corners, minimum height=3em]
\tikzstyle{decision} = [diamond, draw, fill=accent!20, text width=4.5em, text badly centered, inner sep=0pt]
\tikzstyle{cloud} = [ellipse, draw, fill=background, text width=5em, text centered, minimum height=2.5em]

% ------------------------------------------------------------------------------
% BOXES & CALLOUTS (tcolorbox)
% ------------------------------------------------------------------------------
\usepackage{tcolorbox}
\tcbuselibrary{skins, breakable, raster}

% Key Concept Box (Blue)
\newtcolorbox{keypoint}{
    enhanced,
    colback=primary!10,
    colframe=primary,
    fonttitle=\bfseries,
    title=\faLightbulb\ Key Concept,
    attach boxed title to top left={yshift=-2mm, xshift=5mm},
    boxed title style={colback=primary},
    breakable
}

% Example Box (Green)
\newtcolorbox{example}{
    enhanced,
    colback=secondary!10,
    colframe=secondary,
    fonttitle=\bfseries,
    title=\faCode\ Example,
    attach boxed title to top left={yshift=-2mm, xshift=5mm},
    boxed title style={colback=secondary},
    breakable
}

% Warning Box (Red)
\newtcolorbox{warning}{
    enhanced,
    colback=warning!10,
    colframe=warning,
    fonttitle=\bfseries,
    title=\faExclamationTriangle\ Common Pitfall,
    attach boxed title to top left={yshift=-2mm, xshift=5mm},
    boxed title style={colback=warning},
    breakable
}

% Tip Box (Orange)
\newtcolorbox{tip}{
    enhanced,
    colback=accent!10,
    colframe=accent,
    fonttitle=\bfseries,
    title=\faLightbulb\ Pro Tip,
    attach boxed title to top left={yshift=-2mm, xshift=5mm},
    boxed title style={colback=accent},
    breakable
}

% Summary Box (Light gray)
\newtcolorbox{summary}{
    enhanced,
    colback=background,
    colframe=codeblock,
    fonttitle=\bfseries,
    title=\faListUl\ Quick Summary,
    attach boxed title to top left={yshift=-2mm, xshift=5mm},
    boxed title style={colback=codeblock},
    breakable
}

% ------------------------------------------------------------------------------
% ICONS & SYMBOLS (fontawesome5)
% ------------------------------------------------------------------------------
\usepackage{fontawesome5}

% Custom icon commands for consistency
\newcommand{\iconKey}{\textcolor{primary}{\faLightbulb}}
\newcommand{\iconCode}{\textcolor{secondary}{\faCode}}
\newcommand{\iconWarning}{\textcolor{warning}{\faExclamationTriangle}}
\newcommand{\iconTip}{\textcolor{accent}{\faInfoCircle}}
\newcommand{\iconLink}{\textcolor{primary}{\faLink}}
\newcommand{\iconBook}{\textcolor{secondary}{\faBook}}
\newcommand{\iconTarget}{\textcolor{accent}{\faBullseye}}

% ------------------------------------------------------------------------------
% TYPOGRAPHY & FONTS
% ------------------------------------------------------------------------------
\usepackage{lmodern}
\usepackage[T1]{fontenc}
\usepackage[utf8]{inputenc}

% Section styling
\usepackage{titlesec}
\titleformat{\section}{\Large\bfseries\color{primary}}{\thesection}{1em}{}[\titlerule]
\titleformat{\subsection}{\large\bfseries\color{secondary}}{\thesubsection}{1em}{}

% ------------------------------------------------------------------------------
% CODE LISTINGS
% ------------------------------------------------------------------------------
\usepackage{listings}
\lstset{
    basicstyle=\ttfamily\footnotesize\color{white},
    backgroundcolor=\color{codeblock},
    keywordstyle=\color{primary!80},
    commentstyle=\color{secondary!60}\itshape,
    stringstyle=\color{accent!80},
    numbers=left,
    numberstyle=\tiny\color{white!50},
    stepnumber=1,
    numbersep=8pt,
    frame=single,
    rulecolor=\color{codeblock},
    breaklines=true,
    breakatwhitespace=true,
    tabsize=4,
    captionpos=b
}

% Python-specific styling
\lstdefinestyle{python}{
    language=Python,
    morekeywords={self, def, class, import, from, as, return, if, else, elif, for, while, True, False, None}
}

% YAML-specific styling
\lstdefinestyle{yaml}{
    basicstyle=\ttfamily\footnotesize,
    showstringspaces=false,
    commentstyle=\color{secondary!60}\itshape,
    keywordstyle=\color{primary!80}
}

% ------------------------------------------------------------------------------
% HYPERLINKS
% ------------------------------------------------------------------------------
\usepackage{hyperref}
\hypersetup{
    colorlinks=true,
    linkcolor=primary,
    urlcolor=secondary,
    citecolor=accent
}

% ------------------------------------------------------------------------------
% HEADER & FOOTER CUSTOMIZATION
% ------------------------------------------------------------------------------
\fancyhf{}
\fancyhead[L]{\textcolor{primary}{\textbf{DIP-SMC-PSO Podcast Cheatsheet}}}
\fancyhead[R]{\textcolor{secondary}{\episodetitle}}
\fancyfoot[C]{\textcolor{primary}{\thepage}}
\renewcommand{\headrulewidth}{0.5pt}
\renewcommand{\footrulewidth}{0.5pt}

% ------------------------------------------------------------------------------
% CUSTOM COMMANDS
% ------------------------------------------------------------------------------

% Episode title command (to be defined in each episode file)
\newcommand{\episodetitle}{Episode Title}

% Learning objective command
\newcommand{\learningobjective}[1]{%
    \begin{center}
    \begin{tcolorbox}[colback=highlight!30, colframe=accent, width=0.9\textwidth]
    \iconTarget\ \textbf{Learning Objective:} #1
    \end{tcolorbox}
    \end{center}
}

% Quick reference command (for formulas/code snippets)
\newcommand{\quickref}[2]{%
    \begin{tcolorbox}[colback=background, colframe=primary, title=\faBookmark\ #1]
    #2
    \end{tcolorbox}
}

% Resource link command
\newcommand{\resourcelink}[2]{%
    \iconLink\ \href{#1}{\textcolor{secondary}{#2}}
}

% ------------------------------------------------------------------------------
% TITLE PAGE FORMATTING
% ------------------------------------------------------------------------------
\usepackage{afterpage}

\newcommand{\makeepisodetitle}[4]{%
    \begin{titlepage}
    \begin{tikzpicture}[remember picture, overlay]
        % Background gradient
        \fill[primary!20] (current page.south west) rectangle (current page.north east);

        % Title box
        \node[
            fill=white,
            rounded corners=10pt,
            drop shadow,
            text width=0.8\paperwidth,
            align=center
        ] at (current page.center) {
            \Huge\bfseries\color{primary} #1 \\[1em]
            \Large\color{secondary} #2 \\[2em]
            \large\color{codeblock} Part #3 $\cdot$ Duration: #4 \\[1em]
            \normalsize\color{codeblock} \textit{Beginner-Friendly Visual Study Guide}
        };

        % Footer
        \node[
            anchor=south,
            text width=0.9\paperwidth,
            align=center
        ] at (current page.south) {
            \color{primary}\rule{0.8\paperwidth}{0.5pt} \\[0.5em]
            \small\color{codeblock}
            \textbf{Repository:} \url{https://github.com/theSadeQ/dip-smc-pso} \\
            \textbf{Documentation:} academic/paper/presentations/podcasts/episodes/ \\[1em]
        };
    \end{tikzpicture}
    \end{titlepage}
}

% ------------------------------------------------------------------------------
% MATH & EQUATIONS
% ------------------------------------------------------------------------------
\usepackage{amsmath, amssymb, amsthm}
\usepackage{mathtools}

% Equation box (highlighted equations)
\newcommand{\eqbox}[1]{%
    \begin{tcolorbox}[colback=primary!5, colframe=primary, boxrule=1pt]
    \begin{equation}
    #1
    \end{equation}
    \end{tcolorbox}
}

% ------------------------------------------------------------------------------
% TABLES
% ------------------------------------------------------------------------------
\usepackage{booktabs}
\usepackage{array}
\usepackage{multirow}
\usepackage{colortbl}

% Custom table colors
\newcommand{\tableheadcolor}{\rowcolor{primary!30}}

% ------------------------------------------------------------------------------
% END OF PREAMBLE
% ------------------------------------------------------------------------------


% Episode Metadata
\title{\textbf{E017: Multi-agent Orchestration}}
\def\episodenumber{017}
\def\episodetitle{Multi-Agent Orchestration \& Checkpoint Recovery}
\def\episodecategory{Professional Practice}
\def\difficulty{Professional}

\begin{document}
\makeepisodetitle

% =============================================================================
% SECTION 1: OVERVIEW
% =============================================================================
\section{Overview}

\subsection{What You'll Learn}
\begin{itemize}
  \item \textbf{Ultimate Orchestrator Pattern}: Coordinate 6 specialized agents for complex tasks
  \item \textbf{Checkpoint System}: Prevent work loss on token limits/crashes
  \item \textbf{Recovery Workflow}: Resume multi-agent tasks across sessions
  \item \textbf{Quality Gates}: Automated validation (8 gates, 7/8 pass required)
\end{itemize}

\subsection{Why This Matters}
\begin{highlightbox}
\textbf{Problem}: Large refactorings (e.g., 34-issue UI overhauls) can exceed 200K token limits mid-execution, losing hours of work.

\textbf{Solution}: Checkpoint system + multi-agent orchestration survives crashes and enables resumption with zero context loss.

\textbf{Impact}: Phase 3 UI completion (34 issues) achieved via checkpointed 6-agent orchestration, surviving 3 token limit events.
\end{highlightbox}

% =============================================================================
% SECTION 2: THE ULTIMATE ORCHESTRATOR PATTERN
% =============================================================================
\section{The Ultimate Orchestrator Pattern}

\subsection{Architecture Overview}
\begin{center}
\begin{tikzpicture}[
  node distance=1.5cm,
  orchestrator/.style={rectangle, draw=primaryblue, fill=primaryblue!10, thick, minimum width=3cm, minimum height=0.8cm},
  agent/.style={rectangle, draw=secondarygreen, fill=secondarygreen!10, thick, minimum width=2.5cm, minimum height=0.7cm},
  checkpoint/.style={diamond, draw=errorred, fill=errorred!10, thick, minimum width=1.5cm, minimum height=1cm}
]
  % Orchestrator
  \node[orchestrator] (orch) at (0,0) {\textbf{Ultimate Orchestrator}};

  % Agents (6 total)
  \node[agent] (int) at (-5,-2) {Integration Agent};
  \node[agent] (ctrl) at (-2.5,-2) {Control Systems};
  \node[agent] (pso) at (0,-2) {PSO Agent};
  \node[agent] (doc) at (2.5,-2) {Documentation};
  \node[agent] (beauty) at (5,-2) {Code Beauty};
  \node[agent] (test) at (0,-3.5) {Testing Agent};

  % Checkpoints
  \node[checkpoint] (chk1) at (-5,-0.5) {C1};
  \node[checkpoint] (chk2) at (5,-0.5) {C2};

  % Connections
  \draw[->, thick] (orch) -- (chk1);
  \draw[->, thick] (orch) -- (chk2);
  \draw[->, thick] (chk1) -- (int);
  \draw[->, thick] (orch) -- (ctrl);
  \draw[->, thick] (orch) -- (pso);
  \draw[->, thick] (orch) -- (doc);
  \draw[->, thick] (chk2) -- (beauty);
  \draw[->, thick] (orch) -- (test);
\end{tikzpicture}
\end{center}

\subsection{Agent Roles \& Responsibilities}
\begin{enumerate}
  \item \textbf{Integration Agent}: Merge changes across 5+ modules, resolve conflicts
  \item \textbf{Control Systems Agent}: Implement controllers, validate Lyapunov stability
  \item \textbf{PSO Agent}: Optimize hyperparameters, run benchmarks
  \item \textbf{Documentation Agent}: Generate API docs, update guides
  \item \textbf{Code Beautification Agent}: Apply formatters, fix linting issues
  \item \textbf{Testing Agent}: Write pytest tests, achieve 95\% coverage
\end{enumerate}

\subsection{Orchestrator Responsibilities}
\begin{itemize}
  \item \textbf{Task Decomposition}: Break 34-issue roadmap into 6 agent workstreams
  \item \textbf{Dependency Resolution}: Launch agents in correct order (e.g., Integration before Documentation)
  \item \textbf{Checkpoint Management}: Save agent state every 5-10 minutes
  \item \textbf{Quality Validation}: Run 8 quality gates, fail if <7 pass
  \item \textbf{Recovery Coordination}: Resume from last checkpoint on token limit
\end{itemize}

% =============================================================================
% SECTION 3: CHECKPOINT SYSTEM ARCHITECTURE
% =============================================================================
\section{Checkpoint System Architecture}

\subsection{Core Concepts}
\begin{itemize}
  \item \textbf{Checkpoint}: Snapshot of agent state (task ID, progress, deliverables, current phase)
  \item \textbf{Granularity}: Per-agent, per-task, timestamped
  \item \textbf{Storage}: JSON files in \texttt{.ai\_workspace/state/checkpoints/}
  \item \textbf{Triggers}: Manual (API call), automatic (every 5-10 min), event-driven (pre-commit)
\end{itemize}

\subsection{Checkpoint Lifecycle}
\begin{enumerate}
  \item \textbf{Plan Approved}: User approves multi-agent plan, checkpoint created
  \item \textbf{Agent Launched}: Each agent gets unique checkpoint on spawn
  \item \textbf{Progress Updates}: Agent reports completion every 5-10 min
  \item \textbf{Agent Complete/Failed}: Final checkpoint with success/failure status
  \item \textbf{Recovery}: New session loads latest checkpoints, resumes work
\end{enumerate}

\subsection{Checkpoint Data Structure}
\begin{lstlisting}[language=json]
{
  "task_id": "LT-4",
  "agent_id": "agent_control_systems",
  "timestamp": "2025-01-29T17:30:00Z",
  "status": "in_progress",
  "hours_allocated": 12.0,
  "hours_completed": 8.5,
  "deliverables_completed": [
    "Lyapunov proof for classical SMC",
    "Boundary layer optimization"
  ],
  "current_phase": "Implementing adaptive SMC proofs",
  "dependencies": ["agent_pso"],
  "artifacts": [
    ".artifacts/LT-4/lyapunov_classical.pdf",
    ".artifacts/LT-4/boundary_layer_results.json"
  ]
}
\end{lstlisting}

% =============================================================================
% SECTION 4: CHECKPOINT API \& USAGE
% =============================================================================
\section{Checkpoint API \& Usage}

\subsection{Mandatory Checkpoint Calls}
\begin{lstlisting}[style=pythonstyle]
from .ai_workspace.tools.checkpoints.agent_checkpoint import (
    checkpoint_plan_approved,
    checkpoint_agent_launched,
    checkpoint_agent_progress,
    checkpoint_agent_complete,
    checkpoint_agent_failed
)

# 1. User Approves Plan
checkpoint_plan_approved(
    task_id="LT-4",
    plan_summary="Lyapunov stability proofs for 7 controllers",
    hours=46.0,
    agents=["control_systems", "pso", "documentation"],
    deliverables=["proofs.pdf", "benchmarks.json", "theory_guide.md"]
)

# 2. Launch Each Agent
checkpoint_agent_launched(
    task_id="LT-4",
    agent_id="agent_control_systems",
    role="Implement Lyapunov proofs",
    hours_allocated=18.0
)

# 3. Progress Updates (every 5-10 minutes)
checkpoint_agent_progress(
    task_id="LT-4",
    agent_id="agent_control_systems",
    hours_completed=8.5,
    deliverables_completed=["Lyapunov proof for classical SMC"],
    current_phase="Implementing adaptive SMC proofs"
)

# 4. Agent Completes
checkpoint_agent_complete(
    task_id="LT-4",
    agent_id="agent_control_systems",
    hours_completed=18.0,
    deliverables=["lyapunov_proofs.pdf"],
    summary="7 controllers validated, 3 new theorems proven"
)

# Or Agent Fails
checkpoint_agent_failed(
    task_id="LT-4",
    agent_id="agent_control_systems",
    hours_completed=12.5,
    reason="Token limit exceeded at 180K tokens",
    recovery_recommendation="Resume from checkpoint_0012 at adaptive SMC proof"
)
\end{lstlisting}

\subsection{Recovery Commands}
\begin{lstlisting}[style=bashstyle]
# Standard Recovery Workflow
/recover                     # Load project state, list incomplete tasks

/resume LT-4 agent_control_systems
                            # Resume specific agent from last checkpoint

# Advanced Recovery
python .ai_workspace/tools/recovery/recover_project.sh
python .ai_workspace/tools/checkpoints/analyze_checkpoints.py --task LT-4
\end{lstlisting}

% =============================================================================
% SECTION 5: QUALITY GATES SYSTEM
% =============================================================================
\section{Quality Gates System}

\subsection{The 8 Quality Gates}
\begin{enumerate}
  \item \textbf{Test Coverage}: Critical paths $\geq 95\%$, overall $\geq 85\%$
  \item \textbf{Critical Issues}: Zero high-severity bugs allowed
  \item \textbf{Memory Safety}: All controllers use weakref, no circular refs
  \item \textbf{Documentation}: All public APIs have docstrings + examples
  \item \textbf{Linting}: Ruff/Pylint score $\geq 9.0/10$
  \item \textbf{Type Safety}: MyPy strict mode, no \texttt{Any} types
  \item \textbf{Performance}: Benchmarks within 5\% of baseline
  \item \textbf{Integration}: All 11 MCP servers operational
\end{enumerate}

\subsection{Pass Requirements}
\begin{center}
\begin{tabular}{lcc}
\toprule
\textbf{Phase} & \textbf{Min Gates} & \textbf{Critical Gates} \\
\midrule
Development & 5/8 & Test Coverage \\
Pre-Merge & 7/8 & Coverage + Critical Issues \\
Production & 8/8 & All gates mandatory \\
\bottomrule
\end{tabular}
\end{center}

\subsection{Automated Validation}
\begin{lstlisting}[style=bashstyle]
# Run Quality Gates
python .ai_workspace/tools/quality/validate_quality_gates.py

# Output Example
[OK] Gate 1: Test Coverage (87% overall, 96% critical) PASS
[OK] Gate 2: Critical Issues (0 found) PASS
[OK] Gate 3: Memory Safety (11/11 tests pass) PASS
[OK] Gate 4: Documentation (98% API coverage) PASS
[WARNING] Gate 5: Linting (score 8.7/10) FAIL
[OK] Gate 6: Type Safety (MyPy strict, 0 errors) PASS
[OK] Gate 7: Performance (baselines within 3%) PASS
[OK] Gate 8: MCP Integration (11/11 servers) PASS

[INFO] Result: 7/8 gates pass (PRE-MERGE READY)
[ACTION] Fix linting issues to achieve 8/8 for production
\end{lstlisting}

% =============================================================================
% SECTION 6: CASE STUDY - PHASE 3 UI OVERHAUL
% =============================================================================
\section{Case Study: Phase 3 UI Overhaul}

\subsection{Task Overview}
\begin{itemize}
  \item \textbf{Goal}: Achieve WCAG 2.1 Level AA accessibility across Streamlit UI
  \item \textbf{Scope}: 34 issues (GitHub milestones UI-001 through UI-034)
  \item \textbf{Duration}: 8 days (October 9-17, 2025)
  \item \textbf{Complexity}: 18 design tokens, 4 breakpoints, 12 color palettes
\end{itemize}

\subsection{Orchestration Strategy}
\begin{enumerate}
  \item \textbf{Week 1 (Days 1-3)}: Foundation work (Integration + Documentation agents)
  \begin{itemize}
    \item Integration Agent: Create design system tokens
    \item Documentation Agent: Write accessibility guidelines
  \end{itemize}

  \item \textbf{Week 2 (Days 4-6)}: Implementation (Control Systems + PSO agents)
  \begin{itemize}
    \item Control Systems Agent: Refactor UI components to use tokens
    \item PSO Agent: Optimize color contrast ratios
  \end{itemize}

  \item \textbf{Week 3 (Days 7-8)}: Validation (Testing + Code Beauty agents)
  \begin{itemize}
    \item Testing Agent: Puppeteer WCAG audits
    \item Code Beauty Agent: Apply formatters, fix linting
  \end{itemize}
\end{enumerate}

\subsection{Checkpoint Events}
\begin{itemize}
  \item \textbf{Checkpoint 1 (Day 2, 3h)}: Token limit during design token creation
  \item \textbf{Recovery 1}: Resumed from \texttt{checkpoint\_0008}, lost 0 work
  \item \textbf{Checkpoint 2 (Day 5, 6h)}: Token limit during component refactoring
  \item \textbf{Recovery 2}: Resumed from \texttt{checkpoint\_0015}, lost 0 work
  \item \textbf{Checkpoint 3 (Day 7, 2h)}: Intentional save before WCAG audits
\end{itemize}

\subsection{Final Results}
\begin{itemize}
  \item \textbf{Success}: 34/34 issues resolved, WCAG AA achieved
  \item \textbf{Quality Gates}: 8/8 passing (production-ready)
  \item \textbf{Token Efficiency}: 3 recoveries, 0 work lost
  \item \textbf{Documentation}: 47 pages of UI guidelines generated
\end{itemize}

% =============================================================================
% SECTION 7: MULTI-ACCOUNT RECOVERY
% =============================================================================
\section{Multi-Account Recovery}

\subsection{Problem Statement}
\begin{itemize}
  \item Each Claude Code account has 200K token limit per conversation
  \item Large refactorings (e.g., 46-hour research roadmaps) exceed single account capacity
  \item Need to resume work across accounts WITHOUT losing context
\end{itemize}

\subsection{Solution: Cross-Account Checkpoints}
\begin{enumerate}
  \item \textbf{Account A}: Works until 180K tokens, hits checkpoint
  \item \textbf{Checkpoint System}: Saves state to \texttt{.ai\_workspace/state/checkpoints/}
  \item \textbf{Git Commit}: Push checkpoint files to remote
  \item \textbf{Account B}: Pulls latest commits, runs \texttt{/recover}
  \item \textbf{Recovery}: Loads checkpoint, resumes work with full context
\end{enumerate}

\subsection{Multi-Account Workflow}
\begin{lstlisting}[style=bashstyle]
# Account A (reaches token limit)
checkpoint_agent_progress(...)   # Save progress
git add .ai_workspace/state/checkpoints/
git commit -m "chore: Save checkpoint for LT-4 agent_control_systems"
git push origin main

# Switch to Account B
D:\Tools\Claude\Switch-ClaudeAccount.ps1
cd D:\Projects\main
git pull origin main

# Account B (resume work)
/recover                         # Load checkpoints
/resume LT-4 agent_control_systems
\end{lstlisting}

\subsection{Cross-Account Safety}
\begin{itemize}
  \item \textbf{Conflict Prevention}: Never run same agent on multiple accounts simultaneously
  \item \textbf{Version Control}: All checkpoints tracked via git, merge conflicts resolved
  \item \textbf{State Validation}: Recovery script verifies checkpoint integrity before resume
\end{itemize}

% =============================================================================
% SECTION 8: ADVANCED ORCHESTRATION PATTERNS
% =============================================================================
\section{Advanced Orchestration Patterns}

\subsection{Pattern 1: Sequential Pipeline}
\textbf{Use Case}: Tasks with strict dependencies (e.g., Integration -> Documentation -> Testing)

\begin{lstlisting}[style=pythonstyle]
# Launch agents sequentially
integration_result = checkpoint_agent_launched(...)
wait_for_completion(integration_result)

doc_result = checkpoint_agent_launched(...)
wait_for_completion(doc_result)

test_result = checkpoint_agent_launched(...)
\end{lstlisting}

\subsection{Pattern 2: Parallel Fan-Out}
\textbf{Use Case}: Independent tasks (e.g., PSO optimization + Documentation)

\begin{lstlisting}[style=pythonstyle]
# Launch multiple agents in parallel
agents = [
    checkpoint_agent_launched("LT-4", "agent_pso", ...),
    checkpoint_agent_launched("LT-4", "agent_doc", ...),
    checkpoint_agent_launched("LT-4", "agent_beauty", ...)
]

# Wait for all to complete
wait_for_all(agents)
\end{lstlisting}

\subsection{Pattern 3: Hierarchical Delegation}
\textbf{Use Case}: Sub-agent spawning (e.g., Documentation agent spawns Sphinx builder)

\begin{lstlisting}[style=pythonstyle]
# Parent agent launches child sub-agents
doc_agent = checkpoint_agent_launched("LT-4", "agent_doc", ...)

# Within Documentation Agent
sphinx_builder = spawn_sub_agent("sphinx_builder")
api_generator = spawn_sub_agent("api_doc_generator")

# Parent waits for children
wait_for_children([sphinx_builder, api_generator])
checkpoint_agent_complete("LT-4", "agent_doc", ...)
\end{lstlisting}

% =============================================================================
% SECTION 9: MONITORING \& DEBUGGING
% =============================================================================
\section{Monitoring \& Debugging}

\subsection{Real-time Monitoring}
\begin{lstlisting}[style=bashstyle]
# Watch checkpoint updates in real-time
watch -n 5 'ls -lh .ai_workspace/state/checkpoints/'

# Monitor agent progress
python .ai_workspace/tools/checkpoints/analyze_checkpoints.py --live

# Output Example
[INFO] Active agents: 3
  - agent_control_systems (LT-4): 8.5h / 18h (47% complete)
  - agent_pso (LT-4): 12h / 12h (100% complete)
  - agent_doc (LT-4): 6h / 15h (40% complete)

[WARNING] agent_control_systems: No checkpoint in 12 minutes (expected <10 min)
[ACTION] Check if agent is blocked or needs manual intervention
\end{lstlisting}

\subsection{Debugging Failed Recoveries}
\begin{itemize}
  \item \textbf{Symptom}: \texttt{/recover} fails to find checkpoints
  \item \textbf{Cause 1}: Checkpoints not committed to git
  \item \textbf{Fix}: \texttt{git add .ai\_workspace/state/checkpoints/ \&\& git commit}

  \item \textbf{Cause 2}: Checkpoint file corrupted (invalid JSON)
  \item \textbf{Fix}: Restore from previous commit (\texttt{git checkout HEAD\textasciitilde 1 -- .ai\_workspace/state/})

  \item \textbf{Cause 3}: Agent dependencies not resolved
  \item \textbf{Fix}: Launch dependency agents first (\texttt{/resume LT-4 agent\_pso} before \texttt{agent\_control\_systems})
\end{itemize}

% =============================================================================
% SECTION 10: BEST PRACTICES
% =============================================================================
\section{Best Practices}

\subsection{Checkpoint Frequency}
\begin{itemize}
  \item \textbf{Too Frequent (<5 min)}: Overhead slows agents, git bloat
  \item \textbf{Too Infrequent (>15 min)}: Risk losing significant work on crash
  \item \textbf{Sweet Spot}: 5-10 minutes OR after completing each deliverable
\end{itemize}

\subsection{Task Granularity}
\begin{itemize}
  \item \textbf{Too Large (>20h)}: Agent gets lost, hard to resume
  \item \textbf{Too Small (<2h)}: Orchestration overhead dominates
  \item \textbf{Sweet Spot}: 8-15 hours per agent task
\end{itemize}

\subsection{Agent Coordination}
\begin{itemize}
  \item \textbf{DO}: Use explicit dependency tracking in checkpoints
  \item \textbf{DO}: Launch independent agents in parallel for speed
  \item \textbf{DON'T}: Assume agents can communicate directly (they can't)
  \item \textbf{DON'T}: Retry failed agents without fixing root cause
\end{itemize}

\subsection{Git Hygiene}
\begin{itemize}
  \item Commit checkpoints immediately after creation
  \item Use descriptive messages: \texttt{"chore: Checkpoint LT-4 agent\_pso at 8h"}
  \item Prune old checkpoints after task completion (keep last 3)
  \item Never gitignore \texttt{.ai\_workspace/state/checkpoints/}
\end{itemize}

% =============================================================================
% SECTION 11: TROUBLESHOOTING
% =============================================================================
\section{Troubleshooting}

\subsection{Common Issues}
\begin{center}
\begin{tabular}{l}p{6cm}}
\toprule
\textbf{Symptom} & \textbf{Solution} \\
\midrule
Agent stuck, no progress & Check if waiting on dependency agent, launch it first \\
Checkpoint not found & Verify \texttt{git pull} completed, check \texttt{.ai\_workspace/state/checkpoints/} exists \\
Recovery loads wrong task & Specify task ID: \texttt{/resume LT-4 agent\_pso} \\
Quality gates fail & Run \texttt{validate\_quality\_gates.py}, fix failing gates \\
\bottomrule
\end{tabular}
\end{center}

\subsection{Emergency Recovery}
\begin{lstlisting}[style=bashstyle]
# If all else fails, manual recovery
cd .ai_workspace/state/checkpoints/
ls -lt | head -5              # Find most recent checkpoint
cat checkpoint_0023.json      # Read agent state manually
# Resume work based on "current_phase" and "deliverables_completed"
\end{lstlisting}

% =============================================================================
% CHECKLIST: MULTI-AGENT ORCHESTRATION
% =============================================================================
\section*{Checklist: Multi-Agent Orchestration}
\begin{itemize}
  \item[$\square$] \textbf{Plan}: Decompose task into 3-6 agent workstreams
  \item[$\square$] \textbf{Checkpoint}: Call \texttt{checkpoint\_plan\_approved()} when user approves
  \item[$\square$] \textbf{Launch}: Start each agent with \texttt{checkpoint\_agent\_launched()}
  \item[$\square$] \textbf{Monitor}: Agent calls \texttt{checkpoint\_agent\_progress()} every 5-10 min
  \item[$\square$] \textbf{Complete}: Finalize with \texttt{checkpoint\_agent\_complete()} or \texttt{checkpoint\_agent\_failed()}
  \item[$\square$] \textbf{Validate}: Run quality gates (\texttt{validate\_quality\_gates.py})
  \item[$\square$] \textbf{Commit}: Push checkpoints to git after each session
  \item[$\square$] \textbf{Recover}: Use \texttt{/recover} + \texttt{/resume} on token limit
\end{itemize}

% =============================================================================
% NEXT STEPS
% =============================================================================
\section*{Next Steps}
\begin{itemize}
  \item \textbf{E018}: Testing philosophy - coverage standards and validation strategies
  \item \textbf{E019}: Production safety - memory management and thread safety
  \item \textbf{E020}: MCP integration - auto-trigger strategy and server orchestration
\end{itemize}

\end{document}
