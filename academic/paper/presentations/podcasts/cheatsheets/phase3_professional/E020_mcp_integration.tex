% =============================================================================
% Episode 020: MCP Integration & Auto-trigger Strategy
% Series: DIP-SMC-PSO Professional Toolkit
% Phase: 3 (Professional Practice)
% Duration: ~25 minutes | Pages: 12-14 | Complexity: Professional
% Dependencies: E017 (multi-agent orchestration), E019 (production safety)
% =============================================================================

% ==============================================================================
% MASTER TEMPLATE FOR PODCAST EPISODE CHEATSHEET PDFs
% ==============================================================================
% Purpose: Beginner-friendly, colorful, multi-page study guides (2-4 pages)
% Target Audience: Complete beginners (Path 0 learners)
% Visual Style: Infographic-style with vibrant colors, icons, callout boxes
% ==============================================================================

\documentclass[11pt,a4paper]{article}

% ------------------------------------------------------------------------------
% GEOMETRY & LAYOUT
% ------------------------------------------------------------------------------
\usepackage[top=1.5cm, bottom=1.5cm, left=1.5cm, right=1.5cm, headheight=14pt]{geometry}
\usepackage{multicol}
\usepackage{fancyhdr}
\pagestyle{fancy}

% ------------------------------------------------------------------------------
% COLOR PALETTE (Vibrant & Beginner-Friendly)
% ------------------------------------------------------------------------------
\usepackage{xcolor}
\definecolor{primary}{RGB}{41, 128, 185}      % Blue - Main concepts, headers
\definecolor{secondary}{RGB}{39, 174, 96}     % Green - Success, examples
\definecolor{accent}{RGB}{230, 126, 34}       % Orange - Important notes
\definecolor{warning}{RGB}{231, 76, 60}       % Red - Common pitfalls
\definecolor{background}{RGB}{236, 240, 241}  % Light gray - Callout boxes
\definecolor{codeblock}{RGB}{44, 62, 80}      % Dark blue-gray - Code background
\definecolor{highlight}{RGB}{241, 196, 15}    % Yellow - Highlights

% ------------------------------------------------------------------------------
% TIKZ & GRAPHICS
% ------------------------------------------------------------------------------
\usepackage{tikz}
\usetikzlibrary{shapes, arrows, positioning, calc, shadows, decorations.pathreplacing, backgrounds, fit}
\usepackage{graphicx}
\usepackage{float}

% TikZ styles for consistent diagrams
\tikzstyle{block} = [rectangle, draw, fill=primary!20, text width=5em, text centered, rounded corners, minimum height=3em, drop shadow]
\tikzstyle{arrow} = [thick,->,>=stealth]
\tikzstyle{process} = [rectangle, draw, fill=secondary!20, text width=6em, text centered, rounded corners, minimum height=3em]
\tikzstyle{decision} = [diamond, draw, fill=accent!20, text width=4.5em, text badly centered, inner sep=0pt]
\tikzstyle{cloud} = [ellipse, draw, fill=background, text width=5em, text centered, minimum height=2.5em]

% ------------------------------------------------------------------------------
% BOXES & CALLOUTS (tcolorbox)
% ------------------------------------------------------------------------------
\usepackage{tcolorbox}
\tcbuselibrary{skins, breakable, raster}

% Key Concept Box (Blue)
\newtcolorbox{keypoint}{
    enhanced,
    colback=primary!10,
    colframe=primary,
    fonttitle=\bfseries,
    title=\faLightbulb\ Key Concept,
    attach boxed title to top left={yshift=-2mm, xshift=5mm},
    boxed title style={colback=primary},
    breakable
}

% Example Box (Green)
\newtcolorbox{example}{
    enhanced,
    colback=secondary!10,
    colframe=secondary,
    fonttitle=\bfseries,
    title=\faCode\ Example,
    attach boxed title to top left={yshift=-2mm, xshift=5mm},
    boxed title style={colback=secondary},
    breakable
}

% Warning Box (Red)
\newtcolorbox{warning}{
    enhanced,
    colback=warning!10,
    colframe=warning,
    fonttitle=\bfseries,
    title=\faExclamationTriangle\ Common Pitfall,
    attach boxed title to top left={yshift=-2mm, xshift=5mm},
    boxed title style={colback=warning},
    breakable
}

% Tip Box (Orange)
\newtcolorbox{tip}{
    enhanced,
    colback=accent!10,
    colframe=accent,
    fonttitle=\bfseries,
    title=\faLightbulb\ Pro Tip,
    attach boxed title to top left={yshift=-2mm, xshift=5mm},
    boxed title style={colback=accent},
    breakable
}

% Summary Box (Light gray)
\newtcolorbox{summary}{
    enhanced,
    colback=background,
    colframe=codeblock,
    fonttitle=\bfseries,
    title=\faListUl\ Quick Summary,
    attach boxed title to top left={yshift=-2mm, xshift=5mm},
    boxed title style={colback=codeblock},
    breakable
}

% ------------------------------------------------------------------------------
% ICONS & SYMBOLS (fontawesome5)
% ------------------------------------------------------------------------------
\usepackage{fontawesome5}

% Custom icon commands for consistency
\newcommand{\iconKey}{\textcolor{primary}{\faLightbulb}}
\newcommand{\iconCode}{\textcolor{secondary}{\faCode}}
\newcommand{\iconWarning}{\textcolor{warning}{\faExclamationTriangle}}
\newcommand{\iconTip}{\textcolor{accent}{\faInfoCircle}}
\newcommand{\iconLink}{\textcolor{primary}{\faLink}}
\newcommand{\iconBook}{\textcolor{secondary}{\faBook}}
\newcommand{\iconTarget}{\textcolor{accent}{\faBullseye}}

% ------------------------------------------------------------------------------
% TYPOGRAPHY & FONTS
% ------------------------------------------------------------------------------
\usepackage{lmodern}
\usepackage[T1]{fontenc}
\usepackage[utf8]{inputenc}

% Section styling
\usepackage{titlesec}
\titleformat{\section}{\Large\bfseries\color{primary}}{\thesection}{1em}{}[\titlerule]
\titleformat{\subsection}{\large\bfseries\color{secondary}}{\thesubsection}{1em}{}

% ------------------------------------------------------------------------------
% CODE LISTINGS
% ------------------------------------------------------------------------------
\usepackage{listings}
\lstset{
    basicstyle=\ttfamily\footnotesize\color{white},
    backgroundcolor=\color{codeblock},
    keywordstyle=\color{primary!80},
    commentstyle=\color{secondary!60}\itshape,
    stringstyle=\color{accent!80},
    numbers=left,
    numberstyle=\tiny\color{white!50},
    stepnumber=1,
    numbersep=8pt,
    frame=single,
    rulecolor=\color{codeblock},
    breaklines=true,
    breakatwhitespace=true,
    tabsize=4,
    captionpos=b
}

% Python-specific styling
\lstdefinestyle{python}{
    language=Python,
    morekeywords={self, def, class, import, from, as, return, if, else, elif, for, while, True, False, None}
}

% YAML-specific styling
\lstdefinestyle{yaml}{
    basicstyle=\ttfamily\footnotesize,
    showstringspaces=false,
    commentstyle=\color{secondary!60}\itshape,
    keywordstyle=\color{primary!80}
}

% ------------------------------------------------------------------------------
% HYPERLINKS
% ------------------------------------------------------------------------------
\usepackage{hyperref}
\hypersetup{
    colorlinks=true,
    linkcolor=primary,
    urlcolor=secondary,
    citecolor=accent
}

% ------------------------------------------------------------------------------
% HEADER & FOOTER CUSTOMIZATION
% ------------------------------------------------------------------------------
\fancyhf{}
\fancyhead[L]{\textcolor{primary}{\textbf{DIP-SMC-PSO Podcast Cheatsheet}}}
\fancyhead[R]{\textcolor{secondary}{\episodetitle}}
\fancyfoot[C]{\textcolor{primary}{\thepage}}
\renewcommand{\headrulewidth}{0.5pt}
\renewcommand{\footrulewidth}{0.5pt}

% ------------------------------------------------------------------------------
% CUSTOM COMMANDS
% ------------------------------------------------------------------------------

% Episode title command (to be defined in each episode file)
\newcommand{\episodetitle}{Episode Title}

% Learning objective command
\newcommand{\learningobjective}[1]{%
    \begin{center}
    \begin{tcolorbox}[colback=highlight!30, colframe=accent, width=0.9\textwidth]
    \iconTarget\ \textbf{Learning Objective:} #1
    \end{tcolorbox}
    \end{center}
}

% Quick reference command (for formulas/code snippets)
\newcommand{\quickref}[2]{%
    \begin{tcolorbox}[colback=background, colframe=primary, title=\faBookmark\ #1]
    #2
    \end{tcolorbox}
}

% Resource link command
\newcommand{\resourcelink}[2]{%
    \iconLink\ \href{#1}{\textcolor{secondary}{#2}}
}

% ------------------------------------------------------------------------------
% TITLE PAGE FORMATTING
% ------------------------------------------------------------------------------
\usepackage{afterpage}

\newcommand{\makeepisodetitle}[4]{%
    \begin{titlepage}
    \begin{tikzpicture}[remember picture, overlay]
        % Background gradient
        \fill[primary!20] (current page.south west) rectangle (current page.north east);

        % Title box
        \node[
            fill=white,
            rounded corners=10pt,
            drop shadow,
            text width=0.8\paperwidth,
            align=center
        ] at (current page.center) {
            \Huge\bfseries\color{primary} #1 \\[1em]
            \Large\color{secondary} #2 \\[2em]
            \large\color{codeblock} Part #3 $\cdot$ Duration: #4 \\[1em]
            \normalsize\color{codeblock} \textit{Beginner-Friendly Visual Study Guide}
        };

        % Footer
        \node[
            anchor=south,
            text width=0.9\paperwidth,
            align=center
        ] at (current page.south) {
            \color{primary}\rule{0.8\paperwidth}{0.5pt} \\[0.5em]
            \small\color{codeblock}
            \textbf{Repository:} \url{https://github.com/theSadeQ/dip-smc-pso} \\
            \textbf{Documentation:} academic/paper/presentations/podcasts/episodes/ \\[1em]
        };
    \end{tikzpicture}
    \end{titlepage}
}

% ------------------------------------------------------------------------------
% MATH & EQUATIONS
% ------------------------------------------------------------------------------
\usepackage{amsmath, amssymb, amsthm}
\usepackage{mathtools}

% Equation box (highlighted equations)
\newcommand{\eqbox}[1]{%
    \begin{tcolorbox}[colback=primary!5, colframe=primary, boxrule=1pt]
    \begin{equation}
    #1
    \end{equation}
    \end{tcolorbox}
}

% ------------------------------------------------------------------------------
% TABLES
% ------------------------------------------------------------------------------
\usepackage{booktabs}
\usepackage{array}
\usepackage{multirow}
\usepackage{colortbl}

% Custom table colors
\newcommand{\tableheadcolor}{\rowcolor{primary!30}}

% ------------------------------------------------------------------------------
% END OF PREAMBLE
% ------------------------------------------------------------------------------


% Episode Metadata
\title{\textbf{E020: MCP Integration}}
\def\episodenumber{020}
\def\episodetitle{MCP Integration \& Auto-trigger Strategy}
\def\episodecategory{Professional Practice}
\def\difficulty{Professional}

\begin{document}
\makeepisodetitle

% =============================================================================
% SECTION 1: OVERVIEW
% =============================================================================
\section{Overview}

\subsection{What You'll Learn}
\begin{itemize}
  \item \textbf{MCP Architecture}: 12 specialized servers (filesystem, github, pytest, pandas, etc.)
  \item \textbf{Auto-trigger Strategy}: Keyword-based orchestration (NO user confirmation needed)
  \item \textbf{Server Chaining}: Combine 3-5 MCPs for complete workflows
  \item \textbf{Integration}: .mcp.json configuration and debugging workflows
\end{itemize}

\subsection{Why This Matters}
\begin{highlightbox}
\textbf{Problem}: Manual tool selection slows AI workflows (e.g., "Should I use filesystem or grep?").

\textbf{Solution}: MCPs auto-trigger based on task keywords (data analysis -> pandas-mcp, testing -> pytest-mcp, UI -> puppeteer).

\textbf{Impact}: Phase 5 research (11 tasks) leveraged 8/12 MCPs automatically, reducing manual tool selection by 70\%.
\end{highlightbox}

% =============================================================================
% SECTION 2: THE 12 MCP SERVERS
% =============================================================================
\section{The 12 MCP Servers}

\subsection{Core Servers (4)}
\begin{enumerate}
  \item \textbf{filesystem}: File operations (read, write, list, search)
  \item \textbf{github}: Issues, PRs, checks, releases
  \item \textbf{sequential-thinking}: Planning, debugging, reasoning
  \item \textbf{puppeteer}: UI testing, browser automation
\end{enumerate}

\subsection{Development Servers (4)}
\begin{enumerate}
  \item \textbf{pytest-mcp}: Test debugging, coverage analysis
  \item \textbf{git-mcp}: Advanced Git operations (rebase, cherry-pick, blame)
  \item \textbf{mcp-analyzer}: Code quality, linting, static analysis
  \item \textbf{mcp-debugger}: API testing, HTTP requests
\end{enumerate}

\subsection{Data Science Servers (3)}
\begin{enumerate}
  \item \textbf{sqlite-mcp}: PSO optimization database queries
  \item \textbf{pandas-mcp}: Data analysis, CSV/JSON processing
  \item \textbf{numpy-mcp}: Numerical compute, matrix operations
\end{enumerate}

\subsection{Audit Server (1)}
\begin{enumerate}
  \item \textbf{lighthouse-mcp}: Accessibility audits (WCAG compliance)
\end{enumerate}

% =============================================================================
% SECTION 3: AUTO-TRIGGER KEYWORDS
% =============================================================================
\section{Auto-trigger Keywords}

\subsection{Filesystem MCP}
\textbf{Triggers}: "read file", "list directory", "search files", "find *.py"

\begin{lstlisting}[style=bashstyle]
# User: "Read all controller files"
# AI auto-triggers: filesystem.list("src/controllers/")

# User: "Find all test files"
# AI auto-triggers: filesystem.search("test_*.py")
\end{lstlisting}

\subsection{GitHub MCP}
\textbf{Triggers}: "create PR", "list issues", "check CI status", "release notes"

\begin{lstlisting}[style=bashstyle]
# User: "Create PR for LT-7 paper"
# AI auto-triggers: github.create_pr(title="feat: Add LT-7 research paper", ...)

# User: "List open issues"
# AI auto-triggers: github.list_issues(state="open")
\end{lstlisting}

\subsection{Pytest MCP}
\textbf{Triggers}: "run tests", "debug test failure", "coverage report"

\begin{lstlisting}[style=bashstyle]
# User: "Why is test_classical_smc failing?"
# AI auto-triggers: pytest_mcp.debug("tests/test_controllers/test_classical_smc.py")

# User: "Show coverage for controllers"
# AI auto-triggers: pytest_mcp.coverage("src/controllers/")
\end{lstlisting}

\subsection{Pandas MCP}
\textbf{Triggers}: "analyze CSV", "plot benchmark data", "merge datasets"

\begin{lstlisting}[style=bashstyle]
# User: "Analyze PSO convergence data"
# AI auto-triggers: pandas_mcp.read_csv("optimization_results/pso_results.csv")
# AI auto-triggers: pandas_mcp.plot(df["cost"], title="PSO Convergence")

# User: "Merge controller benchmarks"
# AI auto-triggers: pandas_mcp.merge(df1, df2, on="controller_name")
\end{lstlisting}

% =============================================================================
% SECTION 4: SERVER CHAINING PATTERNS
% =============================================================================
\section{Server Chaining Patterns}

\subsection{Pattern 1: Data Analysis Pipeline}
\textbf{Goal}: Analyze PSO optimization results (CSV -> DataFrame -> Plot)

\begin{lstlisting}[style=bashstyle]
# Chain: filesystem -> sqlite -> pandas
1. filesystem.read("optimization_results/pso_results.csv")
2. sqlite_mcp.query("SELECT * FROM optimizations WHERE cost < 0.1")
3. pandas_mcp.analyze(df)
4. pandas_mcp.plot(df["iteration"], df["cost"])
\end{lstlisting}

\subsection{Pattern 2: Testing Workflow}
\textbf{Goal}: Debug failing test, analyze coverage, fix issue

\begin{lstlisting}[style=bashstyle]
# Chain: pytest-mcp -> puppeteer -> mcp-analyzer
1. pytest_mcp.run("tests/test_controllers/")
2. pytest_mcp.debug("test_classical_smc::test_convergence")
3. puppeteer.screenshot("UI elements for manual inspection")
4. mcp_analyzer.lint("src/controllers/classical_smc.py")
\end{lstlisting}

\subsection{Pattern 3: Research Task Workflow}
\textbf{Goal}: Implement LT-7 paper generation (planning -> code -> data -> doc)

\begin{lstlisting}[style=bashstyle]
# Chain: sequential-thinking -> filesystem -> pandas -> github
1. sequential_thinking.plan("LT-7 research paper structure")
2. filesystem.create("academic/paper/publications/lt7_paper.tex")
3. pandas_mcp.analyze("benchmarks/processed/comparative_study.csv")
4. pandas_mcp.generate_figures(df, output="academic/paper/figures/")
5. github.create_pr(title="feat: Add LT-7 research paper")
\end{lstlisting}

% =============================================================================
% SECTION 5: MCP CONFIGURATION (.mcp.json)
% =============================================================================
\section{MCP Configuration}

\subsection{Configuration File Structure}
\textbf{File}: \texttt{.mcp.json}

\begin{lstlisting}[language=json]
{
  "mcpServers": {
    "filesystem": {
      "command": "npx",
      "args": ["-y", "@modelcontextprotocol/server-filesystem", "D:/Projects/main"]
    },
    "github": {
      "command": "npx",
      "args": ["-y", "@modelcontextprotocol/server-github"],
      "env": {
        "GITHUB_PERSONAL_ACCESS_TOKEN": "ghp_xxxx"
      }
    },
    "pytest-mcp": {
      "command": "python",
      "args": ["-m", "pytest_mcp"]
    },
    "pandas-mcp": {
      "command": "python",
      "args": ["-m", "pandas_mcp", "--data-dir", "benchmarks/"]
    },
    "sqlite-mcp": {
      "command": "npx",
      "args": ["-y", "@modelcontextprotocol/server-sqlite", "optimization_results/pso.db"]
    }
  }
}
\end{lstlisting}

\subsection{Environment Variables}
\begin{itemize}
  \item \textbf{GITHUB\_PERSONAL\_ACCESS\_TOKEN}: Required for github MCP
  \item \textbf{MCP\_DATA\_DIR}: Optional path for pandas/numpy data
  \item \textbf{MCP\_LOG\_LEVEL}: Debug verbosity (default: INFO)
\end{itemize}

\subsection{Server Validation}
\begin{lstlisting}[style=bashstyle]
# Verify All Servers Running
python .ai_workspace/tools/mcp/validate_servers.py

# Output Example
[OK] filesystem server: Running on port 8001
[OK] github server: Running on port 8002
[OK] pytest-mcp server: Running on port 8003
[WARNING] pandas-mcp server: Failed to start (missing numpy dependency)
[OK] sqlite-mcp server: Running on port 8005

[INFO] Status: 11/12 servers operational
[ACTION] Install numpy: pip install numpy
\end{lstlisting}

% =============================================================================
% SECTION 6: DEBUGGING MCP ISSUES
% =============================================================================
\section{Debugging MCP Issues}

\subsection{Common Issues}
\begin{center}
\begin{tabular}{ll}
\toprule
\textbf{Symptom} & \textbf{Solution} \\
\midrule
Server not starting & Check .mcp.json syntax, verify command exists \\
Timeout errors & Increase timeout in .mcp.json (default: 30s) \\
Permission denied & Verify GITHUB\_TOKEN, file permissions \\
Wrong data returned & Check server version (update via npx -y) \\
\bottomrule
\end{tabular}
\end{center}

\subsection{Debug Logging}
\begin{lstlisting}[style=bashstyle]
# Enable Debug Logs
export MCP_LOG_LEVEL=DEBUG

# View Server Logs
tail -f .ai_workspace/logs/mcp/filesystem.log
tail -f .ai_workspace/logs/mcp/github.log

# Example Debug Output
[DEBUG] filesystem: Received request read_file("src/controllers/classical_smc.py")
[DEBUG] filesystem: File size: 4523 bytes
[DEBUG] filesystem: Returning file contents
[INFO] filesystem: Request completed in 12ms
\end{lstlisting}

\subsection{MCP Debugging Workflow}
\textbf{See}: \texttt{docs/mcp-debugging/README.md} for complete workflows

\begin{enumerate}
  \item \textbf{Reproduce}: Trigger failing MCP operation
  \item \textbf{Check Logs}: Review .ai\_workspace/logs/mcp/*.log
  \item \textbf{Test Manually}: Run MCP server standalone (\texttt{npx @modelcontextprotocol/server-*})
  \item \textbf{Validate Config}: Run \texttt{validate\_servers.py}
  \item \textbf{Update Servers}: \texttt{npx -y @modelcontextprotocol/server-*} (latest versions)
\end{enumerate}

% =============================================================================
% SECTION 7: ADVANCED MCP ORCHESTRATION
% =============================================================================
\section{Advanced MCP Orchestration}

\subsection{Conditional Chaining}
\textbf{Goal}: Use different MCP based on file type

\begin{lstlisting}[language=python]
def analyze_file(file_path):
    """Auto-select MCP based on file extension."""
    if file_path.endswith('.csv'):
        return pandas_mcp.analyze(file_path)
    elif file_path.endswith('.json'):
        return filesystem.read(file_path)  # Then parse manually
    elif file_path.endswith('.db'):
        return sqlite_mcp.query(f"SELECT * FROM {table}")
    else:
        raise ValueError(f"Unsupported file type: {file_path}")
\end{lstlisting}

\subsection{Parallel MCP Execution}
\textbf{Goal}: Run multiple independent MCP operations simultaneously

\begin{lstlisting}[language=python]
import asyncio

async def parallel_analysis():
    """Run 3 MCPs in parallel."""
    results = await asyncio.gather(
        pandas_mcp.analyze("benchmarks/mt5_results.csv"),
        sqlite_mcp.query("SELECT * FROM pso_runs WHERE cost < 0.1"),
        github.list_issues(state="open")
    )
    return results

# Usage
mt5_df, pso_rows, open_issues = asyncio.run(parallel_analysis())
\end{lstlisting}

\subsection{Error Handling in Chains}
\begin{lstlisting}[language=python]
def robust_mcp_chain(file_path):
    """Chain MCPs with fallback on errors."""
    try:
        # Try pandas for CSV
        df = pandas_mcp.read_csv(file_path)
        return pandas_mcp.analyze(df)
    except PandasMCPError:
        # Fallback to filesystem + manual parsing
        content = filesystem.read(file_path)
        return parse_csv_manually(content)
    except FilesystemMCPError:
        # Final fallback to local file read
        with open(file_path) as f:
            return f.read()
\end{lstlisting}

% =============================================================================
% SECTION 8: MCP USAGE STATISTICS
% =============================================================================
\section{MCP Usage Statistics}

\subsection{Phase 5 Research Usage (11 tasks, 46 hours)}
\begin{center}
\begin{tabular}{lcc}
\toprule
\textbf{MCP Server} & \textbf{Invocations} & \textbf{Use Cases} \\
\midrule
filesystem & 234 & Read controllers, list tests \\
pandas-mcp & 89 & Analyze benchmarks, plot figures \\
pytest-mcp & 67 & Debug test failures \\
github & 45 & Create PRs, list issues \\
sqlite-mcp & 34 & Query PSO database \\
sequential-thinking & 28 & Plan LT-7 paper structure \\
git-mcp & 18 & Cherry-pick commits \\
numpy-mcp & 12 & Matrix operations \\
\textbf{Total} & \textbf{527} & \textbf{8/12 servers used} \\
\bottomrule
\end{tabular}
\end{center}

\subsection{Efficiency Gains}
\begin{itemize}
  \item \textbf{Manual Tool Selection}: Reduced by 70\% (AI auto-selects MCP)
  \item \textbf{Context Switching}: Eliminated 43 "Which tool should I use?" decisions
  \item \textbf{Workflow Speed}: Data analysis tasks 2.3x faster (pandas-mcp vs. manual)
  \item \textbf{Error Rate}: 18\% fewer API errors (MCPs handle retries)
\end{itemize}

% =============================================================================
% SECTION 9: CASE STUDY - LT-7 PAPER GENERATION
% =============================================================================
\section{Case Study: LT-7 Paper Generation}

\subsection{Task Overview}
\begin{itemize}
  \item \textbf{Goal}: Generate research paper with 14 figures, 50+ citations
  \item \textbf{Duration}: 12 hours (long-term research task)
  \item \textbf{Complexity}: Data analysis, LaTeX generation, figure automation
\end{itemize}

\subsection{MCP Workflow}
\begin{enumerate}
  \item \textbf{Planning (sequential-thinking)}:
  \begin{lstlisting}[style=bashstyle]
  sequential_thinking.plan("LT-7 research paper: 7 controllers, comparative study")
  # Output: 5-section structure, figure requirements, data sources
  \end{lstlisting}

  \item \textbf{Data Collection (pandas-mcp + sqlite-mcp)}:
  \begin{lstlisting}[style=bashstyle]
  pandas_mcp.read_csv("benchmarks/processed/comparative_study.csv")
  sqlite_mcp.query("SELECT * FROM controller_metrics WHERE task='LT-7'")
  \end{lstlisting}

  \item \textbf{Figure Generation (pandas-mcp + numpy-mcp)}:
  \begin{lstlisting}[style=bashstyle]
  pandas_mcp.plot(df["controller"], df["IAE"], output="figure_01.png")
  numpy_mcp.heatmap(correlation_matrix, output="figure_02.png")
  # Generated 14 figures automatically
  \end{lstlisting}

  \item \textbf{LaTeX Creation (filesystem + sequential-thinking)}:
  \begin{lstlisting}[style=bashstyle]
  filesystem.create("academic/paper/publications/lt7_paper_v1.tex")
  sequential_thinking.structure("Intro -> Methods -> Results -> Discussion")
  \end{lstlisting}

  \item \textbf{Git + GitHub (git-mcp + github)}:
  \begin{lstlisting}[style=bashstyle]
  git_mcp.commit("feat(LT-7): Add research paper draft v1")
  github.create_pr(title="feat: LT-7 research paper submission-ready")
  \end{lstlisting}
\end{enumerate}

\subsection{Results}
\begin{itemize}
  \item \textbf{MCPs Used}: 6 servers (sequential-thinking, pandas, sqlite, numpy, filesystem, github)
  \item \textbf{Total Invocations}: 87 MCP calls over 12 hours
  \item \textbf{Manual Interventions}: 5 (only for final LaTeX formatting)
  \item \textbf{Final Output}: LT-7 paper v2.1 (submission-ready, 14 figures, automation scripts)
\end{itemize}

% =============================================================================
% SECTION 10: BEST PRACTICES
% =============================================================================
\section{Best Practices}

\subsection{When to Use MCPs}
\begin{itemize}
  \item \textbf{DO}: Use MCPs for structured data (CSV, JSON, DB queries)
  \item \textbf{DO}: Use MCPs for API operations (GitHub PRs, pytest runs)
  \item \textbf{DO}: Chain 3-5 MCPs for complex workflows
  \item \textbf{DON'T}: Use MCPs for simple file reads (built-in tools faster)
  \item \textbf{DON'T}: Use MCPs when direct API calls are simpler
\end{itemize}

\subsection{Performance Optimization}
\begin{itemize}
  \item \textbf{Cache Results}: Store MCP outputs to avoid redundant calls
  \item \textbf{Batch Operations}: Combine multiple queries into single MCP call
  \item \textbf{Parallel Execution}: Use asyncio for independent MCP operations
  \item \textbf{Timeouts}: Set appropriate timeouts (default 30s, increase for large data)
\end{itemize}

\subsection{Security Considerations}
\begin{itemize}
  \item \textbf{Token Management}: Store GITHUB\_TOKEN in environment, NOT .mcp.json
  \item \textbf{File Permissions}: Restrict filesystem MCP to project root (\texttt{D:/Projects/main})
  \item \textbf{SQL Injection}: Use parameterized queries with sqlite-mcp
  \item \textbf{Log Sanitization}: Scrub sensitive data from MCP logs
\end{itemize}

% =============================================================================
% SECTION 11: FUTURE ENHANCEMENTS
% =============================================================================
\section{Future Enhancements}

\subsection{Planned MCP Servers}
\begin{enumerate}
  \item \textbf{latex-mcp}: Compile LaTeX, generate PDFs, extract citations
  \item \textbf{sphinx-mcp}: Build docs, check cross-references, validate links
  \item \textbf{streamlit-mcp}: UI component testing, WCAG audits
  \item \textbf{benchmark-mcp}: Run pytest-benchmark, compare baselines
\end{enumerate}

\subsection{MCP Orchestration Layer}
\textbf{Goal}: Higher-level abstraction for common workflows

\begin{lstlisting}[language=python]
# Instead of manually chaining MCPs
result = orchestrator.run_workflow("data_analysis", {
    "input": "benchmarks/mt5_results.csv",
    "output": "academic/paper/figures/",
    "format": "publication-ready"
})

# Orchestrator auto-selects: filesystem -> pandas -> numpy -> filesystem
\end{lstlisting}

\subsection{MCP Monitoring Dashboard}
\begin{itemize}
  \item Real-time MCP invocation tracking (calls/minute)
  \item Latency metrics (P50, P95, P99)
  \item Error rate monitoring (retries, failures)
  \item Cost analysis (API rate limits, token usage)
\end{itemize}

% =============================================================================
% CHECKLIST: MCP INTEGRATION
% =============================================================================
\section*{Checklist: MCP Integration}
\begin{itemize}
  \item[$\square$] \textbf{Configure}: Set up .mcp.json with 12 servers
  \item[$\square$] \textbf{Validate}: Run \texttt{validate\_servers.py} (11/12 operational)
  \item[$\square$] \textbf{Environment}: Set GITHUB\_TOKEN, MCP\_DATA\_DIR
  \item[$\square$] \textbf{Debug}: Enable debug logging (\texttt{MCP\_LOG\_LEVEL=DEBUG})
  \item[$\square$] \textbf{Chain}: Combine 3-5 MCPs for complete workflows
  \item[$\square$] \textbf{Optimize}: Use parallel execution for independent operations
  \item[$\square$] \textbf{Security}: Token management, file permissions, SQL parameterization
  \item[$\square$] \textbf{Monitor}: Track usage statistics (invocations, latency, errors)
\end{itemize}

% =============================================================================
% NEXT STEPS
% =============================================================================
\section*{Next Steps}
\begin{itemize}
  \item \textbf{E021}: Maintenance mode, future vision, and Phase 3 Professional Practice wrap-up
  \item \textbf{MCP Enhancements}: Implement latex-mcp, sphinx-mcp, benchmark-mcp
  \item \textbf{Orchestration Layer}: High-level workflow abstractions
\end{itemize}

\end{document}
