\documentclass[11pt,a4paper]{article}
\usepackage[utf8]{inputenc}
\usepackage[margin=1in]{geometry}
\usepackage{hyperref}
\usepackage{enumitem}
\usepackage{fancyhdr}

\pagestyle{fancy}
\fancyhf{}
\rhead{E014}
\lhead{DIP-SMC-PSO Learning Episode}
\rfoot{\thepage}

\title{\Large\textbf{E014: Development Infrastructure}}
\author{DIP-SMC-PSO Educational Series}
\date{\today}

\begin{document}

\maketitle

\section*{Overview}

This episode covers development infrastructure from the DIP-SMC-PSO project.

\textbf{Part:} Part3 Advanced

\textbf{Duration:} 15-20 minutes

\textbf{Source:} Comprehensive Presentation Materials

\newpage

\section{Educational System Overview}

**Mission:** Democratize access to advanced control theory

        44 episodes covering Phases 1-4, ~40 hours audio content \\
        Convert written materials to commute-friendly learning format

\section{Beginner Roadmap: Path 0}

**Target:** Zero prerequisites (no coding/control theory background)

    **Duration:** 125-150 hours over 4-6 months

    **Phase Breakdown:**
    
        - **Computing Fundamentals (30 hrs)**
        
            - Terminal/command line basics
            - Git version control
            - Package management (pip, conda)

        - **Python Programming (40 hrs)**
        
            - Variables, functions, classes
            - NumPy/SciPy fundamentals
            - Matplotlib visualization

        - **Physics \& Mathematics (35 hrs)**
        
            - Classical mechanics (pendulum dynamics)
            - Linear algebra (matrices, eigenvalues)
            - Differential equations (ODEs)

        - **Control Theory (20 hrs)**
        
            - PID control introduction
            - State-space representation
            - Lyapunov stability basics

\section{Tutorial System Architecture}

**Progressive Learning Structure:**

    \begin{tabular}{lll}
        \toprule
        **Tutorial** & **Topic** & **Duration** \\
        \midrule
        Tutorial 01 & Getting Started & 1-2 hrs \\
        & CLI basics, first simulation & \\
        \midrule
        Tutorial 02 & Controller Comparison & 3-4 hrs \\
        & All 7 controllers, PSO tuning & \\
        \midrule
        Tutorial 03 & Advanced Features & 4-5 hrs \\
        & Batch simulation, monitoring & \\
        \midrule
        Tutorial 04 & Web Interface & 2-3 hrs \\
        & Streamlit dashboard, real-time plots & \\
        \midrule
        Tutorial 05 & Research Workflow & 5-8 hrs \\
        & Reproducible experiments, paper figures & \\
        \bottomrule
    \end{tabular}

        Every tutorial includes runnable code, expected outputs, troubleshooting tips

\section{NotebookLM Podcast Series}

**Innovation:** Convert documentation to podcast-style audio

    **Series Statistics:**
    
        - **44 episodes** covering Phases 1-4
        - **~40 hours** total audio content
        - **125 hours** equivalent learning material
        - TTS optimization for commute/exercise listening

    **Episode Structure:**
    
        - **Phase 1:** Foundations (Python, Git, Physics) -- 12 episodes
        - **Phase 2:** Control Theory (SMC, PSO) -- 10 episodes
        - **Phase 3:** Implementation (Controllers, Simulation) -- 14 episodes
        - **Phase 4:** Advanced Topics (HIL, Monitoring, Research) -- 8 episodes

        Episode templates, TTS optimization checklist, phase-specific examples \\
        \textit{See:} `.ai\_workspace/guides/notebooklm\_guide.md`

\section{Learning Path Integration}

**Seamless Progression Across Paths:**

    [Visual diagram - see PDF]

\newpage

\section*{{Resources}}

\begin{itemize}
    \item \textbf{Repository:} \url{https://github.com/theSadeQ/dip-smc-pso.git}
    \item \textbf{Documentation:} See \texttt{docs/} directory
    \item \textbf{Getting Started:} \texttt{docs/guides/getting-started.md}
\end{itemize}

\vfill

\noindent\textit{Educational podcast episode generated from comprehensive presentation materials}

\end{document}
