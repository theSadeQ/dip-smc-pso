\documentclass[11pt,a4paper]{article}
\usepackage[utf8]{inputenc}
\usepackage[margin=1in]{geometry}
\usepackage{hyperref}
\usepackage{enumitem}
\usepackage{fancyhdr}

\pagestyle{fancy}
\fancyhf{}
\rhead{E025}
\lhead{DIP-SMC-PSO Learning Episode}
\rfoot{\thepage}

\title{\Large\textbf{E025: Appendix Reference Part 1}}
\author{DIP-SMC-PSO Educational Series}
\date{\today}

\begin{document}

\maketitle

\section*{Overview}

This episode covers appendix reference part 1 from the DIP-SMC-PSO project.

\textbf{Part:} Appendix

\textbf{Duration:} 15-20 minutes

\textbf{Source:} Comprehensive Presentation Materials

\newpage

\section{Collaboration Workflow}

**Multi-Developer Best Practices:**

    **Branch Strategy (Future):**
    
        - **main:** Stable, production-ready code
        - **develop:** Integration branch for features
        - **feature/*:** Individual feature branches
        - **hotfix/*:** Urgent bug fixes

    **Pull Request Process:**
    
        - Create feature branch from `develop`
        - Implement changes, write tests
        - Run full test suite (`python run\_tests.py`)
        - Submit PR with description, test plan
        - Code review (≥1 approval required)
        - Merge to `develop` (squash commits)

    **Current Status:**
    
        - Single developer (main branch only)
        - Future: Multi-developer branching strategy

\newpage

\section*{{Resources}}

\begin{itemize}
    \item \textbf{Repository:} \url{https://github.com/theSadeQ/dip-smc-pso.git}
    \item \textbf{Documentation:} See \texttt{docs/} directory
    \item \textbf{Getting Started:} \texttt{docs/guides/getting-started.md}
\end{itemize}

\vfill

\noindent\textit{Educational podcast episode generated from comprehensive presentation materials}

\end{document}
