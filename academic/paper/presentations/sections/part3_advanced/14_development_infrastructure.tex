% ============================================================================
% SECTION 9: EDUCATIONAL MATERIALS
% ============================================================================
\section{Educational Materials}

\begin{frame}{Educational System Overview}
    \textbf{Mission:} Democratize access to advanced control theory

    \vspace{0.3cm}

    \begin{columns}
        \begin{column}{0.5\textwidth}
            \textbf{Learning Paths:}
            \begin{itemize}
                \item \textbf{Path 0:} Complete beginners (125-150 hrs)
                \item \textbf{Path 1:} Quick start (1-2 hrs)
                \item \textbf{Path 2:} Intermediate (5-8 hrs)
                \item \textbf{Path 3:} Advanced (8-12 hrs)
                \item \textbf{Path 4:} Research level (12+ hrs)
            \end{itemize}
        \end{column}

        \begin{column}{0.5\textwidth}
            \textbf{Content Types:}
            \begin{itemize}
                \item Written tutorials (5 comprehensive)
                \item Theory documentation (~2,000 lines)
                \item Podcast series (44 episodes)
                \item Interactive notebooks
                \item Code examples (100+)
            \end{itemize}
        \end{column}
    \end{columns}

    \vspace{0.3cm}

    \begin{block}{Key Feature: NotebookLM Podcast Series}
        \success{44 episodes} covering Phases 1-4, ~40 hours audio content \\
        Convert written materials to commute-friendly learning format
    \end{block}
\end{frame}

\begin{frame}{Beginner Roadmap: Path 0}
    \textbf{Target:} Zero prerequisites (no coding/control theory background)

    \vspace{0.3cm}

    \textbf{Duration:} 125-150 hours over 4-6 months

    \vspace{0.3cm}

    \textbf{Phase Breakdown:}
    \begin{enumerate}
        \item \textbf{Computing Fundamentals (30 hrs)}
        \begin{itemize}
            \item Terminal/command line basics
            \item Git version control
            \item Package management (pip, conda)
        \end{itemize}

        \item \textbf{Python Programming (40 hrs)}
        \begin{itemize}
            \item Variables, functions, classes
            \item NumPy/SciPy fundamentals
            \item Matplotlib visualization
        \end{itemize}

        \item \textbf{Physics \& Mathematics (35 hrs)}
        \begin{itemize}
            \item Classical mechanics (pendulum dynamics)
            \item Linear algebra (matrices, eigenvalues)
            \item Differential equations (ODEs)
        \end{itemize}

        \item \textbf{Control Theory (20 hrs)}
        \begin{itemize}
            \item PID control introduction
            \item State-space representation
            \item Lyapunov stability basics
        \end{itemize}
    \end{enumerate}
\end{frame}

\begin{frame}{Tutorial System Architecture}
    \textbf{Progressive Learning Structure:}

    \vspace{0.3cm}

    \begin{tabular}{lll}
        \toprule
        \textbf{Tutorial} & \textbf{Topic} & \textbf{Duration} \\
        \midrule
        Tutorial 01 & Getting Started & 1-2 hrs \\
        & CLI basics, first simulation & \\
        \midrule
        Tutorial 02 & Controller Comparison & 3-4 hrs \\
        & All 7 controllers, PSO tuning & \\
        \midrule
        Tutorial 03 & Advanced Features & 4-5 hrs \\
        & Batch simulation, monitoring & \\
        \midrule
        Tutorial 04 & Web Interface & 2-3 hrs \\
        & Streamlit dashboard, real-time plots & \\
        \midrule
        Tutorial 05 & Research Workflow & 5-8 hrs \\
        & Reproducible experiments, paper figures & \\
        \bottomrule
    \end{tabular}

    \vspace{0.3cm}

    \begin{exampleblock}{Hands-On Philosophy}
        Every tutorial includes runnable code, expected outputs, troubleshooting tips
    \end{exampleblock}
\end{frame}

\begin{frame}{NotebookLM Podcast Series}
    \textbf{Innovation:} Convert documentation to podcast-style audio

    \vspace{0.3cm}

    \textbf{Series Statistics:}
    \begin{itemize}
        \item \textbf{44 episodes} covering Phases 1-4
        \item \textbf{~40 hours} total audio content
        \item \textbf{125 hours} equivalent learning material
        \item TTS optimization for commute/exercise listening
    \end{itemize}

    \vspace{0.3cm}

    \textbf{Episode Structure:}
    \begin{enumerate}
        \item \textbf{Phase 1:} Foundations (Python, Git, Physics) -- 12 episodes
        \item \textbf{Phase 2:} Control Theory (SMC, PSO) -- 10 episodes
        \item \textbf{Phase 3:} Implementation (Controllers, Simulation) -- 14 episodes
        \item \textbf{Phase 4:} Advanced Topics (HIL, Monitoring, Research) -- 8 episodes
    \end{enumerate}

    \vspace{0.3cm}

    \begin{block}{Quality Validation}
        Episode templates, TTS optimization checklist, phase-specific examples \\
        \textit{See:} \texttt{.ai\_workspace/guides/notebooklm\_guide.md}
    \end{block}
\end{frame}

\begin{frame}{Learning Path Integration}
    \textbf{Seamless Progression Across Paths:}

    \vspace{0.3cm}

    \begin{tikzpicture}[
        node distance=1.5cm,
        every node/.style={font=\small},
        box/.style={rectangle, draw, fill=dipblue!20, minimum width=2.5cm, minimum height=0.8cm}
    ]
        \node[box] (path0) {Path 0: Beginner Roadmap};
        \node[box, below of=path0] (path1) {Path 1: Quick Start};
        \node[box, below of=path1] (path2) {Path 2: Intermediate};
        \node[box, below of=path2] (path3) {Path 3: Advanced};
        \node[box, below of=path3] (path4) {Path 4: Research};

        \draw[->, thick] (path0) -- node[right] {125-150 hrs} (path1);
        \draw[->, thick] (path1) -- node[right] {1-2 hrs} (path2);
        \draw[->, thick] (path2) -- node[right] {5-8 hrs} (path3);
        \draw[->, thick] (path3) -- node[right] {8-12 hrs} (path4);

        \node[right=3cm of path0] (edu) {\texttt{.ai\_workspace/edu/}};
        \node[right=3cm of path1] (tut1) {Tutorial 01};
        \node[right=3cm of path2] (tut23) {Tutorials 02-03};
        \node[right=3cm of path3] (tut45) {Tutorials 04-05};
        \node[right=3cm of path4] (research) {Research docs};

        \draw[->, dashed, dipgreen] (path0.east) -- (edu.west);
        \draw[->, dashed, dipgreen] (path1.east) -- (tut1.west);
        \draw[->, dashed, dipgreen] (path2.east) -- (tut23.west);
        \draw[->, dashed, dipgreen] (path3.east) -- (tut45.west);
        \draw[->, dashed, dipgreen] (path4.east) -- (research.west);
    \end{tikzpicture}
\end{frame}

\begin{frame}[fragile]{Educational Code Examples}
    \textbf{Example: First Simulation (Tutorial 01)}

    \vspace{0.3cm}

    \begin{lstlisting}
# Step 1: Run classical SMC simulation
python simulate.py --ctrl classical_smc --plot

# Step 2: View performance metrics
# Output includes:
#   - Settling time: ~2.5 seconds
#   - Final angle error: <0.01 radians
#   - Control effort: integral of u^2

# Step 3: Save results
python simulate.py --ctrl classical_smc --save results.json
    \end{lstlisting}

    \vspace{0.3cm}

    \begin{exampleblock}{Expected Output}
        \texttt{Simulation complete! Settling time: 2.47s} \\
        \texttt{Angles stabilized within 0.01 rad tolerance} \\
        \texttt{Results saved to: results.json}
    \end{exampleblock}
\end{frame}
