% ============================================================================
% SECTION 5: SIMULATION ENGINE
% ============================================================================
\section{Simulation Engine}

\begin{frame}{Simulation Architecture Overview}
    \textbf{Core Components:}

    \vspace{0.3cm}

    \begin{enumerate}
        \item \textbf{SimulationRunner} -- Main orchestration interface
        \begin{itemize}
            \item \texttt{src/core/simulation\_runner.py}
            \item Coordinates plant, controller, data logging
        \end{itemize}

        \item \textbf{Unified Simulation Context} -- State management
        \begin{itemize}
            \item \texttt{src/core/simulation\_context.py}
            \item Thread-safe state updates
            \item 3 re-export locations (backward compatibility)
        \end{itemize}

        \item \textbf{Batch Simulator} -- Numba-accelerated parallel execution
        \begin{itemize}
            \item \texttt{src/core/vector\_sim.py}
            \item JIT compilation for performance
        \end{itemize}

        \item \textbf{Integrators} -- Numerical ODE solvers
        \begin{itemize}
            \item RK4, RK45, adaptive schemes
            \item \texttt{src/core/integrators/}
        \end{itemize}
    \end{enumerate}
\end{frame}

\begin{frame}{Simulation Loop: Control Cycle}
    \textbf{Execution Flow (100 Hz control rate):}

    \vspace{0.3cm}

    \begin{tikzpicture}[node distance=1.5cm, auto,
        block/.style={rectangle, draw, fill=dipblue!20, text width=3cm, text centered, minimum height=1cm},
        arrow/.style={->, thick, dipblue}
    ]
        \node[block] (init) {Initialize State $\statevec_0$};
        \node[block, below of=init] (sense) {Read State $\statevec(t)$};
        \node[block, below of=sense] (control) {Compute Control $u(t)$};
        \node[block, below of=control] (plant) {Plant Dynamics $\dot{\statevec} = f(\statevec, u)$};
        \node[block, below of=plant] (integrate) {Integrate $\statevec(t+\Delta t)$};
        \node[block, below of=integrate] (log) {Log Data};

        \draw[arrow] (init) -- (sense);
        \draw[arrow] (sense) -- (control);
        \draw[arrow] (control) -- (plant);
        \draw[arrow] (plant) -- (integrate);
        \draw[arrow] (integrate) -- (log);
        \draw[arrow] (log.west) -- ++(-1,0) |- (sense.west);
    \end{tikzpicture}
\end{frame}

\begin{frame}{Real-Time Simulation Parameters}
    \textbf{Default Configuration:}

    \vspace{0.3cm}

    \begin{tabular}{ll}
        \toprule
        \textbf{Parameter} & \textbf{Value} \\
        \midrule
        Time step ($\Delta t$) & 0.01 s (100 Hz) \\
        Simulation duration & 10 s \\
        Total steps & 1000 \\
        Integrator & RK4 (4th-order Runge-Kutta) \\
        \midrule
        \multicolumn{2}{l}{\textit{Safety Guards:}} \\
        Max angle deviation & $\pm 45^\circ$ \\
        Max cart position & $\pm 2.0$ m \\
        NaN detection & Enabled \\
        \bottomrule
    \end{tabular}

    \vspace{0.3cm}

    \begin{block}{Performance}
        \textbf{Single simulation:} ~10-50 ms (depending on controller complexity) \\
        \textbf{100 Monte Carlo runs:} ~5-10 seconds (with Numba acceleration)
    \end{block}
\end{frame}

\begin{frame}[fragile]{Batch Simulation: Numba Acceleration}
    \textbf{Vectorized Execution:}

    \vspace{0.3cm}

    \begin{lstlisting}[language=Python,basicstyle=\ttfamily\tiny]
from src.core.vector_sim import run_batch_simulation
import numpy as np

# Generate 100 initial conditions (Monte Carlo)
initial_states = np.random.uniform(
    low=[-0.1, -0.1, -0.1, -0.1, -0.1, -0.1],
    high=[0.1, 0.1, 0.1, 0.1, 0.1, 0.1],
    size=(100, 6)
)

# Run batch simulation (parallelized, JIT-compiled)
results = run_batch_simulation(
    controller=controller,
    dynamics=dynamics,
    initial_conditions=initial_states,
    sim_params={'dt': 0.01, 'duration': 10.0}
)

# Results shape: (100 runs, 1000 steps, 6 states)
# Speedup: ~50x compared to sequential Python loop
    \end{lstlisting}

    \vspace{0.3cm}

    \begin{exampleblock}{MT-5 Benchmark}
        100 Monte Carlo runs × 7 controllers = 700 simulations in ~2 minutes
    \end{exampleblock}
\end{frame}

\begin{frame}[fragile]{Hardware-in-the-Loop (HIL) Support}
    \textbf{Architecture:}

    \vspace{0.3cm}

    \begin{columns}
        \begin{column}{0.5\textwidth}
            \textbf{Plant Server:}
            \begin{itemize}
                \item Runs plant dynamics
                \item Listens on network port
                \item Sends state $\statevec(t)$
                \item Receives control $u(t)$
            \end{itemize}

            \vspace{0.3cm}

            \textbf{Usage:}
            \begin{lstlisting}[language=bash,basicstyle=\ttfamily\tiny]
python simulate.py --run-hil-server --port 8888
            \end{lstlisting}
        \end{column}

        \begin{column}{0.5\textwidth}
            \textbf{Controller Client:}
            \begin{itemize}
                \item Runs controller algorithm
                \item Connects to plant server
                \item Receives state $\statevec(t)$
                \item Sends control $u(t)$
            \end{itemize}

            \vspace{0.3cm}

            \textbf{Usage:}
            \begin{lstlisting}[language=bash,basicstyle=\ttfamily\tiny]
python simulate.py --run-hil --host localhost --port 8888
            \end{lstlisting}
        \end{column}
    \end{columns}

    \vspace{0.3cm}

    \begin{alertblock}{Validation}
        \statusok Thread-safety validated (100\% tests passing) \\
        Real-time latency monitoring enabled
    \end{alertblock}
\end{frame}

% ============================================================================
% PART II: INFRASTRUCTURE
% ============================================================================
\part{Infrastructure}
\frame{\partpage}
