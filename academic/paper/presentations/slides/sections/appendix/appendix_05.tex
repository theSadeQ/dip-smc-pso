% ============================================================================
% SECTION 24: LESSONS LEARNED
% ============================================================================
\section{Lessons Learned}

\begin{frame}{What Worked Well}
    \textbf{Successful Strategies \& Practices:}

    \vspace{0.3cm}

    \begin{enumerate}
        \item \textbf{Configuration-First Philosophy}
        \begin{itemize}
            \item Define all parameters in \texttt{config.yaml} before coding
            \item Prevented scattered magic numbers
            \item Enabled rapid experimentation
        \end{itemize}

        \item \textbf{Automated Checkpoint System}
        \begin{itemize}
            \item Survived 100\% of token limit interruptions
            \item Zero loss of agent work during Phase 5 research
            \item Recovery time: <30 seconds
        \end{itemize}

        \item \textbf{Multi-Agent Orchestration}
        \begin{itemize}
            \item 6-agent system completed complex tasks efficiently
            \item Clear role separation (integration, control, PSO, docs, beautification)
            \item Quality gates enforced systematically
        \end{itemize}

        \item \textbf{Comprehensive Documentation}
        \begin{itemize}
            \item 985 files ensured no knowledge loss
            \item Multiple navigation systems accommodated different user needs
            \item Beginner roadmap (125-150 hrs) democratized access
        \end{itemize}
    \end{enumerate}
\end{frame}

\begin{frame}{Technical Challenges Overcome}
    \textbf{Problem-Solving Highlights:}

    \vspace{0.3cm}

    \begin{enumerate}
        \item \textbf{MT-6: Boundary Layer Optimization}
        \begin{itemize}
            \item \textbf{Challenge:} Initial claims of 66.5\% chattering reduction
            \item \textbf{Discovery:} Biased "combined\_legacy" metric penalized $d\epsilon/dt$
            \item \textbf{Resolution:} Deep dive validation with unbiased frequency-domain metrics
            \item \textbf{Result:} 3.7\% actual improvement → Fixed boundary layer is near-optimal
            \item \textbf{Value:} Negative result prevents future wasted effort
        \end{itemize}

        \item \textbf{Coverage Measurement Breakage}
        \begin{itemize}
            \item \textbf{Challenge:} Coverage tools stopped working mid-project
            \item \textbf{Impact:} Quality gates 1/8 passing
            \item \textbf{Mitigation:} Thread safety tests (11/11), browser tests (17/17) maintained
            \item \textbf{Status:} Research-ready despite coverage issue
        \end{itemize}
    \end{enumerate}
\end{frame}

\begin{frame}{Organizational Lessons}
    \textbf{Workspace \& Process Improvements:}

    \vspace{0.3cm}

    \begin{enumerate}
        \item \textbf{Three-Category Structure (Dec 2025)}
        \begin{itemize}
            \item \texttt{academic/paper/} (research outputs)
            \item \texttt{academic/logs/} (runtime logs)
            \item \texttt{academic/dev/} (development artifacts)
            \item \textbf{Impact:} Root directory clutter eliminated (73\% reduction)
        \end{itemize}

        \item \textbf{Centralized Log Paths}
        \begin{itemize}
            \item Single source of truth: \texttt{src/utils/logging/paths.py}
            \item NEVER hardcode "logs/" paths
            \item \textbf{Impact:} Zero scattered log files at root
        \end{itemize}

        \item \textbf{Automated Tracking via Git Hooks}
        \begin{itemize}
            \item Pre-commit hooks detect task IDs (QW-*, MT-*, LT-*)
            \item Auto-update project state JSON
            \item \textbf{Impact:} Zero manual status updates, 100\% accuracy
        \end{itemize}
    \end{enumerate}
\end{frame}

\begin{frame}{Critical Discoveries}
    \textbf{Unexpected Insights That Shaped The Project:}

    \vspace{0.3cm}

    \begin{enumerate}
        \item \textbf{Negative Results Are Valuable}
        \begin{itemize}
            \item MT-6 boundary layer optimization revealed marginal benefit (3.7\%)
            \item Fixed boundary layer ($\epsilon = 0.02$) is near-optimal
            \item \textbf{Lesson:} Publish negative results to prevent redundant research
        \end{itemize}

        \item \textbf{Checkpoint System Reliability}
        \begin{itemize}
            \item Git commits (10/10), project state (9/10), agent checkpoints (9/10)
            \item Background bash processes (0/10) → expected, not critical
            \item \textbf{Lesson:} Git-based persistence is bulletproof for recovery
        \end{itemize}

        \item \textbf{Documentation Navigation is Critical}
        \begin{itemize}
            \item 985 files require multiple entry points (11 navigation systems)
            \item Persona-based ("I'm a student...") beats category-based
            \item \textbf{Lesson:} Users need intent-driven navigation, not just hierarchical
        \end{itemize}

        \item \textbf{Automation Prevents Errors}
        \begin{itemize}
            \item Git hooks for task tracking: 100\% accuracy vs. manual updates
            \item Automated cleanup policies prevent root clutter
            \item \textbf{Lesson:} If humans can forget it, automate it
        \end{itemize}
    \end{enumerate}
\end{frame}

\begin{frame}{Recommendations for Future Projects}
    \textbf{Best Practices Distilled:}

    \vspace{0.3cm}

    \begin{enumerate}
        \item \textbf{Start with Recovery Infrastructure}
        \begin{itemize}
            \item Implement checkpoints from day 1
            \item Don't wait until first token limit crash
        \end{itemize}

        \item \textbf{Configuration Before Code}
        \begin{itemize}
            \item Define all parameters in YAML/JSON first
            \item Validate with Pydantic before implementation
        \end{itemize}

        \item \textbf{Automate Tracking \& Status}
        \begin{itemize}
            \item Git hooks for task detection
            \item Pre-commit checks for quality gates
            \item Never rely on manual status updates
        \end{itemize}

        \item \textbf{Document for Multiple Audiences}
        \begin{itemize}
            \item Beginners (Path 0), quick starters (Path 1), researchers (Path 4)
            \item Provide multiple navigation styles (persona, intent, category)
        \end{itemize}

        \item \textbf{Embrace Negative Results}
        \begin{itemize}
            \item MT-6 taught us fixed boundary layer is optimal
            \item Publish to prevent redundant research
        \end{itemize}
    \end{enumerate}
\end{frame}

% ============================================================================
% APPENDIX
% ============================================================================
\appendix

\begin{frame}[fragile]{Quick Reference: Essential Commands}
    \textbf{Most Common Operations:}

    \vspace{0.3cm}

    \textbf{Simulation:}
    \begin{lstlisting}[language=bash]
python simulate.py --ctrl classical_smc --plot
python simulate.py --ctrl sta_smc --run-pso --save gains.json
python simulate.py --load tuned_gains.json --plot
    \end{lstlisting}

    \vspace{0.3cm}

    \textbf{Testing:}
    \begin{lstlisting}[language=bash]
python run_tests.py
python -m pytest tests/ -v --cov=src --cov-report=html
    \end{lstlisting}

    \vspace{0.3cm}

    \textbf{Web UI:}
    \begin{lstlisting}[language=bash]
streamlit run streamlit_app.py
    \end{lstlisting}

    \vspace{0.3cm}

    \textbf{Recovery:}
    \begin{lstlisting}[language=bash]
# Windows
.ai_workspace\tools\recovery\quick_recovery.bat

# Linux/Mac
bash .ai_workspace/tools/recovery/recover_project.sh
    \end{lstlisting}
\end{frame}

\begin{frame}{Quick Reference: Key Files}
    \textbf{Essential Project Files:}

    \vspace{0.3cm}

    \begin{tabular}{ll}
        \toprule
        \textbf{File/Directory} & \textbf{Purpose} \\
        \midrule
        \texttt{simulate.py} & Main CLI entry point \\
        \texttt{streamlit\_app.py} & Web UI entry point \\
        \texttt{config.yaml} & Central configuration \\
        \texttt{requirements.txt} & Python dependencies \\
        \midrule
        \texttt{src/controllers/} & 7 SMC controller variants \\
        \texttt{src/core/} & Dynamics, simulation engine \\
        \texttt{src/optimizer/} & PSO tuner \\
        \texttt{src/utils/} & Validation, monitoring, viz \\
        \midrule
        \texttt{tests/} & 85 test files (pytest) \\
        \texttt{docs/} & 814 documentation files \\
        \texttt{scripts/} & 173 automation scripts \\
        \midrule
        \texttt{.ai\_workspace/} & AI configs, tools, guides \\
        \texttt{academic/} & Research outputs (paper, logs, dev) \\
        \bottomrule
    \end{tabular}
\end{frame}

\begin{frame}{Bibliography Overview}
    \textbf{39 Academic References Organized by Topic:}

    \vspace{0.3cm}

    \textbf{Foundational SMC (8 refs):}
    \begin{itemize}
        \item Utkin (1977, 1992), Slotine \& Li (1991), Edwards \& Spurgeon (1998)
    \end{itemize}

    \vspace{0.3cm}

    \textbf{Higher-Order SMC (6 refs):}
    \begin{itemize}
        \item Levant (1993, 2005, 2007), Moreno \& Osorio (2008)
    \end{itemize}

    \vspace{0.3cm}

    \textbf{Adaptive SMC (5 refs):}
    \begin{itemize}
        \item Slotine \& Coetsee (1986), Plestan et al. (2010)
    \end{itemize}

    \vspace{0.3cm}

    \textbf{PSO Optimization (7 refs):}
    \begin{itemize}
        \item Kennedy \& Eberhart (1995), Shi \& Eberhart (1998), Clerc \& Kennedy (2002)
    \end{itemize}

    \vspace{0.3cm}

    \textbf{Inverted Pendulum Control (13 refs):}
    \begin{itemize}
        \item Bogdanov (2004), Graichen et al. (2007), Zhang et al. (2015)
    \end{itemize}
\end{frame}

\begin{frame}[fragile]{Repository Structure (Condensed)}
    \textbf{Top-Level Organization:}

    \vspace{0.3cm}

    \begin{lstlisting}[basicstyle=\ttfamily\tiny]
dip-smc-pso/
|-- src/                    # Production code (15,000 lines)
|   |-- controllers/        # 7 SMC variants
|   |-- core/               # Dynamics, simulation
|   |-- optimizer/          # PSO tuner
|   |-- utils/              # Validation, monitoring, viz
|   `-- hil/                # Hardware-in-the-loop
|-- tests/                  # 85 test files (8,000 lines)
|-- docs/                   # 814 documentation files
|-- scripts/                # 173 automation scripts
|-- academic/               # Research outputs (262 MB)
|   |-- paper/              # Papers, thesis, experiments (203 MB)
|   |-- logs/               # Runtime logs (13 MB)
|   `-- dev/                # QA audits, coverage (46 MB)
|-- .ai_workspace/          # AI configs, tools, guides (HIDDEN)
|-- simulate.py             # CLI entry point
|-- streamlit_app.py        # Web UI entry point
|-- config.yaml             # Central configuration
`-- requirements.txt        # Python dependencies
    \end{lstlisting}
\end{frame}

\begin{frame}{Contact \& Resources}
    \textbf{Project Information:}

    \vspace{0.3cm}

    \begin{itemize}
        \item \textbf{Author:} Sadegh Naderi
        \item \textbf{Repository:} \url{https://github.com/theSadeQ/dip-smc-pso.git}
        \item \textbf{License:} MIT (open for academic \& commercial use)
    \end{itemize}

    \vspace{0.3cm}

    \textbf{Documentation Entry Points:}
    \begin{itemize}
        \item \textbf{Getting Started:} \texttt{docs/guides/getting-started.md}
        \item \textbf{Beginner Roadmap:} \texttt{.ai\_workspace/edu/beginner-roadmap.md}
        \item \textbf{Navigation Hub:} \texttt{docs/NAVIGATION.md}
        \item \textbf{Research Completion:} \texttt{.ai\_workspace/planning/research/RESEARCH\_COMPLETION\_SUMMARY.md}
    \end{itemize}

    \vspace{0.3cm}

    \textbf{Key Documentation Files:}
    \begin{itemize}
        \item \texttt{CLAUDE.md} -- Project instructions for Claude Code
        \item \texttt{README.md} -- Project overview
        \item \texttt{CHANGELOG.md} -- Version history
    \end{itemize}
\end{frame}

\begin{frame}[plain]
    \begin{center}
        \vspace{2cm}
        {\Huge Thank You!}

        \vspace{1cm}

        {\large Questions \& Discussion}

        \vspace{2cm}

        \textbf{Repository:} \\
        \url{https://github.com/theSadeQ/dip-smc-pso.git}

        \vspace{1cm}

        \textit{Double-Inverted Pendulum Sliding Mode Control \\
        with PSO Optimization}
    \end{center}
\end{frame}
