\documentclass[11pt]{article}
\usepackage[utf8]{inputenc}
\usepackage{amsmath,amssymb}
\usepackage[margin=1in]{geometry}

\title{Comparative Analysis of Sliding Mode Control Variants for \\
       Double-Inverted Pendulum Systems: \\
       Performance, Stability, and Robustness \\
       \vspace{0.5cm}
       \large Enhanced Version - All 10 Sections}

\author{Research Paper}
\date{December 25, 2025}

\begin{document}
\maketitle
\tableofcontents
\newpage

\begin{abstract}

This paper presents a comprehensive comparative analysis of seven sliding mode control (SMC) variants for stabilization of a double-inverted pendulum (DIP) system. We evaluate \textbf{Classical SMC}, Super-Twisting Algorithm (STA), \textbf{Adaptive SMC}, Hybrid Adaptive \textbf{STA-SMC}, Swing-Up SMC, Model Predictive Control (MPC), and their combinations across multiple performance dimensions: computational efficiency, transient response, chattering reduction, energy consumption, and robustness to model uncertainty and external disturbances. Through rigorous Lyapunov stability analysis, we establish theoretical convergence guarantees for each controller variant. Performance benchmarking with 400+ Monte Carlo simulations reveals that \textbf{STA-SMC} achieves superior overall performance (1.82s settling time, 2.3percent overshoot, 11.8J energy), while \textbf{Classical SMC} provides the fastest computation (18.5 microseconds). PSO-based optimization demonstrates significant performance improvements but reveals critical generalization limitations: parameters optimized for small perturbations ($\pm$0.05 rad) exhibit 50.4x chattering degradation and 90.2percent failure rate under realistic disturbances ($\pm$0.3 rad). Robustness analysis with $\pm$20percent model parameter errors shows Hybrid Adaptive \textbf{STA-SMC} offers best uncertainty tolerance (16percent mismatch before instability), while \textbf{STA-SMC} excels at disturbance rejection (91percent attenuation). Our findings provide evidence-based controller selection guidelines for practitioners and identify critical gaps in current optimization approaches for real-world deployment.

Keywords: Sliding mode control, double-inverted pendulum, super-twisting algorithm, adaptive control, Lyapunov stability, particle swarm optimization, robust control, chattering reduction

---

