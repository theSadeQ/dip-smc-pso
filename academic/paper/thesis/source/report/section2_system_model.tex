\section{System Model and Problem Formulation}
\label{sec:model}

% Content to be extracted from docs/theory/system_dynamics.md

\subsection{Physical Description}
The double-inverted pendulum (DIP) system consists of three main components:
\begin{itemize}
\item \textbf{Cart} (mass $m_0$): Moves horizontally on a frictionless track
\item \textbf{First Pendulum} (mass $m_1$, length $l_1$): Revolute joint attached to cart
\item \textbf{Second Pendulum} (mass $m_2$, length $l_2$): Revolute joint attached to first pendulum tip
\end{itemize}

The control objective is to stabilize both pendulums in the upright position ($\theta_1 = \theta_2 = 0$) using horizontal cart force $u$, while regulating cart position within bounds.

\subsection{System Challenges}
The DIP presents four key challenges for control design:

\textbf{Underactuation}: The system has 3 degrees of freedom (DOF) but only 1 control input, requiring exploitation of dynamic coupling.

\textbf{Unstable Equilibrium}: The upright position is inherently unstable—small perturbations cause pendulums to fall without active stabilization.

\textbf{Nonlinear Dynamics}: Trigonometric functions in the equations of motion complicate controller design, especially for large-angle swings.

\textbf{System Coupling}: Cart motion affects both pendulums, and second pendulum motion influences the first through inertial coupling.

\subsection{State-Space Representation}
The system state vector is:
\begin{equation}
\vect{x} = [x, \theta_1, \theta_2, \dot{x}, \dot{\theta}_1, \dot{\theta}_2]^T \in \Real^6
\label{eq:state_vector}
\end{equation}

The nonlinear dynamics are:
\begin{equation}
\vect{M}(\vect{q})\ddot{\vect{q}} + \vect{C}(\vect{q}, \dot{\vect{q}}) + \vect{G}(\vect{q}) = \vect{B}u
\label{eq:dynamics}
\end{equation}

where $\vect{q} = [x, \theta_1, \theta_2]^T$ represents generalized coordinates, $u$ is the control force applied to the cart, and $\vect{M}$, $\vect{C}$, $\vect{G}$ are inertia, Coriolis, and gravity terms.

\subsection{Control Objectives}
\begin{enumerate}
\item Stabilization: $\lim_{t \to \infty} [\theta_1(t), \theta_2(t)] = [0, 0]$
\item Cart regulation: $|x(t)| \leq x_{max}$
\item Bounded control effort: $|u(t)| \leq u_{max}$
\end{enumerate}

\subsection{Lagrangian Formulation}
The equations of motion are derived using the Euler-Lagrange approach. The Lagrangian $L = T - V$ consists of kinetic energy $T$ and potential energy $V$.

\subsubsection{Kinetic Energy}
The total kinetic energy includes contributions from the cart and both pendulums:
\begin{equation}
T = \frac{1}{2}m_0\dot{x}^2 + \frac{1}{2}m_1(\dot{x}_1^2 + \dot{y}_1^2) + \frac{1}{2}I_1\dot{\theta}_1^2 + \frac{1}{2}m_2(\dot{x}_2^2 + \dot{y}_2^2) + \frac{1}{2}I_2\dot{\theta}_2^2
\label{eq:kinetic_energy}
\end{equation}

where $(x_i, y_i)$ are Cartesian coordinates of pendulum centers of mass, and $I_i$ are moments of inertia. After substituting kinematic constraints and simplifying, the kinetic energy depends only on generalized coordinates $\vect{q} = [x, \theta_1, \theta_2]^T$ and their derivatives.

\subsubsection{Potential Energy}
Gravitational potential energy (with cart height as reference):
\begin{equation}
V = m_1 g l_{c1}\cos(\theta_1) + m_2 g (l_1\cos(\theta_1) + l_{c2}\cos(\theta_2))
\label{eq:potential_energy}
\end{equation}

where $l_{c1}$ and $l_{c2}$ are distances from joints to centers of mass, and $g$ is gravitational acceleration.

\subsubsection{Equations of Motion}
Applying the Euler-Lagrange equation $\frac{d}{dt}\left(\frac{\partial L}{\partial \dot{q}_i}\right) - \frac{\partial L}{\partial q_i} = Q_i$ yields the compact form:
\begin{equation}
\vect{M}(\vect{q})\ddot{\vect{q}} + \vect{C}(\vect{q}, \dot{\vect{q}})\dot{\vect{q}} + \vect{G}(\vect{q}) = \vect{B}u
\label{eq:manipulator_form}
\end{equation}

The inertia matrix $\vect{M}(\vect{q}) \in \Real^{3 \times 3}$ is configuration-dependent and positive definite. The Coriolis/centrifugal matrix $\vect{C}(\vect{q}, \dot{\vect{q}})$ captures velocity-dependent forces. The gravity vector $\vect{G}(\vect{q})$ contains gravitational torques. The input matrix $\vect{B} = [1, 0, 0]^T$ maps cart force to generalized forces.

\subsection{Nominal System Parameters}
The nominal physical parameters used throughout this work are listed in Table \ref{tab:parameters}.

\begin{table}[htbp]
\centering
\caption{Double-Inverted Pendulum Nominal Parameters}
\label{tab:parameters}
\begin{tabular}{lccl}
\toprule
Parameter & Symbol & Value & Unit \\
\midrule
Cart mass & $m_0$ & 1.5 & kg \\
First pendulum mass & $m_1$ & 0.2 & kg \\
Second pendulum mass & $m_2$ & 0.15 & kg \\
First pendulum length & $l_1$ & 0.4 & m \\
Second pendulum length & $l_2$ & 0.3 & m \\
First pendulum COM & $l_{c1}$ & 0.2 & m \\
Second pendulum COM & $l_{c2}$ & 0.15 & m \\
First pendulum inertia & $I_1$ & 0.0081 & kg·m$^2$ \\
Second pendulum inertia & $I_2$ & 0.0034 & kg·m$^2$ \\
Gravitational acceleration & $g$ & 9.81 & m/s$^2$ \\
Control force limit & $u_{max}$ & 50 & N \\
Sampling time & $\Delta t$ & 0.01 & s \\
\bottomrule
\end{tabular}
\end{table}

These parameters represent a realistic laboratory-scale DIP system. The inertias satisfy the parallel-axis theorem constraints: $I_i \geq m_i l_{ci}^2$.

\subsection{Linearization and Controllability}
To analyze controllability, the nonlinear system is linearized around the upright equilibrium $\vect{x}_0 = [0, 0, 0, 0, 0, 0]^T$. The linearized state-space representation is:
\begin{equation}
\dot{\vect{x}} = \mat{A}\vect{x} + \vect{b}u
\label{eq:linearized_system}
\end{equation}

where $\mat{A} \in \Real^{6 \times 6}$ is the system matrix and $\vect{b} \in \Real^6$ is the input vector.

The system is controllable if the controllability matrix $\mat{C} = [\vect{b}, \mat{A}\vect{b}, \mat{A}^2\vect{b}, \ldots, \mat{A}^5\vect{b}]$ has full rank. Numerical verification confirms $\text{rank}(\mat{C}) = 6$, establishing complete controllability.

The open-loop system is unstable with two eigenvalues having positive real parts (corresponding to unstable pendulum modes). This confirms the necessity of active feedback control for stabilization.
