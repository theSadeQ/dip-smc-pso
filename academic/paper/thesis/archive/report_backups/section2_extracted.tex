\chapter{Double-Inverted Pendulum Dynamics}

\textbf{Understanding the Physics and Mathematics of the DIP System}

This guide derives the mathematical model of the double-inverted pendulum from first principles. You'll learn how Lagrangian mechanics produces the equations of motion, when linearization is valid, and why the system is controllable despite underactuation.



\section{Table of Contents}
\label{section:table-of-contents}

\begin{itemize}
\item [System Overview](\#system-overview)
\item [Lagrangian Derivation](\#lagrangian-derivation)
\item [Equations of Motion](\#equations-of-motion)
\item [Linearization for Control](\#linearization-for-control)
\item [Controllability Analysis](\#controllability-analysis)
\end{itemize}



\section{System Overview}
\label{section:system-overview}

\subsection{Physical Description}
\label{subsection:physical-description}

The double-inverted pendulum (DIP) consists of:
\begin{itemize}
\item \textbf{Cart} (mass \texttt{m₀}): Moves horizontally on track
\item \textbf{First Pendulum} (mass \texttt{m₁}, length \texttt{l₁}): Hinged to cart
\item \textbf{Second Pendulum} (mass \texttt{m₂}, length \texttt{l₂}): Hinged to tip of first pendulum
\end{itemize}

\textbf{Control Objective}: Balance both pendulums upright (\texttt{θ₁ = θ₂ = 0}) using horizontal cart force \texttt{u}.

\textbf{Physical System Diagram}:

\begin{lstlisting}[language=mermaid]
graph TD
    subgraph "Double-Inverted Pendulum System"
        CART["Cart (m₀)<br/>Position: x"] -->|Hinge| P1["First Pendulum (m₁, l₁)<br/>Angle: θ₁ from vertical"]
        P1 -->|Hinge| P2["Second Pendulum (m₂, l₂)<br/>Angle: θ₂ from vertical"]

        FORCE["Control Force u<br/>(Horizontal)"] -->|Applied to| CART

        TRACK["Frictionless Track"] -.->|Constrained| CART
    end

    style CART fill:\#ccccff
    style P1 fill:\#ffcccc
    style P2 fill:\#ccffcc
    style FORCE fill:\#ffffcc
\end{lstlisting}

\textbf{System Components}:
\begin{itemize}
\item 🔵 \textbf{Cart}: Movable platform (1 DOF: position x)
\item 🔴 \textbf{Pendulum 1}: First link (1 DOF: angle θ₁)
\item 🟢 \textbf{Pendulum 2}: Second link (1 DOF: angle θ₂)
\item 🟡 \textbf{Control}: Horizontal force u (single actuator)
\end{itemize}

\subsection{State Variables}
\label{subsection:state-variables}

\textbf{Generalized Coordinates} (3 DOF):
\begin{verbatim}
q = [x, θ₁, θ₂]ᵀ
\end{verbatim}yaml

Where:
\begin{itemize}
\item \texttt{x}: Cart position (m)
\item \texttt{θ₁}: First pendulum angle from vertical (rad, 0 = upright)
\item \texttt{θ₂}: Second pendulum angle from vertical (rad, 0 = upright)
\end{itemize}

\textbf{Full State Vector} (6 dimensions):
\begin{verbatim}
state = [x, ẋ, θ₁, θ̇₁, θ₂, θ̇₂]ᵀ
\end{verbatim}

\textbf{Control Input}:
\begin{verbatim}
u = horizontal force on cart (N)
\end{verbatim}

\textbf{State-Space Representation}:

\begin{lstlisting}[language=mermaid]
flowchart LR
    subgraph STATE["State Vector (6D)"]
        direction TB
        X["x<br/>(position)"]
        DX["ẋ<br/>(velocity)"]
        T1["θ₁<br/>(angle 1)"]
        DT1["θ̇₁<br/>(angular vel 1)"]
        T2["θ₂<br/>(angle 2)"]
        DT2["θ̇₂<br/>(angular vel 2)"]
    end

    CONTROL["u<br/>(force)"] --> DYNAMICS["DIP Dynamics<br/>M(q)q̈ + C(q,q̇)q̇ + G(q) = Qu"]
    STATE --> DYNAMICS
    DYNAMICS --> NEXT["Next State<br/>(Integration)"]
    NEXT --> STATE

    style CONTROL fill:\#ffffcc
    style DYNAMICS fill:\#ffcccc
    style STATE fill:\#ccffcc
\end{lstlisting}

\textbf{Dimensions}:
\begin{itemize}
\item \textbf{State space}: 6D (position + velocity for 3 coordinates)
\item \textbf{Control space}: 1D (horizontal force only)
\item \textbf{Underactuated}: 6 states, 1 control → dynamic coupling essential
\end{itemize}

\subsection{Why This System is Challenging}
\label{subsection:why-this-system-is-challenging}

\textbf{Underactuated}: 3 DOF, 1 control input
\begin{itemize}
\item Cannot independently control all coordinates
\item Must use dynamic coupling
\end{itemize}

\textbf{Unstable Equilibrium}: Upright position
\begin{itemize}
\item Small perturbations → pendulums fall
\item Requires active stabilization
\end{itemize}

\textbf{Nonlinear Dynamics}: Trigonometric functions
\begin{itemize}
\item Linear control design requires approximations
\item Full dynamics needed for large angles
\end{itemize}

\textbf{Coupled System}: Motion of one affects others
\begin{itemize}
\item Cart motion affects both pendulums
\item Second pendulum motion affects first
\end{itemize}



\section{Lagrangian Derivation}
\label{section:lagrangian-derivation}

\subsection{Why Lagrangian Mechanics?}
\label{subsection:why-lagrangian-mechanics}

\textbf{Newton's Laws}: Force-based approach
\begin{itemize}
\item Require constraint forces (reaction forces at joints)
\item Complex for multi-body systems
\end{itemize}

\textbf{Lagrangian Mechanics}: Energy-based approach
\begin{itemize}
\item Automatically handles constraints
\item Natural for deriving equations of motion
\item Uses \texttt{L = T - V} (kinetic - potential energy)
\end{itemize}

\textbf{Lagrangian Derivation Process}:

\begin{lstlisting}[language=mermaid]
flowchart TD
    START["System Description<br/>(Masses, lengths, constraints)"] --> KE["Compute Kinetic Energy T<br/>½m₀ẋ² + ½m₁(ẋ₁²+ż₁²) + ½m₂(ẋ₂²+ż₂²)"]
    KE --> PE["Compute Potential Energy V<br/>m₁gl₁cos(θ₁) + m₂g[l₁cos(θ₁)+l₂cos(θ₂)]"]

    PE --> LAG["Form Lagrangian<br/>L = T - V"]

    LAG --> EL["Apply Euler-Lagrange<br/>d/dt(∂L/∂q̇ᵢ) - ∂L/∂qᵢ = Qᵢ"]

    EL --> EOM["Equations of Motion<br/>M(q)q̈ + C(q,q̇)q̇ + G(q) = Qu"]

    style START fill:\#ccccff
    style LAG fill:\#ffffcc
    style EOM fill:\#ccffcc
\end{lstlisting}

\textbf{Advantages}:
\begin{itemize}
\item 🔵 \textbf{Systematic}: No need to derive constraint forces
\item 🟡 \textbf{Lagrangian} L = T - V: Single scalar function
\item 🟢 \textbf{Result}: Configuration-dependent dynamics M(q)
\end{itemize}

\subsection{Kinetic Energy}
\label{subsection:kinetic-energy}

\textbf{Cart}:
\begin{verbatim}
T₀ = ½m₀ẋ²
\end{verbatim}

\textbf{First Pendulum} (center of mass at \texttt{l₁/2}):

Position of COM:
\begin{verbatim}
x₁ = x + (l₁/2)sin(θ₁)
z₁ = (l₁/2)cos(θ₁)
\end{verbatim}yaml

Velocity:
\begin{verbatim}
ẋ₁ = ẋ + (l₁/2)θ̇₁cos(θ₁)
ż₁ = -(l₁/2)θ̇₁sin(θ₁)
\end{verbatim}

Kinetic energy:
\begin{verbatim}
T₁ = ½m₁(ẋ₁² + ż₁²) + ½I₁θ̇₁²
   = ½m₁[ẋ² + (l₁/2)²θ̇₁² + l₁ẋθ̇₁cos(θ₁)] + ½I₁θ̇₁²
\end{verbatim}yaml

Where \texttt{I₁ = (1/3)m₁l₁²} (inertia of rod about end)

\textbf{Second Pendulum} (center of mass at \texttt{l₁ + l₂/2} from cart):

Position:
\begin{verbatim}
x₂ = x + l₁sin(θ₁) + (l₂/2)sin(θ₂)
z₂ = l₁cos(θ₁) + (l₂/2)cos(θ₂)
\end{verbatim}

Velocity (chain rule):
\begin{verbatim}
ẋ₂ = ẋ + l₁θ̇₁cos(θ₁) + (l₂/2)θ̇₂cos(θ₂)
ż₂ = -l₁θ̇₁sin(θ₁) - (l₂/2)θ̇₂sin(θ₂)
\end{verbatim}

Kinetic energy:
\begin{verbatim}
T₂ = ½m₂(ẋ₂² + ż₂²) + ½I₂θ̇₂²
\end{verbatim}

\textbf{Total Kinetic Energy}:
\begin{verbatim}
T = T₀ + T₁ + T₂
\end{verbatim}

\subsection{Potential Energy}
\label{subsection:potential-energy}

Choose zero at cart level:

\textbf{First Pendulum}:
\begin{verbatim}
V₁ = m₁g(l₁/2)cos(θ₁)
\end{verbatim}

\textbf{Second Pendulum}:
\begin{verbatim}
V₂ = m₂g[l₁cos(θ₁) + (l₂/2)cos(θ₂)]
\end{verbatim}

\textbf{Total Potential Energy}:
\begin{verbatim}
V = V₁ + V₂
  = (m₁l₁/2 + m₂l₁)g·cos(θ₁) + (m₂l₂/2)g·cos(θ₂)
\end{verbatim}

\subsection{Lagrangian}
\label{subsection:lagrangian}

\begin{verbatim}
L = T - V
  = (kinetic energy) - (potential energy)
\end{verbatim}

After substitution and simplification:
\begin{verbatim}
L = ½(m₀ + m₁ + m₂)ẋ²
  + ½(I₁ + m₁l₁²/4 + m₂l₁²)θ̇₁²
  + ½(I₂ + m₂l₂²/4)θ̇₂²
  + (m₁l₁/2 + m₂l₁)ẋθ̇₁cos(θ₁)
  + (m₂l₂/2)ẋθ̇₂cos(θ₂)
  + (m₂l₁l₂/2)θ̇₁θ̇₂cos(θ₁ - θ₂)
  - (m₁l₁/2 + m₂l₁)g·cos(θ₁)
  - (m₂l₂/2)g·cos(θ₂)
\end{verbatim}

\textbf{Note}: This is a complex expression due to coupling terms!



\section{Equations of Motion}
\label{section:equations-of-motion}

\subsection{Euler-Lagrange Equations}
\label{subsection:euler-lagrange-equations}

For each generalized coordinate \texttt{qᵢ}:
\begin{verbatim}
d/dt(∂L/∂q̇ᵢ) - ∂L/∂qᵢ = Qᵢ
\end{verbatim}

Where \texttt{Qᵢ} are generalized forces.

\textbf{For DIP}:
\begin{itemize}
\item \texttt{q₁ = x}: Generalized force = \texttt{u} (cart force)
\item \texttt{q₂ = θ₁}: Generalized force = \texttt{0} (no direct torque)
\item \texttt{q₃ = θ₂}: Generalized force = \texttt{0} (no direct torque)
\end{itemize}

\subsection{Matrix Form}
\label{subsection:matrix-form}

After applying Euler-Lagrange and simplifying:

\begin{verbatim}
M(q)q̈ + C(q, q̇)q̇ + G(q) = Qu
\end{verbatim}yaml

Where:
\begin{itemize}
\item \texttt{M(q)}: \textbf{Inertia matrix} (3×3, configuration-dependent)
\item \texttt{C(q, q̇)}: \textbf{Coriolis/centrifugal matrix} (3×3)
\item \texttt{G(q)}: \textbf{Gravitational vector} (3×1)
\item \texttt{Q}: \textbf{Input matrix} (3×1)
\item \texttt{u}: \textbf{Control force} (scalar)
\end{itemize}

\textbf{Inertia Matrix} (simplified):
\begin{verbatim}
M = [m_total           m_c1·cos(θ₁)      m_c2·cos(θ₂)    ]
    [m_c1·cos(θ₁)      I_eff1            m_12·cos(θ₁-θ₂)]
    [m_c2·cos(θ₂)      m_12·cos(θ₁-θ₂)   I_eff2          ]
\end{verbatim}

\textbf{Gravitational Vector}:
\begin{verbatim}
G = [    0                            ]
    [(m₁l₁/2 + m₂l₁)g·sin(θ₁)        ]
    [(m₂l₂/2)g·sin(θ₂)                ]
\end{verbatim}

\textbf{Input Matrix}:
\begin{verbatim}
Q = [1, 0, 0]ᵀ
\end{verbatim}

\subsection{Properties of the Dynamics}
\label{subsection:properties-of-the-dynamics}

\textbf{1. M(q) is Symmetric and Positive Definite}
\begin{itemize}
\item Always invertible (except at singularities)
\item Energy considerations guarantee this
\end{itemize}

\textbf{2. Configuration-Dependent Inertia}
\begin{itemize}
\item \texttt{M} depends on \texttt{θ₁, θ₂} (not constant!)
\item Coupling changes with pendulum angles
\end{itemize}

\textbf{3. Coriolis Matrix is Skew-Symmetric}
\begin{itemize}
\item \texttt{Ṁ - 2C} is skew-symmetric
\item Important for energy-based control
\end{itemize}



\section{Linearization for Control}
\label{section:linearization-for-control}

\subsection{Small Angle Approximation}
\label{subsection:small-angle-approximation}

Near upright position (\texttt{θ₁ ≈ 0, θ₂ ≈ 0}):

\textbf{Trigonometric Approximations}:
\begin{verbatim}
sin(θ) ≈ θ
cos(θ) ≈ 1
sin(θ)cos(θ) ≈ θ
\end{verbatim}

\textbf{Validity}:
\begin{itemize}
\item |θ| < 0.2 rad (≈ 11°): Error < 2\%
\item |θ| < 0.3 rad (≈ 17°): Error < 5\%
\item |θ| > 0.5 rad (≈ 29°): Approximation breaks down
\end{itemize}

\subsection{Simplified Dynamics}
\label{subsection:simplified-dynamics}

\textbf{Assumptions}:
\begin{enumerate}
\item Small angles: \texttt{θ₁, θ₂ ≪ 1}
\item Small velocities: \texttt{θ̇₁, θ̇₂ ≪ 1}
\item Neglect products: \texttt{θᵢθⱼ ≈ 0}
\end{enumerate}

\textbf{Result}: Linearized equations
\begin{verbatim}
M_lin·q̈ + G_lin·q = Q·u
\end{verbatim}yaml

Where:
\begin{itemize}
\item \texttt{M_lin}: Constant inertia matrix
\item \texttt{G_lin}: Constant gravitational matrix (no sin/cos)
\end{itemize}

\textbf{State-Space Form}:
\begin{verbatim}
ẋ = Ax + Bu

A = [0   I  ]    B = [  0      ]
    [M⁻¹G 0]        [M⁻¹Q    ]
\end{verbatim}

\textbf{Control Design Benefits}:
\begin{itemize}
\item Linear system → linear control methods (LQR, pole placement)
\item Constant matrices → easier analysis
\item Superposition applies → simple design
\end{itemize}

\subsection{Limitations of Linearization}
\label{subsection:limitations-of-linearization}

\textbf{Invalid for}:
\begin{enumerate}
\item \textbf{Large angles}: Swing-up control (θ > 90°)
\item \textbf{Fast motion}: Large θ̇ violates small velocity assumption
\item \textbf{Extreme conditions}: Actuator saturation, disturbances
\end{enumerate}

\textbf{When to Use Full Nonlinear}:
\begin{itemize}
\item Final validation
\item Hardware testing
\item Large disturbance scenarios
\item Research publications
\end{itemize}



\section{Controllability Analysis}
\label{section:controllability-analysis}

\subsection{Controllability Matrix}
\label{subsection:controllability-matrix}

\textbf{Definition}: System is controllable if we can reach any state from any initial state.

\textbf{Test}: Controllability matrix has full rank:
\begin{verbatim}
C = [B  AB  A²B  ...  A⁵B]
\end{verbatim}bash

For 6-state system, need rank(C) = 6.

\subsection{DIP Controllability}
\label{subsection:dip-controllability}

\textbf{Result}: DIP is controllable (rank = 6) despite underactuation!

\textbf{Why?}
\begin{itemize}
\item Cart force \texttt{u} affects cart acceleration: \texttt{ẍ ∝ u}
\item Cart motion couples to pendulums via inertia matrix
\item Dynamic coupling propagates control to all states
\end{itemize}

\textbf{Physical Intuition}:
\begin{itemize}
\item Move cart right → first pendulum tilts left (inertia)
\item First pendulum motion → affects second pendulum
\item Clever sequencing → control all angles
\end{itemize}

\textbf{Proof Sketch}:
\begin{enumerate}
\item Direct effect: \texttt{u → ẍ} (cart acceleration)
\item First-order coupling: \texttt{ẍ → θ̈₁} (first pendulum)
\item Second-order coupling: \texttt{θ̈₁ → θ̈₂} (second pendulum)
\item Three integration steps → reach all position states
\item Controllability matrix full rank ✓
\end{enumerate}

\textbf{Control Propagation Diagram}:

\begin{lstlisting}[language=mermaid]
flowchart TD
    U["Control Force u"] --> CART_ACC["Cart Acceleration ẍ<br/>(Direct effect)"]

    CART_ACC --> INERTIA1["Inertial Coupling<br/>to Pendulum 1"]
    INERTIA1 --> P1_ACC["θ̈₁<br/>(First pendulum acceleration)"]

    P1_ACC --> INERTIA2["Inertial Coupling<br/>to Pendulum 2"]
    INERTIA2 --> P2_ACC["θ̈₂<br/>(Second pendulum acceleration)"]

    CART_ACC --> CART_VEL["∫ → ẋ<br/>(Cart velocity)"] --> CART_POS["∫ → x<br/>(Cart position)"]
    P1_ACC --> P1_VEL["∫ → θ̇₁<br/>(Angular velocity 1)"] --> P1_POS["∫ → θ₁<br/>(Angle 1)"]
    P2_ACC --> P2_VEL["∫ → θ̇₂<br/>(Angular velocity 2)"] --> P2_POS["∫ → θ₂<br/>(Angle 2)"]

    style U fill:\#ffffcc
    style CART_ACC fill:\#ccffcc
    style P1_ACC fill:\#ffcccc
    style P2_ACC fill:\#ccccff
\end{lstlisting}

\textbf{Key Insight}: Single control input reaches all 6 states via:
\begin{itemize}
\item 🟡 \textbf{Direct} effect on cart
\item 🟢 \textbf{Inertial coupling} to pendulum 1
\item 🔴 \textbf{Secondary coupling} to pendulum 2
\item 🔵 \textbf{Integration} for position control
\end{itemize}

\textbf{Controllability Matrix}:
\begin{verbatim}
rank([B  AB  A²B  A³B  A⁴B  A⁵B]) = 6  ✓
\end{verbatim}

System is \textbf{fully controllable} despite underactuation!

\subsection{Observability}
\label{subsection:observability}

\textbf{State Measurement}:
\begin{itemize}
\item Typically measure: \texttt{[x, θ₁, θ₂]} (positions only)
\item Velocities: Computed from position derivatives
\end{itemize}

\textbf{Observability Test}:
\begin{verbatim}
O = [C  CA  CA²  ...  CA⁵]ᵀ
\end{verbatim}

Where \texttt{C = [1 0 0 0 0 0; 0 0 1 0 0 0; 0 0 0 0 1 0]} (measure positions)

\textbf{Result}: Observable (rank = 6)

\textbf{Meaning}: Can estimate full state from position measurements



\section{Summary}
\label{section:summary}

\textbf{Key Takeaways}:

\begin{enumerate}
\item \textbf{Lagrangian Derivation}: Energy method naturally handles constraints
\item \textbf{Matrix Form}: \texttt{M(q)q̈ + C(q,q̇)q̇ + G(q) = Qu} (configuration-dependent)
\item \textbf{Linearization}: Valid for |θ| < 15°, enables linear control design
\item \textbf{Controllability}: Underactuated but controllable via dynamic coupling
\item \textbf{Observability}: Full state observable from position measurements
\end{enumerate}

\textbf{Design Implications}:
\begin{itemize}
\item \textbf{Small angles}: Use simplified dynamics for fast control design
\item \textbf{Large angles}: Must use full nonlinear dynamics
\item \textbf{Control}: Exploit coupling to control all DOFs with one input
\item \textbf{Estimation}: State observer feasible from position measurements
\end{itemize}

\textbf{Next Steps}:
\begin{itemize}
\item See linearization in action: [Tutorial 01](../tutorials/tutorial-01-first-simulation.md)
\item Implement dynamics models: [Plant Models API](../api/plant-models.md)
\item Deep dive: [Dynamics Derivations](../../mathematical_foundations/dynamics_derivations.md)
\end{itemize}



\textbf{Further Reading}:
\begin{itemize}
\item Goldstein, H. (2002). \textit{Classical Mechanics}. Addison-Wesley.
\item Spong, M. W., et al. (2006). \textit{Robot Modeling and Control}. Wiley.
\item Khalil, H. K. (2002). \textit{Nonlinear Systems}. Prentice Hall.
\end{itemize}



\textbf{Last Updated}: October 2025
