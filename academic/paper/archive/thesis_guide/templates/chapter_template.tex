% ==============================================================================
% CHAPTER TEMPLATE - Reusable Chapter Structure
% ==============================================================================
% Copy this template for each new chapter.
% Replace placeholders with your content.
% ==============================================================================

\chapter{Chapter Title Here}
\label{chap:chapter-name}

% ==============================================================================
% CHAPTER INTRODUCTION (1-2 paragraphs)
% ==============================================================================

This chapter introduces [topic]. The main objectives are:
\begin{itemize}
    \item Objective 1: [Brief description]
    \item Objective 2: [Brief description]
    \item Objective 3: [Brief description]
\end{itemize}

The chapter is organized as follows: \cref{sec:section1} presents [topic 1], \cref{sec:section2} discusses [topic 2], and \cref{sec:section3} concludes with [topic 3].

% ==============================================================================
% SECTION 1
% ==============================================================================

\section{First Section Title}
\label{sec:section1}

Lorem ipsum dolor sit amet, consectetur adipiscing elit. This is an example of citing a reference~\cite{Utkin1977}. Multiple citations can be grouped~\cite{Utkin1977,Khalil2002,Slotine1991}.

% Subsection 1.1
\subsection{First Subsection}
\label{subsec:subsection1-1}

Content for subsection 1.1. Here is an example of a mathematical equation:
\begin{equation}
    \dot{\vect{x}} = \mat{A}\vect{x} + \mat{B}\vect{u}
    \label{eq:state-space}
\end{equation}
where $\vect{x} \in \Real^n$ is the state vector, $\vect{u} \in \Real^m$ is the control input, and $\mat{A}$, $\mat{B}$ are system matrices.

% Subsection 1.2
\subsection{Second Subsection}
\label{subsec:subsection1-2}

Content for subsection 1.2. Example of referencing an equation: As shown in \cref{eq:state-space}, the system dynamics...

% Example: Multi-line equation
\begin{align}
    \dot{x}_1 &= x_2 \label{eq:system-1} \\
    \dot{x}_2 &= f(\vect{x}) + g(\vect{x})u \label{eq:system-2}
\end{align}

% ==============================================================================
% SECTION 2 - With Figure Example
% ==============================================================================

\section{Second Section Title}
\label{sec:section2}

This section discusses [topic]. \cref{fig:example-figure} illustrates the concept.

% Single figure example
\begin{figure}[htbp]
    \centering
    \includegraphics[width=0.8\textwidth]{figures/example_figure.pdf}
    \caption{Example figure caption. Describe what the figure shows and its significance.}
    \label{fig:example-figure}
\end{figure}

% Subsection 2.1 - With subfigures
\subsection{Comparison Analysis}
\label{subsec:comparison}

\cref{fig:comparison} compares two approaches.

\begin{figure}[htbp]
    \centering
    \begin{subfigure}[b]{0.45\textwidth}
        \centering
        \includegraphics[width=\textwidth]{figures/approach_a.pdf}
        \caption{Approach A}
        \label{fig:approach-a}
    \end{subfigure}
    \hfill
    \begin{subfigure}[b]{0.45\textwidth}
        \centering
        \includegraphics[width=\textwidth]{figures/approach_b.pdf}
        \caption{Approach B}
        \label{fig:approach-b}
    \end{subfigure}
    \caption{Comparison of two approaches. (a) Approach A shows... (b) Approach B demonstrates...}
    \label{fig:comparison}
\end{figure}

% ==============================================================================
% SECTION 3 - With Table Example
% ==============================================================================

\section{Third Section Title}
\label{sec:section3}

This section presents results summarized in \cref{tab:example-table}.

% Table example with booktabs
\begin{table}[htbp]
    \centering
    \caption{Example table caption. Describe the data and its source.}
    \label{tab:example-table}
    \begin{tabular}{lcccc}
        \toprule
        \textbf{Controller} & \textbf{Settling Time (s)} & \textbf{Overshoot (\%)} & \textbf{Energy (J)} \\
        \midrule
        Classical SMC       & 3.45                       & 12.3                    & 245                 \\
        STA-SMC            & 2.78                       & 8.5                     & 198                 \\
        Adaptive SMC       & 2.91                       & 9.2                     & 210                 \\
        Hybrid             & 2.62                       & 7.8                     & 185                 \\
        \bottomrule
    \end{tabular}
\end{table}

% Subsection 3.1 - With algorithm
\subsection{Algorithm Description}
\label{subsec:algorithm}

\cref{alg:example-algorithm} presents the pseudocode.

\begin{algorithm}[htbp]
    \caption{Example Algorithm}
    \label{alg:example-algorithm}
    \begin{algorithmic}[1]
        \Require State $\vect{x}$, Gains $\vect{K}$
        \Ensure Control input $u$
        \State Compute sliding surface: $s = \vect{c}^T \vect{x}$
        \If{$|s| > \phi$}
            \State $u = -K \cdot \sign(s)$
        \Else
            \State $u = -K \cdot s / \phi$
        \EndIf
        \Return $u$
    \end{algorithmic}
\end{algorithm}

% ==============================================================================
% SECTION 4 - With Code Listing
% ==============================================================================

\section{Implementation Details}
\label{sec:implementation}

The implementation in Python is shown in \cref{lst:example-code}.

\begin{lstlisting}[language=Python, caption={Example Python code}, label={lst:example-code}, style=python]
def compute_control(state, gains):
    """
    Compute SMC control input.

    Args:
        state: State vector [x1, x2, ...]
        gains: Control gains [c1, c2, ..., K]

    Returns:
        u: Control input
    """
    # Compute sliding surface
    s = np.dot(gains[:-1], state)

    # Compute control (sign function)
    u = -gains[-1] * np.sign(s)

    return u
\end{lstlisting}

% ==============================================================================
% SECTION 5 - With Theorem/Proof
% ==============================================================================

\section{Theoretical Analysis}
\label{sec:theory}

\begin{theorem}[Stability of Sliding Mode]
\label{thm:smc-stability}
Consider the system described by \cref{eq:state-space}. If the sliding surface is designed such that $s = \vect{c}^T \vect{x} = 0$ is stable, and the control law satisfies $s\dot{s} < 0$, then the system converges to the sliding surface in finite time.
\end{theorem}

\begin{proof}
Consider the Lyapunov function candidate:
\begin{equation}
    V = \frac{1}{2}s^2
\end{equation}

Taking the time derivative:
\begin{align}
    \dot{V} &= s\dot{s} \\
    &< 0 \quad \text{(by assumption)}
\end{align}

Thus, $V$ is strictly decreasing, proving convergence to $s = 0$.
\end{proof}

% ==============================================================================
% CHAPTER SUMMARY (1-2 paragraphs)
% ==============================================================================

\section{Summary}
\label{sec:summary}

This chapter presented [summary of key points]:
\begin{itemize}
    \item Key finding 1: [Brief description]
    \item Key finding 2: [Brief description]
    \item Key finding 3: [Brief description]
\end{itemize}

The next chapter will discuss [preview of next chapter].

% ==============================================================================
% END OF CHAPTER
% ==============================================================================
